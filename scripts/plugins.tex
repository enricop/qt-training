\subsection{Plug-ins}
\label{plugins}


%----------------------------------------------------------------------
\begin{slide}{0365}\frametitle{Conceptual View}
\xConcept{Plug-ins!Concept}
\begin{itemize}
\item A plugin is a dynamically loadable module (\texttt{.dll} on Microsoft
  Windows, \texttt{.so} on UNIX, and \texttt{.dylib} on Macintosh).
\item A host application loads the plugins into its address space, and can
  control when they are supposed to do their job.
\end{itemize}
% \image{figures/pluginview} %PENDING(blackie) Readd
\end{slide}


%----------------------------------------------------------------------
\begin{slide}[fragile]{0366}
\frametitle{Overview}
\begin{itemize}
\item It is possible to extend Qt in several ways using plugins, these
  include plugins for the image loader, styles, new SQL drivers, and
  plugins for Qt designer.
\item It is also possible to develop a plugin mechanism for your
  own application using building blocks from Qt.
\end{itemize}
\end{slide}

%----------------------------------------------------------------------
\begin{slide}{0367}\frametitle{Extending Qt with Plugins}
\begin{itemize}
\item Developing a plugin for Qt consists of three steps:
  \begin{enumerate}
  \item Inherit an abstract base class and implement it.
  \item Add a factory macro to your implementation file.
  \item Set up an appropriate QMake file.
\end{enumerate}
\end{itemize}
\end{slide}

%----------------------------------------------------------------------
\begin{slide}{0368}\frametitle{Extending Qt with Plugins - Step 1}
\xConcept{Plug-ins!Qt's own}
\begin{itemize}
\item The following base classes exist: \iCls{QAccessibleBridgePlugin},
  \iCls{QAccessiblePlugin}, \iCls{QDecorationPlugin},
  \iCls{QGfxDriverPlugin}, \iCls{QIconEnginePlugin},
  \iCls{QImageIOPlugin}, \iCls{QInputContextPlugin},
  \iCls{QKbdDriverPlugin}, \iCls{QMouseDriverPlugin},
  \iCls{QPictureFormatPlugin}, \iCls{QSqlDriverPlugin},
  \iCls{QStylePlugin}, \iCls{QTextCodecPlugin}.
\item To implement e.g. a new style, you must inherit
  \iCls{QStylePlugin}, and implement the two pure virtual methods
  \iClsFn{QStylePlugin}{keys} and \iClsFn{QStylePlugin}{create}.
\end{itemize}
\end{slide}

%----------------------------------------------------------------------
\begin{slide}{0369}\frametitle{Extending Qt with Plugins - Step 2}
\xConcept{Plug-ins!Exporting}
\begin{itemize}
\item In your implementation file, you must add the macro
  \iCls{Q\_EXPORT\_PLUGIN}, passing the class name as a parameter.
\item This allows Qt to load the macro from the dynamic library, into which
  it will be compiled.
\item Example: \texttt{Q\_EXPORT\_PLUGIN(MyStylePlugin)}
\end{itemize}
\end{slide}

%----------------------------------------------------------------------
\begin{slide}[fragile]{0370} 
\frametitle{Extending Qt with Plugins - Step 3}
\xConcept{Plug-ins!QMake}\xConcept{QMake!Plug-ins}
\begin{itemize}
\item The QMake project file should contain the following two lines:
\begin{alltt}
  TEMPLATE = lib
  CONFIG += plugin
\end{alltt}
\item The QMake project file should place the dynamic library in a special
  directory, as specified on the next page.
\item Do so using one of QMake's \texttt{DESTDIR} or \texttt{INSTALL}
  variables.
\end{itemize}
\end{slide}

%----------------------------------------------------------------------
\begin{slide}{0371}\frametitle{Searching for Plugins}
\xConcept{Plug-ins!Searching for}
\xConcept{Plug-ins!Loading}
\begin{itemize}
\item Qt will search for plugins in several different directory trees:
  \begin{itemize}
  \item The directory of the application executable. (As returned from \iClsFn{QCoreApplication}{applicationDirPath}).
  \item The directory returned by
    \iClsFnPar{QLibraryInfo}{location}{QLibraryInfo::PluginsPath} - usually
    \texttt{plugins}.
  \item Directories added by \iClsFn{QCoreApplication}{addLibraryPath} or
    set by \iClsFn{QCoreApplication}{setLibraryPaths}
  \end{itemize}

\item Each plugin type has its own subdirectory, style plugins are e.g.\ located in
  \texttt{styles} subdirectory: \texttt{plugins/styles/}
\end{itemize}
\end{slide}

%----------------------------------------------------------------------
\begin{slide}{0372}
\frametitle{The Build Key}
\xConcept{Plug-ins!The Build Key}
\begin{itemize}
\item To avoid that Qt loads a plugin not compatible with the
  application, a build key is compiled into the plugin.
\item The following aspects are compared between the
  current Qt version and the plugin:
  \begin{itemize}
  \item The Qt version.
  \item The architecture, operating system, and compiler.
  \item The compilation flags for Qt. 
  \item An optional string that can be specified when running Qt's
    configure script (\texttt{--buildkey} option).
  \end{itemize}
\end{itemize}
\end{slide}

%----------------------------------------------------------------------
\begin{slide}[fragile]{0373}
\frametitle{Extending Your Own Applications Using Plugins}
\xConcept{Plug-ins!Extending your own App.}
\begin{itemize}
\item Using the building blocks from Qt, it is easy to develop
  your own plugin mechanism, which allows your application to be extended
  using plugins. 
\item There are two sides to plugins:
  \begin{itemize}
  \item Making the application ready for plugins.
  \item Implementing actual plugins.
  \end{itemize}
\end{itemize}
\end{slide}

%----------------------------------------------------------------------
\begin{slide}{0374}
\frametitle{Offering and Loading Plugins}
\begin{itemize}
\item Steps to making the application extensible with plugins:
  \begin{enumerate}
  \item Define a set of interfaces (classes with pure virtual functions).
  \item Use the \iCls{Q\_DECLARE\_INTERFACE()} macro to tell Qt about the
    interface.
  \item Use \iCls{QPluginLoader} in the application to load the plugin.
  \item Use \texttt{qobject\_cast()} to test whether a plugin implements a
    certain interface. (see \pageHyperlink{casting}).
  \end{enumerate}
\end{itemize}
\end{slide}

%----------------------------------------------------------------------
\begin{slide}[fragile]{0375}
\frametitle{Interface Example}
\xExample{Plug-Ins!Interface}
\begin{alltt}\small
class FilterInterface
\{
public:
    virtual QStringList filters() const = 0;
    virtual QImage filterImage(const QString &filter, 
                               const QImage &image,
                               QWidget *parent) = 0;
\};

Q_DECLARE_INTERFACE(FilterInterface,
    "com.trolltech.PlugAndPaint.FilterInterface/1.0")
  \end{alltt}
\end{slide}

%----------------------------------------------------------------------
\begin{slide}{0376}\frametitle{Offering and Loading Plugins}
\begin{itemize}
\item The interface may only contain pure virtual classes, otherwise we
  would have to deal with exporting symbols to get it working on Windows
  and newer GCC's.
\item The first argument to the \iCls{Q\_DECLARE\_INTERFACE()} is the
  class name implementing the interface.
\item The second argument must be a unique string identifying the
  interface.
\item The interface may not inherit \iCls{QObject}, as this, among other
  things, would make the class non-pure virtual.
\end{itemize}
\end{slide}

%----------------------------------------------------------------------
\begin{slide}[fragile]{0377}
\frametitle{Loading Plugins Example}
\xExample{Plug-Ins!Loading}
  \begin{alltt}\small
foreach (QString fileName, pluginsDir.entryList(QDir::Files)) \{
    QPluginLoader loader(pluginsDir.absoluteFilePath(fileName));
    QObject* plugin = loader.instance();
    if (plugin) \{
        FilterInterface* iFilter =
              qobject_cast<FilterInterface*>(plugin);
        if (iFilter)
            \ldots
    \}
\}
\end{alltt}
\end{slide}



%----------------------------------------------------------------------
\begin{slide}{0378}
\frametitle{Implementing Actual Plugins}
\xConcept{Plug-ins!Implementing}
\begin{itemize}
\item Steps to implement an actual plugin:
  \begin{enumerate}
  \item Create a class inheriting from QObject and the required interface.
  \item Use \iCls{Q\_INTERFACES()} to tell Qt about the implemented interface.
  \item Export the plugin using \iCls{Q\_EXPORT\_PLUGIN}
  \item Build the plugin using a suitable \texttt{.pro} file.
 \end{enumerate}
\end{itemize}
\end{slide}



%----------------------------------------------------------------------
\begin{slide}[fragile]{0379}
\frametitle{Example Plugin}
\xExample{Plug-Ins!Example plug-in}
  \begin{alltt}\small
class ExtraFiltersPlugin : public QObject, 
                           public FilterInterface
\{
    Q_OBJECT
    Q_INTERFACES(FilterInterface)

public:
    QStringList filters() const;
    QImage filterImage(const QString &filter, 
                       const QImage &image,
                       QWidget *parent);
\};
  \end{alltt}
\end{slide}

%----------------------------------------------------------------------
\begin{slide}{0380}\frametitle{Implementing Actual Plugins}
\begin{itemize}
\item In case the plugin implemented several interfaces, simply list them as a
  space-separated list using \iCls{Q\_INTERFACES}.
\item The interface class may not inherit \iCls{QObject}, but the actual
  plugin must.
\item In one of the implementation files for the plugin, the following code
  must be placed: \texttt{Q\_EXPORT\_PLUGIN(ExtraFiltersPlugin)}.
\end{itemize}
\end{slide}

%----------------------------------------------------------------------
\begin{slide}{0381}\frametitle{Signals/Slots}
\xConcept{Plug-ins!Signal/Slots}
\xConcept{Signals and Slots!Plug-ins}
\begin{itemize}
\item Interfaces may not inherit \iCls{QObject}, so you can expect that
  it is not possible to use signal/slots with interfaces.
\item The introspection of QObject works, however, with dynamic types
  rather than static types, so even if you only have a pointer to a
  QObject, you can actually connect to signals of the real type of the object.
\end{itemize}
\end{slide}

%----------------------------------------------------------------------
\begin{slide}[fragile]{0382}
\frametitle{Signals/Slots}
\begin{itemize}
\item 
\begin{alltt}
class FilterInterface
\{
public:
     ...
    // slots:
    virtual void someSlot() = 0;
    // signals:
    // void someSignal();
\};
\end{alltt}
\item The above is only an informal contract that the users of the interface
  must fulfill.
\end{itemize}
\end{slide}


%----------------------------------------------------------------------
\begin{slide}[fragile]{0383} \frametitle{\iConcept{Casting} using qobject\_cast et. al.}
\label{casting}
\begin{itemize}
\item C++ has four casting operators \iConcept{static\_cast},
  \iConcept{dynamic\_cast}, \iConcept{const\_cast},
  \iConcept{reinterpret\_cast}.
\item \iConcept{dynamic\_cast} can check if what is being casted
  to actually is of the given type:
\begin{alltt}
X* a = dynamic\_cast<X*>( b );
if ( a )
  \ldots
\end{alltt}
\item \texttt{dynamic\_cast} does, however, not work across dynamic library
  boundaries on all compilers/platforms.
\end{itemize}
\end{slide}

%----------------------------------------------------------------------
\begin{slide}[fragile]{0384}\frametitle{Casting using qobject\_cast et. al.}
\begin{itemize}
\item Qt comes with \iConcept{qobject\_cast}, \iConcept{qvariant\_cast}, and
  \iConcept{qstyleoption\_cast}. 
\item These cast methods use internals from the
  respective classes to do the cast, and thus solve the dynamic library
  boundary problems.
\item Example:
\begin{alltt}
\iCls{QPushButton}* button = qobject_cast<QPushButton *>(obj);
if ( button )
  \ldots
\end{alltt}
\item \pleaseNote the casting functions only work within their respective
  classes, thus you can only use \texttt{qobject\_cast} with
  \iCls{QObject} subclasses.
\end{itemize}
\end{slide}

%----------------------------------------------------------------------
\begin{slide}{0385}\frametitle{Project Task: Add Plugins to the Paint Program}
\xProject{Plug-Ins}
\begin{itemize}
\item In the handout directory \emph{other-topics/lab-plugins}, you
  can find a slightly modified version of the, by now well-known, paint
  program.
\item Add plugin capabilities to the application and implement circle
  drawing in the plugin.
\item Optional: Develop other plugins, e.g. a house drawing plugin.
\end{itemize}
\end{slide}
