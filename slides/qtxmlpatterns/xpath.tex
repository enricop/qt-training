\label{xpath}
\subsubsection{XPath}

%----------------------------------------------------------------------
\begin{slide}{8001}
\frametitle{What is XPath?}
\begin{itemize}
\item Syntax for defining parts of an XML document
\item Uses path expressions to navigate
\item Provides a library of standard functions
\item Major element in \iConcept{XQuery} and \iConcept{XSLT}
\item W3C recommendation
\end{itemize}
\end{slide}


%----------------------------------------------------------------------
\begin{slide}{8002}
\frametitle{XPath Path Expressions and Functions}
\begin{itemize}
\item Path expressions select nodes or node-sets
\item Path expressions look similar to file system paths
\item Includes over 100 built-in functions for
\begin{itemize}
 \item strings
 \item numeric values
 \item date and time comparison
 \item node manipulation
 \item and much more...
\end{itemize}
\end{itemize}
\end{slide}


%----------------------------------------------------------------------
\begin{slide}{8003}
\frametitle{Why Learn XPath?}
\begin{itemize}
\item XPath is used in XQuery and XLST
\item XQuery is built on XPath expressions
\item XQuery and XPath share the same data model, functions and operators
\item Without XPath you will not be able to create XSLT documents
\end{itemize}
\end{slide}


%----------------------------------------------------------------------
\begin{slide}[fragile]{8004}
\frametitle{XPath Terminology}
\begin{itemize}
\lstset{basicstyle=\fontsize{8}{9}\selectfont}
\item[] 
\begin{xml}
<?xml version="1.0" encoding="ISO-8859-1"?>
<!DOCTYPE bookstore>
<?MyBookStoreApp master-price-list?>
<bookstore xmlns="http://www.kdab.com/namespaces/bookstore">
    <!-- Books go here -->
    <book>
        <title lang="en">Harry Potter</title>
        <author>J K. Rowling</author>
        <price>29.99</price>
    </book>
</bookstore>
\end{xml}
\vspace{4pt}
 \item Element
 \item Attribute
 \item Text
 \item Namespace
 \item Processing-instruction
\end{itemize}
\end{slide}


%----------------------------------------------------------------------
\begin{slide}{8005}
\frametitle{XPath Terminology Cont'd.}
\begin{itemize}

\item Comment
\item Document node
\item Atomic values
\begin{itemize}
 \item \emph{Atomic values} are nodes with no children or parent.
 \item Example of \emph{atomic values:} ``J K. Rowling'', ``en''
\end{itemize}

\item Atomic values and nodes referred to as \emph{items}.
\end{itemize}
\end{slide}


%----------------------------------------------------------------------
\begin{slide}[fragile]{8006}
\frametitle{Relationships Between Nodes}
\begin{itemize}
\item[]
\begin{xml}
<bookstore>
  <book>
    <title>Harry Potter</title>
    <author>J K. Rowling</author>
    <year>2005</year>
    <price>29.99</price>
  </book>
</bookstore>
\end{xml}
\vspace{8pt}
\item Parent
\item Child
\item Siblings
\item Ancestor
\item Descendants
\end{itemize}
\end{slide}


%----------------------------------------------------------------------
\begin{slide}[fragile]{8007}
\frametitle{Example: XPath Syntax}
\begin{itemize}
\lstset{basicstyle=\fontsize{8}{9}\selectfont}
\item[]
\begin{xml}
<?xml version="1.0" encoding="ISO-8859-1"?>
<bookstore>
    <book>
        <title lang="eng">Harry Potter</title>
        <price>29.99</price>
    </book>
    <book>
        <title lang="eng">Learning XML</title>
        <price>39.95</price>
    </book>
</bookstore>
\end{xml}
\setfont[12]{8}
\begin{center}
\begin{tabular}{|l|l|}
\hline
\textbf{Expression} & \textbf{Result} \\ \hline
bookstore           & The child nodes of the bookstore element \\ \hline
/bookstore          & The root element bookstore \\ \hline
bookstore/book      & Books that are children of bookstore \\ \hline
//book              & Book elements from anywhere\\ \hline
bookstore//book     & Book descendants of the bookstore \\ \hline
//@lang             & All lang attributes \\ \hline
\end{tabular}
\end{center}
\setfont{10}
\end{itemize}
\end{slide}


%----------------------------------------------------------------------
\begin{slide}[fragile]{8008}
\frametitle{XPath Syntax}
\begin{itemize}
\item Uses path expressions to select nodes or node-sets
\item Node selected by following a path (sequence of steps)
\item The most useful path expressions are:
\setfont[12]{8}
\begin{center}
\begin{tabular}{|l|l|}
\hline
\textbf{Expression}  & \textbf{Description} \\ \hline
nodename    & Selects all child nodes of the named node \\ \hline
/           & Selects from the root node \\ \hline
//          & Selects nodes that match the selection \\
            & anywhere in the document \\ \hline
.           & Selects the current node \\ \hline
..          & Selects the parent of the current node \\ \hline
@           & Selects attributes \\ \hline
\end{tabular}
\end{center}
\setfont{10}
\end{itemize}
\end{slide}


%----------------------------------------------------------------------
\begin{slide}[fragile]{8009}
\frametitle{Predicates}
\begin{itemize}
\item Used to find a specific node
\item Used to find a node containing a specific value
\item Loosely analogous to WHERE clause in SQL
\item Appear within square brackets e.g.:
\vspace{8pt}
\begin{xml}
 book[value<30.0]
\end{xml}
\end{itemize}
\end{slide}


%----------------------------------------------------------------------
\begin{slide}[fragile]{8010}
\frametitle{Example: Predicates}
\begin{itemize}
\item Some expressions using predicates and their results:
\setfont[12]{8}
\begin{center}
\begin{tabular}{|l|l|}
\hline
\textbf{Expression}                 & \textbf{Result} \\ \hline
/bookstore/book[1]                  & First book child of bookstore \\ \hline
/bookstore/book[last()]             & Last book child of bookstore \\ \hline
/bookstore/book[last()-1]           & Penultimate book child of bookstore \\ \hline
/bookstore/book[position()<3]       & First two book children of bookstore \\ \hline
//title[@lang]                      & Titles with a lang attribute \\ \hline
//title[@lang='eng']                & Titles with a lang attribute of `eng' \\ \hline
/bookstore/book[price>35.00]        & Books costing > 35.00 \\ \hline
/bookstore/book[price>35.00]/title  & Titles of books costing > 35.00 \\ \hline
\end{tabular}
\end{center}
\setfont{10}
\item Note the 1-based indexing
\end{itemize}
\end{slide}


%----------------------------------------------------------------------
\begin{slide}[fragile]{8019}
\frametitle{XPath Functions}
\begin{itemize}
\item Used to manipulate node-sets, perform actions or type conversions
\item Important functions include:
\begin{itemize}
 \item doc(fileName)
 \item[]
 \lstset{basicstyle=\fontsize{8}{9}\selectfont}
 \begin{xml}
  doc('bookstore.xml')/bookstore/book
 \end{xml}
 \item number(arg)
 \item[]
 \lstset{basicstyle=\fontsize{8}{9}\selectfont}
 \begin{xml}
  number('100')
 \end{xml}
 \item concat(arg1, arg2, ...)
 \item[]
 \lstset{basicstyle=\fontsize{8}{9}\selectfont}
 \begin{xml}
  concat('Hello ', 'World!')
 \end{xml}
\end{itemize}
\item Function reference: \url{http://www.w3.org/TR/xpath/#corelib}
\end{itemize}
\end{slide}


%----------------------------------------------------------------------
\begin{slide}[fragile]{8011}
\frametitle{Unknown Nodes}
\begin{itemize}
\item Wildcards can be used to select unknown nodes
\setfont[12]{8}
\begin{center}
\begin{tabular}{|l|l|}
\hline
\textbf{Wildcard}   & \textbf{Description} \\ \hline
*                   & Matches any element node \\ \hline
@*                  & Matches any attribute node \\ \hline
node()              & Matches any node of any kind \\ \hline
\end{tabular}
\end{center}
\setfont{10}

\item Some examples:

\setfont[12]{8}
\begin{center}
\begin{tabular}{|l|l|}
\hline
\textbf{Expression}     & \textbf{Result} \\ \hline
/bookstore/*            & All the child nodes of the bookstore element \\ \hline
//*                     & All elements in the document \\ \hline
//title[@*]             & All title elements which have any attribute \\ \hline
\end{tabular}
\end{center}
\setfont{10}

\end{itemize}
\end{slide}


%----------------------------------------------------------------------
\begin{slide}[fragile]{8012}
\frametitle{Multiple Paths}
\begin{itemize}
\item Use the | operator to select multiple expressions:
\setfont[12]{8}
\begin{center}
\begin{tabular}{|l|l|}
\hline
\textbf{Expression}             & \textbf{Result} \\ \hline
//book/title | //book/price     & All the title AND \\
                                & price elements of all books \\ \hline
//title | //price               & All the title AND \\
                                & price elements in the document \\ \hline
/bookstore/book/title | //price & All titles of the books of the bookstore AND \\
                                & all price elements in the document  \\ \hline
\end{tabular}
\end{center}
\setfont{10}
\end{itemize}
\end{slide}


%----------------------------------------------------------------------
\begin{slide}[fragile]{8013}
\frametitle{Example: Location Path Expressions}
\begin{itemize}
\item Syntax for a location step is:
\vspace{4pt}
\begin{cpp}
 axisname::nodetest[predicate]
\end{cpp}

\setfont[12]{8}
\begin{center}
\begin{tabular}{|l|l|}
\hline
\textbf{Example}        & \textbf{Result} \\ \hline
child::book             & All book nodes that are children of the current node \\ \hline
attribute::lang         & lang attribute of the current node \\ \hline
child::*                & All element children of the current node \\ \hline
attribute::*            & All attributes of the current node \\ \hline
child::text()           & All text node children of the current node \\ \hline
child::node()           & All children of the current node \\ \hline
descendant::book        & All book descendants of the current node \\ \hline
ancestor::book          & All book ancestors of the current node \\ \hline
ancestor-or-self::book  & All book ancestors of the current node \\
                        & and the current node if it is a book \\ \hline
child::*/child::price   & All price grandchildren of the current node \\ \hline
\end{tabular}
\end{center}
\setfont{10}

\end{itemize}
\end{slide}


%----------------------------------------------------------------------
\begin{slide}[fragile]{8014}
\frametitle{XPath Axes}
\begin{itemize}
\item An axis defines a node-set relative to the current node:
\setfont[12]{8}
\begin{center}
\begin{tabular}{|l|l|}
\hline
\textbf{Axis Name}  & \textbf{Result} \\ \hline
ancestor            & All ancestors of the current node \\ \hline
ancestor-or-self    & All ancestors of the current node and the current node itself \\ \hline
attribute           & All attributes of the current node \\ \hline
child               & All children of the current node \\ \hline
descendant          & All descendants of the current node \\ \hline
descendant-or-self  & All descendants of the current node and the current node itself \\ \hline
following           & Everything after the closing tag of the current node \\ \hline
following-sibling   & All siblings after the current node \\ \hline
namespace           & All namespace nodes of the current node \\ \hline
parent              & The parent of the current node \\ \hline
preceding           & Everything before the start tag of the current node \\ \hline
preceding-sibling   & All siblings before the current node \\ \hline
self                & The current node \\ \hline
\end{tabular}
\end{center}
\setfont{10}
\end{itemize}
\end{slide}


%----------------------------------------------------------------------
\begin{slide}[fragile]{8015}
\frametitle{XPath Examples}
\begin{itemize}
\lstset{basicstyle=\fontsize{8}{9}\selectfont}
\item[]
\begin{xml}
<?xml version="1.0" encoding="ISO-8859-1"?>
<bookstore>
    <book category="Cooking">
        <title lang="en">Everyday Italian</title>
        <author>Giada De Laurentiis</author>
        <year>2005</year>
        <price>30.00</price>
    </book>
    ...
    <book category="Web">
        <title lang="en">Learning XML</title>
        <author>Erik T. Ray</author>
        <year>2003</year>
        <price>39.95</price>
    </book>
</bookstore>
\end{xml}
\vspace{8pt}
\item Select all titles:
\begin{verbatim}
 /bookstore/book/title
 //title
\end{verbatim}
\end{itemize}
\end{slide}


%----------------------------------------------------------------------
\begin{slide}[fragile]{8016}
\frametitle{XPath Examples Cont'd.}
\begin{itemize}
\lstset{basicstyle=\fontsize{8}{9}\selectfont}
\item[]
\begin{xml}
<?xml version="1.0" encoding="ISO-8859-1"?>
<bookstore>
    <book category="Cooking">
        <title lang="en">Everyday Italian</title>
        <author>Giada De Laurentiis</author>
        <year>2005</year>
        <price>30.00</price>
    </book>
    ...
    <book category="Web">
        <title lang="en">Learning XML</title>
        <author>Erik T. Ray</author>
        <year>2003</year>
        <price>39.95</price>
    </book>
</bookstore>
\end{xml}
\vspace{8pt}
\item Select the title of the first book:
\begin{verbatim}
 /bookstore/book[1]/title
\end{verbatim}
\end{itemize}
\end{slide}


%----------------------------------------------------------------------
\begin{slide}[fragile]{8017}
\frametitle{XPath Examples Cont'd.}
\begin{itemize}
\lstset{basicstyle=\fontsize{8}{9}\selectfont}
\item[]
\begin{xml}
<?xml version="1.0" encoding="ISO-8859-1"?>
<bookstore>
    <book category="Cooking">
        <title lang="en">Everyday Italian</title>
        <author>Giada De Laurentiis</author>
        <year>2005</year>
        <price>30.00</price>
    </book>
    ...
    <book category="Web">
        <title lang="en">Learning XML</title>
        <author>Erik T. Ray</author>
        <year>2003</year>
        <price>39.95</price>
    </book>
</bookstore>
\end{xml}
\vspace{8pt}
\item Select the book titles where category = 'Web'
\begin{verbatim}
 /bookstore/book[@category='Web']/title
\end{verbatim}
\end{itemize}
\end{slide}


%----------------------------------------------------------------------
\begin{slide}[fragile]{8018}
\frametitle{XPath Examples Cont'd.}
\begin{itemize}
\lstset{basicstyle=\fontsize{8}{9}\selectfont}
\item[]
\begin{xml}
<?xml version="1.0" encoding="ISO-8859-1"?>
<bookstore>
    <book category="Cooking">
        <title lang="en">Everyday Italian</title>
        <author>Giada De Laurentiis</author>
        <year>2005</year>
        <price>30.00</price>
    </book>
    ...
    <book category="Web">
        <title lang="en">Learning XML</title>
        <author>Erik T. Ray</author>
        <year>2003</year>
        <price>39.95</price>
    </book>
</bookstore>
\end{xml}
\vspace{8pt}
\item Select title nodes with price > 35 and price < 40
\begin{verbatim}
 /bookstore/book[price>35 and price<40]/title
\end{verbatim}
\end{itemize}
\end{slide}
