%%%%%%%%%%%%%%%%%%%%%%%%%%%%%%%%%%%%%%%%%%%%%%%%%%%%%%%%%%%%%%%%%%%%%%%%%%
%
% Copyright (c) 2013, Integrated Computer Solutions
% All rights reserved.
%
% This work, unless otherwise expressly stated, is licensed under a
% Creative Commons Attribution-ShareAlike 2.5.
%
% The full license document is available from
% http://creativecommons.org/licenses/by-sa/2.5/legalcode .
%
%%%%%%%%%%%%%%%%%%%%%%%%%%%%%%%%%%%%%%%%%%%%%%%%%%%%%%%%%%%%%%%%%%%%%%%%%%

\subsection{Qt Multithreading}

%----------------------------------------------------------------------

\begin{slide}
\frametitle{Terminology}

\begin{itemize}
\item \textbf{Multi-tasking} - Multiple programs running on the same computer
\item \textbf{Multi-threading} - Multiple threads running in the same program
\item \textbf{Multi-processing} - Multiple threads running on multiple processors
\vspace{1em}
\item \textbf{Reentrant function}
  \begin{itemize}
  \item May be called simultaneously from multiple threads
  \item ...but only if each invocation references unique data
  \end{itemize}
\item \textbf{Thread safe function}
  \begin{itemize}
  \item May be called simultaneously from multiple threads
  \item ...even if each invocation references shared data
  \end{itemize}
\end{itemize}

\end{slide}

%----------------------------------------------------------------------

\begin{slide}
\frametitle{Disclaimers}

\begin{itemize}
\item Multithreading is not yet mainstream
  \begin{itemize}
  \item Multi-processor systems are only recently commonplace
  \item Programming classes tend not to cover multi-threading
  \end{itemize}
\item Most threading facilities are very low level
\item Race conditions, thread starvation, and deadlocks, oh my!
\item "Multi-threading is like chess"
  \begin{itemize}
  \item Easy to learn but hard to master
  \end{itemize}
\end{itemize}

\end{slide}

%----------------------------------------------------------------------

\begin{slide}
\frametitle{How Threads Work}

\begin{itemize}
\item Independent streams of instructions (threads of execution)
\item Does not require multi-processors
  \begin{itemize}
  \item While one thread waits on IO, a second thread can use the CPU
  \end{itemize}
\item Each thread has its own context
  \begin{itemize}
  \item Stack, stack pointer, instruction pointer, registers
  \end{itemize}
\item Threads share resources with other threads
  \begin{itemize}
  \item Shared data, locks, conditions
  \end{itemize}
\vspace{1em}
\item Modern operating systems use preemption to schedule threads
  \begin{itemize}
  \item Interrupt thread and save its context
  \item Restore context of next thread and resume execution
  \item Threads can also relinquish execution by waiting
  \end{itemize}
\end{itemize}

\end{slide}

%----------------------------------------------------------------------

\begin{slide}
\frametitle{How Threads Work}

\vspace{3em}
\centeredImage{multithreading/images/multithreaded_process}

\end{slide}

%----------------------------------------------------------------------

\begin{slide}
\frametitle{Qt Threading Guidelines}

\begin{itemize}
\item The main thread is the GUI thread
  \begin{itemize}
  \item Only access QWidget and QPixmap objects from the main thread
  \item It is safe to use \iCls{QPainter} with \iCls{QImage}, however
  \end{itemize}
\item Beware of blocking in the main thread
  \begin{itemize}
  \item Waiting in the main thread is inappropriate
  \item But it's perfectly fine in other threads
  \end{itemize}
\item All implicitly shared Qt classes are \textit{reentrant}
\item Do not mistake thread objects for the actual thread of execution
  \begin{itemize}
  \item Objects are data structures not instruction streams
  \end{itemize}
\end{itemize}

\end{slide}

%----------------------------------------------------------------------

\begin{slide}
\frametitle{Simple Threading - QThreadPool}

\begin{itemize}
\item \iCls{QThreadPool} manages a collection of threads
  \begin{itemize}
  \item One thread per core (real or virtual) by default
  \item On some system this may be as low as one thread
  \end{itemize}
\item Can change the number of available threads with:
  \begin{itemize}
  \item \iClsFn{QThreadPool}{setMaxThreadCount} - set the thread count
  \item \iClsFn{QThreadPool}{reserveThread} - increment thread count
  \item \iClsFn{QThreadPool}{releaseThread} - decrement thread count
  \item \textit{A higher number of threads is not necessarily efficient}
  \end{itemize}
\item Suitable for CPU-bound tasks, not general purpose threading
\item Each application has a global thread pool accessible with
      \iClsFn{QThreadPool}{globalInstance}
\end{itemize}

\end{slide}

%----------------------------------------------------------------------


\begin{slide}[fragile]
\frametitle{Thread Pool Tasks - QRunnable}

\begin{itemize}
\item Tasks are implemented by subclassing \iCls{QRunnable}
\begin{cpp}
class HelloWorldTask : public QRunnable
{
    void run() { qDebug() << "Hello world!"; }
}
\end{cpp}
\item Start a \iCls{QRunnable} task using:
  \begin{itemize}
  \item \iClsFnPar{QThreadPool}{start}{QRunnable *task)}
    \begin{itemize}
    \item Will queue task of no thread is available
    \end{itemize}
  \item \iClsFnPar{QThreadPool}{tryStart}{QRunnable *task)}
    \begin{itemize}
    \item Will return false if no thread is available
    \end{itemize}
  \end{itemize}
\item \iCls{QThreadPool} takes ownership of \iCls{QRunnable} tasks
  \begin{itemize}
  \item Task deleted by default when \hClsFn{QRunnable}{run} exits
  \end{itemize}
\end{itemize}

\end{slide}

%----------------------------------------------------------------------

\begin{slide}
\frametitle{Thread Pool Notes}

\begin{itemize}
\item \iCls{QRunnable} is kept small and efficient
  \begin{itemize}
  \item No communication mechanisms
  \item No way to cancel a running task
  \item Won't be informed when task completes
  \end{itemize}
\item Do not block in tasks
  \begin{itemize}
  \item There are a limited number of threads in the pool
  \end{itemize}
\item Good for simple "Fire and Forget" style tasks
\vspace{1em}
\item \iCls{QThreadPool} is the foundation for \iCls{QtConcurrent} module
\end{itemize}

\end{slide}

%----------------------------------------------------------------------

\begin{slide}
\frametitle{Threading Workhorse - QThread}

Two models for using QThread:
\begin{itemize}
\item Java model: subclass and re-implement the \hClsFn{QThread}{run} method
  \begin{itemize}
  \item Traditional model for using \iCls{QThread}
  \item Using \iCls{QRunnable} is easier and less error prone
  \end{itemize}
\item Event-Driven model: manage threads through events and signals
  \begin{itemize}
  \item The new accepted model for using \iCls{QThread}
  \item Requires an understanding of thread affinity and communications
  \end{itemize}
\end{itemize}

\vspace{1em}
How to use QThread has changed over the years \\
\linkbutton{http://blog.qt.digia.com/blog/2010/06/17/youre-doing-it-wrong/}
\linkbutton{http://mayaposch.wordpress.com/2011/11/01/how-to-really-truly-use-qthreads-the-full-explanation/}

\end{slide}

%----------------------------------------------------------------------

\begin{slide}
\frametitle{Threads, Objects, and Events}

\begin{itemize}
\item Every \iCls{QThread} has its own event loop
  \begin{itemize}
  \item The main thread event loop is the application event loop
  \item Threads communicate with each other using events
  \item Only the receiving thread needs to have a running event loop
  \end{itemize}
\item Every \iCls{QObject} is associated with a \iCls{QThread} ("thread affinity")
  \begin{itemize}
  \item A \iCls{QObject} has affinity to the thread that created it
  \item Can be moved to another thread using \iClsFn{QObject}{moveToThread}
  \item Event handling is performed in the context of the owning thread
    \begin{itemize}
    \item Events are delivered to the object's owning thread
    \end{itemize}
  \end{itemize}
\end{itemize}
\imageFullWidth{multithreading/images/threadsandobjects}

\end{slide}

%----------------------------------------------------------------------

\begin{slide}
\frametitle{Controlling the Thread}

\begin{itemize}
\item Control the associated thread with:
  \begin{itemize}
  \item \textbf{\hClsFn{QThread}{run}} - the starting point for the thread.
  Contains the code to be run. The default implementation simply calls \hClsFn{QThread}{exec}.
  \item \textbf{\hClsFn{QThread}{start}} - begin execution by calling \hClsFn{QThread}{run}
  \item \textbf{\hClsFn{QThread}{quit}} - exit the thread's event loop
  \item \textbf{\hClsFn{QThread}{wait}} - block the current thread until the
  associated thread has finished, or an optional timeout expires
  \end{itemize}
\item \iCls{QThreads} signals:
  \begin{itemize}
  \item \textbf{\hClsFn{QThread}{started}} - when the associated thread starts
  \item \textbf{\hClsFn{QThread}{finished}} - when the associated thread finishes
  \end{itemize}
\item \iCls{QThread} states:
  \begin{itemize}
  \item \textbf{\hClsFn{QThread}{isRunning}} - true if the thread is running
  \item \textbf{\hClsFn{QThread}{isFinished}} - true if the thread has finished
  \end{itemize}
\end{itemize}

\end{slide}

%----------------------------------------------------------------------

\begin{slide}[fragile]
\frametitle{Cross-Thread Signals and Slots}

\begin{itemize}
\item Connections can be queued, or asynchronous
  \begin{itemize}
  \item Signals are serialized into events and sent to the receiver's queue
  \item Slots are executed in the context of the owning thread
  \end{itemize}
\item \begin{cpp}
connect( sender, &Sender::signal, receiver, &Receiver::slot,
         Qt::QueuedConnection );
\end{cpp}
\item The receiving thread must have a running event loop
  \begin{itemize}
  \item True in default implementation of \hClsFn{QThread}{run}
  \end{itemize}
\item \texttt{Qt::DirectConnection} - slot is invoked immediately
\item \texttt{Qt::AutoConnection} - (default) makes a direct connection if the
sender and receiver are on the same thread, otherwise makes a queued connection
\end{itemize}

\end{slide}

%----------------------------------------------------------------------

\begin{slide}[fragile]
\frametitle{Cross-Thread Signal Parameters}

\begin{itemize}
\item Queued parameters must be known to the meta-object system
\item Use the \texttt{Q\_DECLARE\_METATYPE} macro
\begin{cpp}
// in the header file
struct MyStruct
{
    ...
};
Q_DECLARE_METATYPE( MyStruct )

// in the implementation file
qRegisterMetaType<MyStruct>( "MyStruct" );
\end{cpp}
\item Note: Don't pass pointers that you do not own
\end{itemize}

\end{slide}

%----------------------------------------------------------------------

\begin{slide}[fragile]
\frametitle{QThread Example}

\begin{cpp}
QThread* thread = new QThread;
Worker* worker = new Worker();
worker->moveToThread(thread);

connect(thread, SIGNAL(started()),  worker, SLOT(process()));
connect(worker, SIGNAL(finished()), thread, SLOT(quit()));
...
connect(worker, SIGNAL(finished()), worker, SLOT(deleteLater()));
connect(thread, SIGNAL(finished()), thread, SLOT(deleteLater()));
        
thread->start();
\end{cpp}

%% TODO: need example app here

\end{slide}

%----------------------------------------------------------------------

\begin{slide}
\frametitle{QThread Pitfalls}

\begin{itemize}
\item A \iCls{QThread} object never has affinity with itself!
  \begin{itemize}
  \item ...because it was not created in the context of its associated thread
  \end{itemize}
\item \iCls{Objects} must be created in their parent's thread
  \begin{itemize}
  \item Implies that \iCls{QThread} should rarely be a parent (see above)
  \end{itemize}
\item Event driven objects may only be used in a single thread
  \begin{itemize}
  \item Applies to \iCls{QTimer}, and \iCls{QAbstractSocket} subclasses
  \item Start timers and connect sockets in their owning threads
  \end{itemize}
\item Objects created in a thread must be deleted before deleting that thread
\end{itemize}

\end{slide}

%----------------------------------------------------------------------
