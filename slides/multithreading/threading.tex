%%%%%%%%%%%%%%%%%%%%%%%%%%%%%%%%%%%%%%%%%%%%%%%%%%%%%%%%%%%%%%%%%%%%%%%%%%
%
% Copyright (c) 2013, Integrated Computer Solutions
% All rights reserved.
%
% This work, unless otherwise expressly stated, is licensed under a
% Creative Commons Attribution-ShareAlike 2.5.
%
% The full license document is available from
% http://creativecommons.org/licenses/by-sa/2.5/legalcode .
%
%%%%%%%%%%%%%%%%%%%%%%%%%%%%%%%%%%%%%%%%%%%%%%%%%%%%%%%%%%%%%%%%%%%%%%%%%%

\subsection{Qt Multithreading}

%----------------------------------------------------------------------

\begin{slide}
\frametitle{Terminology}

\begin{itemize}
\item \textbf{Multi-tasking} - Multiple programs running on the same computer
\item \textbf{Multi-threading} - Multiple threads running in the same program
\item \textbf{Multi-processing} - Multiple threads running on multiple processors
\vspace{1em}
\item \textbf{Reentrant function}
  \begin{itemize}
  \item May be called simultaneously from multiple threads
  \item ...but only if each invocation references unique data
  \end{itemize}
\item \textbf{Thread safe function}
  \begin{itemize}
  \item May be called simultaneously from multiple threads
  \item ...even if each invocation references shared data
  \end{itemize}
\end{itemize}

\end{slide}

%----------------------------------------------------------------------
\begin{slide}
\frametitle{High-Level or Low-Level?}
There are many different ways to use threads in Qt: 
\begin{itemize}
\item Manage threads yourself (Extending, Creating and Destroying QThread)
    \begin{itemize}
        \item Used in event-driven approaches, as well as in long-running
        "server" tasks 
        \item You must: Protect access to data using mutexes, etc.
        \item You must: Worry about deadlocks, thread affinity, etc.
    \end{itemize} 
\item Use threads from \iClsFn{QThreadPool}{globalInstance}
    \begin{itemize}
        \item Simpler, but must still manage threading issues (data access, etc.)
    \end{itemize}
\item Define algorithms that can be run from QtConcurrent
    \begin{itemize}
    \item Also uses the \iClsFn{QThreadPool}{globalInstance} 
    \item Don't write thread management code.
    \item Jobs automatically use correct number of threads (cores) for CPU
    \item No worries about mutexes, deadlocks, starvation, etc
    \end{itemize}
\end{itemize}
\end{slide}

\begin{slide}
\frametitle{Do I need threads?}
\begin{itemize}
    \item Not for driving animations. 
    \begin{itemize}
        \item \iCls{QTimeLine} or \iCls{QPropertyAnimation} are better
    \end{itemize}
    \item Not for playing audio.
        \begin{itemize}
        \item Use higher-level APIs like Qt Multimedia instead.
        \end{itemize}
    \item Not for GUI code
        \begin{itemize}
        \item \iCls{QWidget}, \iCls{QPixmap}, should only be used in main GUI thread
        \end{itemize}
    \item Not for network clients
        \begin{itemize}
        \item Use non-blocking APIs and signals/slots from main thread
        \end{itemize}
    \item Code that does not manage threads is simpler to understand and debug.
\end{itemize}
\end{slide}

%----------------------------------------------------------------
\begin{slide}[fragile]
\frametitle{How Threads Work}

\begin{itemize}
\item Independent streams of instructions (threads of execution)
\item Does not require multiple processors
  \begin{itemize}
  \item While one thread waits on IO, a second thread can use the CPU
  \end{itemize}
\item Each thread has its own context
  \begin{itemize}
  \item Stack, stack pointer, instruction pointer, registers
  \end{itemize}
\item Threads share resources with other threads
  \begin{itemize}
  \item Shared data, locks, conditions
  \end{itemize}
\vspace{1em}
\item Modern operating systems use preemption to schedule threads
  \begin{itemize}
  \item Schedule threads on different cores if available 
  \item Interrupt threads and save their contexts
  \item Restore contexts of next threads and resume execution
  \item Threads can also relinquish execution by waiting
  \end{itemize}
\end{itemize}

\end{slide}

%----------------------------------------------------------------------

\begin{slide}
\frametitle{How Threads Work}

\vspace{3em}
\centeredImage{multithreading/images/multithreaded_process}

\end{slide}

%----------------------------------------------------------------------
%----------------------------------------------------------------------

\begin{slide}
\frametitle{Qt Threading Guidelines}

\begin{itemize}
\item The main thread is the GUI thread
  \begin{itemize}
  \item Only access QWidget and QPixmap objects from the main thread
  \item It is safe to use \iCls{QPainter} with \iCls{QImage}, however
  \end{itemize}
\item Beware of blocking in the main thread
  \begin{itemize}
  \item Waiting in the main thread is inappropriate
  \item But it's perfectly fine in other threads
  \end{itemize}
\item All implicitly shared Qt classes are \textit{reentrant}
\item Do not mistake thread objects for the actual thread of execution
  \begin{itemize}
  \item Objects are data structures not instruction streams
  \end{itemize}
\end{itemize}

\end{slide}

%----------------------------------------------------------------------

\begin{slide}
\frametitle{Thread Affinity}

\begin{itemize}
\item Every \iCls{QThread} can have its own event loop
  \begin{itemize}
  \item Started by \iClsFn{QThread}{exec}
  \item Called by the default implementation of \iClsFn{QThread}{run}
  \item Signal/Slot connections require an event loop running on the receiving thread.
  \end{itemize}
\item Every \iCls{QObject} is associated with a \iCls{QThread} ("thread affinity")
  \begin{itemize}
  \item A \iCls{QObject} has affinity to the thread that created it
  \item Can be moved to another thread using \iClsFn{QObject}{moveToThread}
  \item Event handling is performed in the context of the owning thread
    \begin{itemize}
    \item Events are delivered to the object's owning thread
    \end{itemize}
  \end{itemize}
\end{itemize}
\imageFullWidth{multithreading/images/threadsandobjects}

\end{slide}

%----------------------------------------------------------------------

\begin{slide}
\frametitle{Simple Threading - QThreadPool}

\begin{itemize}
\item \iCls{QThreadPool} manages a collection of {\it n} threads
  \begin{itemize}
  \item One thread per core (real or virtual) by default
  \item On single core systems, {\it n}=1 
  \end{itemize}
\item Can change {\it n} via:
  \begin{itemize}
  \item \iClsFn{QThreadPool}{setMaxThreadCount} - sets n
  \item \iClsFn{QThreadPool}{reserveThread} - increments n
  \item \iClsFn{QThreadPool}{releaseThread} - decrements n
  \item \textit{A higher number of threads is not necessarily better}
  \end{itemize}
\item Suitable for CPU-bound tasks, not general purpose threading
\item Each QApplication has a global thread pool accessible with
      \iClsFn{QThreadPool}{globalInstance}
\end{itemize}

\end{slide}

%----------------------------------------------------------------------


\begin{slide}[fragile] 
\frametitle{Thread Pool Tasks - QRunnable}

\begin{itemize}
\item Tasks are implemented by subclassing \iCls{QRunnable}
\begin{cpp}
class HelloWorldTask : public QRunnable
{
    void run() { qDebug() << "Hello world!"; }
}
\end{cpp}
\item Start a \iCls{QRunnable} task using:
  \begin{itemize}
  \item \iClsFnPar{QThreadPool}{start}{QRunnable *task)}
    \begin{itemize}
    \item Will queue task of no thread is available
    \end{itemize}
  \item \iClsFnPar{QThreadPool}{tryStart}{QRunnable *task)}
    \begin{itemize}
    \item Will return false if no thread is available
    \end{itemize}
  \end{itemize}
\item \iCls{QThreadPool} takes ownership of \iCls{QRunnable} tasks
  \begin{itemize}
  \item Task deleted by default when \hClsFn{QRunnable}{run} exits
  \end{itemize}
\end{itemize}

\end{slide}

%----------------------------------------------------------------------

\begin{slide}
\frametitle{QRunnable Notes}

\begin{itemize}
\item \iCls{QRunnable} is kept small and efficient
  \begin{itemize}
  \item No communication mechanisms
  \item No way to cancel a running task
  \item Won't be informed when task completes
  \end{itemize}
\item Do not block in tasks
  \begin{itemize}
  \item There are a limited number of threads in the pool
  \end{itemize}
\item Good for simple "Fire and Forget" style tasks
\vspace{1em}
\item \iCls{QThreadPool} is the foundation for \iCls{QtConcurrent} module
\end{itemize}

\end{slide}

%----------------------------------------------------------------------

\begin{slide}
\frametitle{Threading Workhorse - QThread}

Two models for using QThread:
\begin{itemize}
\item Subclassing model: subclass and re-implement the \hClsFn{QThread}{run} method
  \begin{itemize}
  \item Traditional model for using \iCls{QThread}
  \item Good for CPU-bound tasks
  \item Using \iCls{QRunnable} is easier and less error prone
  \end{itemize}
\item Event-Driven model: manage threads through events and signals
  \begin{itemize}
  \item Newer model for using \iCls{QThread}
  \item Good for communicating with threads
  \item Requires an understanding of thread affinity
  \end{itemize}
\end{itemize}

\vspace{1em}
The Great \iCls{QThread} Controversy: to subclass or not to subclass \\
\linkbutton{http://blog.qt.digia.com/blog/2010/06/17/youre-doing-it-wrong/}
\linkbutton{http://woboq.com/blog/qthread-you-were-not-doing-so-wrong.html}

\end{slide}


%----------------------------------------------------------------------

\begin{slide}
\frametitle{Controlling the Thread}

\begin{itemize}
\item Control the associated thread with:
  \begin{itemize}
  \item \textbf{\hClsFn{QThread}{run}} - the starting point for the thread.
  Contains the code to be run. The default implementation simply calls \hClsFn{QThread}{exec}.
  \item \textbf{\hClsFn{QThread}{start}} - begin execution by calling \hClsFn{QThread}{run}
  \item \textbf{\hClsFn{QThread}{quit}} - exit the thread's event loop
  \item \textbf{\hClsFn{QThread}{wait}} - block the current thread until the
  associated thread has finished, or an optional timeout expires
  \end{itemize}
\item \iCls{QThreads} signals:
  \begin{itemize}
  \item \textbf{\hClsFn{QThread}{started}} - when the associated thread starts
  \item \textbf{\hClsFn{QThread}{finished}} - when the associated thread finishes
  \end{itemize}
\item \iCls{QThread} states:
  \begin{itemize}
  \item \textbf{\hClsFn{QThread}{isRunning}} - true if the thread is running
  \item \textbf{\hClsFn{QThread}{isFinished}} - true if the thread has finished
  \end{itemize}
\end{itemize}

\end{slide}

%----------------------------------------------------------------------

\begin{slide}[fragile]
\frametitle{Cross-Thread Signals and Slots}

\begin{itemize}
\item Connections can be queued, or asynchronous
  \begin{itemize}
  \item Signals are serialized into events and sent to the receiver's queue
  \item Slots are executed in the context of the owning thread
  \end{itemize}
\item \begin{cpp}
connect( sender, &Sender::signal, receiver, &Receiver::slot,
         Qt::QueuedConnection );
\end{cpp}
\item The receiving thread must have a running event loop
  \begin{itemize}
  \item True in default implementation of \hClsFn{QThread}{run}
  \end{itemize}
\item \texttt{Qt::DirectConnection} - slot is invoked immediately
\item \texttt{Qt::AutoConnection} - (default) makes a direct connection if the
sender and receiver are on the same thread, otherwise makes a queued connection
\end{itemize}

\end{slide}

%----------------------------------------------------------------------

\begin{slide}[fragile]
\frametitle{Cross-Thread Signal Parameters}

\begin{itemize}
\item Queued parameters must be known to the meta-object system
\item Use the \texttt{Q\_DECLARE\_METATYPE} macro
\begin{cpp}
// in the header file
struct MyStruct
{
    ...
};
Q_DECLARE_METATYPE( MyStruct )

// in the implementation file
qRegisterMetaType<MyStruct>( "MyStruct" );
\end{cpp}
\item Note: Don't pass pointers that you do not own
\end{itemize}

\end{slide}

%----------------------------------------------------------------------

\begin{slide}[fragile]
\frametitle{Subclassing Example}

\begin{cpp}
class WorkerThread : public QThread
{
    Q_OBJECT
public:
    void run() {
        QString result;
        // expensive or blocking operation
        // ...
        emit resultReady(result);
    }
    
signals:
    void resultReady(const QString &result);
};\end{cpp}

\vspace{1em}
\demo{multithreading/ex-workqueue}

\end{slide}

%----------------------------------------------------------------------

\begin{slide}[fragile]
\frametitle{Event-Driven Example}

\begin{cpp}
QThread* thread = new QThread;
Worker* worker = new Worker();
worker->moveToThread(thread);

connect(thread, SIGNAL(started()),  worker, SLOT(process()));
connect(worker, SIGNAL(finished()), thread, SLOT(quit()));
...
connect(worker, SIGNAL(finished()), worker, SLOT(deleteLater()));
connect(thread, SIGNAL(finished()), thread, SLOT(deleteLater()));
        
thread->start();
\end{cpp}

\vspace{1em}
\demo{multithreading/ex-mandelbrot}

\end{slide}

%----------------------------------------------------------------------

\begin{slide}
\frametitle{QThread Pitfalls}

\begin{itemize}
\item \iCls{QThread} methods are meant to be called from the creating thread
\item A \iCls{QThread} object should never have affinity with itself!
  \begin{itemize}
  \item ...because it was not created in the context of its associated thread
  \end{itemize}
\item \iCls{QObjects} must be created in their parent's thread
  \begin{itemize}
  \item Implies that \iCls{QThread} should rarely be a parent (see above)
  \end{itemize}
\item Event driven objects may only be used in a single thread
  \begin{itemize}
  \item Applies to \iCls{QTimer}, and \iCls{QAbstractSocket} subclasses
  \item Start timers and connect sockets in their owning threads
  \end{itemize}
\item Objects created in a thread must be deleted before deleting that thread
\end{itemize}

\end{slide}

%----------------------------------------------------------------------
