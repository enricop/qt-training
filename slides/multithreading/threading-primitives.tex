%%%%%%%%%%%%%%%%%%%%%%%%%%%%%%%%%%%%%%%%%%%%%%%%%%%%%%%%%%%%%%%%%%%%%%%%%%
%
% Copyright (c) 2013, Integrated Computer Solutions
% All rights reserved.
%
% This work, unless otherwise expressly stated, is licensed under a
% Creative Commons Attribution-ShareAlike 2.5.
%
% The full license document is available from
% http://creativecommons.org/licenses/by-sa/2.5/legalcode .
%
%%%%%%%%%%%%%%%%%%%%%%%%%%%%%%%%%%%%%%%%%%%%%%%%%%%%%%%%%%%%%%%%%%%%%%%%%%

\subsection{Threading Primitives}



\begin{slide}
\frametitle{Classic Threading Problem: Race Condition}

\begin{itemize}
\item Program behavior is dependent on the order of thread execution
  \begin{itemize}
  \item Producer thread increments count, consumer thread decrements count
  \item But increments and decrements are \textbf{not} atomic operations!
    \item [] \vspace{0.5em}
    \begin{tabular}{l|l}
    \footnotesize{producer fetches value} & \footnotesize{consumer fetches value} \\
    \footnotesize{producer increments value} & \footnotesize{consumer decrements value} \\
    \footnotesize{producer stores value} & \footnotesize{consumer stores value} \\
    \end{tabular}
  \item \vspace{0.5em} Final count value is dependent on order of operations
  \end{itemize}

\vspace{1em}
\item Solution: Synchronization via locks
  \begin{itemize}
  \item Access to data is serialized
  \item Locking semantics provided by:
  \begin{itemize}
  \item \iCls{QMutex}, \iCls{QReadLock}, and \iCls{QReadWriteLock}.
  \end{itemize}
  \end{itemize}
\end{itemize}

\end{slide}

%----------------------------------------------------------------------
\begin{slide}
\frametitle{Classic Threading Problem: Deadlock}

Two or more threads wait on each other, thus neither can progress
  \begin{itemize}
  \item Alice and Bob transfer money to each other
  \item Each dutifully tries to lock both bank accounts
  \item [] \vspace{0.5em}
  \footnotesize{Alice locks Alice's account} \\
  \hspace{1cm}\footnotesize{Bob locks Bob's account} \\
  \footnotesize{Alice waits on a lock for Bob's account} \\
  \hspace{1cm}\footnotesize{Bob waits on a lock for Alice's account} \\
  \item Both will wait forever
  \end{itemize}
\vspace{1em}
Solution: Test-And-Lock, with timeout or busy-waiting may be needed
\begin{itemize}
  \item \iClsFn{QMutex}{tryLock} can help
  \item \iClsFn{QAtomicInt}{testAndSetAcquire},
  \iClsFn{QAtomicInt}{testAndSetRelease}
\end{itemize}

\end{slide}


% ----------------------------------------------------------------------
\begin{slide}[fragile]
\frametitle{QMutex}
\begin{itemize}
\item Mutex (mututal exclusion) is implemented by \iCls{QMutex}
\item Important methods: \iClsFn{QMutex}{lock} and \iClsFn{QMutex}{unlock}
\begin{itemize}
\item \texttt{lock()} waits indefinitely (non-busy) for the lock
\end{itemize}
\item \texttt{tryLock()} returns \texttt{false} immediately if already locked.
\item Test-And-Lock: \texttt{tryLock(int timeOut)} waits milliseconds
before returning \texttt{false} if locked.
\end{itemize}

\begin{cpp}
void method2()
{
    mutex.lock();
    number *= 3;
    number /= 2;
    mutex.unlock();
}
\end{cpp}
\end{slide}
% ----------------------------------------------------------------------
\begin{slide}[fragile]
\frametitle{Automatically (Un)locking QMutex}

\begin{itemize}
\item Must ensure \texttt{unlock()} is called after each \texttt{lock()}
\item In functions, this means checking for multiple exit points
\item Instead, one can use \textbf{scoped mutex} \iCls{QMutexLocker}
\item A \textbf{scoped mutex} unlocks its mutex in the destructor
\item If created on the stack, limits the scope of mutex lock to a block
\end{itemize}
\begin{cpp}
void method2()
{
    QMutexLocker locker(&mutex);  // locked here
    number *= 3;
    number /= 2;
}                                 // unlocked on function return
\end{cpp}
\end{slide}

% ----------------------------------------------------------------------
\begin{slide}
\frametitle{Multiple Readers or One Writer}

\iCls{QReadWriteLock} supports the use case of many readers and few writers
\begin{itemize}
\item Multiple readers may safely access the data at the same time
    \begin{itemize}
    \item Only if the data is not locked for writing
    \end{itemize}
\item Only one writer may access the data at a time
    \begin{itemize}
    \item Only if the data is not locked for reading or writing
    \end{itemize}

\item \texttt{lockForRead()} and \texttt{lockForWrite()} to acquire
\item \texttt{unlock()} to release
\item \texttt{tryLockForRead()} and \texttt{tryLockForWrite()} to test/acquire
    \begin{itemize}
    \item Optional \texttt{timeOut} in milliseconds
    \end{itemize}
\item \iCls{QReadLocker} and \iCls{QWriteLocker}: \textbf{scoped locker} objects
    \begin{itemize}
    \item Create on stack
    \item Automatically acquire and release locks in a block
    \end{itemize}
\end{itemize}

\end{slide}
% ----------------------------------------------------------------------
\begin{slide}[fragile]
\frametitle{Thread-local data}
In General:
\begin{itemize}
\item Static variables are always "global"
\item Local variables are always "thread-local"
\end{itemize}
Sometimes you need to define data that is local to a thread but not local
to a function.
\begin{itemize}
\item \iCls{QThreadStorage} can help
\item Parametrized template class for value types
\item \textit{In Qt 4: restricted to pointer types}
\end{itemize}
\begin{cpp}
QThreadStorage<int*> localIntPtr;
QThreadStorage<QCache<QString, SomeClass> > threadCache;
\end{cpp}

\end{slide}
% ----------------------------------------------------------------------
\begin{slide}[fragile]
\frametitle{QThreadStorage methods}
Similar to, but not really a smart-pointer.
\begin{itemize}
\item \texttt{setLocalData()} to store value (usually for pointers to heap
    objects)
\item \texttt{hasLocalData()} to check if value was set
\item \texttt{localData()} to get it
\end{itemize}
\begin{cpp}
QThreadStorage<QCache<QString, SomeClass> > caches;
void cacheObject(const QString &key, SomeClass *object) {
    caches.localData().insert(key, object);
}
void removeFromCache(const QString &key) {
    if (!caches.hasLocalData())
        return;
    caches.localData().remove(key);
}
\end{cpp}
\end{slide}
%----------------------------------------------------------------------

% ----------------------------------------------------------------------
\begin{slide}[fragile]
\frametitle{QWaitCondition}
\begin{itemize}
\item \texttt{QWaitCondition::wait(QMutex* lockedMutex, int timeout)} releases a
lock, waits for a \texttt{wake()}
\item Can use \iCls{QReadWriteLock} or \iCls{QMutex} here
\item When \texttt{wait()} returns \texttt{true}, the mutex is locked by your
thread again
\item Other threads wake a waiting thread with \iClsFn{QWaitCondition}{wakeOne}
\item Wake all other threads with \iClsFn{QWaitCondition}{wakeAll}
\end{itemize}
\demo{multithreading/ex-workqueue}
\end{slide}


\begin{slide}
\frametitle{Classic Threading Problem: Starvation}

\textbf{The Dining Philosophers Problem}
\begin{itemize}
\item 4 philosophers sit at a table.
\item Each has 2 states: \textit{eating} and \textit{thinking}
\item 4 bowls, and 4 chopsticks are on the table
\item To switch from \textit{thinking} to \textit{eating},
      each philosopher must "acquire" 2 chopsticks
\end{itemize}
Imagine each philosopher doing this:
\begin{itemize}
\item  Lock left chopstick
\item  Lock right chopstick
\item  Eat
\end{itemize}

A careful redesign is required
\begin{itemize}
\item \iCls{QSemaphore} can help here.
\end{itemize}
\end{slide}

% ----------------------------------------------------------------------
\begin{slide}
\frametitle{Semaphores}

\iCls{QSemaphore} is a generalized mutex.
\begin{itemize}
\item \iCls{QSemaphore} guards multiple resources (in contrast to \iCls{QMutex}
      which can only be locked once)
\item The maximum number of resources is set in the constructor
\item \texttt{QSemaphore::acquire(int n)} to acquire \textit{n} resources guarded
      by the semaphore
\item \texttt{release()} releases the resources
\item \texttt{tryAcquire(int n)} returns false immediately if \textit{n} resources
      are not available
\item \texttt{tryAcquire(int n, int timeOut)} waits milliseconds before
      returning \texttt{false}.
\item \texttt{available()} returns the number of resources available
\end{itemize}
\end{slide}

%----------------------------------------------------------------------
\begin{slide}
\frametitle{Mutexes vs. Semaphores: Summary}
\begin{center}
\begin{tabular}{|l||c|c|}
\hline
& \iCls{Mutex} & \iCls{Semaphore}\\
\hline
Terminology & Locks & Resources\\
At any time & one & many\\
Locking & \hClsFn{QThread}{lock} & \hClsFn{QSemaphore}{acquire}\\
Unlocking & \hClsFn{QThread}{unlock} & \hClsFn{QSemaphore}{release}\\
Testing & \hClsFn{QThread}{tryLock} & \hClsFn{QSemaphore}{tryAcquire}\\
\hline
\end{tabular}
\end{center}
\end{slide}
% ----------------------------------------------------------------------
% \begin{slide}[fragile]
% \frametitle{Lab: Threaded Progress Bar}
% Write a program that displays the progress of a running thread:
% \begin{itemize}
% \item A partial program is provided in
% \iExample{multithreading/lab-threaded-progress}
% \item The thread should perform a CPU intensive task one thousand times,
% reporting the progress back to the main GUI thread. 
% \item The function busy\_work() is provided in file work.cpp as the task to be
% performed. 
% \item QThread should not be subclassed.
% \end{itemize}
% \centeredImage{multithreading/images/multithreaded_progress}
% 
% \end{slide}

% ----------------------------------------------------------------------
\begin{slide}[fragile]
\frametitle{Lab: Finding Prime Numbers (Step 1)}
Write a program that finds prime numbers utilizing multiple threads.
\flushedImage{multithreading/images/primethreads}
\vspace{5.0em}
\begin{itemize}
\item Start with \iExample{multithreading/lab-prime}
\item Add a progress bar 
\item Add a label + spinbox for number of threads. 
\item Move calculation code into a separate \iCls{QObject}, \texttt{PrimeFinder}.
\item Hide \texttt{Ok} and \texttt{Close} and show a working \texttt{Cancel} button
while it is finding primes.
\end{itemize}

\end{slide}

% ----------------------------------------------------------------------
\begin{slide}
\frametitle{Lab: Finding Prime Numbers (step 2)}
\begin{itemize}
\item Create worker threads and distribute the calculation across them
\item It is your choice whether to extend \iCls{QThread}, or use
\iCls{QRunnable} with \iCls{QThreadPool}.
\item If you have \textit{n} cores, report the primes calculated per second when 
using \textit{1}, \textit{n/2}, and \textit{n} threads. 
\end{itemize}
Is there a significant speedup?
\end{slide}


% ----------------------------------------------------------------------
% :wrap=hard:maxLineLen=80:
