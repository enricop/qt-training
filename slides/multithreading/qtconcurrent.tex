%%%%%%%%%%%%%%%%%%%%%%%%%%%%%%%%%%%%%%%%%%%%%%%%%%%%%%%%%%%%%%%%%%%%%%%%%%
%
% Parts Copyright (c) 2008-2011 Nokia
% Parts Copyright (c) 2010-2013 Integrated Computer Solutions
%
% This work, unless otherwise expressly stated, is licensed under a
% Creative Commons Attribution-ShareAlike 2.5.
%
% The full license document is available from
% http://creativecommons.org/licenses/by-sa/2.5/legalcode.
%
%%%%%%%%%%%%%%%%%%%%%%%%%%%%%%%%%%%%%%%%%%%%%%%%%%%%%%%%%%%%%%%%%%%%%%%%%%

\subsection{QtConcurrent}

%----------------------------------------------------------------------
\begin{slide}{0329}
\frametitle{QtConcurrent}\label{QtConcurrent}
  \xConcept{Map-Reduce}
  \xConcept{Multithreading!Map-Reduce}
  \begin{itemize}
  \item \iNs{QtConcurrent} is a high-level framework for parallel algorithms.
  \item Work is automatically distributed over an optimal number of
    threads, determined at runtime.
  \item Currently supported are combinations of the following:
    \begin{itemize}
    \item \hNsFn{QtConcurrent}{map} or \hNsFn{QtConcurrent}{filter}
          (\textit{transform} and \textit{copy\_if} in STL)


    \item in-place (\hNsFn{QtConcurrent}{map},
      \hNsFn{QtConcurrent}{filter}) or\\
      copying (\hNsFn{QtConcurrent}{mapped}, \hNsFn{QtConcurrent}{filtered}) or\\
      with \emph{reduction} (\hNsFn{QtConcurrent}{mappedReduced}, \hNsFn{QtConcurrent}{filteredReduced})\\
      (\emph{folding} in functional languages, \emph{accumulate} in STL)
    \item blocking (\hNsFn{QtConcurrent}{blockingMapped}, returns the result) or\\
      asynchronous (\hNsFn{QtConcurrent}{mapped}, returns a \emph{future})
    \end{itemize}
  \item The algorithms are defined in namespace \iNs{QtConcurrent}
    (\iNs{QtConcurrent} is \emph{not} a class).
  \end{itemize}
\end{slide}

%----------------------------------------------------------------------
\begin{slide}{0330}
\frametitle{QtConcurrent~\textemdash~Blocking Operations}
  \xConcept{QtConcurrent!Mapping!Blocking}
  \xConcept{QtConcurrent!Blocking!Mapping}
  \xConcept{Multithreading!QtConcurrent!Mapping!Blocking}
  \xConcept{Multithreading!QtConcurrent!Blocking!Mapping}
  \begin{itemize}
  \item \iNs{QtConcurrent} can transform (map) sequences based on a
    user-defined \emph{\iConcept{mapping function}}.
  \item Only random access sequences (\iCls{QVector}, \iCls{QList})
    should be used with \iNs{QtConcurrent}, forward sequences
    (\iCls{QLinkedList}, \iCls{QMap},\ldots) can be used, but incur a
    performance penalty.
  \item Instead of complete sequences, a pair of iterators is allowed
    where it makes sense.
  \item The mapping function either takes one element of the sequence
    as an argument and returns the modified element
    (\hNsFn{QtConcurrent}{mapped}), or modifies the argument directly
    (\hNsFn{QtConcurrent}{map})
  \item The order in which elements are processed is undefined, though
    the sequence is never reordered.
  \item See \qtexample{qtconcurrent/map}.
  \end{itemize}
\end{slide}

%----------------------------------------------------------------------
\begin{slide}[fragile]{0331}
\frametitle{QtConcurrent~\textemdash~Blocking Operations}
  \xConcept{QtConcurrent!Filtering!Blocking}
  \xConcept{QtConcurrent!Blocking!Filtering}
  \xConcept{Multithreading!QtConcurrent!Filtering!Blocking}
  \xConcept{Multithreading!QtConcurrent!Blocking!Filtering}
  \begin{itemize}
  \item \iNs{QtConcurrent} can filter (grep) sequences based on a
    user-defined \emph{\iConcept{filter function}}.
  \item The filter function takes one element of the sequence as an
    argument and returns \texttt{true} (keep element) or
    \texttt{false} (drop element). Filter functions are ``unary predicates''.
  \item Filter and mapping functions may also be \emph{member
      functions} of the elements in the sequence.
    Example:%
    \xExample{Multithreading!QtConcurrent!blockingMapped}
    \xExample{QtConcurrent!blockingMapped}
    \xExample{blockingMapped (QtConcurrent)}
\begin{cpp}
QStringList input = ...;
QStringList lower =
  QtConcurrent::blockingMapped(input, &QString::toLower);
\end{cpp}
  \item Filtering and mapping are very similar, so in the following,
    we talk about mapping, and point out where filtering differs.
  \end{itemize}
\end{slide}
%----------------------------------------------------------------------
\begin{slide}[fragile]{0332}
\frametitle{QtConcurrent~\textemdash~Blocking Operations}
  \xConcept{QtConcurrent!Map-Reduce!Blocking}
  \xConcept{QtConcurrent!Blocking!Map-Reduce}
  \xConcept{QtConcurrent!Filter-Reduce!Blocking}
  \xConcept{QtConcurrent!Blocking!Filter-Reduce}
  \xConcept{Multithreading!QtConcurrent!Map-Reduce!Blocking}
  \xConcept{Multithreading!QtConcurrent!Blocking!Map-Reduce}
  \xConcept{Multithreading!QtConcurrent!Filter-Reduce!Blocking}
  \xConcept{Multithreading!QtConcurrent!Blocking!Filter-Reduce}
  \begin{itemize}
  \item In addition to mapping/filtering, \iNs{QtConcurrent} can optionally
    \emph{reduce} the sequence with a user-defined \emph{\iConcept{reduce function}}.
  \item The reduce function takes the partial result by reference,
    and the next element of the sequence as arguments and modifies the
    partial result to incorparate the new element. The return value is
    ignored. Example:%
    \xExample{Multithreading!Reduce Function}
    \xExample{Reduce Function}
\begin{cpp}
  void join( QString &result, const QString &next ) {
    result += next;
  }
\end{cpp}
\item \iNsEnum{QtConcurrent}{ReduceOptions} specify how exactly the
  reduction is applied.
\item Currently, reduction is never parallelized (the mapping part is).
  \end{itemize}
\end{slide}
%----------------------------------------------------------------------
\begin{slide}[fragile]{0333}
\frametitle{\iConcept{QtConcurrent Cheat Sheet}}
  \label{qtconcurrent-cheat-sheet}
  \xConcept{QtConcurrent!Cheat Sheet}
  \xConcept{Multithreading!QtConcurrent!Cheat Sheet}
  \xConcept{mapping function}
  \xConcept{reduce function}
  \xConcept{filter function}
  \begin{center}
    \footnotesize%
    Here is how the arguments of \iNs{QtConcurrent}
    functions map to return values (assuming the sequence passed is
    \texttt{\iCls{QList}<T>}): \\
    \medskip
    \begin{tabular}{r|c|c|c}
     \textbf{Function} & \textbf{Map Func.} & \textbf{Reduce Func.} & \textbf{Returns} \\
     \hline
      \multirow2*{\hNsFn{QtConcurrent}{map}} & void f(T\&)                       & \multirow2*{---} & void / \\
                                              & void T::f() \strikethrough{const} &                  & QFut.<void> \\
      \hline
      \multirow2*{\hNsFn{QtConcurrent}{mapped}} & \highlight{U} f(const T\&) & \multirow2*{---} & QList<\highlight{U}> / \\
                                                 & \highlight{U} T::f() const &                  & QFuture<\highlight{U}> \\
      \hline
      \multirow2*{\hNsFn{QtConcurrent}{mappedReduced}} & \highlight{U} f(const T\&) & f(\highlight{V}\&, const U\&) & \\
                                                        & \highlight{U} T::f() const & \highlight{V}::f(const U\&)   & QList<\highlight{V}> / \\
      \hhline{===|~}
      \multirow2*{\hNsFn{QtConcurrent}{filteredReduced}} &  & f(\highlight{V}\&, const T\&) & QFuture<\highlight{V}> \\
                                                          &  & \highlight{V}::f(const T\&)   & \\
      \cline{1-1}\cline{3-4}
      \multirow2*{\hNsFn{QtConcurrent}{filtered}} & bool f(const T\&) & \multirow2*{---} & QList<T> / \\
                                                   & bool T::f() const &                  & QFuture<T> \\
      \cline{1-1}\cline{3-4}
      \multirow2*{\hNsFn{QtConcurrent}{filter}}   &  & \multirow2*{---} & void / \\
                                                   &  &                  & QFut.<void> \\
     \hline
      \textbf{Function} &\textbf{ Filter Func.} & \textbf{Reduce Func} &\textbf{ (block/async)} \\
   \end{tabular}
    {\tiny(for \iCls{QFuture} cf.~\contentHyperlink{QFuture}{QFuture} below)}
  \end{center}
\end{slide}
%----------------------------------------------------------------------
\begin{slide}[fragile]{0334}
\frametitle{Interlude: \iConcept{Function Objects}}
  A mapping function that transforms elements of type \iCls{QString} to
  values of type \texttt{int} can be any of the following:
  \begin{itemize}
  \item a free function:
    \texttt{int myfunc( const QString \& )}
  \item a \texttt{const} member function of \iCls{QString}:
    \texttt{\&QString::length}
  \item a hand-written \emph{function object}:
    \begin{cpp}
struct Functor : std::unary_function<QString,int> {
  const QString ref;
  explicit Functor( const QString &r ) : ref( r ) {}
  int operator()( const QString &arg ) const {
    return QString::localeAwareCompare( ref, arg );
  }
};\end{cpp}
  \item a (TR1/C++0x/Boost) bind expression\xConcept{Bind Expression}:\\
    \texttt{bind(\&QString::localeAwareCompare, reference, \_1)}
    \xConcept{Boost}\xConcept{TR1}\xConcept{C++0x}
  \end{itemize}
\end{slide}

%----------------------------------------------------------------------

\begin{slide}{0337}
\frametitle{QFuture}\label{QFuture}
  \begin{itemize}
  \item \iCls{QFuture} is Qt's implementation of the Future concept.
  \item You can't create \iCls{QFuture}s yourself.
  \item They're returned from \emph{non-blocking} \iNs{QtConcurrent} functions.
  \item With \iCls{QFuture}, you can:
      \begin{itemize}
      \item copy
      \item pass by value,
      \item ask for progress
      \item pause
      \item cancel it
      \end{itemize}
  \item Can be \emph{single-valued}, or \emph{multi-valued}.
  \item A template proxy for an actual value containing the result.
  \end{itemize}
\end{slide}

%----------------------------------------------------------------------
\begin{slide}[fragile]{0338}
\frametitle{QFuture Example}
  \xExample{Multithreading!QFuture}
  \xExample{QFuture}
\begin{center}
\begin{cpp}
int someCalculation() {
  const int a = ...; // lengthy calculation
  const int b = ...;
  return a + b; // immediate
}
\end{cpp}
{\Huge$\Downarrow$}
\begin{cpp}
int someCalculation() {
  const QFuture<int> a = QtConcurrent::...; // imm.
  const QFuture<int> b = QtConcurrent::...; // imm.
  return a + b; // might block
}
\end{cpp}
\end{center}
\end{slide}
%----------------------------------------------------------------------
\begin{slide}{0339}
\frametitle{QFuture (Single-Valued)}
  \xConcept{QFuture!single-valued}
  \begin{itemize}
  \item A single-valued \iCls{QFuture} acts much like the value itself
  \item Can be used in arithmetic with other compatible QFuture types without blocking.
  \item Check for the result with
    \begin{itemize}
    \item \hClsFn{QFuture}{isFinished}
    \item \hClsFn{QFuture}{waitForFinished},
    \item \hClsFn{QFuture}{result} (might block)
    \item \emph{implicit} conversion of \texttt{QFuture<T>} to \texttt{T} (calls \hClsFn{QFuture}{result})
    \end{itemize}
  \end{itemize}
\end{slide}
%----------------------------------------------------------------------
\begin{slide}{0340}
\frametitle{QFuture (Multi-Valued)}
  \xConcept{QFuture!multi-valued}
  \xConcept{Iterators!Java style!QFuture}
  \xConcept{Iterators!STL style!QFuture}
  \begin{itemize}
  \item A \iCls{QFuture} is \emph{multi-valued} if
    $\hbox{\hClsFn{QFuture}{resultCount}}>1$.
  \item A multi-valued \iCls{QFuture} acts much like a
    \iCls{QList}, but iteration might block, waiting for more elements
    to become available.
  \item Check for progress:
    \begin{itemize}
    \item \hClsFn{QFuture}{progressValue}, \hClsFn{QFuture}{progressRangeChanged}
    \end{itemize}
  \item Cancel (\hClsFn{QFuture}{cancel}), pause
    (\hClsFn{QFuture}{pause}), and resume (\hClsFn{QFuture}{resume})
    the task.
  \item Check results:
    \begin{itemize}
    \item \hClsFn{QFuture}{resultCount}, \hClsFn{QFuture}{isFinished},
      \hClsFnPar{QFuture}{isResultReadyAt}{int}
    \item \hClsFn{QFuture}{results}, \hClsFnPar{QFuture}{resultAt}{int
        index} (might block)
    \end{itemize}
  \item For iteration, both Java-style \iCls{QFutureIterator}, and
    STL-style \iClsTypedef{QFuture}{const\_iterator} are available.
\end{itemize}
\end{slide}
%----------------------------------------------------------------------
\begin{slide}{0341}
\frametitle{QFutureWatcher}
\begin{itemize}
\item Extends \iCls{QObject}, and implements most of the \iCls{QFuture} API.
\item Set a future with \iClsFn{QFutureWatcher}{setFuture}, get it
  back with \iClsFn{QFutureWatcher}{future}.
\item Interesting signals:
  \begin{itemize}
  \item%
    \hClsSig{QFutureWatcher}{finished},
    \hClsSig{QFutureWatcher}{paused},
    \hClsSig{QFutureWatcher}{canceled}
  \item%
    \hClsSigPar{QFutureWatcher}{resultReadyAt}{int~index},
    \hClsSigPar{QFutureWatcher}{resultReadyAt}{int~begin,~int~end}

  \item Useful for connecting to \iCls{QProgressBar}:
    \hClsSig{QFutureWatcher}{progressValueChanged},
    \hClsSig{QFutureWatcher}{progressRangeChanged},
    \hClsSig{QFutureWatcher}{progressTextChanged}
  \end{itemize}
\item Interesting slots:
  \begin{itemize}
  \item %
    \hClsSlt{QFutureWatcher}{cancel}
  \item %
    \hClsSltPar{QFutureWatcher}{setPaused}{bool},
    \hClsSlt{QFutureWatcher}{pause},
    \hClsSlt{QFutureWatcher}{resume},
    \hClsSlt{QFutureWatcher}{togglePaused}
  \end{itemize}
\end{itemize}
\end{slide}



%----------------------------------------------------------------------
\begin{slide}[fragile]{0342}
\frametitle{QFutureWatcher Example}
  \xExample{Multithreading!QFutureWatcher}
  \xExample{QFutureWatcher}
\begin{center}\footnotesize
\begin{cpp}
int someCalculation() {
  const int a = ...; // lengthy calculation
  doUse( a );
}
\end{cpp}
{\Huge$\Downarrow$}
\begin{cpp}
QFutureWatcher m_watcher;
// in constructor:
connect( &m_watcher, SIGNAL(finished()),
         this, SLOT(slotSomeCalculation()) );
void startSomeCalculation() {
  m_watcher.setFuture( QtConcurrent::... );
}
void slotSomeCalculation() {
  doUse( m_a.result<int>() );
}
\end{cpp}
\end{center}
\end{slide}

%----------------------------------------------------------------------

\begin{slide}
\frametitle{Lab: Finding Prime Numbers (concurrent)}
You can start with \iExample{multithreading/lab-prime}
\begin{itemize}
\item Rewrite the prime number finder to use a QtConcurrent algorithm instead of
QThread or QRunnable.
\item If you have \textit{n} cores, report the primes calculated per second when
using \textit{1}, \textit{n/2}, and \textit{n} threads.
\end{itemize}
\centeredImage{multithreading/images/primethreads-concurrent}

Is there a significant speedup?

How does it compare to the previous labs in performance?

\end{slide}


%----------------------------------------------------------------------
\begin{slide}[fragile]{0343}
\frametitle{QtConcurrent and Exceptions}
  \xConcept{Exceptions!QtConcurrent}
  \xConcept{Exceptions!Multithreading}
  \xConcept{QtConcurrent!Exceptions}
  \xConcept{Multithreading!QtConcurrent!Exceptions}
  \xConcept{Multithreading!Exceptions}
  \begin{itemize}
  \item What happens when one of the mapping or filtering functions
    throws an exception? How will it travel back to the calling
    thread?
  \item Exceptions inheriting \iNsCls{QtConcurrent}{Exception} can
    travel across thread boundaries to be delivered to the calling
    thread.
  \item Exceptions \emph{not} inheriting
    \hNsCls{QtConcurrent}{Exception} are caught and delivered as
    \iNsCls{QtConcurrent}{UnhandledException}.
  \end{itemize}
\end{slide}
%----------------------------------------------------------------------
\begin{slide}[fragile]{0344}
\frametitle{QtConcurrent and Exceptions}
  \xConcept{Exceptions!QtConcurrent}
  \xConcept{Exceptions!Multithreading}
  \xConcept{QtConcurrent!Exceptions}
  \xConcept{Multithreading!QtConcurrent!Exceptions}
  \xConcept{Multithreading!Exceptions}
 \begin{itemize}
  \begin{cpp}
class Exception : public QtConcurrent::Exception {
  virtual Exception *clone() const;
  virtual void raise() const;
};
\end{cpp}
  \item \hhNsClsFn{QtConcurrent}{Exception}{clone} makes it possible
    to store exceptions derived from \iNsCls{QtConcurrent}{Exception}
    (by copying/cloning them).
  \item \hhNsClsFn{QtConcurrent}{Exception}{raise} makes it possible
    to re-throw these stored exceptions.
  \item \pleaseNote Even though you \emph{can} throw
    \iNsCls{QtConcurrent}{Exception} as-is, you usually want to
    subclass it, so you can transport some information from the throw
    to the catch site.
  \end{itemize}
\end{slide}
%----------------------------------------------------------------------
\begin{slide}[fragile]{0345}
\frametitle{QtConcurrent and Exceptions}
  \xConcept{Exceptions!QtConcurrent}
  \xConcept{Exceptions!Multithreading}
  \xConcept{QtConcurrent!Exceptions}
  \xConcept{Multithreading!QtConcurrent!Exceptions}
  \xConcept{Multithreading!Exceptions}
  \xExample{QtConcurrent::Exceptions}%
  \xExample{Exceptions (QtConcurrent)}%
  \xExample{Multithreading!QtConcurrent::Exceptions}%
  Example: An \hNsCls{QtConcurrent}{Exception} subclass should be
  implemented like this:
\begin{cpp}
class MyException : public QtConcurrent::Exception {

  MyException *clone() const {
    return new MyException( *this );
  }

  void raise() const { throw *this; }

  const char * what() const throw() {
    return "some descriptive string";
  }
};
\end{cpp}
\end{slide}

%----------------------------------------------------------------------
\begin{slide}[fragile]{0346}
\frametitle{QtConcurrent~\textemdash~Asynchronous Operations}
  \xConcept{Multithreading!Non-Blocking QtConcurrent}
  \xConcept{Multithreading!QtConcurrent!Non-Blocking}
  \xConcept{QtConcurrent!Asynchronous!Filter-Reduce}
  \xConcept{QtConcurrent!Asynchronous!Filtering}
  \xConcept{QtConcurrent!Asynchronous!Map-Reduce}
  \xConcept{QtConcurrent!Asynchronous!Mapping}
  \xConcept{QtConcurrent!Filter-Reduce!Asynchronous}
  \xConcept{QtConcurrent!Filtering!Asynchronous}
  \xConcept{QtConcurrent!Map-Reduce!Asynchronous}
  \xConcept{QtConcurrent!Mapping!Asynchronous}
  \begin{itemize}
  \item All blocking operations in \iNs{QtConcurrent} are also available in asynchronous form
  \item Asynchronous versions return a future instead of the value
  \item Example:%
    \xExample{Multithreading!QtConcurrent!mapped}
    \xExample{QtConcurrent!mapped}
    \xExample{mapped (QtConcurrent)}
    \begin{alltt}
\highlight{QStringList} input = ...;
QFuture<\highlight{QString}> lower =
  \iNsFnPar{QtConcurrent}{mapped}{ input, \&QString::toLower };
// do something else...
QStringList result = lower.\hClsFn{QFuture}{results};
    \end{alltt}
  \end{itemize}
\end{slide}
%----------------------------------------------------------------------
\begin{slide}{0348}
\frametitle{QtConcurrent::run}
  \xConcept{Function Object}
  \begin{itemize}
  \item You will find that you often already \emph{have} a function
    that you want to run as a task. Or you are porting from a native thread
    API.
  \item Then defining \iCls{QRunnable} subclasses yourself is trivial,
    but tedious.
  \item \iNsFn{QtConcurrent}{run} is for you.
  \item It takes a function (or \emph{function object}) and up to 5
    arguments and runs it on \iClsFn{QThreadPool}{globalInstance}.
  \item It returns a (single-valued) \iCls{QFuture} that you can plug
    into a \iCls{QFutureWatcher} to be informed about task completion.
  \item \pleaseNote Unlike the other \iNs{QtConcurrent} functions,
    this one does \emph{not} auto-parallelize the task, nor can the
    future be canceled or queried for progress.
  \end{itemize}
\end{slide}
%----------------------------------------------------------------------
\begin{slide}[fragile]{0349}
\frametitle{QtConcurrent::run}
  \xConcept{Function Object}
  \begin{itemize}
  \item The type of \iCls{QFuture} returned is determined by the
    return type of the function\ldots
\begin{alltt}\footnotesize
  \highlight{QString} func1();
  \iCls{QFuture}<\highlight{QString}> fut1 = \iClsFnPar{QtConcurrent}{run}{func1};

  \highlight{QStringList} func2( int arg1 ); int i;
  \iCls{QFuture}<\highlight{QStringList}> fut2 = \iClsFnPar{QtConcurrent}{run}{func2, i};
\end{alltt}
  \item \ldots or the \texttt{result\_type} typedef of the function object:
\begin{alltt}\footnotesize
  struct MyFunctionObject : ... \{
    typedef \highlight{int} result_type;
    int operator()( const QString \& arg ) const { ... }
  \};
  \iCls{QFuture}<\highlight{int}> fut2 = \iClsFnPar{QtConcurrent}{run}{MyFunctionObject()};\end{alltt}
  \end{itemize}
\end{slide}

%----------------------------------------------------------------------

\begin{slide}[fragile]
\frametitle{Lab: Parallel Game Of Life}
See if you can improve the performance of Conway's Game of Life by using
concurrent algorithms.
\begin{itemize}
\item You can start with \iExample{multithreading/lab-life} which solves the problem for a single thread.
\end{itemize}
\end{slide}
\centeredImage{multithreading/images/lab-life}
\end{frame}
% ----------------------------------------------------------------------
