%%%%%%%%%%%%%%%%%%%%%%%%%%%%%%%%%%%%%%%%%%%%%%%%%%%%%%%%%%%%%%%%%%%%%%%%%%
%
% Copyright (c) 2008-2011, Nokia Corporation and/or its subsidiary(-ies).
% All rights reserved.
%
% This work, unless otherwise expressly stated, is licensed under a
% Creative Commons Attribution-ShareAlike 2.5.
%
% The full license document is available from
% http://creativecommons.org/licenses/by-sa/2.5/legalcode .
%
%%%%%%%%%%%%%%%%%%%%%%%%%%%%%%%%%%%%%%%%%%%%%%%%%%%%%%%%%%%%%%%%%%%%%%%%%%

\subsection{Mouse Input}
%----------------------------------------------------------------------
\begin{slide}{1373}\frametitle{Mouse Areas}

Mouse areas define parts of the screen where cursor input occurs

\begin{itemize}
\item Placed and resized like ordinary items
  \begin{itemize}
  \item using anchors if necessary
  \end{itemize}
\item Two ways to monitor mouse input:
  \begin{itemize}
  \item handle signals
  \item dynamic property bindings
  \end{itemize}
\end{itemize}

% signals - need to figure out the order of slides for this
% properties - containsMouse with hoverEnabled, for example

\doc{qml-mousearea.html}{QML MouseArea Element Reference}
\end{slide}

%----------------------------------------------------------------------

\begin{slide}{1372}\frametitle{Clickable Mouse Area}

\flushedImage{qml-user-interaction/images/mouse-pressed.png}
% declarative-uis/user-interaction/mouse-pressed-signals.qml
\inputqml{qml-user-interaction/colorized/mouse-pressed-signals}
                                                            
\demo{qml-user-interaction/ex-mouse-input/mouse-pressed-signals.qml}
\end{slide}

%----------------------------------------------------------------------

\begin{slide}{1369}\frametitle{Mouse Hover and Properties}

\flushedImage{qml-user-interaction/images/hover-property.png}
% declarative-uis/user-interaction/hover-property.qml
\inputqml{qml-user-interaction/colorized/hover-property}
                                                     
\demo{qml-user-interaction/ex-mouse-input/hover-property.qml}
\end{slide}

%----------------------------------------------------------------------

\begin{slide}{1368}\frametitle{Enabling Mouse Hover}

\begin{qml}
\qtt{\qc{lightgray}{Rectangle~\{}}\\
\qtt{~~~~\qc{lightgray}{...}}\\
\qtt{~~~~\qc{lightgray}{color:~mouse\_area.containsMouse~?~"green"~:~"white"}}\\
\vspace*{0.5em}
\qtt{~~~~\qc{class}{MouseArea}~\{}\\
\qtt{~~~~~~~~\qc{lightgray}{id:~mouse\_area}}\\
\qtt{~~~~~~~~\qc{type}{anchors.fill}:~parent}\\
\qtt{~~~~~~~~\qc{type}{hoverEnabled}:~\qc{number}{true}}\\
\qtt{~~~~\}}\\
\qtt{\qc{lightgray}{\}}}
\end{qml}

\begin{itemize}
\item Ensure that the mouse area fills the parent's area
\item Set the \qic{type}{hoverEnabled} property
\end{itemize}

\end{slide}

%----------------------------------------------------------------------

\begin{slide}{1367}\frametitle{Naming the Mouse Area}

\begin{qml}
\qtt{\qc{lightgray}{Rectangle~\{}}\\
\qtt{~~~~\qc{lightgray}{...}}\\
\qtt{~~~~\qc{lightgray}{color:~mouse\_area.containsMouse~?~"green"~:~"white"}}\\
\vspace*{0.5em}
\qtt{~~~~\qc{class}{MouseArea}~\{}\\
\qtt{~~~~~~~~\qc{type}{id}:~mouse\_area}\\
\qtt{~~~~~~~~\qc{lightgray}{anchors.fill:~parent}}\\
\qtt{~~~~~~~~\qc{lightgray}{hoverEnabled:~true}}\\
\qtt{~~~~\}}\\
\qtt{\qc{lightgray}{\}}}
\end{qml}

\begin{itemize}
\item Give the \qic{class}{MouseArea} an \qic{type}{id} property
\item Refer to this mouse area as \qtt{mouse\_area}
  \begin{itemize}
  \item Enables access to the mouse area's properties
  \end{itemize}
\end{itemize}

\end{slide}

%----------------------------------------------------------------------

\begin{slide}{1366}\frametitle{Using a Mouse Area Property}

\begin{qml}
\qtt{\qc{class}{Rectangle}~\{}\\
\qtt{~~~~\qc{lightgray}{...}}\\
\qtt{~~~~\qc{type}{color}:~mouse\_area.containsMouse~?~\qc{string}{"green"}~:~\qc{string}{"white"}}\\
\vspace*{0.5em}
\qtt{~~~~\qc{lightgray}{MouseArea~\{}}\\
\qtt{~~~~~~~~\qc{lightgray}{id~:~mouse\_area}}\\
\qtt{~~~~~~~~\qc{lightgray}{anchors.fill:~parent}}\\
\qtt{~~~~~~~~\qc{lightgray}{hoverEnabled:~true}}\\
\qtt{~~~~\qc{lightgray}{\}}}\\
\qtt{\}}
\end{qml}

\begin{itemize}
\item \qic{type}{color} is updated when \qtt{mouse\_area.}\qic{type}{containsMouse} changes
\item Uses a simple JavaScript test
\item \qic{string}{"green"} if true
\item \qic{string}{"white"} if false
\end{itemize}

\end{slide}

%----------------------------------------------------------------------

\begin{slide}{1365}\frametitle{Mouse Area Hints and Tips}

\begin{itemize}
\item A mouse area only responds to its \qic{type}{acceptedButtons}
  \begin{itemize}
  \item the handlers are not called for other buttons, but
  \item any click involving an allowed button is reported
  \item the \qic{type}{pressedButtons} property contains \textit{all} buttons
  \item even non-allowed buttons, if an allowed button is also pressed
  \end{itemize}
\item With \qic{type}{hoverEnabled} set to \qic{number}{false}
  \begin{itemize}
  \item \qic{type}{containsMouse} can be \qic{number}{true} if the mouse area
  is clicked
  \end{itemize}
\end{itemize}

\end{slide}

%----------------------------------------------------------------------

\begin{slide}{1364}\frametitle{Signals vs. Property Bindings}

Which to use?

\begin{itemize}
\item Signals can be easier to use in some cases
  \begin{itemize}
  \item when a signal only affects one other item
  \end{itemize}
\item Property bindings rely on named elements
  \begin{itemize}
  \item many items can react to a change by referring to a property
  \end{itemize}
\item Use the most intuitive approach for the use case
\item Favor simple assignments over complex scripts
\end{itemize}

% Signal handlers are easier to use when only modifying one other object
% Bindings are easier to set up when more than one object is affected by
% a change to an item

\end{slide}

%----------------------------------------------------------------------

