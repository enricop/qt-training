%%%%%%%%%%%%%%%%%%%%%%%%%%%%%%%%%%%%%%%%%%%%%%%%%%%%%%%%%%%%%%%%%%%%%%%%%%
%
% Copyright (c) 2008-2011, Nokia Corporation and/or its subsidiary(-ies).
% All rights reserved.
%
% This work, unless otherwise expressly stated, is licensed under a
% Creative Commons Attribution-ShareAlike 2.5.
%
% The full license document is available from
% http://creativecommons.org/licenses/by-sa/2.5/legalcode .
%
%%%%%%%%%%%%%%%%%%%%%%%%%%%%%%%%%%%%%%%%%%%%%%%%%%%%%%%%%%%%%%%%%%%%%%%%%%

\subsection{Keyboard Input}  
%----------------------------------------------------------------------

\begin{slide}{1384}\frametitle{Keyboard Input}

Basic keyboard input is handled in two different use cases:

\begin{itemize}
\item Accepting text input
  \begin{itemize}
  \item \qic{class}{TextInput} and \qic{class}{TextEdit}
  \end{itemize}
\item Navigation between elements
  \begin{itemize}
  \item changing the focused element
  \item directional (arrow keys), tab and backtab
  \end{itemize}
\end{itemize}

On page \pageref{raw_keyboard_input} we will see how to handle raw keyboard
input.
\end{slide}

%----------------------------------------------------------------------

\begin{slide}{1383}\frametitle{Assigning Focus}

\flushedImageDoubleWidth{qml-user-interaction/images/textinputs.png}
\begin{itemize}
\item UIs with just one \qic{class}{TextInput}
  \begin{itemize}
  \item focus assigned automatically
  \end{itemize}
\item More than one \qic{class}{TextInput}
  \begin{itemize}
  \item need to change focus by clicking
  \end{itemize}
\item What happens if a \qic{class}{TextInput} has no text?
  \begin{itemize}
  \item no way to click on it
  \item unless it has a \qic{type}{width} or uses anchors
  \end{itemize}
\item Set the \qic{type}{focus} property to assign focus
\end{itemize}

\end{slide}

%----------------------------------------------------------------------

\begin{slide}{1382}\frametitle{Using TextInputs}

\flushedImageDoubleWidth{qml-user-interaction/images/textinputs.png}
% declarative-uis/user-interaction/textinputs.qml
\inputqml{qml-user-interaction/colorized/textinputs}

% Tab handling                                   

\end{slide}

%----------------------------------------------------------------------

\begin{slide}{1381}\frametitle{Getting the Focus}

% declarative-uis/user-interaction/textinputs.qml
\begin{qml}
\qtt{~~~~\qc{class}{TextInput}~\{}\\
\qtt{~~~~~~~~\qc{type}{anchors.left}:~parent.left;~\qc{lightgray}{y:~16}}\\
\qtt{~~~~~~~~\qc{type}{anchors.right}:~parent.right}\\
\qtt{~~~~~~~~\qc{lightgray}{text:~"Field~1";~font.pixelSize:~32}}\\
\qtt{~~~~~~~~\qc{lightgray}{color:~focus~?~"black"~:~"gray"}}\\
\qtt{~~~~~~~~\qc{type}{focus}:~\qc{number}{true}}\\
\qtt{~~~~\}}
\end{qml}

\begin{itemize}
\item Set the \qic{type}{focus} property
\item Use anchors to ensure there is something to click on
\end{itemize}

\demo{qml-user-interaction/ex-key-input/textinputs.qml}

\end{slide}

%----------------------------------------------------------------------

\begin{slide}{1380}\frametitle{Focus Navigation}

\flushedImageDoubleWidth{qml-user-interaction/images/tab-navigation.png}
% declarative-uis/user-interaction/tab-navigation.qml
\begin{qml}
\qtt{~~~~\qc{class}{TextInput}~\{}\\
\qtt{~~~~~~~~\qc{type}{id}:~name\_field}\\
\qtt{~~~~~~~~...}\\
\qtt{~~~~~~~~\qc{type}{focus}:~\qc{number}{true}}\\
\qtt{~~~~~~~~\qc{type}{KeyNavigation.tab}:~address\_field}\\
\qtt{~~~~\}}\\
\vspace*{0.5em}
\qtt{~~~~\qc{class}{TextInput}~\{}\\
\qtt{~~~~~~~~\qc{type}{id}:~address\_field}\\
\qtt{~~~~~~~~...}\\
\qtt{~~~~~~~~\qc{type}{KeyNavigation.backtab}:~name\_field}\\
\qtt{~~~~\}}
\end{qml}

\vspace*{0.5em}
\begin{itemize}
\item The \qtt{name\_field} item defines \qic{type}{KeyNavigation.tab}
  \begin{itemize}
  \item pressing \textbf{Tab} moves focus to the \qtt{address\_field} item
  \end{itemize}
\item The \qtt{address\_field} item defines \qic{type}{KeyNavigation.backtab}
  \begin{itemize}
  \item pressing \textbf{Shift+Tab} moves focus to the \qtt{name\_field} item
  \end{itemize}
\end{itemize}     

\demo{qml-user-interaction/ex-key-input/tab-navigation.qml}


\end{slide}

%----------------------------------------------------------------------

\begin{slide}{1379}\frametitle{Key Navigation}

\flushedImage{qml-user-interaction/images/key-navigation-red.png}
% declarative-uis/user-interaction/key-navigation.qml
\inputqml{qml-user-interaction/colorized/key-navigation}

\begin{itemize}
\item Using cursor keys with non-text items
\item Non-text items can have focus, too
\end{itemize}

\end{slide}

%----------------------------------------------------------------------

\begin{slide}{1378}\frametitle{Key Navigation}

\flushedImage{qml-user-interaction/images/key-navigation-red.png}
% declarative-uis/user-interaction/key-navigation.qml
\begin{qml}
\qtt{\qc{class}{Rectangle}~\{~\qc{type}{id}:~left\_rect}\\
\qtt{~~~~~~~~~~~~\qc{lightgray}{x:~25;~y:~25;~width:~150;~height:~150}}\\
\qtt{~~~~~~~~~~~~\qc{lightgray}{color:~focus~?~"red"~:~"darkred"}}\\
\qtt{~~~~~~~~~~~~\qc{type}{KeyNavigation.right}:~right\_rect}\\
\qtt{~~~~~~~~~~~~\qc{type}{focus}:~\qc{number}{true}~\}}\\
\end{qml}
\flushedImage{qml-user-interaction/images/key-navigation-green.png}
\vspace*{0.5em}
\begin{qml}
\qtt{\qc{class}{Rectangle}~\{~\qc{type}{id}:~right\_rect}\\
\qtt{~~~~~~~~~~~~\qc{lightgray}{x:~225;~y:~25;~width:~150;~height:~150}}\\
\qtt{~~~~~~~~~~~~\qc{type}{color}:~focus~?~\qc{string}{"\#00ff00"}~:~\qc{string}{"green"}}\\
\qtt{~~~~~~~~~~~~\qc{type}{KeyNavigation.left}:~left\_rect~\}}
\end{qml}

\vspace*{1em}
\begin{itemize}
\item \qtt{left\_rect} has the focus initially
\item Define relationships between \qtt{left\_rect} and \qtt{right\_rect}
  \begin{itemize}
  \item using \qic{type}{id} and \qic{type}{KeyNavigation}
  \end{itemize}
\end{itemize}       

\demo{qml-user-interaction/ex-key-input/key-navigation.qml}

\end{slide}

%----------------------------------------------------------------------

\begin{slide}{1377}\frametitle{Summary}

Mouse and cursor input handling:

\begin{itemize}
\item \qic{class}{MouseArea} receives clicks and other events
\item Use anchors to fill objects and make them clickable
\item Respond to user input:
  \begin{itemize}
  \item give the area a name and refer to its properties, or
  \item use handlers in the area and change other named items
  \end{itemize}
\end{itemize}

Key handling:

\begin{itemize}
\item \qic{class}{TextInput} and \qic{class}{TextEdit} provide text entry features
\item Set the \qic{type}{focus} property to start receiving key input
\item Use anchors to make items clickable
  \begin{itemize}
  \item lets the user set the focus
  \end{itemize}
\item \qic{class}{KeyNavigation} defines relationships between items
  \begin{itemize}
  \item enables focus to be moved
  \item using cursor keys, tab and backtab
  \item works with non-text-input items
  \end{itemize}
\end{itemize}

\end{slide}       


%----------------------------------------------------------------------
\begin{slide}{1376}\frametitle{Exercise~\textendash~User~Input}

\begin{itemize}
\item Which element is used to receive mouse clicks?
% MouseArea.
\item Name two ways \qic{class}{TextInput} can obtain the input focus?
% User clicks and by setting its focus property.
\item How do you define keyboard navigation between items?
% Define KeyNavigation.<key> for each key used for navigation, referring to the
% items they give the focus to.
\end{itemize}

\end{slide}

%----------------------------------------------------------------------
\begin{slide}{1375}\frametitle{Lab~\textendash~Menu~Screen}

\centeredImage{qml-user-interaction/images/menu-screen.png}

\begin{itemize}
\item Using the partial solution as a starting point, create a user interface
similar to the one shown above with these features:
  \begin{itemize}
  \item items that change color when they have the focus
  \item clicking an item gives it the focus
  \item the current focus can be moved using the cursor keys
  \end{itemize}
\end{itemize}       

\lab{qml-user-interaction/lab-menu-screen}

\end{slide}                                 
