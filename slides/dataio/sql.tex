%%%%%%%%%%%%%%%%%%%%%%%%%%%%%%%%%%%%%%%%%%%%%%%%%%%%%%%%%%%%%%%%%%%%%%%%%%
%
% Copyright (c) 2008-2011, Nokia Corporation and/or its subsidiary(-ies).
% All rights reserved.
%
% This work, unless otherwise expressly stated, is licensed under a
% Creative Commons Attribution-ShareAlike 2.5.
%
% The full license document is available from
% http://creativecommons.org/licenses/by-sa/2.5/legalcode .
%
%%%%%%%%%%%%%%%%%%%%%%%%%%%%%%%%%%%%%%%%%%%%%%%%%%%%%%%%%%%%%%%%%%%%%%%%%%

\subsection{SQL Database API} \label{sql}

%----------------------------------------------------------------------
\begin{slide}{0516}\frametitle{Overview}
\begin{itemize}
\item We will cover
 \begin{itemize}
  \item Connecting to a database.
  \item Invoking SQL queries using \iCls{QSqlQuery}.
  \item SQL Models
  \end{itemize}
  \item Not covered
    \begin{itemize}
    \item Developing database drivers
    \end{itemize}
\end{itemize}
\end{slide}


%----------------------------------------------------------------------
\begin{slide}{0514}\frametitle{SQL Drivers}
\xConcept{SQL!Supported databases}
\begin{itemize}
\item \textbf{Qt's SQL}
  \begin{itemize}
  \item Cross platform
  \item Database independent
  \end{itemize}
\item \textbf{Supported drivers are:}
  \begin{multicols}{2}

     \begin{itemize}
    \begin{footnotesize}
      \item MySQL (QMYSQL)
      \item PostgreSQL (QPSQL)
      \item SQLite (QSQLITE)
      \item ODBC (QODBC)
      \item IBM DB2 (QDB2)
      \item Borland Interbase (QIBASE)
      \item Oracle Call Interface (QOCI)
      \item Sybase Adaptive Server (QTDS)
        \medskip
  \end{footnotesize}
      \end{itemize}
   \end{multicols}
\end{itemize}
\begin{itemize}
\item Extra drivers can be developed as Qt plug-ins.
\item Enable SQL module, with \texttt{QT += sql} in qmake file
\end{itemize}
 
\doc{sql-driver.html}{SQL Database Drivers}

\end{slide}

%----------------------------------------------------------------------
\begin{slide}[fragile]{0518}\frametitle{Connecting to a Database}
\begin{itemize}
\item Connect to PostgreSQL database
  \begin{cpp}
// choose driver by name
QSqlDatabase db = QSqlDatabase::addDatabase("QPSQL");
db.setHostName("acidalia");
db.setDatabaseName("customdb");
db.setUserName("mojito");
db.setPassword("J0a1m8");
bool ok = db.open();    
  \end{cpp}

\item Supporting multi databases
  \begin{cpp}
QSqlDatabase::addDatabase("QPSQL", "db1");
// retrieve database by name
QSqlDatabase db1 = QSqlDatabase::database("db1");
  \end{cpp}
\item By default all SQL API uses the default connection 
\end{itemize}
\end{slide}

%----------------------------------------------------------------------
\begin{slide}[fragile]{0520}\frametitle{Obtaining Error Messages}
\begin{itemize}
\item \textbf{\iCls{QSqlError}}
  \begin{cpp}
QSqlDatabase db ...
QSqlError error = db.lastError();    
qDebug() << "database error: " << error.databaseText();
qDebug() << "driver error: " << error.driverText();
qDebug() << "combined message: " << error.text();
  \end{cpp}
\item \iCls{QSqlQuery} supports same error reporting
\item \textit{Note: } Messages are most likely not localized. 
\end{itemize}
\end{slide}

%----------------------------------------------------------------------
\begin{slide}[fragile]{0522}
\frametitle{Invoking SQL Queries}
\xConcept{SQL!Simple queries}
\begin{itemize}
\item \iCls{QSqlQuery} - used to run SQL queries
  \begin{cpp}
QSqlQuery query;
if ( query.exec( "SELECT name FROM author" ) ) {
  while( query.next() ) {
  // ask for field by index ( 0 = name )
  QString name = query.value(0).toString();
  // do something useful with name
}  else {
  qDebug() << query.lastError();
}
  \end{cpp}
\item Alternative query with sql statement, specific database
  \begin{cpp}
QSqlQuery query("SELECT name FROM author", db1);
// query is automatically executed; error checking with isActive()
while( query.next() ) {
...
  \end{cpp}
\end{itemize}
\end{slide}

%----------------------------------------------------------------------
\begin{slide}[fragile]{0523}\frametitle{Navigating and Querying Results}
\begin{itemize}
\item \textbf{Navigating result}
  \begin{cpp}
QSqlQuery query("SELECT name from author");
// [ Anders, Relative, Chris, Donald ] 
query.first(); // "Anders"
query.last(); // "Donald"
query.prev(); // "Chris"
query.seek(-1, true); // "Relative" 
 \end{cpp}
\item \textbf{Querying result}
  \begin{cpp}
query.size(); // number of rows
query.numRowsAffected(); // e.g. update statement
  \end{cpp}
\item \demo{dataio/ex-query}
\end{itemize}
\end{slide}


%----------------------------------------------------------------------
\begin{slide}[fragile]{0525}\frametitle{Inserting Records with Prepared Queries}
\begin{itemize}
\item \textbf{Named Bindings}
  \begin{cpp}
QSqlQuery query;
query.prepare("INSERT INTO person (id, forename, surname) "
              "VALUES (:id, :forename, :surname)");
// for each data you want to insert
    query.bindValue(":id", 1001);
    query.bindValue(":forename", "Bart");
    query.bindValue(":surname", "Simpson");
    query.exec();  
\end{cpp}
\item \textbf{Positional Bindings}
  \begin{cpp}
QSqlQuery query;
query.prepare("INSERT INTO person (id, forename, surname) "
              "VALUES (?, ?, ?)");
// for each data you want to insert
    query.bindValue(0, 1001);
    query.bindValue(1, "Bart");
    query.bindValue(2, "Simpson");
    query.exec();
  \end{cpp}
\end{itemize}
\end{slide}

\subsection{SQL Models}

%----------------------------------------------------------------------
\begin{slide}[fragile]{0529}\frametitle{QSqlQueryModel}
\xConcept{SQL!Model/View}
\xConcept{Model/View!SQL}
\begin{itemize}
\item Read-only data model for SQL result set
\item Derived from \iCls{QAbstractTableModel}
  \begin{itemize}
  \item Wraps a \iCls{QSqlQuery}
  \end{itemize}
\item By default headers taken from database
  \begin{itemize}
  \item \iClsFn{QSqlQueryModel}{setHeaderData} - custom headers
  \end{itemize}
\item \textbf{Using \iCls{QSqlQueryModel}}
  \begin{cpp}
QSqlQueryModel *model = new QSqlQueryModel;
model->setQuery("SELECT name, salary FROM employee");
model->setHeaderData(0, Qt::Horizontal, tr("Name"));
model->setHeaderData(1, Qt::Horizontal, tr("Salary"));
QQuickView *view = new QQuickView;
view->engine()->rootContext()->setContextProperty("_model", model);
view->show();    
  \end{cpp}
\item \demo{dataio/ex-query-model}
\end{itemize}
\end{slide}

%----------------------------------------------------------------------
\begin{slide}[fragile]{0530}\frametitle{Working on a SQL Table - QSqlTableModel}
\xConcept{SQL!Tables}
\begin{itemize}
\item Wraps a single table in a model
\item Allows editing of data
\item Using \iCls{QSqlTableModel}
  \begin{itemize}
  \begin{cpp}
QSqlTableModel *model = new QSqlTableModel;
model->setTable("employee");
model->setEditStrategy(QSqlTableModel::OnManualSubmit);
model->select(); // execute query
model->removeColumn(0); // don't show the ID

QQuickView *view = new QQuickView;
view->engine()->rootContext()->setContextProperty("_model", model);
view->show();    
  \end{cpp}
\item Optionally
  \begin{itemize}
  \item \texttt{setFilter(...)} - \texttt{WHERE} part of query
  \item \texttt{setSort(...)} - sort by column
  \end{itemize}
 \end{itemize}
\end{itemize}
\end{slide}

%----------------------------------------------------------------------
\begin{slide}[fragile]{0531}\frametitle{Convenience Methods of QSqlTableModel}
\begin{itemize}
\item All \iCls{QAbstractItemModel} methods  are available
\item Additional record based access
  \begin{itemize}
  \item \texttt{record( row )}
  \item \texttt{setRecord( row, record )}
  \item \texttt{insertRecord(row, record)}
  \end{itemize}
\item \iCls{QSqlRecord}
  \begin{itemize}
  \item Encapsulates database record (row in table) 
  \end{itemize}
\item Updating price in table
  \begin{cpp}
for ( int row = 0; row < model->rowCount(); ++row ) {
  QSqlRecord record = model->record( row );
  double price = record.value( "price" ).toDouble();
  price *= 1.1;
  record.setValue( "price", price );
  model->setRecord( row, record );
}
model->submitAll();
    \end{cpp}
\end{itemize}
\end{slide}


%----------------------------------------------------------------------
\begin{slide}{0533}\frametitle{Commit Strategies}
\begin{itemize}
\item \hClsFn{QSqlTableModel}{setEditStrategy}
  \begin{itemize}
  \item Strategy when changes are commited to database
  \end{itemize}
\medskip
\item \textbf{OnFieldChange}
  \begin{itemize}
  \item Changes applied immediately to database
  \end{itemize}
\item \textbf{OnRowChange}
  \begin{itemize}
  \item Changes applied when user selects a different row
  \item Discard with \texttt{revert()}
  \end{itemize}
\item \textbf{OnManualSubmit}
  \begin{itemize}
  \item Changes only applied on \texttt{submit()}
  \item Discard with \texttt{revertAll()}
  \end{itemize}
\end{itemize}
\medskip
\begin{footnotesize}
\emph{Note:} \texttt{OnFieldChange} has performance hit (frequent update)
\end{footnotesize}
\end{slide}

%----------------------------------------------------------------------
\begin{slide}[fragile]{0535}\frametitle{Relations between Table QSqlRelationalTableModel}
\begin{itemize}
\item Editable model for database table, with foreign key support
  \begin{cpp}
model->setTable("employee");
// map column 2 to city table column id
// and show name from city table in view
model->setRelation(2, QSqlRelation("city", "id", "name"));
// map column 3 to country table column id 
// and show name from country table in view
model->setRelation(3, QSqlRelation("country", "id", "name"));    
  \end{cpp}
\medskip

\item \demo{dataio/ex-relational-table-model}
\end{itemize}
\end{slide}

%----------------------------------------------------------------------
\begin{slide}{0536}\frametitle{Editable Queries Using QSqlQueryModel}
\begin{itemize}
\item \iCls{QSqlQueryModel} is read only by default
\item \iCls{QSqlTableModel} works only on single table
\medskip
\item To enable arbitrary editable query models, override
  \begin{itemize}
  \item \iClsFn{QAbstractItemModel}{setData}
    \begin{itemize}
    \item to update the data yourself
    \end{itemize}
  \item \iClsFn{QAbstractItemModel}{flags}
    \begin{itemize}
    \item to specify that the table is editable
    \end{itemize}  
  \end{itemize}

\item \demo{dataio/ex-editable-query}
\end{itemize}
\end{slide}

%----------------------------------------------------------------------
\begin{slide}{0537}\frametitle{Database Transactions}
\xConcept{SQL!Transactions}
\begin{itemize}
\item \iClsFn{QSqlDatabase}{transaction}
  \begin{itemize}
  \item starts a transaction
  \end{itemize}
\item \iClsFn{QSqlDatabase}{commit} \\
  \iClsFn{QSqlDatabase}{rollback}
  \begin{itemize}
  \item Ends the transaction
  \item Returns true on success
  \end{itemize}
\medskip
\item \iClsFnPar{QSqlDriver}{hasFeature}{QSqlDriver::Transactions}
  \begin{itemize}
  \item Check if database supports transactions
  \end{itemize} 
\end{itemize}
\end{slide}

%----------------------------------------------------------------------
\begin{slide}{0538}\frametitle{Lab: Bookstore}
\flushedImage{dataio/images/sql-exercise}\xProject{SQL}
\begin{itemize}
\item Author table in upper view
\item Book table in lower view
\item Only books from current author
\item Follow these steps:
  \begin{enumerate}
  \item Setup the author table
  \item Use current item change
  \item Setup book table with \iCls{QSqlQueryModel}
  \item Rerun the query
  \item Provide edit support for both tables
  \item Optional
    \begin{itemize}
    \item Support add/delete rows
    \end{itemize}
 \end{enumerate}
\end{itemize}
\lab{dataio/lab-bookstore}
\end{slide}
