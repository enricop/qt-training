%%%%%%%%%%%%%%%%%%%%%%%%%%%%%%%%%%%%%%%%%%%%%%%%%%%%%%%%%%%%%%%%%%%%%%%%%%
%
% Copyright (c) 2008-2011, Nokia Corporation and/or its subsidiary(-ies).
% All rights reserved.
%
% This work, unless otherwise expressly stated, is licensed under a
% Creative Commons Attribution-ShareAlike 2.5.
%
% The full license document is available from
% http://creativecommons.org/licenses/by-sa/2.5/legalcode .
%
%%%%%%%%%%%%%%%%%%%%%%%%%%%%%%%%%%%%%%%%%%%%%%%%%%%%%%%%%%%%%%%%%%%%%%%%%%

\subsection{Basic XML APIs}


%----------------------------------------------------------------------
\begin{slide}[fragile]{0009}
\frametitle{\iConcept{XML} APIs in Qt} \label{xml}
\begin{itemize}
\item \textbf{SAX (version 2)}
  \begin{itemize}
  \item Serial stream parser API
  \item User defines callbacks
  \item Callbacks called when XML events occur
  \item Suited when data is processed in one-pass
  \end{itemize}
\item\textbf{ DOM (level 1 and 2)}
  \begin{itemize}
  \item Tree node view on data
  \item Suited when data is accessed repeatedly or out of sequence 
  \end{itemize}
\item \textbf{XML Stream Reader/Writer}
  \begin{itemize}
  \item Serial stream parser API
  \item User pulls token on own convenience
  \item Easy to build recursive descent parsers
  \item Extremly fast and memory conservative
  \item Preferred XML reader/writer if no DOM is required
  \end{itemize}
\item For SAX and DOM APIs, add \verb!QT += xml!. XML Stream APIs are in QtCore
\end{itemize}
\end{slide}

%----------------------------------------------------------------------
\subsubsection{SAX}
\begin{slide}[fragile]{1724}\frametitle{Using \iConcept{SAX}}
  \begin{itemize}
  \item Simple usage of SAX with a custom xml handler
  \item Preparing the reader
 \begin{cpp}
// Own implementation of QXmlDefaultHandler
CustomXmlHandler handler;
// Standard xml parser
QXmlSimpleReader reader;
// register handler with the reader
reader.setContentHandler(&handler);
reader.setErrorHandler(&handler);
\end{cpp}
\item Feeding xml file to reader
\begin{cpp}
QFile xmlFile ...;
QXmlInputSource source(&xmlFile);
// parse reports xml tokens via callbacks to your handler
reader.parse(source);
  \end{cpp}
  \end{itemize}

\end{slide}
%----------------------------------------------------------------------
\begin{slide}{0011}\frametitle{SAX Handlers}
\begin{itemize}
  \item Several abstract handler classes that define SAX events:
    \begin{itemize}
    \item \textbf{\iCls{QXmlContentHandler}}
      \begin{itemize}
      \item for content-related events
      \end{itemize}
    \item \textbf{\iCls{QXmlErrorHandler}}
      \begin{itemize}
      \item for errors, fatal errors, and warnings
      \end{itemize}

    \item \textbf{\iCls{QXmlEntityResolver}}
      \begin{itemize}
      \item for external entities that need resolution
      \end{itemize}
    \item \textbf{\iCls{QXmlDTDHandler}}
      \begin{itemize}
      \item for DTD-related events
      \end{itemize}
    \item \textbf{\iCls{QXmlDeclHandler}}
      \begin{itemize}
      \item for rarely used DTD-related events
      \end{itemize}
    \item  \textbf{\iCls{QXmlLexicalHandler}}
      \begin{itemize}
      \item for events related to the lexical structure. (start/end of DTD, comments,...)
      \end{itemize}
    \end{itemize}
  \item \textbf{\iCls{QXmlDefaultHandler}} - easiest to subclass 
    \begin{itemize}
    \item provides empty implementation for all callbacks
    \end{itemize}

  \end{itemize}
\end{slide}
%----------------------------------------------------------------------
\begin{slide}[fragile]{1723}\frametitle{Using QXmlDefaultHandler}
  \begin{itemize}
  \item Register it for each role it has
  \item Out custom handler
    \begin{cpp}
class CustomXmlHandler : public QXmlDefaultHandler { ... }
    \end{cpp}
  \item Registration for xml event roles
    \begin{cpp}
CustomXmlHandler handler;
QXmlSimpleReader reader;
// register handler for the roles
reader.setContentHandler( &handler );
reader.setErrorHandler( &handler );
    \end{cpp}
  \item See \qtdemo{examples/xml/saxbookmarks}
  \item or \doc{xml-saxbookmarks.html}{Sax Bookmarks}
  \end{itemize}
\end{slide}


%----------------------------------------------------------------------
\subsubsection{DOM}
\begin{slide}[fragile]{0017}
\label{qdom}
        \frametitle{\iConcept{DOM} - The \iConcept{Document Object Model}}\label{qdom}
\begin{itemize}
  \item Accessing XML in tree form
  \item Entire XML document is kept memory
  \item Everything is a \emph{node} (in Qt: \iCls{QDomNode}).
  \item DOM is specified by the W3C
  \item DOM is awkward in places, but has the advantage of being standardized.
  \end{itemize}
\end{slide}
%----------------------------------------------------------------------
\begin{slide}[fragile]{1722}\frametitle{Working with DOM}
\begin{itemize}
\item Print tag names of all elements
   \begin{cpp}
QFile xmlFile ...;
QDomDocument doc;      
doc.setContent(&xmlFile);
// retrieve root element
QDomElement root = doc.documentElement();
QDomNode node = root.firstChild();
while(!node.isNull()) { // always test a node
  QDomElement element = node.toElement();
  if(!element.isNull()) { // test if node is an element
    qDebug() << element.tagName();
  }
  node = node.nextSibling();
}
    \end{cpp}
  \item Iterate over elements
    \begin{itemize}
    \item \hClsFn{QDomNode}{firstChildElement}, \hClsFn{QDomNode}{nextSiblingElement}
    \end{itemize}
  \item \demo{dataio/ex-qdom}
  \end{itemize}
\end{slide}
%----------------------------------------------------------------------
\begin{slide}[fragile]{1721}\frametitle{Writing XML with DOM}
\begin{itemize}
\item Create HTML page with one anchor
  \begin{cpp}
QDomDocument doc;
QDomElement html = doc.createElement("html");
doc.appendChild(html);
QDomElement body = doc.createElement("body");
html.appendChild(body);
QDomElement anchor = doc.createElement("a");
anchor.setAttribute("href", "http://qt.nokia.com/training/");
QDomText text = doc.createTextNode("Qt Training");
anchor.appendChild(text);
body.appendChild(anchor);
qDebug() << doc.toString(); // print DOM tree
  \end{cpp}
\item Writing the DOM
  \begin{cpp}
QTextStream stream(&file); // stream for writeable file
doc.save(stream, 4); // write to file with indent = 4    
  \end{cpp}
 \item \demo{dataio/ex-qdom-write}
  \end{itemize}
\end{slide}

\subsubsection{XML Streaming API}
%----------------------------------------------------------------------
\begin{slide}[fragile]{1720}\frametitle{Pull Parser - QXmlStreamReader} \label{QXmlStreamReader}
\begin{itemize}
\item Provides fast parser for reading via a simple streaming API
\item Faster and more convenient replacement for SAX parser
  \begin{cpp}
QXmlStreamReader xml;
...
while ( !xml.atEnd() ) {
  xml.readNext(); // pulls next token
  ... // do processing
}
if ( xml.hasError() ) {
  ... // do error handling
}    
  \end{cpp}
\item Determine type of Token
  \begin{itemize}
  \item \hClsFn{QXmlStreamReader}{isStartDocument}
  \item \hClsFn{QXmlStreamReader}{isStartElement}
  \item \hClsFn{QXmlStreamReader}{isEndElement}
  \item \hClsFn{QXmlStreamReader}{isCharacters}
 \end{itemize}
\end{itemize}
\end{slide}

%----------------------------------------------------------------------
\begin{slide}[fragile]{1719}\frametitle{QXmlStreamReader - Errors}
\begin{itemize}
\item \textbf{If an error occurs}
  \begin{itemize}
  \item \texttt{readNext()} returns \texttt{EndDocument}
  \item and \texttt{atEnd()} returns \texttt{true}
  \end{itemize}
\item \textbf{\texttt{hasError()} - test if error exists}
  \begin{itemize}
  \item \texttt{error()} - Returns type of error
  \item \texttt{errorString()} - Returns error message
  \end{itemize}
\item \textbf{\texttt{raiseError( msg )} - Raises a custom error}
\end{itemize}
\end{slide}

%----------------------------------------------------------------------
\begin{slide}{0015}\frametitle{QXmlStreamReader - Providing XML}
\begin{itemize}
\item \textbf{Provide XML content via}
  \begin{itemize}
  \item \texttt{addData(...)} as \iCls{QString} or \iCls{QByteArray}
  \item \texttt{setDevice(...)} as IO device (e.g. file, socket, ...)
  \end{itemize}
\item \textbf{If data ends prematurely}
  \begin{itemize}
  \item \texttt{readNext} returns \texttt{PrematureEndOfDocumentError}
  \item and \texttt{atEnd()} return \texttt{false}
  \item Possible to resume once data is available
  \end{itemize}
\end{itemize}
\qtdemo{examples/xml/streambookmarks}
\end{slide}

%----------------------------------------------------------------------
\begin{slide}{0016}\frametitle{Writing XML with \iCls{QXmlStreamWriter}}
\begin{itemize}
\item Allows to write XML using high level functions
\item \textbf{Methods for writing}
  \begin{itemize}
  \item \hClsFn{QXmlStreamWriter}{writeStartDocument}
  \item \hClsFn{QXmlStreamWriter}{writeStartElement}
  \item \hClsFn{QXmlStreamWriter}{writeEndElement}
  \item \hClsFn{QXmlStreamWriter}{writeAttribute}
  \item \hClsFn{QXmlStreamWriter}{writeCharacters}.
  \end{itemize}
\item \textbf{\texttt{setDevice(QIODevice *)}} - device to write to
\item  \textbf{\texttt{setAutoFormatting(true)}}
  \begin{itemize}
  \item Generates human readable XML
  \end{itemize}
\item \demo{dataio/ex-xmlstreamwriter} 
\end{itemize}
\end{slide}

%----------------------------------------------------------------------

\begin{slide}[fragile]{1718}\frametitle{Lab: Reading and writing XML keys}
  \begin{itemize}
  \item KeyEngine class allows storing key-value pairs
  \item Your task to write XML read/write backends
  \item XML Format shall be:
    \begin{xml}
<?xml version="1.0" encoding="UTF-8"?>
<keys version="1.0">
  <item key="Key-0">Value-0</item>
  <item key="Key-1">Value-1</item>
  ...
  <item key="Key-9">Value-9</item>
</keys>      
    \end{xml}
  \item \lab{dataio/lab-xmlkeys}
  \item \textbf{Optional}
    \begin{itemize}
  \item No optional tasks
    \end{itemize}
\end{itemize}

\end{slide}





