%%%%%%%%%%%%%%%%%%%%%%%%%%%%%%%%%%%%%%%%%%%%%%%%%%%%%%%%%%%%%%%%%%%%%%%%%%
%
% Copyright (c) 2008-2011, Nokia Corporation and/or its subsidiary(-ies).
% All rights reserved.
%
% This work, unless otherwise expressly stated, is licensed under a
% Creative Commons Attribution-ShareAlike 2.5.
%
% The full license document is available from
% http://creativecommons.org/licenses/by-sa/2.5/legalcode .
%
%%%%%%%%%%%%%%%%%%%%%%%%%%%%%%%%%%%%%%%%%%%%%%%%%%%%%%%%%%%%%%%%%%%%%%%%%%

\subsection{Declarative Environment}

\begin{slide}{2075}\frametitle{Overview}

Qt Quick is a combination of technologies:

\begin{itemize}
\item A set of components, some graphical
\item A declarative language: QML
  \begin{itemize}
  \item based on JavaScript
  \item running on a virtual machine
  \end{itemize}
\item A C++ API for managing and interacting with components
  \begin{itemize}
  \item the \textbf{QtDeclarative} module
  \end{itemize}
\end{itemize}

\end{slide}

%----------------------------------------------------------------------
\begin{slide}[fragile]{1526}\frametitle{Setting up a Declarative View}

\begin{lstlisting}
#include <QGuiApplication>
#include <QQuickView>
#include <QUrl>

int main(int argc, char *argv[])
{
    QGuiApplication app(argc, argv);
    QQuickView view;
    view.setSource(QUrl("qrc:animation.qml"));
    view.show();
    return app.exec();
}
\end{lstlisting}

\demo{qml-cpp-integration/ex-simpleviewer}
\end{slide}

%----------------------------------------------------------------------
\begin{slide}[fragile]{1525}\frametitle{Setting up QtDeclarative}

\begin{qmake}
QT       += quick
RESOURCES = simpleviewer.qrc
SOURCES   = main.cpp
\end{qmake}

\end{slide}

\subsection{Exporting C++ objects to QML}
%----------------------------------------------------------------------
\begin{slide}[fragile]{1527}\frametitle{Exporting C++ objects to QML}
\begin{itemize}
\item C++ objects can be exported to QML
\begin{cpp}
class User : public QObject
{
    Q_OBJECT
    Q_PROPERTY(QString name READ name WRITE setName
               NOTIFY nameChanged)
    Q_PROPERTY(int age READ age WRITE setAge NOTIFY ageChanged)

public:
    User(const QString &name, int age, QObject *parent = 0);
    ...
}
\end{cpp}
\item The notify signal is needed for correct property bindings!
\item Q\_PROPERTY must be at top of class
\end{itemize}

\end{slide}

%----------------------------------------------------------------------

\begin{slide}[fragile]{1528}\frametitle{Exporting C++ objects to QML}
\begin{itemize}
\item \iCls{QQmlContext} exports the instance to QML.\\[4mm]

\begin{cpp}
void main( int argc, char* argv[] ) {
    ...
    User *currentUser = new User("Alice", 29);

    QAbstractItemModel *thingsModel = createModel();

    QQuickView *view = new QQuickView;
    QQmlContext *context = view->engine()->rootContext();

    context->setContextProperty("_currentUser", currentUser);
    context->setContextProperty("_favoriteThings", thingsModel);

    ...
}
\end{cpp}
\end{itemize}
\end{slide}

%----------------------------------------------------------------------

\begin{slide}[fragile]{2077}\frametitle{Using the object in QML}

%Text {
%    text : _currentUser.name
%    ...
%}
%
%ListView {
%    model : _favoriteThings
%    ...
%}
\begin{itemize}
\item Use the instances like any other QML object\\[5mm]

\footnotesize
\qtt{\qc{class}{Text}~\{}\\
\qtt{~~~~\qc{type}{text}~:~\_currentUser.name}\\
\qtt{~~~~...}\\
\qtt{\}}\\
\vspace*{0.5em}
\qtt{\qc{class}{ListView}~\{}\\
\qtt{~~~~\qc{type}{model}~:~\_favoriteThings}\\
\qtt{~~~~...}\\
\qtt{\}}\\
\end{itemize}
\end{slide}

%----------------------------------------------------------------------
\begin{slide}[fragile]{2150}\frametitle{What is exported?}
 Accessible from QML:
    \begin{itemize}
    \item Properties
    \item Signals
    \item Slots
    \item Methods marked with Q\_INVOKABLE
    \item Enums registered with Q\_ENUMS
      \begin{cpp}
class Circle {
  Q_ENUMS(Style)

public:
  enum Style { Outline, Filled };
  ...
};
      \end{cpp}
    \end{itemize}
\end{slide}
