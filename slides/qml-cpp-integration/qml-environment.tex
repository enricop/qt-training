%%%%%%%%%%%%%%%%%%%%%%%%%%%%%%%%%%%%%%%%%%%%%%%%%%%%%%%%%%%%%%%%%%%%%%%%%%
%
% Copyright (c) 2008-2011, Nokia Corporation and/or its subsidiary(-ies).
% All rights reserved.
%
% This work, unless otherwise expressly stated, is licensed under a
% Creative Commons Attribution-ShareAlike 2.5.
%
% The full license document is available from
% http://creativecommons.org/licenses/by-sa/2.5/legalcode .
%
%%%%%%%%%%%%%%%%%%%%%%%%%%%%%%%%%%%%%%%%%%%%%%%%%%%%%%%%%%%%%%%%%%%%%%%%%%

\subsection{Declarative Environment}

\begin{slide}{2075}\frametitle{Overview}

Qt Quick is a combination of technologies:

\begin{itemize}
\item A set of components, some graphical
\item A declarative language: QML
  \begin{itemize}
  \item based on JavaScript
  \item running on a virtual machine
  \end{itemize}
\item A C++ API for managing and interacting with components
  \begin{itemize}
  \item the \textbf{QtDeclarative} module
  \end{itemize}
\end{itemize}

\end{slide}

%----------------------------------------------------------------------
\begin{slide}[fragile]{1526}\frametitle{Setting up a Declarative View}

\begin{lstlisting}
#include <QApplication>
#include <QDeclarativeView>
#include <QUrl>

int main(int argc, char *argv[])
{
    QApplication app(argc, argv);
    QDeclarativeView view;
    view.setSource(QUrl("qrc:files/animation.qml"));
    view.show();
    return app.exec();
}
\end{lstlisting}

\demo{qml-cpp-integration/ex-simpleviewer}
\end{slide}

%----------------------------------------------------------------------
\begin{slide}[fragile]{1525}\frametitle{Setting up QtDeclarative}

\begin{qmake}
QT       += declarative
RESOURCES = simpleviewer.qrc
SOURCES   = main.cpp
\end{qmake}

\end{slide}
