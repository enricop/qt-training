%%%%%%%%%%%%%%%%%%%%%%%%%%%%%%%%%%%%%%%%%%%%%%%%%%%%%%%%%%%%%%%%%%%%%%%%%%
%
% Copyright (c) 2008-2011, Nokia Corporation and/or its subsidiary(-ies).
% All rights reserved.
%
% This work, unless otherwise expressly stated, is licensed under a
% Creative Commons Attribution-ShareAlike 2.5.
%
% The full license document is available from
% http://creativecommons.org/licenses/by-sa/2.5/legalcode .
%
%%%%%%%%%%%%%%%%%%%%%%%%%%%%%%%%%%%%%%%%%%%%%%%%%%%%%%%%%%%%%%%%%%%%%%%%%%

%----------------------------------------------------------------------
\subsection{Custom Items and Properties}

%----------------------------------------------------------------------
\begin{slide}{1576}\frametitle{Overview}

Interaction between C++ and QML occurs via objects exposed to the
QML environment as new types.

\vspace*{0.5em}
\begin{itemize}
\item Non-visual types are \iCls{QObject} subclasses
\item Visual types (items) are \iCls{QDeclarativeItem} subclasses
  \begin{itemize}
  \item \iCls{QDeclarativeItem} is the C++ equivalent of \qic{class}{Item}
  \end{itemize}
\end{itemize}

\vspace*{0.5em}
To define a new type:

\begin{itemize}
\item In C++: Subclass either \iCls{QObject} or \iCls{QDeclarativeItem}
\item In C++: Register the type with the QML environment
\item In QML: Import the module containing the new item
\item In QML: Use the item like any other standard item
\end{itemize}

\end{slide}

%----------------------------------------------------------------------
\begin{slide}[fragile]{1575}\frametitle{A Custom Item}

\flushedImage{qml-cpp-integration/images/ellipse1.png}
In the \textit{ellipse1.qml} file:

\vspace*{0.5em}
% qml-cpp-integration/ex-simple-item/files/ellipse1.qml
\inputqml{qml-cpp-integration/colorized/ellipse1}

\vspace*{0.5em}
\begin{itemize}
\item A custom ellipse item
  \begin{itemize}
  \item instantiated using the \qtt{Ellipse} element
  \item from the custom \qtt{Shapes} module
  \end{itemize}
\item Draws an ellipse with the specified geometry
\end{itemize}

\end{slide}

%----------------------------------------------------------------------
\begin{slide}[fragile]{1574}\frametitle{Declaring the Item}

In the \textit{ellipseitem.h} file:

\medskip
\begin{lstlisting}
#include <QDeclarativeItem>

class EllipseItem : public QDeclarativeItem
{
    Q_OBJECT
public:
    EllipseItem(QDeclarativeItem *parent = 0);
    void paint(QPainter *painter,
               const QStyleOptionGraphicsItem *option,
               QWidget *widget = 0);
};
\end{lstlisting}

\begin{itemize}
\item \qtt{EllipseItem} is a \iCls{QDeclarativeItem} subclass
\item As with Qt widgets, each item can have a parent
\item Each custom item needs to paint itself
\end{itemize}

\end{slide}

%----------------------------------------------------------------------
\begin{slide}[fragile]{1573}\frametitle{Implementing an Item}

In the \textit{ellipseitem.cpp} file:

\vspace*{0.5em}
\begin{lstlisting}
#include <QtGui>
#include "ellipseitem.h"

EllipseItem::EllipseItem(QDeclarativeItem *parent)
    : QDeclarativeItem(parent)
{
    setFlag(QGraphicsItem::ItemHasNoContents, false);
}
...
\end{lstlisting}

\begin{itemize}
\item As usual, call the base class's constructor
\item It is necessary to clear the \hClsEnum{QGraphicsItem}{ItemHasNoContents}
      flag
\end{itemize}

\end{slide}

%----------------------------------------------------------------------
\begin{slide}[fragile]{1572}\frametitle{Implementing an Item (Continued)}

In the \textit{ellipseitem.cpp} file:

\vspace*{0.5em}
\begin{lstlisting}
...
void EllipseItem::paint(QPainter *painter,
    const QStyleOptionGraphicsItem *option, QWidget *widget)
{
    painter->drawEllipse(option->rect);
}
\end{lstlisting}

\begin{itemize}
\item A simple \hClsFn{QDeclarativeItem}{paint} function implementation
\item Ignore style information and use the default pen
\end{itemize}

\end{slide}

%----------------------------------------------------------------------
\begin{slide}[fragile]{1571}\frametitle{Registering the Item}

\begin{lstlisting}
#include <QApplication>
#include <QDeclarativeView>
#include "ellipseitem.h"

int main(int argc, char *argv[])
{
    QApplication app(argc, argv);
    qmlRegisterType<EllipseItem>("Shapes", 1, 0, "Ellipse");

    QDeclarativeView view;
    view.setSource(QUrl("qrc:files/ellipse1.qml"));
    view.show();
    return app.exec();
}
\end{lstlisting}

\begin{itemize}
\item Custom item registered as a module and element
\item Automatically available to the \qtt{ellipse1.qml} file
\end{itemize}

\end{slide}

%----------------------------------------------------------------------
\begin{slide}[fragile]{1570}\frametitle{Reviewing the Registration}

\medskip
\begin{lstlisting}
qmlRegisterType<EllipseItem>("Shapes", 1, 0, "Ellipse");
\end{lstlisting}

\medskip
\medskip

\begin{itemize}
\item  This registers the \qtt{EllipseItem} C++ class
\medskip
\item Available from the \qtt{Shapes} QML module
  \begin{itemize}
  \item version 1.0 (first number is major; second is minor)
  \end{itemize}
\item Available as the \qtt{Ellipse} element
  \begin{itemize}
  \item the \qtt{Ellipse} element is an visual item
  \item a subtype of \qic{class}{Item}
  \end{itemize}
\end{itemize}
\demo{qml-cpp-integration/ex-simple-item}

\end{slide}

%----------------------------------------------------------------------
\begin{slide}[fragile]{1569}\frametitle{Adding Properties}

\flushedImage{qml-cpp-integration/images/ellipse2.png}
In the \textit{ellipse2.qml} file:

\medskip
% qml-cpp-integration/ex-simple-item/files/ellipse2.qml
\inputqml{qml-cpp-integration/colorized/ellipse2}

\medskip
\begin{itemize}
\item A new \qic{type}{color} property
\item[] \demo{qml-cpp-integration/ex-properties}
\end{itemize}

\end{slide}

%----------------------------------------------------------------------
\begin{slide}[fragile]{1568}\frametitle{Declaring a Property}

In the \textit{ellipseitem.h} file:

\vspace*{0.5em}
\begin{lstlisting}
class EllipseItem : public QDeclarativeItem
{
    Q_OBJECT
    Q_PROPERTY(QColor color READ color WRITE setColor
               NOTIFY colorChanged)
    ...
\end{lstlisting}

\begin{itemize}
\item Use a \iCls{Q\_PROPERTY} macro to define a new property
  \begin{itemize}
  \item named \qic{type}{color} with \iCls{QColor} type
  \item with getter and setter, \hClsFn{EllipseItem}{color} and \hClsFn{EllipseItem}{setColor}
  \item emits the \hClsSig{EllipseItem}{colorChanged} signal when the value changes
  \end{itemize}
  \vspace*{0.5em}
\item The signal is just a notification
  \begin{itemize}
  \item it contains no value
  \item we must emit it to make property bindings work
  \end{itemize}
\end{itemize}

\end{slide}

%----------------------------------------------------------------------
\begin{slide}[fragile]{1567}\frametitle{Declaring Getter and Setter}

In the \textit{ellipseitem.h} file:

\vspace*{0.5em}
\begin{lstlisting}
public:
    ...
    const QColor &color() const;
    void setColor(const QColor &newColor);

signals:
    void colorChanged();

private:
    QColor m_color;
};
\end{lstlisting}

\begin{itemize}
\item Declare the getter and setter
\item Declare the private variable containing the color
\end{itemize}

\end{slide}

%----------------------------------------------------------------------
\begin{slide}[fragile]{1566}\frametitle{Implementing Getter and Setter}

In the \textit{ellipseitem.cpp} file:

\vspace*{0.25em}
\begin{lstlisting}
const QColor &EllipseItem::color() const {
    return m_color;
}

void EllipseItem::setColor(const QColor &newColor) {
  // always check if new value differs from current
  if (m_color != newColor) {
    m_color = newColor; // set the new value
    update(); // update painting
    emit colorChanged(); // notify about changes
  }
}
\end{lstlisting}

\end{slide}

%----------------------------------------------------------------------
\begin{slide}[fragile]{1565}\frametitle{Updated Paint}

In the \textit{ellipseitem.cpp} file:

\vspace*{0.5em}
\begin{lstlisting}
void EllipseItem::paint(QPainter *painter,
    const QStyleOptionGraphicsItem *option, QWidget *widget) {
    painter->save(); // save painter state
    painter->setPen(Qt::NoPen); // we use no pen
    // setup brush as described by property 'color'
    painter->setBrush(m_color); 
    painter->drawEllipse(option->rect);
    painter->restore(); // restore painter state
}
\end{lstlisting}

\end{slide}

%----------------------------------------------------------------------
\begin{slide}{1564}\frametitle{Summary of Items and Properties}

\begin{itemize}
\item Register new QML types using \iCls{qmlRegisterType}
  \begin{itemize}
  \item new types are subclasses of \iCls{QDeclarativeItem}
  \end{itemize}
\vspace*{0.5em}
\item Add QML properties
  \begin{itemize}
  \item define C++ properties with \qtt{NOTIFY} signals
  \item notifications are used to maintain the bindings between items
  \end{itemize}
\end{itemize}

\end{slide}       

%----------------------------------------------------------------------
\subsection{Signals \& Slots and Methods}

%----------------------------------------------------------------------
\begin{slide}[fragile]{1563}\frametitle{Adding Signals}

\flushedImage{qml-cpp-integration/images/ellipse3a.png}
In the \textit{ellipse3.qml} file:

\vspace*{0.5em}
% qml-cpp-integration/ex-simple-item/files/ellipse3.qml
\begin{qml}
\qtt{\qc{keyword}{import}~\qc{class}{QtQuick}~\qc{number}{1.0}}\\
\qtt{\qc{keyword}{import}~Shapes~\qc{number}{3.0}}\\
\vspace*{0.5em}
\qtt{\qc{class}{Item}~\{}\\
\qtt{~~~~\qc{type}{width}:~\qc{number}{300};~\qc{type}{height}:~\qc{number}{200}}\\
\vspace*{0.5em}
\qtt{~~~~Ellipse~\{}\\
\qtt{~~~~~~~~\qc{type}{x}:~\qc{number}{50};~\qc{type}{y}:~\qc{number}{35};~\qc{type}{width}:~\qc{number}{200};~\qc{type}{height}:~\qc{number}{100}}\\
\qtt{~~~~~~~~\qc{type}{color}:~\qc{string}{"blue"}}\\
\qtt{~~~~~~~~\qc{type}{onReady}:~label.text~\qc{operator}{=}~\qc{string}{"Ready"}}\\
\qtt{~~~~\}}\\
\qtt{~~~~...}
\end{qml}

\vspace*{0.5em}
\begin{itemize}
\item A new \qic{type}{onReady} signal handler
  \begin{itemize}
  \item changes the \qic{type}{text} property of the \qtt{label} item
  \end{itemize}
\end{itemize}

\end{slide}

%----------------------------------------------------------------------
\begin{slide}[fragile]{1562}\frametitle{Declaring a Signal}

In the \textit{ellipseitem.h} file:

\vspace*{0.5em}
\begin{lstlisting}
...
signals:
    void colorChanged();
    void ready();
...
\end{lstlisting}

\begin{itemize}
\item Add a \hClsSig{EllipseItem}{ready} signal
  \begin{itemize}
  \item this will have a corresponding \qic{type}{onReady} handler in QML
  \item we will emit this 2 seconds after the item is created
  \end{itemize}
\end{itemize}

\end{slide}

%----------------------------------------------------------------------
\begin{slide}[fragile]{1561}\frametitle{Emitting the Signal}

In the \textit{ellipseitem.cpp} file:

\vspace*{0.5em}
\begin{lstlisting}
EllipseItem::EllipseItem(QDeclarativeItem *parent)
    : QDeclarativeItem(parent)
{
    setFlag(QGraphicsItem::ItemHasNoContents, false);
    QTimer::singleShot(2000, this, SIGNAL(ready()));
}
\end{lstlisting}

\begin{itemize}
\item Change the constructor
  \begin{itemize}
  \item start a 2 second single-shot timer
  \item this emits the \hClsSig{EllipseItem}{ready} signal when it times out
  \end{itemize}
\end{itemize}

\end{slide}

%----------------------------------------------------------------------
\begin{slide}[fragile]{1560}\frametitle{Handling the Signal}

\flushedImage{qml-cpp-integration/images/ellipse3b.png}
In the \textit{ellipse3.qml} file:

\vspace*{0.5em}
\begin{qml}
\qtt{~~~~Ellipse~\{}\\
\qtt{~~~~~~~~\qc{type}{x}:~\qc{number}{50};~\qc{type}{y}:~\qc{number}{35};~\qc{type}{width}:~\qc{number}{200};~\qc{type}{height}:~\qc{number}{100}}\\
\qtt{~~~~~~~~\qc{type}{color}:~\qc{string}{"blue"}}\\
\qtt{~~~~~~~~\qc{type}{onReady}:~label.text~\qc{operator}{=}~\qc{string}{"Ready"}}\\
\qtt{~~~~\}}\\
\vspace*{0.5em}
\qtt{~~~~\qc{class}{Text}~\{}\\
\qtt{~~~~~~~~...}\\
\qtt{~~~~~~~~\qc{type}{text}:~\qc{string}{"Not~ready"}}\\
\qtt{~~~~\}}
\end{qml}

\begin{itemize}
\item In C++:
  \begin{itemize}
  \item the \iClsSig{EllipseItem}{ready} signal is emitted
  \end{itemize}
\item In QML:
  \begin{itemize}
  \item the \qic{class}{Ellipse} item's \qic{type}{onReady} handler is called
  \item sets the text of the \qic{class}{Text} item
  \end{itemize}
\end{itemize}

\end{slide}

%----------------------------------------------------------------------
\begin{slide}{1559}\frametitle{Adding Methods to Items}

Two ways to add methods that can be called from QML:

\vspace*{0.5em}
\begin{enumerate}
\item Create C++ slots
  \begin{itemize}
  \item automatically exposed to QML
  \item useful for methods that do not return values
  \end{itemize}
\item Mark regular C++ functions as invokable
  \begin{itemize}
  \item allows values to be returned
  \end{itemize}
\end{enumerate}

\end{slide}

%----------------------------------------------------------------------
\begin{slide}[fragile]{1558}\frametitle{Adding Slots}

\flushedImage{qml-cpp-integration/images/ellipse4.png}
In the \textit{ellipse4.qml} file:

\vspace*{0.5em}
% qml-cpp-integration/ex-slots/files/ellipse4.qml
\begin{qml}
\qtt{Ellipse~\{}\\
\qtt{~~~~\qc{type}{x}:~\qc{number}{50};~\qc{type}{y}:~\qc{number}{35};~\qc{type}{width}:~\qc{number}{200};~\qc{type}{height}:~\qc{number}{100}}\\
\qtt{~~~~\qc{type}{color}:~\qc{string}{"blue"}}\\
\qtt{~~~~\qc{type}{onReady}:~label.text~\qc{operator}{=}~\qc{string}{"Ready"}}\\
\vspace*{0.5em}
\qtt{~~~~\qc{class}{MouseArea}~\{}\\
\qtt{~~~~~~~~\qc{type}{anchors.fill}:~parent}\\
\qtt{~~~~~~~~\qc{type}{onClicked}:~parent.setColor(\qc{string}{"darkgreen"});}\\
\qtt{~~~~\}}\\
\qtt{\}}
\end{qml}

\vspace*{0.5em}
\begin{itemize}
\item \qtt{Ellipse} now has a \qtt{setColor()} method
  \begin{itemize}
  \item accepts a color
  \item also accepts strings containing colors
  \end{itemize}
 \item Normally, could just use properties to change colors...
\end{itemize}

\end{slide}

%----------------------------------------------------------------------
\begin{slide}[fragile]{1557}\frametitle{Declaring a Slot}

In the \textit{ellipseitem.h} file:

\vspace*{0.25em}
\begin{lstlisting}
...
    const QColor &color() const;

signals:
    void colorChanged();
    void ready();

public slots:
    void setColor(const QColor &newColor);
...
\end{lstlisting}

\begin{itemize}
\item Moved the \hClsFn{EllipseItem}{setColor} setter function
  \begin{itemize}
  \item now in the \iCls{public slots} section of the class
  \end{itemize}
\vspace*{0.25em}
\item Accepts \iCls{QColor} values
  \begin{itemize}
  \item the string passed from QML is converted to a color
  \item for custom types, make sure that type conversion is supported
  \end{itemize}
\end{itemize}

\end{slide}

%----------------------------------------------------------------------
\begin{slide}[fragile]{1556}\frametitle{Adding Methods}

\flushedImage{qml-cpp-integration/images/ellipse5.png}
In the \textit{ellipse5.qml} file:

\vspace*{0.5em}
% qml-cpp-integration/ex-methods/files/ellipse5.qml
\begin{qml}
\qtt{Ellipse~\{}\\
\qtt{~~~~\qc{type}{x}:~\qc{number}{50};~\qc{type}{y}:~\qc{number}{35};~\qc{type}{width}:~\qc{number}{200};~\qc{type}{height}:~\qc{number}{100}}\\
\qtt{~~~~\qc{type}{color}:~\qc{string}{"blue"}}\\
\qtt{~~~~\qc{type}{onReady}:~label.text~\qc{operator}{=}~\qc{string}{"Ready"}}\\
\vspace*{0.5em}
\qtt{~~~~\qc{class}{MouseArea}~\{}\\
\qtt{~~~~~~~~\qc{type}{anchors.fill}:~parent}\\
\qtt{~~~~~~~~\qc{type}{onClicked}:~parent.color~\qc{operator}{=}~parent.randomColor()}\\
\qtt{~~~~\}}\\
\qtt{\}}
\end{qml}

\vspace*{0.5em}
\begin{itemize}
\item \qtt{Ellipse} now has a \qtt{randomColor()} method
  \begin{itemize}
  \item obtain a random color using this method
  \item set the color using the \qic{type}{color} property
  \end{itemize}
\end{itemize}

\end{slide}

%----------------------------------------------------------------------
\begin{slide}[fragile]{1555}\frametitle{Declaring a Method}

In the \textit{ellipseitem.h} file:

\vspace*{0.25em}
\begin{lstlisting}
...
public:
    EllipseItem(QDeclarativeItem *parent = 0);
    void paint(QPainter *painter,
      const QStyleOptionGraphicsItem *option, QWidget *widget = 0);

    const QColor &color() const;
    void setColor(const QColor &newColor);

    Q_INVOKABLE QColor randomColor() const;
...
\end{lstlisting}

\begin{itemize}
\item Define the \hClsFn{EllipseItem}{randomColor} function
  \begin{itemize}
  \item add the \iCls{Q\_INVOKABLE} macro before the declaration
  \item returns a \iCls{QColor} value
  \item \textit{cannot} return a const reference
  \end{itemize}
\end{itemize}

\end{slide}

%----------------------------------------------------------------------
\begin{slide}[fragile]{1554}\frametitle{Implementing a Method}

In the \textit{ellipseitem.cpp} file:

\vspace*{0.5em}
\begin{lstlisting}
QColor EllipseItem::randomColor() const
{
    return QColor(qrand() & 0xff, qrand() & 0xff, qrand() & 0xff);
}
\end{lstlisting}

\begin{itemize}
\item Define the new \hClsFn{EllipseItem}{randomColor} function
  \begin{itemize}
  \item the pseudo-random number generator has already been seeded
  \item simply return a color
  \item do not use the \iCls{Q\_INVOKABLE} macro in the source file
  \end{itemize}
\end{itemize}

\end{slide}

%----------------------------------------------------------------------
\begin{slide}{1553}\frametitle{Summary of Signals, Slots and Methods}

\begin{itemize}
\item Define signals
  \begin{itemize}
  \item connect to Qt signals with the \qic{type}{onSignal} syntax
  \end{itemize}
\item Define QML-callable methods
  \begin{itemize}
  \item reuse slots as QML-callable methods
  \item methods that return values are marked using \iMacro{Q\_INVOKABLE}
  \end{itemize}
\end{itemize}

\end{slide}
                             

%----------------------------------------------------------------------
\subsection{Using Custom Types}

%----------------------------------------------------------------------
\begin{slide}{1552}\frametitle{Defining Custom Property Types}

\begin{itemize}
\item Items can be used as property types
  \begin{itemize}
  \item allows rich description of properties
  \item subclass \iCls{QDeclarativeItem} (as before)
  \item requires registration of types (as before)
  \end{itemize}

\vspace*{0.5em}
\item A simpler way to define custom property types:
  \begin{itemize}
  \item use simple enums and flags
  \item easy to declare and use
  \end{itemize}

\vspace*{0.5em}
\item Collections of custom types:
  \begin{itemize}
  \item define a new custom item
  \item use with a \iCls{QDeclarativeListProperty} template type
  \end{itemize}
\end{itemize}

\end{slide}

%----------------------------------------------------------------------
\begin{slide}[fragile]{1551}\frametitle{Enums as Property Types}

\flushedImage{qml-cpp-integration/images/ellipse6.png}
% qml-cpp-integration/ex-methods/files/ellipse6.qml
\inputqml{qml-cpp-integration/colorized/ellipse6}

\begin{itemize}
\item A new \qic{type}{style} property with a custom value type
\end{itemize}

\end{slide}

%----------------------------------------------------------------------
\begin{slide}[fragile]{1550}\frametitle{Declaring Enums as Property Types}

In the \textit{ellipseitem.h} file:

\vspace*{0.25em}
\begin{lstlisting}
class EllipseItem : public QDeclarativeItem
{
    Q_OBJECT
    Q_PROPERTY(QColor color READ color WRITE setColor
               NOTIFY colorChanged)
    Q_PROPERTY(Style style READ style WRITE setStyle
               NOTIFY styleChanged)
    Q_ENUMS(Style)

public:
    enum Style { Outline, Filled };
...
\end{lstlisting}

\begin{itemize}
\item Declare the \qic{type}{style} property using the \iCls{Q\_PROPERTY}
      macro
\item Declare the \hClsEnum{EllipseItem}{Style} enum using the \iCls{Q\_ENUMS}
      macro
\item Define the \hClsEnum{EllipseItem}{Style} enum
\end{itemize}

\end{slide}

%----------------------------------------------------------------------
\begin{slide}[fragile]{1549}\frametitle{Using Enums in Getters and Setters}

In the \textit{ellipseitem.h} file:

\vspace*{0.25em}
\begin{lstlisting}
...
    const Style &style() const;
    void setStyle(const Style &newStyle);

signals:
    void colorChanged();
    void styleChanged();

private:
    QColor m_color;
    Style m_style;
...
\end{lstlisting}

\begin{itemize}
\item Define the usual getter, setter, signal and private member for the
      \qic{type}{style} property
\end{itemize}

\end{slide}

%----------------------------------------------------------------------
\begin{slide}[fragile]{1548}\frametitle{Using Enums in Getters and Setters}

In the \textit{ellipseitem.cpp} file:

\vspace*{0.25em}
\begin{lstlisting}
const EllipseItem::Style &EllipseItem::style() const
{
    return m_style;
}

void EllipseItem::setStyle(const Style &newStyle)
{
    if (m_style != newStyle) {
        m_style = newStyle;
        update();
        emit styleChanged();
    }
}
\end{lstlisting}

\begin{itemize}
\item Similar getter and setter as used for the \qic{type}{color} property
\end{itemize}

\end{slide}

%----------------------------------------------------------------------
\begin{slide}[fragile]{1547}\frametitle{Custom Items as Property Types}

\flushedImage{qml-cpp-integration/images/ellipse7.png}
% qml-cpp-integration/ex-methods/files/ellipse7.qml
\inputqml{qml-cpp-integration/colorized/ellipse7}

\begin{itemize}
\item A new \qic{type}{style} property with a custom item type
\end{itemize}

\end{slide}

%----------------------------------------------------------------------
\begin{slide}[fragile]{1546}\frametitle{Declaring the Style Item}

In the \textit{style.h} file:

\vspace*{0.5em}
\begin{lstlisting}
class Style : public QDeclarativeItem
{
    Q_OBJECT
    Q_PROPERTY(QColor color READ color WRITE setColor
               NOTIFY colorChanged)
    Q_PROPERTY(bool filled READ filled WRITE setFilled
               NOTIFY filledChanged)

public:
    Style(QDeclarativeItem *parent = 0);
...
\end{lstlisting}

\begin{itemize}
\item \qtt{Style} is a \iCls{QDeclarativeItem} subclass
\item Defined like the \qtt{EllipseItem} class
  \begin{itemize}
  \item implemented in the same way
  \end{itemize}
\end{itemize}

\end{slide}

%----------------------------------------------------------------------
\begin{slide}[fragile]{1545}\frametitle{Declaring the Style Property}

In the \textit{ellipseitem.h} file:

\vspace*{0.25em}
\begin{lstlisting}
class EllipseItem : public QDeclarativeItem
{
    Q_OBJECT
    Q_PROPERTY(Style *style READ style WRITE setStyle
               NOTIFY styleChanged)
public:
...
    Style *style() const;
    void setStyle(Style *newStyle);

signals:
    void styleChanged();
...
\end{lstlisting}

\begin{itemize}
\item Declare the \qic{type}{style} property
  \begin{itemize}
  \item with the \qtt{Style} pointer as its type
  \item declare the associated getter, setter and signal
  \end{itemize}
\end{itemize}

\end{slide}

%----------------------------------------------------------------------
\begin{slide}[fragile]{1544}\frametitle{Using the Style Property}

In the \textit{ellipseitem.cpp} file:

\vspace*{0.25em}
\begin{lstlisting}
void EllipseItem::paint(QPainter *painter,
    const QStyleOptionGraphicsItem *option, QWidget *widget)
{
    painter->save();
    if (!m_style->filled()) {
        painter->setPen(m_style->color());
        painter->setBrush(Qt::NoBrush);
    } else {
        painter->setPen(Qt::NoPen);
        painter->setBrush(m_style->color());
    }
    painter->drawEllipse(option->rect);
    painter->restore();
}
\end{lstlisting}

\begin{itemize}
\item Use the internal \qtt{Style} object to decide what to paint
  \begin{itemize}
  \item using the getters to read its \qic{type}{color} and \qic{type}{filled}
        properties
  \end{itemize}
\end{itemize}

\end{slide}

%----------------------------------------------------------------------
\begin{slide}[fragile]{1543}\frametitle{Registering the Style Property}

In the \textit{ellipseitem.cpp} file:

\vspace*{0.5em}
\begin{lstlisting}
int main(int argc, char *argv[])
{
    QApplication app(argc, argv);
    qmlRegisterType<EllipseItem>("Shapes", 7, 0, "Ellipse");
    qmlRegisterType<Style>("Shapes", 7, 0, "Style");
...
\end{lstlisting}

\vspace*{1.0em}
\begin{itemize}
\item Register the \qtt{Style} class
  \begin{itemize}
  \item in the same way as the \qtt{EllipseItem} class
  \item in the same module: \qtt{Shapes} 7.0
  \end{itemize}
\end{itemize}

\end{slide}

%----------------------------------------------------------------------
\begin{slide}[fragile]{1542}\frametitle{Collections of Custom Types}

\flushedImage{qml-cpp-integration/images/chart1.png}
% qml-cpp-integration/ex-methods/files/chart1.qml
\inputqml{qml-cpp-integration/colorized/chart1}

\begin{itemize}
\item A \qtt{Chart} item
  \begin{itemize}
  \item with a \qic{type}{bars} list property
  \item accepting custom \qtt{Bar} items
  \end{itemize}
\end{itemize}

\end{slide}

%----------------------------------------------------------------------
\begin{slide}[fragile]{1541}\frametitle{Declaring the List Property}

In the \textit{chartitem.h} file:

\vspace*{0.5em}
\begin{lstlisting}
class BarItem;

class ChartItem : public QDeclarativeItem
{
    Q_OBJECT
    Q_PROPERTY(QDeclarativeListProperty<BarItem> bars READ bars
               NOTIFY barsChanged)
public:
    ChartItem(QDeclarativeItem *parent = 0);
    void paint(QPainter *painter,
      const QStyleOptionGraphicsItem *option, QWidget *widget = 0);
...
\end{lstlisting}

\vspace*{0.5em}
\begin{itemize}
\item Define the \qic{type}{bars} property
  \begin{itemize}
  \item in theory, read-only but with a notification signal
  \item in reality, writable as well as readable
  \end{itemize}
\end{itemize}

\end{slide}

%----------------------------------------------------------------------
\begin{slide}[fragile]{1540}\frametitle{Declaring the List Property}

In the \textit{chartitem.h} file:

\vspace*{0.5em}
\begin{lstlisting}
...
    QDeclarativeListProperty<BarItem> bars();

signals:
    void barsChanged();

private:
    static void append_bar(QDeclarativeListProperty<BarItem> *list,
                           BarItem *bar);
    QList<BarItem*> m_bars;
};
\end{lstlisting}

\vspace*{0.5em}
\begin{itemize}
\item Define the getter function and notification signal
\item Define an append function for the list property
\end{itemize}

\end{slide}

%----------------------------------------------------------------------
\begin{slide}[fragile]{1539}\frametitle{Defining the Getter Function}

In the \textit{chartitem.cpp} file:

\vspace*{0.5em}
\begin{lstlisting}
QDeclarativeListProperty<BarItem> ChartItem::bars()
{
    return QDeclarativeListProperty<BarItem>(this, 0,
        &ChartItem::append_bar);
}
\end{lstlisting}

\vspace*{0.5em}
\begin{itemize}
\item Defines and returns a list of \qtt{BarItem} objects
  \begin{itemize}
  \item with an append function
  \end{itemize}
\end{itemize}

\end{slide}

%----------------------------------------------------------------------
\begin{slide}[fragile]{1538}\frametitle{Defining the Append Function}

\begin{lstlisting}
void ChartItem::append_bar(QDeclarativeListProperty<BarItem> *list,
                           BarItem *bar)
{
    ChartItem *chart = qobject_cast<ChartItem *>(list->object);
    if (chart) {
        bar->setParentItem(chart);
        chart->m_bars.append(bar);
        chart->barsChanged();
    }
}
\end{lstlisting}

\vspace*{0.5em}
\begin{itemize}
\item Static function, accepts
  \begin{itemize}
  \item the list to operate on
  \item each \qtt{BarItem} to append
  \end{itemize}
\vspace*{0.5em}
\item When a \qtt{BarItem} is appended
  \begin{itemize}
  \item emits the \hClsSig{ChartItem}{barsChanged} signal
  \end{itemize}
\end{itemize}

\end{slide}

%----------------------------------------------------------------------
\begin{slide}[fragile]{1537}\frametitle{Summary of Custom Property Types}

\begin{itemize}
\item Define items as property types:
  \begin{itemize}
  \item declare and implement a new \iCls{QDeclarativeItem} subclass
  \item declare properties to use a pointer to the new type
  \item register the item with \iCls{qmlRegisterType}
  \end{itemize}
\vspace*{1em}
\item Use enums as simple custom property types:
  \begin{itemize}
  \item use \iCls{Q\_ENUMS} to declare a new enum type
  \item declare properties as usual
  \end{itemize}
\vspace*{1em}
\item Define collections of custom types:
  \begin{itemize}
  \item using a custom item that has been declared and registered
  \item declare properties with \iCls{QDeclarativeListProperty}
  \item implement a getter and an append function for each property
  \item read-only properties, but read-write containers
  \item read-only containers define append functions that simply return
  \end{itemize}
\end{itemize}

\end{slide}

%----------------------------------------------------------------------
\begin{slide}{1536}\frametitle{Creating Extension Plugins}

\begin{itemize}
\item Declarative extensions can be deployed as plugins 
  \begin{itemize}
  \item using source and header files for a working custom type
  \item developed separately then deployed with an application
  \item write QML-only components then rewrite in C++
  \item use placeholders for C++ components until they are ready
  \end{itemize}
\vspace*{0.5em}
\item Plugins can be loaded by the \qtt{qmlviewer} tool
  \begin{itemize}
  \item with an appropriate \qtt{qmldir} file
  \end{itemize}
\vspace*{0.5em}
\item Plugins can be loaded by C++ applications
  \begin{itemize}
  \item some work is required to load and initialize them
  \end{itemize}
\end{itemize}

\end{slide}

%----------------------------------------------------------------------
\begin{slide}[fragile]{1535}\frametitle{Defining an Extension Plugin}

\begin{lstlisting}
#include <QDeclarativeExtensionPlugin>

class EllipsePlugin : public QDeclarativeExtensionPlugin
{
    Q_OBJECT

public:
    void registerTypes(const char *uri);
};
\end{lstlisting}

\vspace*{0.5em}
\begin{itemize}
\item Create a \iCls{QDeclarativeExtensionPlugin} subclass
  \begin{itemize}
  \item only one function to reimplement
  \end{itemize}
\end{itemize}

\end{slide}

%----------------------------------------------------------------------
\begin{slide}[fragile]{1534}\frametitle{Implementing an Extension Plugin}

\begin{lstlisting}
#include <qdeclarative.h>
#include "ellipseplugin.h"
#include "ellipseitem.h"

void EllipsePlugin::registerTypes(const char *uri)
{
    qmlRegisterType<EllipseItem>(uri, 9, 0, "Ellipse");
}

Q_EXPORT_PLUGIN2(ellipseplugin, EllipsePlugin);
\end{lstlisting}

\vspace*{0.5em}
\begin{itemize}
\item Register the custom type using the \qtt{uri} supplied
  \begin{itemize}
  \item the same custom type we started with
  \end{itemize}
\item Export the plugin using
  \begin{itemize}
  \item the name of the plugin library (\qtt{ellipseplugin})
  \item the name of the plugin class (\qtt{EllipsePlugin})
  \end{itemize}
\end{itemize}

\end{slide}

%----------------------------------------------------------------------
\begin{slide}[fragile]{1533}\frametitle{Building an Extension Plugin}

\begin{qmake}
TEMPLATE  = lib
CONFIG   += qt plugin
QT       += declarative

HEADERS  += ellipseitem.h \
            ellipseplugin.h

SOURCES  += ellipseitem.cpp \
            ellipseplugin.cpp

DESTDIR   = ../plugins
\end{qmake}

\vspace*{0.5em}
\begin{itemize}
\item Ensure that the project is built as a Qt plugin
\item Declarative module is added to the Qt configuration
\item Plugin is written to a \qtt{plugins} directory
\end{itemize}

\end{slide}

%----------------------------------------------------------------------
\begin{slide}[fragile]{1532}\frametitle{Using an Extension Plugin}

To use the plugin with the \qtt{qmlviewer} tool:

\flushedImageDoubleWidth{qml-cpp-integration/images/plugin-location.pdf}
\begin{itemize}
\item Write a \qtt{qmldir} file
  \begin{itemize}
  \item include a line to describe the plugin
  \item stored in the \qtt{standalone} directory
  \end{itemize}
\vspace*{0.5em}
\item Write a QML file to show the item
  \begin{itemize}
  \item \qtt{ellipse9s.qml}
  \end{itemize}
\end{itemize}

\vspace*{1em}
The \qtt{qmldir} file contains a declaration:

\begin{alltt}
plugin ellipseplugin ../plugins
\end{alltt}

\begin{itemize}
\item \textbf{\qtt{plugin}} followed by
  \begin{itemize}
  \item the plugin name: \textbf{\qtt{ellipseplugin}}
  \item the plugin path relative to the \qtt{qmldir} file: \textbf{\qtt{../plugins}}
  \end{itemize}
\end{itemize}

\end{slide}

%----------------------------------------------------------------------
\begin{slide}[fragile]{1531}\frametitle{Using an Extension Plugin}

In the \qtt{ellipse9s.qml} file:

\vspace*{0.5em}
% qml-cpp-integration/ex-extension-plugin/standalone/ellipse9s.qml
\inputqml{qml-cpp-integration/colorized/ellipse9s}

\vspace*{0.5em}
\begin{itemize}
\item Use the custom item directly
\item No need to import any custom modules
  \begin{itemize}
  \item \qtt{qmldir} and \qtt{ellipse9s.qml} are in the same project
        directory
  \item \qtt{Ellipse} is automatically imported into the global namespace
  \end{itemize}
\end{itemize}

% uri mentioned but not explained

\end{slide}

%----------------------------------------------------------------------
\begin{slide}[fragile]{1530}\frametitle{Loading an Extension Plugin}

To load the plugin in a C++ application:

\begin{itemize}
\item Locate the plugin
  \begin{itemize}
  \item (perhaps scan the files in the \qtt{plugins} directory)
  \end{itemize}
\item Load the plugin with \iCls{QPluginLoader}
\begin{lstlisting}
QPluginLoader loader(pluginsDir.absoluteFilePath(fileName));
\end{lstlisting}
\vspace*{0.5em}
\item Cast the plugin object to a \iCls{QDeclarativeExtensionPlugin}
\begin{lstlisting}
QDeclarativeExtensionPlugin *plugin =
  qobject_cast<QDeclarativeExtensionPlugin *>(loader.instance());
\end{lstlisting}
\vspace*{0.5em}
\item Register the extension with a URI
\begin{lstlisting}
if (plugin)
  plugin->registerTypes("Shapes");
\end{lstlisting}
  \begin{itemize}
  \item in this example, \qtt{Shapes} is used as a URI
  \end{itemize}
\end{itemize}

\end{slide}

%----------------------------------------------------------------------
\begin{slide}[fragile]{1529}\frametitle{Using an Extension Plugin}

In the \qtt{ellipse9s.qml} file:

\vspace*{0.5em}
% qml-cpp-integration/ex-extension-plugin/standalone/ellipse9.qml
\inputqml{qml-cpp-integration/colorized/ellipse9}

\begin{itemize}
\item The \qtt{Ellipse} item is part of the \qtt{Shapes} module
\item A different URI makes a different import necessary; e.g.,
\begin{lstlisting}
plugin->registerTypes("com.nokia.qt.examples.Shapes");
\end{lstlisting}
\item corresponds to\\
\begin{qml}
\qtt{\qc{keyword}{import}~com.nokia.qt.examples.Shapes~\qc{number}{9.0}}\\
\end{qml}
\end{itemize}

\end{slide}

%----------------------------------------------------------------------
\begin{slide}[fragile]{1528}\frametitle{Summary of Extension Plugins}

\begin{itemize}
\item Extensions can be compiled as plugins
  \begin{itemize}
  \item define and implement a \iCls{QDeclarativeExtensionPlugin} subclass
  \item define the version of the plugin in the extension
  \item build a Qt plugin project with the \qtt{declarative} option enabled
  \end{itemize}
\vspace*{0.5em}
\item Plugins can be loaded by the \qtt{qmlviewer} tool
  \begin{itemize}
  \item write a \qtt{qmldir} file
  \item declare the plugin's name and location relative to the file
  \item no need to import the plugin in QML
  \end{itemize}
\vspace*{0.5em}
\item Plugins can be loaded by C++ applications
  \begin{itemize}
  \item use \iCls{QPluginLoader} to load the plugin
  \item register the custom types with a specific URI
  \item import the same URI and plugin version number in QML
  \end{itemize}
\end{itemize}

\end{slide}
