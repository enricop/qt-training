%%%%%%%%%%%%%%%%%%%%%%%%%%%%%%%%%%%%%%%%%%%%%%%%%%%%%%%%%%%%%%%%%%%%%%%%%%
%
% Copyright (c) 2008-2011, Nokia Corporation and/or its subsidiary(-ies).
% All rights reserved.
%
% This work, unless otherwise expressly stated, is licensed under a
% Creative Commons Attribution-ShareAlike 2.5.
%
% The full license document is available from
% http://creativecommons.org/licenses/by-sa/2.5/legalcode .
%
%%%%%%%%%%%%%%%%%%%%%%%%%%%%%%%%%%%%%%%%%%%%%%%%%%%%%%%%%%%%%%%%%%%%%%%%%%

\subsection{Main Windows}

\begin{slide}{1769}\frametitle{Typical Application Ingredients}
\flushedImage{appcreation/images/mainwindow}
  \begin{itemize}
  \item Main window with
    \begin{itemize}
    \item Menu bar
    \item Tool bar, Status bar
    \item Central widget
    \item Often a dock window
    \end{itemize}
  \item Settings (saving state)
  \item Resources (e.g icons)
  \item Translation
  \item Load/Save documents
  \end{itemize}
  \begin{center} \textit{Not a complete list} \end{center}  
    
\end{slide}

%----------------------------------------------------------------------
\begin{slide}{1768} \frametitle{Main Window}
  \begin{itemize}
  \item \iCls{QMainWindow}: main application window
  \item Has own layout \\
    \flushedImage{appcreation/images/mainwindowlayout}
    \begin{itemize}
    \item Central Widget
    \item QMenuBar
    \item QToolBar
    \item QDockWidget
    \item QStatusBar
   \end{itemize}
  \end{itemize}

  \vfill
  \begin{itemize}
  \item Central Widget
    \begin{itemize}
    \item \lstinline{QMainWindow::setCentralWidget( widget )}
    \item Just any widget object
    \end{itemize}
  \end{itemize}
\end{slide}

%----------------------------------------------------------------------
\begin{slide}[fragile]{1767}
  \frametitle{Creating Actions - \iCls{QAction}}
  \begin{itemize}
  \item \textit{Action is an abstract user interface command}
    \vspace{3mm}
  \item Emits signal triggered on execution
    \begin{itemize}
    \item Connected slot performs action
    \end{itemize}
 \item Added to  menus, toolbar, key shortcuts
  \item Each performs same way
    \begin{itemize}
    \item Regardless of user interface used
    \end{itemize}
\item[] \begin{cpp}
void MainWindow::setupActions() {
  QAction* action = new QAction(tr("Open ..."), this);
  action->setIcon(QIcon(":/images/open.png"));
  action->setShortcut(QKeySequence::Open);
  action->setStatusTip(tr("Open file"));
  connect(action, SIGNAL(triggered()), this, SLOT(onOpen()));

  menu->addAction(action);
  toolbar->addAction(action);
\end{cpp}
%\item[]\footnotesize{\emph{Advice: Create actions as children of the window they are used in}}
  \end{itemize}

\end{slide}

%----------------------------------------------------------------------
\begin{slide}{1766}\frametitle{QAction capabilities}
  \begin{itemize}
  \item \texttt{setEnabled(bool)}
    \begin{itemize}
    \item Enables disables actions
    \item In menu and toolbars, etc...
    \end{itemize}
  \item \texttt{setCheckable(bool)}
    \begin{itemize}
    \item Switches checkable state (on/off)
    \item \texttt{setChecked(bool)} toggles checked state
    \end{itemize}
  \item \texttt{setData(QVariant)}
    \begin{itemize}
    \item  Stores data with the action
    \end{itemize}
  \item \doc{qaction.html}{QAction}

 \end{itemize}

\end{slide}

%----------------------------------------------------------------------
\begin{slide}[fragile]{1765}
  \frametitle{Create Menu Bar}
 \begin{itemize}
  \item \lstinline{QMenuBar}: a horizontal menu bar
\flushedImageDoubleWidth{appcreation/images/menubar}
  \item \lstinline{QMenu}: represents a menu
    \begin{itemize}
    \item indicates action state
    \end{itemize}

  \item \lstinline{QAction}: menu items added to \lstinline{QMenu}
\flushedImage{appcreation/images/menu}
 \item[] \begin{cpp}
void MainWindow::setupMenuBar() {
  QMenuBar* bar = menuBar();
  QMenu* menu = bar->addMenu(tr("&File"));
  menu->addAction(action);
  menu->addSeparator();

  QMenu* subMenu = menu->addMenu(tr("Sub Menu"));
  ...
\end{cpp}
 \end{itemize}
\end{slide}


%----------------------------------------------------------------------
\begin{slide}[fragile]{1764}
  \frametitle{Creating Toolbars - \iCls{QToolBar}}
\flushedImageDoubleWidth{appcreation/images/toolbar}
  \begin{itemize}
  \item Movable panel ...
    \begin{itemize}
    \item Contains set of controls
    \item Can be horizontal or vertical
    \end{itemize}
    \vspace{2mm}
  \item \texttt{QMainWindow::addToolbar( toolbar )}
   \begin{itemize}
   \item Adds toolbar to main window
    \end{itemize}
  \item \iClsFn{QMainWindow}{addToolBarBreak}
    \begin{itemize}
    \item Adds section splitter
    \end{itemize}
  \item \texttt{QToolBar::addAction( action )}
    \begin{itemize}
    \item Adds action to toolbar
      \end{itemize}
    \item \texttt{QToolBar::addWidget(widget)}
      \begin{itemize}
      \item Adds widget to toolbar
      \end{itemize}
    \item[] \begin{cpp}
void MainWindow::setupToolBar() {
  QToolBar* bar = addToolBar(tr("File"));
  bar->addAction(action);
  bar->addSeparator();
  bar->addWidget(new QLineEdit(tr("Find ...")));
  ...
  \end{cpp}
  \end{itemize}
\end{slide}

%----------------------------------------------------------------------
\begin{slide}[fragile]{1763}\frametitle{QToolButton}
  \begin{itemize}
  \item Quick-access button to commands or options
  \item Used when adding action to QToolBar
  \item Can be used instead \texttt{QPushButton}
    \begin{itemize}
    \item Different visual appearance!
    \end{itemize}
 \item Advantage: allows to attach action
  \item[] \begin{cpp}
QToolButton* button = new QToolButton(this);
button->setDefaultAction(action);
// Can have a menu
button->setMenu(menu);      
// Shows menu indicator on  button
button->setPopupMode(QToolButton::MenuButtonPopup);
// Control over text + icon placements
button->setToolButtonStyle(Qt::ToolButtonTextUnderIcon);
    \end{cpp}

  \end{itemize}

\end{slide}

%----------------------------------------------------------------------
\begin{slide}[fragile]{1762}
  \frametitle{The Status Bar - \iCls{QStatusBar}}
  \flushedImage{appcreation/images/statusbar}

  \begin{itemize}
  \item Horizontal bar
    \begin{itemize}
    \item[] Suitable for presenting status information
    \end{itemize}

    \vspace{3mm}
  \item \texttt{showMessage( message, timeout )}
    \begin{itemize}
    \item Displays temporary message for specified milli-seconds
    \end{itemize}
  \item \texttt{clearMessage()}
    \begin{itemize}
    \item Removes any temporary message
    \end{itemize}
  \item \texttt{addWidget()} or \texttt{addPermanentWidget()}
    \begin{itemize}
    \item Normal, permanent messages displayed by widget
    \end{itemize}
\begin{cpp}
void MainWindow::createStatusBar() {
  QStatusBar* bar = statusBar();
  bar->showMessage(tr("Ready"));
  bar->addWidget(new QLabel(tr("Label on StatusBar")));
  \end{cpp}
  \end{itemize}
\end{slide}

%----------------------------------------------------------------------
\begin{slide}[fragile]{1761}
  \frametitle{Creating Dock Windows - \iCls{QDockWidget}}
  \flushedImage{appcreation/images/docks}
  \begin{itemize}
  \item Window docked into main window
    \vspace{3mm}
  \item \iClsEnum{Qt}{DockWidgetArea} enum
    \begin{itemize}
    \item Left, Right, Top, Bottom dock areas
    \end{itemize}
  \item \texttt{QMainWindow::setCorner(corner,area)}
    \begin{itemize}
    \item Sets area to occupy specified corner
    \end{itemize}
  \item \texttt{QMainWindow::setDockOptions(options)}
    \begin{itemize}
    \item Specifies docking behavior (animated, nested, tabbed, ...)
    \end{itemize}

  \begin{cpp}
void MainWindow::createDockWidget() {
  QDockWidget *dock = new QDockWidget(tr("Title"), this);
  dock->setAllowedAreas(Qt::LeftDockWidgetArea);
  QListWidget *widget = new QListWidget(dock);
  dock->setWidget(widget);
  addDockWidget(Qt::LeftDockWidgetArea, dock);    
  \end{cpp}
  \end{itemize}
  
\end{slide}

                 
%----------------------------------------------------------------------
\begin{slide}[fragile]{1760}\frametitle{\iCls{QMenu} and Context Menus}
  \begin{itemize}
 \item Launch via event handler
  \item [] \begin{cpp}
void MyWidget::contextMenuEvent(event) {
 m_contextMenu->exec(event->globalPos());    
    \end{cpp}
\flushedImage{appcreation/images/menu}
  \item or signal \texttt{customContextMenuRequested()}
    \begin{itemize}
    \item Connect to signal to show context menu
    \end{itemize}
\item Or via \iClsFn{QWidget}{actions} list
    \begin{itemize}
    \item \texttt{QWidget::addAction(action)}
    \item \texttt{setContextMenuPolicy(Qt::ActionsContextMenu)}
    \item Displays \iClsFn{QWidget}{actions} as context menu
    \end{itemize}
\end{itemize}
\end{slide}


%----------------------------------------------------------------------
\begin{slide}{1759}
  \frametitle{Typical APIs}
  \begin{columns}[t]
    \begin{column}{5cm}
      \begin{itemize}
      \item QWidget
        \begin{itemize}
        \item setWindowModified(...)
        \item setWindowTitle(...)
        \item addAction(...)
        \item contextMenuEvent(...)
       \end{itemize}

      \item QMainWindow
        \begin{itemize}
        \item setCentralWidget(...)
        \item menuBar()
        \item statusBar()
        \item addToolbar(...)
        \item addToolBarBreak()
        \item addDockWidget(...)
        \item setCorner(...)
        \item setDockOptions(...)
       \end{itemize}

     \end{itemize}
    \end{column}
    
    \begin{column}[T]{5cm}
     \begin{itemize}
      \item QAction
        \begin{itemize}
        \item setShortcuts(...)
        \item setStatusTip(...)
        \item signal triggered()
        \end{itemize}
        
      \item QMenuBar
        \begin{itemize}
        \item addMenu(...)
        \end{itemize}
        
     \item QToolbar
        \begin{itemize}
        \item addAction(...)
        \end{itemize}
        
      \item QStatusBar
        \begin{itemize}
        \item showMessage(...)
        \item clearMessage()
        \item addWidget(...)
        \end{itemize}
      \end{itemize}
    \end{column}
  \end{columns}
\end{slide}

%----------------------------------------------------------------------
\begin{slide}{984}
  \frametitle{Lab: \iProject{Text Editor}}
\begin{itemize}
\item Create a text editor with
  \begin{itemize}
  \item \emph{load}, \emph{save}, \emph{quit}
  \item \emph{about} and \emph{About Qt}
  \end{itemize}
\item A \iCls{QPlainTextEdit} serves for editing the text.
\item Optional:
  \begin{itemize}
  \item Show whether the file is dirty
  \item Ask the user whether to save if file is dirty when application quits
  \item Make sure also to asks when window is closed via window manager
  \item Show the cursor position in the status bar
    \begin{itemize}
    \item Position is determined by cursors block and column count 
    \end{itemize}
  \item Add printing support. \doc{printing.html}{Printing with Qt}
  \end{itemize}
\end{itemize}
\vfill
% \textit{No handout} % FM: Whatfore??
\end{slide}
