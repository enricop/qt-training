%%%%%%%%%%%%%%%%%%%%%%%%%%%%%%%%%%%%%%%%%%%%%%%%%%%%%%%%%%%%%%%%%%%%%%%%%%
%
% Copyright (c) 2008-2011, Nokia Corporation and/or its subsidiary(-ies).
% All rights reserved.
%
% This work, unless otherwise expressly stated, is licensed under a
% Creative Commons Attribution-ShareAlike 2.5.
%
% The full license document is available from
% http://creativecommons.org/licenses/by-sa/2.5/legalcode .
%
%%%%%%%%%%%%%%%%%%%%%%%%%%%%%%%%%%%%%%%%%%%%%%%%%%%%%%%%%%%%%%%%%%%%%%%%%%

\subsection{Settings}

%----------------------------------------------------------------------
\begin{slide}[fragile]{1783}\frametitle{Persistent Settings - \iCls{QSettings}}
  \begin{itemize}
 \item Configure \iCls{QSettings}
  \item[] \begin{cpp}
QCoreApplication::setOrganizationName("MyCompany");
QCoreApplication::setOrganizationDomain("mycompany.com");
QCoreApplication::setApplicationName("My Application");
  \end{cpp}
 \item Typical usage
  \item[] \begin{cpp}
QSettings settings; 
settings.setValue("group/value", 68);    
int value = settings.value("group/value").toInt();
  \end{cpp}
  \item Values are stored as \iCls{QVariant}
  \item Keys form hierarchies using '/'
   \begin{itemize}
   \item or use \texttt{beginGroup(prefix)} / \texttt{endGroup()} 
   \end{itemize}
 \item value() excepts default value
   \begin{itemize}
   \item \lstinline{settings.value("group/value", 68).toInt()}
 \end{itemize}
   \item Returns invalid \texttt{QVariant()} if value not found and no default

   \end{itemize}
\end{slide}

%----------------------------------------------------------------------
\begin{slide}[fragile]{1782}\frametitle{Restoring State of an Application}
  \begin{itemize}
  \item Store geometry of application
 \item[] \begin{cpp}
void MainWindow::writeSettings() {
  QSettings settings;
  settings.setValue("MainWindow/size", size());
  settings.setValue("MainWindow/pos", pos());
}    
  \end{cpp}
  \item Restore geometry of application
  \item[] \begin{cpp}
void MainWindow::readSettings() {
  QSettings settings;
  settings.beginGroup("MainWindow");
  resize(settings.value("size", QSize(400, 400)).toSize());
  move(settings.value("pos", QPoint(200, 200)).toPoint());
  settings.endGroup();
}    
  \end{cpp}
  \end{itemize}
\end{slide}


%----------------------------------------------------------------------
\begin{slide}{1781}\frametitle{Settings - Behind the Scenes}
\begin{itemize}
\item Stored in platform specific format
  \begin{itemize}
  \item Unix: INI files
  \item Windows: System registry
  \item MacOS: CFPreferences API
  \item \doc{qsettings.html\#platform-specific-notes}{Platform-Specific Notes}
  \end{itemize}
\end{itemize}
\begin{itemize}
\item \textit{Value lookup will search several locations}
  \begin{enumerate}
  \item  User-specific location
    \begin{enumerate}
    \item  for application
    \item for applications by organization
    \end{enumerate}
  \item  System-wide location
    \begin{enumerate}
    \item  for application
    \item for applications by organization
    \end{enumerate}
  \end{enumerate}
\item \doc{qsettings.html\#fallback-mechanism}{Fallback Mechanism}
\item \iCls{QSettings} creation is cheap! Use on stack
\end{itemize}
\end{slide}

