%%%%%%%%%%%%%%%%%%%%%%%%%%%%%%%%%%%%%%%%%%%%%%%%%%%%%%%%%%%%%%%%%%%%%%%%%%
%
% Copyright (c) 2008-2011, Nokia Corporation and/or its subsidiary(-ies).
% All rights reserved.
%
% This work, unless otherwise expressly stated, is licensed under a
% Creative Commons Attribution-ShareAlike 2.5.
%
% The full license document is available from
% http://creativecommons.org/licenses/by-sa/2.5/legalcode .
%
%%%%%%%%%%%%%%%%%%%%%%%%%%%%%%%%%%%%%%%%%%%%%%%%%%%%%%%%%%%%%%%%%%%%%%%%%%

\subsection{Translation for Developers}

%----------------------------------------------------------------------
\begin{slide}[fragile]{1776} \frametitle{Internationalization (i18n) }
\begin{itemize}
\item \textbf{This is by no means a complete guide!}
\item Internationalization  (i18n)
  \begin{itemize}
  \item Designing applications to be adaptable to languages and regions.
  \end{itemize}
\item Localization (l10n)
  \begin{itemize}
  \item Adapting applications for region by adding components and translations
  \end{itemize}

\item Qt supports the whole process:
  \begin{itemize}
  \item QString supports unicode
  \item On-screen texts (\iClsFn{QObject}{tr})
  \item Number and date formats (\iCls{QLocale})
  \item Icons loading (Resource System)
  \item Translation tool (Qt Linguist)
  \item LTR and RTL text, layout and widgets (e.g. arabic)
  \item Plural handling (1 file vs 2 files)
  \end{itemize}
\end{itemize}

\doc{internationalization.html}{Internationalization with Qt}

\end{slide}

%----------------------------------------------------------------------
\begin{slide}[fragile]{1775}\frametitle{Text Translation}
  
  \begin{itemize}
  \item \textbf{lupdate} - scan C++ and .ui files for strings. Create/update .ts file
  \item \textbf{linguist} - edit .ts file for adding translations
  \item \textbf{lrelease} - read .ts and creates .qm file for release.     
  \item \textbf{QObject::tr()} - mark translatable strings in C++ code.
    \begin{itemize}
    \item combine with \texttt{QString::arg()} for dynamic text
    \end{itemize}        
 \begin{cpp}
  setWindowTitle(tr("File: %1 Line: %2").arg(f).arg(l));

  //: This comment is seen by translation staff
  label->setText(tr("Name: %1  Date: %2")
                .arg(name, d.toString()));
\end{cpp}
\item \doc{i18n-source-translation.html}{Writing Source Code for Translation}
  \end{itemize}
\end{slide}

%----------------------------------------------------------------------
\begin{slide}{1774}\frametitle{Translation Process}
\centeredImageFullWidth{appcreation/images/translation}
\end{slide}

%----------------------------------------------------------------------
\begin{slide}[fragile]{1773}\frametitle{Other Internationalization}
  \begin{itemize}
  \item Qt classes are locale aware
    \begin{itemize}
    \item[] \begin{cpp}
QLocale::setDefault(QLocale::German); // de_DE
QLocale german; bool ok;
german.toDouble("1234,56", &ok); // ok == true
QLocale::setDefault(QLocale::C); // en_US
value = QString("1234.56").toDouble(&ok) // ok == true
     \end{cpp}
   \end{itemize}
  \item \iCls{QDate}, \iCls{QTime} and \iCls{QDateTime}
    \begin{itemize}
    \item[] \begin{cpp}
qDebug() << QDate().toString();  // prints localized date
      \end{cpp}
   \item \doc{qdate.html}{QDate}

    \end{itemize}
 \item Translating Media (Resource System)
   \begin{itemize}
   \item See \doc{resources.html}{Qt Resource System}
   \end{itemize}
  \item Use \iCls{QKeySequence} for shortcut values
    \begin{itemize}
    \item []\begin{cpp}
action->setShortcut(QKeySequence::New);
action->setShortcut(QKeySequence(tr("Ctrl+N"));
      \end{cpp}
    \end{itemize}
  \end{itemize}
\end{slide}


%----------------------------------------------------------------------
\begin{slide}{1772}
  \frametitle{Lab: Translate Editor to German}
  \begin{itemize}
  \item We use our existing editor
  \item In the handout you will find a list of translation words
  \item Germany country code is \texttt{de}
  \item Tip: You can use Qt Linguist to edit translations
  \end{itemize}
\end{slide}
