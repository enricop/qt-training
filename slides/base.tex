%%%%%%%%%%%%%%%%%%%%%%%%%%%%%%%%%%%%%%%%%%%%%%%%%%%%%%%%%%%%%%%%%%%%%%%%%%
%
% Copyright (c) 2008-2011, Nokia Corporation and/or its subsidiary(-ies).
% All rights reserved.
%
% This work, unless otherwise expressly stated, is licensed under a
% Creative Commons Attribution-ShareAlike 2.5.
%
% The full license document is available from
% http://creativecommons.org/licenses/by-sa/2.5/legalcode .
%
%%%%%%%%%%%%%%%%%%%%%%%%%%%%%%%%%%%%%%%%%%%%%%%%%%%%%%%%%%%%%%%%%%%%%%%%%%

\usepackage{ifthen}
\usepackage{alltt}
\usepackage{comment}
\usepackage{multicol}
\usepackage{multirow}
\usepackage{hhline}
\usepackage{bold-extra}
\usepackage{version}
\usepackage{graphicx}
\usepackage{pgfpages}
\usepackage{picins}
\usepackage{makeidx}
\usepackage{multind}
\usepackage{xspace}
\usepackage{mathptmx}
\usepackage{pdfsync}
\usepackage{listings}
%\usepackage{multimedia}
\usepackage[absolute,overlay]{textpos}

\usetheme{QtTraining}

% \definecolor{code_bg}{RGB}{255, 255, 204}

\lstdefinelanguage{qmake}%
  { morekeywords={SOURCES,HEADERS,RESOURCES,FORMS,TARGET,TEMPLATE},
    morestring=[b]",
    sensitive=true,
    comment=[l]\#,
 }

  \lstdefinelanguage[Qt]{C++}[]{C++}%
  { keywordsprefix=Q
  }

  \lstset{ % 
    language=[Qt]C++,
    basicstyle=\ttfamily\footnotesize,
    keywordstyle=\bfseries,
    stringstyle=\color{midnightblue},
    commentstyle=\color{qtdarkgreen}, 
    %numbers=left,
    %numberstyle=\tiny,
    %stepnumber=1,
    %numbersep=10pt, 
    showspaces=false,
    showstringspaces=false,
    frame=l, 
    framerule=1pt,
    framesep=4pt,
    %rulecolor=\color{black!30},
    rulecolor=\color{qtlightgreen},
    aboveskip=1mm,
    belowskip=1mm,
    showlines=false,
  }

\lstnewenvironment{cpp}{\lstset{language=[Qt]C++}}{}
\lstnewenvironment{qmake}{\lstset{language=qmake}}{}
\lstnewenvironment{shell}{\lstset{language=bash}}{}
\lstnewenvironment{xml}{\lstset{language=html}}{}


\newcommand{\inputqml}[1]{%
\setlength{\lineskip}{0pt}
\scriptsize
\input{#1}
\normalsize
}

\newenvironment{qml}{%
\setlength{\lineskip}{0pt}
\scriptsize
}{%
}

\newcommand{\qthome}[2]{\href{http://qt.-project.org/#1}{\beamerbutton{See #2}}}  
\newcommand{\doc}[2]{\href{http://qt-project.org/doc/latest/#1}{\beamerbutton{See #2 Documentation}}}  
\newcommand{\wiki}[2]{\href{http://en.wikipedia.org/wiki/#1}{\beamerbutton{See #2 Wikipedia Article}}}  
\newcommand{\doccreator}[2]{\href{http://qt-project.org/doc/qtcreator/#1}{\beamerbutton{See #2 Documentation}}}  
\newcommand{\demo}[1]{\href{run:#1}{\beamerbutton{Demo #1}}}  
\newcommand{\qtdemo}[1]{\href{run:#1}{\beamerbutton{Demo \$QTDIR/#1}}}  
\newcommand{\lab}[1]{\href{run:#1}{\beamerbutton{Lab #1}}}
\newcommand\externalLink[1]{{\sffamily\color{blue}{\href{#1}{#1}}}}
\newcommand\externalNamedLink[2]{{\sffamily\color{blue}{\href{#1}{#2}}}}


\date{}

\sloppy
\newenvironment{answers}[2]{%
\begin{frame}[fragile]
\begin{center}
#1\\(page #2)
\end{center}
\tiny
}{%
\end{frame}}

%----------------------------------------------------------------------

% setup for two column printing.
\setlength{\columnseprule}{1pt}
\setlength{\columnsep}{20pt}

%----------------------------------------------------------------------
\newcommand{\secTitle}{}
\newcommand{\setSectionTitle}[1]{\def\secTitle{#1}\section{#1}}
\newcommand{\sectionTitle}{\secTitle}

\newcommand{\contentHyperlink}[2]{\hyperlink{#2}{#1 (page~\pageref{#2})}}
\newcommand{\pageHyperlink}[1]{\hyperlink{#1}{page~\pageref{#1}}}
\newcommand{\itop}[2]{\item \contentHyperlink{#1}{#2}}
\newcommand{\itopStar}[2]{\item<*> \contentHyperlink{#1}{#2}}
\newcommand{\showHyperlink}[2]{\textcolor[rgb]{0,0,1}{\hyperlink{#1}{#2}}}

%----------------------------------------------------------------------
\newcommand{\qtThree}{$^{qt3}$}
\newcommand{\qt}[2]{$^{\mathrm{(Qt\,#1.#2)}}$}
\def\strikethrough#1{%
{%
  \setbox0=\hbox{#1}%
  \dimen0=\ht0 \dimen1=\dp0
  \setbox1=\vbox{
       \box0 \vskip-\dimen1\vskip-0.8ex \hrule
   }%
   \ht1=\dimen0 \dp1=\dimen1
   \box1
  }%
}

% These aren't really up to date currently
%\newcommand{\notQtopia}{$^{\tiny \hbox{\strikethrough{Qtopia}}}$}
%\newcommand{\Qtopia}{$^{\tiny \hbox{Qtopia}}$}
\newcommand{\notQtopia}{}
\newcommand{\Qtopia}{}
%----------------------------------------------------------------------
\newcommand{\pleaseNote}{\underline{\textbf{Note:}}\ }

%----------------------------------------------------------------------
%% All this is really magic to me, I found from pdfinfo slides.pdf that:
%% Page size:      362.835 x 272.126 pts
%% If I device 362.835 with the screen width, I get the magic
%% \paperToScreenFactor factor. Don't ask why or how I got to doing that!

%% Note: The \includegraphics command takes the resolution (dots-per-inch)
%% into account, so screenshots from higher resolution sources will be
%% scaled down even further.

\ifthenelse{\equal{\screenwidth}{1024}}{
  \newcommand{\paperToScreenFactor}{0.35433}
  \newcommand{\doublePaperToScreenFactor}{0.70866}
}{
  \ifthenelse{\equal{\screenwidth}{1280}}{
    \newcommand{\paperToScreenFactor}{0.28346}
    \newcommand{\doublePaperToScreenFactor}{0.56692}
  }{
  \ifthenelse{\equal{\screenwidth}{1400}}{
    \newcommand{\paperToScreenFactor}{0.25916}
    \newcommand{\doublePaperToScreenFactor}{0.51833}
    }{
    \error{Unknown resolution, acceptable screen width are 1024, 1280, and
      1400}
    }}
}

\newcommand{\flushedImage}[1]{\parpic[r]{\includegraphics[origin=t,scale=\paperToScreenFactor]{#1}}}
\newcommand{\flushedImageDoubleWidth}[1]{\parpic[r]{\includegraphics[origin=t,scale=\doublePaperToScreenFactor]{#1}}}
\newcommand{\image}[1]{\includegraphics[scale=\paperToScreenFactor]{#1}}
\newcommand{\imageDoubleWidth}[1]{\includegraphics[scale=\doublePaperToScreenFactor]{#1}}
\newcommand{\imageFullWidth}[1]{\includegraphics[width=0.75\pdfpagewidth,keepaspectratio]{#1}}
\newcommand{\centeredImage}[1]{\strut\hfill\includegraphics[scale=\paperToScreenFactor]{#1}\hfill\strut}
\newcommand{\centeredImageDoubleWidth}[1]{\strut\hfill\includegraphics[scale=\doublePaperToScreenFactor]{#1}\hfill\strut}
\newcommand{\centeredImageFullWidth}[1]{\strut\hfill\includegraphics[width=0.75\pdfpagewidth,keepaspectratio]{#1}\hfill\strut}
\newcommand{\centeredImageFullHeight}[1]{\strut\hfill\includegraphics[height=0.6\pdfpageheight,keepaspectratio]{#1}\hfill\strut}
\newcommand{\inlinedImage}[1]{\includegraphics[height=0.5cm]{#1}\strut}

% Open Office images do not need to be scale down, they are vector
% graphics, so we can use the same size for all resolutions.
\newcommand{\oooFlushedImage}[1]{\parpic[r]{\includegraphics[origin=t,scale=0.354331054688]{ooo/#1}}}
\newcommand{\oooImage}[1]{\includegraphics[scale=0.354331054688]{ooo/#1}}
\newcommand{\oooCenteredImage}[1]{\strut\hfill\includegraphics[scale=0.354331054688]{ooo/#1}\hfill\strut}

\definecolor{exampletitlecolor}{rgb}{0.9,0.9,1.0}
\newcommand{\exampletitle}[1]{\colorbox{exampletitlecolor}{\parbox{1.0\textwidth}{#1}}}


% Syntax coloring definitions
\definecolor{keyword}{rgb}{0.9,0.2,0.2}
\definecolor{class}{rgb}{0.0,0.0,0.9}
\definecolor{number}{rgb}{0.9,0.5,0.0}
\definecolor{type}{rgb}{0.6,0.6,0.0}
\definecolor{comment}{rgb}{0.2,0.5,0.2}
\definecolor{string}{rgb}{0.2,0.5,0.9}
\definecolor{operator}{rgb}{0.5,0.5,0.5}

% ----------- Notes (the boolean is set in setup.tex)
\ifthenelse{\boolean{with-notes}}{
  \includeversion{note}
}
{%else
  \excludeversion{note}
}
\excludeversion{EXCLUDE}

\newenvironment{instructors-notes}{%
\setbeamertemplate{headline} {\includegraphics[width=\textwidth]{images/comment-small}}
\setbeamertemplate{footline} {\includegraphics[width=\textwidth]{images/comment-small}}
    \begin{frame}[fragile,environment=instructors-notes]
}{%
\end{frame}}

% --------------- Watermark ---------------------------------------------
\ifthenelse{\boolean{eval-version}}{%
\let\evaltitle\frametitle
\renewcommand{\frametitle}[1]{\evaltitle{#1}%
  \begin{picture}(0,0)%
    \put(0,0){%
      \rotatebox{-25}{\textcolor[gray]{0.80}{\fontsize{15mm}{15mm}\selectfont Evaluation Copy}}}%
  \end{picture}%
}}{}


%% ------------ Title of the course ----------------------------------------
\newcommand{\frontpageTitleFormat}{\fontfamily{NokiaLarge}\fontseries{b}\fontsize{24pt}{28.8pt} \selectfont }

%------------ Setup of index command -----------------------
\newcommand{\classsep}{::}
\newcommand{\pointerStar}{*\xspace}
\newcommand{\programmingLanguage}{C++\xspace}
\newcommand{\pointerAmpersand}{\&}
\newcommand{\pointerDeref}{->}

%%% \newcommand{\capitalise}[1]{\capitalizefoo#1\end} \def\capitalizefoo#1#2\end{\uppercase{#1}\lowercase{#2}}
\newcommand{\possibleEmptyConcat}[2]{\ifthenelse{\equal{#1}{}}{#2}{#1 #2}}
\newcommand{\xConcept}[1]{\index{Concepts}{#1}}
\newcommand{\xCls}[1]{\index{Classes}{#1}}
\newcommand{\xNs}[1]{\index{Classes}{#1 (namespace)}}
\newcommand{\xNsCls}[2]{\index{Classes}{#1 (namespace)!#2}\index{Classes}{#2 (#1)}}
\newcommand{\xClsFn}[2]{\index{Classes}{#1!#2}\index{Classes}{#2 (#1)}}
\newcommand{\xNsFn}[2]{\index{Classes}{#1 (namespace)!#2}\index{Classes}{#2 (#1)}}
\newcommand{\xNsClsFn}[3]{\index{Classes}{#1 (namespace)!#2!#3}\index{Classes}{#2 (#1)!#3}\index{Classes}{#3 (#1\classsep#2)}}
%PENDING(blackie): have the extra section?
%\newcommand{\xClsSig}[2]{\index{Classes}{#1!#2 (signal)}\index{Classes}{#2 (#1, signal)}\index{Classes}{Signals!#1!#2}\index{Classes}{Signals!#2!#1}}
\newcommand{\xClsSig}[2]{\index{Classes}{#1!#2 (signal)}\index{Classes}{#2 (signal on #1)}}
\newcommand{\xNsClsSig}[3]{\index{Classes}{#1 (namespace)!#2!#3 (signal)}\index{Classes}{#2 (#1)!#3 (signal)}\index{Classes}{#3 (signal on #1\classsep#2)}}
%PENDING(blackie): have the extra section?
%\newcommand{\xClsSlt}[2]{\index{Classes}{#1!#2 (slot)}\index{Classes}{#2 (#1, slot)}\index{Classes}{Slots!#1!#2}\index{Classes}{Slots!#2!#1}}
\newcommand{\xClsSlt}[2]{\index{Classes}{#1!#2 (slot)}\index{Classes}{#2 (slot on #1)}}
\newcommand{\xNsClsSlt}[3]{\index{Classes}{#1 (namespace)!#2!#3 (slot)}\index{Classes}{#2 (#1)!#3 (slot)}\index{Classes}{#3 (slot on #1\classsep#2)}}
\newcommand{\xMacro}[1]{\index{Classes}{#1 (macro)}}
\newcommand{\xExample}[1]{\index{Examples}{#1}}
\newcommand{\xProject}[1]{\index{Projects}{#1}}

\newcommand{\iConcept}[2][]{#1{#2}\xConcept{#2}}
\newcommand{\iMacro}[2][\texttt]{#1{#2}\xMacro{#2}}
\newcommand{\iMacroPar}[3][\texttt]{#1{#2(#3)}\xMacro{#2}}
\newcommand{\iExample}[1]{\emph{#1}\xExample{#1}}
\newcommand{\iProject}[1]{\emph{#1}\xProject{#1}}

% \iCls*
\newcommand{\iCls}[2][\texttt]{#1{#2}\xCls{#2}}
\newcommand{\iClsFn}[3][\texttt]{#1{#2\classsep#3()}\xClsFn{#2}{#3}}
\newcommand{\iClsSig}[3][\texttt]{#1{#2\classsep#3()}\xClsSig{#2}{#3}}
\newcommand{\iClsSlt}[3][\texttt]{#1{#2\classsep#3()}\xClsSlt{#2}{#3}}
\newcommand{\iClsEnum}[3][\texttt]{#1{#2\classsep#3}\xClsFn{#2}{#3}}
\newcommand{\iClsFnPar}[4][\texttt]{#1{#2\classsep#3(#4)}\xClsFn{#2}{#3}}
\newcommand{\iClsSigPar}[4][\texttt]{#1{#2\classsep#3(#4)}\xClsSig{#2}{#3}}
\newcommand{\iClsSltPar}[4][\texttt]{#1{#2\classsep#3(#4)}\xClsSlt{#2}{#3}}

% \iClsT*
\newcommand{\iClsT}[2][T]{\texttt{#2<#1>}\xCls{#2}}
\newcommand{\iClsTFn}[3][T]{\texttt{#2<#1>\classsep#3()}\xClsFn{#2}{#3}}
\newcommand{\iClsTSig}[3][T]{\texttt{#2<#1>\classsep#3()}\xClsSig{#2}{#3}}
\newcommand{\iClsTSlt}[3][T]{\texttt{#2<#1>\classsep#3()}\xClsSlt{#2}{#3}}
\newcommand{\iClsTEnum}[3][T]{\texttt{#2<#1>\classsep#3}\xClsFn{#2}{#3}}
\newcommand{\iClsTFnPar}[4][T]{\texttt{#2<#1>\classsep#3(#4)}\xClsFn{#2}{#3}}
\newcommand{\iClsTSigPar}[4][T]{\texttt{#2<#1>\classsep#3(#4)}\xClsSig{#2}{#3}}
\newcommand{\iClsTSltPar}[4][T]{\texttt{#2<#1>\classsep#3(#4)}\xClsSlt{#2}{#3}}

% \iNs*
\newcommand{\iNs}[2][\texttt]{#1{#2}\xNs{#2}}
\newcommand{\iNsFn}[3][\texttt]{#1{#2\classsep#3()}\xNsFn{#2}{#3}}
\newcommand{\iNsEnum}[3][\texttt]{#1{#2\classsep#3}\xNsFn{#2}{#3}}
\newcommand{\iNsFnPar}[4][\texttt]{#1{#2\classsep#3(#4)}\xNsFn{#2}{#3}}

% \iNsCls*
\newcommand{\iNsCls}[3][\texttt]{#1{#2\classsep#3}\xNsCls{#2}{#3}}
\newcommand{\iNsClsFn}[4][\texttt]{#1{#2\classsep#3\classsep#4()}\xNsClsFn{#2}{#3}{#4}}
\newcommand{\iNsClsSig}[4][\texttt]{#1{#2\classsep#3\classsep#4()}\xNsClsSig{#2}{#3}{#4}}
\newcommand{\iNsClsSlt}[4][\texttt]{#1{#2\classsep#3\classsep#4()}\xNsClsSlt{#2}{#3}{#4}}
\newcommand{\iNsClsEnum}[4][\texttt]{#1{#2\classsep#3\classsep#4}\xNsClsFn{#2}{#3}{#4}}
\newcommand{\iNsClsFnPar}[5][\texttt]{#1{#2\classsep#3\classsep#4(#5)}\xNsClsFn{#2}{#3}{#4}}
\newcommand{\iNsClsSigPar}[5][\texttt]{#1{#2\classsep#3\classsep#4(#5)}\xClsSig{#2}{#3}{#4}}
\newcommand{\iNsClsSltPar}[5][\texttt]{#1{#2\classsep#3\classsep#4(#5)}\xClsSlt{#2}{#3}{#4}}

% like \iCls*, but hide the class:
\newcommand{\hClsFn}[3][\texttt]{#1{#3()}\xClsFn{#2}{#3}}
\newcommand{\hClsSig}[3][\texttt]{#1{#3()}\xClsSig{#2}{#3}}
\newcommand{\hClsSlt}[3][\texttt]{#1{#3()}\xClsSlt{#2}{#3}}
\newcommand{\hClsEnum}[3][\texttt]{#1{#3}\xClsFn{#2}{#3}}
\newcommand{\hClsFnPar}[4][\texttt]{#1{#3(#4)}\xClsFn{#2}{#3}}
\newcommand{\hClsSigPar}[4][\texttt]{#1{#3(#4)}\xClsSig{#2}{#3}}
\newcommand{\hClsSltPar}[4][\texttt]{#1{#3(#4)}\xClsSlt{#2}{#3}}

% like \iNs*, but hide the namespace:
\newcommand{\hNsFn}[3][\texttt]{#1{#3()}\xNsFn{#2}{#3}}
\newcommand{\hNsEnum}[3][\texttt]{#1{#3}\xNsFn{#2}{#3}}
\newcommand{\hNsFnPar}[4][\texttt]{#1{#3(#4)}\xNsFn{#2}{#3}}

% like \iNsCls*, but hide namespace (but not class):
\newcommand{\hNsCls}[3][\texttt]{#1{#3}\xNsCls{#2}{#3}}
\newcommand{\hNsClsFn}[4][\texttt]{#1{#3\classsep#4()}\xNsClsFn{#2}{#3}{#4}}
\newcommand{\hNsClsSig}[4][\texttt]{#1{#3\classsep#4()}\xNsClsSig{#2}{#3}{#4}}
\newcommand{\hNsClsSlt}[4][\texttt]{#1{#3\classsep#4()}\xNsClsSlt{#2}{#3}{#4}}
\newcommand{\hNsClsEnum}[4][\texttt]{#1{#3\classsep#4}\xNsClsFn{#2}{#3}{#4}}
\newcommand{\hNsClsFnPar}[5][\texttt]{#1{#3\classsep#4(#5)}\xNsClsFn{#2}{#3}{#4}}
\newcommand{\hNsClsSigPar}[5][\texttt]{#1{#3\classsep#4(#5)}\xNsClsSig{#2}{#3}{#4}}
\newcommand{\hNsClsSltPar}[5][\texttt]{#1{#3\classsep#4(#5)}\xNsClsSlt{#2}{#3}{#4}}

% like \iNsCls*, but hide namespace /and/ class:
\newcommand{\hhNsClsFn}[4][\texttt]{#1{#4()}\xNsClsFn{#2}{#3}{#4}}
\newcommand{\hhNsClsSig}[4][\texttt]{#1{#4()}\xNsClsSig{#2}{#3}{#4}}
\newcommand{\hhNsClsSlt}[4][\texttt]{#1{#4()}\xNsClsSlt{#2}{#3}{#4}}
\newcommand{\hhNsClsEnum}[4][\texttt]{#1{#4}\xNsClsFn{#2}{#3}{#4}}
\newcommand{\hhNsClsFnPar}[5][\texttt]{#1{#4(#5)}\xNsClsFn{#2}{#3}{#4}}
\newcommand{\hhNsClsSigPar}[5][\texttt]{#1{#4(#5)}\xNsClsSig{#2}{#3}{#4}}
\newcommand{\hhNsClsSltPar}[5][\texttt]{#1{#4(#5)}\xNsClsSlt{#2}{#3}{#4}}

\newcommand{\iClsVar}{\iClsEnum}
\newcommand{\hClsVar}{\hClsEnum}
\newcommand{\iClsTypedef}{\iClsEnum}
\newcommand{\hClsTypedef}{\hClsEnum}

\newcommand{\iNsVar}{\iNsEnum}
\newcommand{\hNsVar}{\hNsEnum}
\newcommand{\iNsTypedef}{\iNsEnum}
\newcommand{\hNsTypedef}{\hNsEnum}

\newcommand{\iNsClsVar}{\iNsClsEnum}
\newcommand{\hNsClsVar}{\hNsClsEnum}
\newcommand{\hhNsClsVar}{\hhNsClsEnum}
\newcommand{\iNsClsTypedef}{\iNsClsEnum}
\newcommand{\hNsClsTypedef}{\hNsClsEnum}
\newcommand{\hhNsClsTypedef}{\hhNsClsEnum}

% expands to "example $QTDIR/examples/#1"
\newcommand{\qtexample}[1]{example \emph{\$QTDIR/examples/\iExample{#1}}}
% expands to "handout/#1"
\newcommand{\projecthandout}[1]{\emph{handout/#1}}
% expands to "solutions/#1"
\newcommand{\projectsolution}[1]{\emph{solutions/#1}}

%-------------------------- Review commands--------------------------------------------
\newcommand{\reviewTT}{%
  \begin{picture}(0,0)%
    \put(313,40){%
      \rotatebox{-90}{\textcolor{red}{\fontsize{10mm}{10mm}\selectfont
          Qt REVIEW}}%
      }%
  \end{picture}%
  \par
 \vspace{-6mm}
}

\newboolean{ignore-slide}
\excludeversion{EXCLUDE}
%--------------------------- Slide environment -------------------------------------------
\newenvironment{slide}[2][1]{%
  \renewcommand{\slideId}{#2}
  \ifthenelse{\equal{#1}{1}}{
    \begin{frame}[environment=slide]
      }{
    \begin{frame}[#1,environment=slide]
  }
}%
{\end{frame}}

\AtBeginSection[] % Do nothing for \section*
{
  \begin{frame}<beamer>  
    \frametitle{Module: \secname}
     \tableofcontents[sectionstyle=hide,subsectionstyle=show/show/hide,subsubsectionstyle=show/show/show]
 \end{frame}
}

\AtBeginSubsection[] % Do nothing for \subsection*
{
  \begin{frame}<beamer>
    \frametitle{Module: \secname}
    \medskip
    \tableofcontents[sectionstyle=hide,subsectionstyle=show/shaded/hide,subsubsectionstyle=show/show/shaded]
  \end{frame}
}

\AtBeginPart
{
  \begin{frame}<beamer>
    \frametitle{\insertshortpart}
    \tableofcontents[sectionstyle=show,subsectionstyle=hide]      
  \end{frame}
}


\newcommand{\todo}[1]{\begin{minipage}{3cm}
        \tiny{#1} \end{minipage}}


\newenvironment{questionize}{\begin{itemize}[<+>]}{\end{itemize}}

% Highlighting in {alltt}, slides style:
\newcommand{\highlight}[1]{\textcolor{red}{#1}}

\newcommand{\correct}{\textcolor{qtdarkgreen}{$\surd$}}
\newcommand{\incorrect}{\textcolor{red}{$X$}}
\newcommand{\incorrecttext}[1]{\textcolor{red}{\strikethrough{#1}}}

\newcommand{\ok}{\textcolor{qtdarkgreen}{$OK$}}
\newcommand{\new}{\textcolor{qtdarkgreen}{$NEW$}}
\newcommand{\notok}{\textcolor{red}{$NOK$}}
\newcommand{\work}{\textcolor{yellow}{$WORK$}}
\newcommand{\unknown}{\textcolor{orange}{$???$}}
\newcommand{\delete}{\textcolor{red}{$DEL$}}

\newcommand{\consolecode}[1]{%
\vspace*{0.25em}
\fcolorbox{gray}{darkgray}{\begin{minipage}{\columnwidth - \leftmargin}
\textcolor{white}{\texttt{#1}}
\end{minipage}}
\vspace*{0.25em}}

\newcommand{\qc}[2]{\textcolor{#1}{#2}}
\newcommand{\qic}[2]{{\small\texttt{\textcolor{#1}{#2}}}}
\newcommand{\qtt}[1]{\texttt{#1}}

\usepackage{fontspec}
\usepackage{xunicode}

% set the default roman font
\setmainfont[Scale=0.9]{Arial}

% set the default sans font
\setsansfont[Scale=0.9]{Arial}

% set the default mono font
\setmonofont[Scale=0.8]{Liberation Mono}

% define new font families for main body, footnote, title
\newfontfamily\slidebodyfont{Arial}

\newfontfamily\slidefootnotefont{Arial}

\newfontfamily\slidetitlefont{Arial Bold}

