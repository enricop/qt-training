%%%%%%%%%%%%%%%%%%%%%%%%%%%%%%%%%%%%%%%%%%%%%%%%%%%%%%%%%%%%%%%%%%%%%%%%%%
%
% Copyright (c) 2008-2012, Nokia Corporation and/or its subsidiary(-ies).
% All rights reserved.
%
% This work, unless otherwise expressly stated, is licensed under a
% Creative Commons Attribution-ShareAlike 2.5.
%
% The full license document is available from
% http://creativecommons.org/licenses/by-sa/2.5/legalcode .
%
%%%%%%%%%%%%%%%%%%%%%%%%%%%%%%%%%%%%%%%%%%%%%%%%%%%%%%%%%%%%%%%%%%%%%%%%%%

\section{Graphics Effects}

%----------------------------------------------------------------------
\begin{slide}{1790}\frametitle{Objectives}

\vspace*{1em}
\begin{itemize}
\item Knowledge on how to create items with your own painting code
\item Use canvas with user interaction of animations
\item Create a complete particle system
  \begin{itemize}
  \item Specify the particles
  \item Provide speed, acceleration or other physics traits
  \end{itemize}
\item Use shaders to modify items rendering
  \begin{itemize}
  \item Fragment shaders for pixel manipulation
  \item Vertex shaders for shape manipulation
  \end{itemize}
\end{itemize}

\end{slide}

%----------------------------------------------------------------------
\begin{slide}{1791}\frametitle{Why use canvas, particles and shaders?}
\flushedImage{qml-graphics-effects/images/effects-showcase.png}

\begin{itemize}
\item Custom painting of components
\item Graphs and plots
\end{itemize}
\vspace*{0.5em}
\begin{itemize}
\item Complex visual effects
\item Simulate some physics during animations
\item Benefit as much as possible from the GPU
\end{itemize}

\demo{qml-graphics-effects/ex-canvas/piechart.qml}

\demo{qml-graphics-effects/ex-flickr/flickr.qml}
\end{slide}

%%%%%%%%%%%%%%%%%%%%%%%%%%%%%%%%%%%%%%%%%%%%%%%%%%%%%%%%%%%%%%%%%%%%%%%%%%
%
% Copyright (c) 2008-2012, Nokia Corporation and/or its subsidiary(-ies).
% All rights reserved.
%
% This work, unless otherwise expressly stated, is licensed under a
% Creative Commons Attribution-ShareAlike 2.5.
%
% The full license document is available from
% http://creativecommons.org/licenses/by-sa/2.5/legalcode .
%
%%%%%%%%%%%%%%%%%%%%%%%%%%%%%%%%%%%%%%%%%%%%%%%%%%%%%%%%%%%%%%%%%%%%%%%%%%

\subsection{Canvas}

%----------------------------------------------------------------------

\begin{slide}{1800}\frametitle{Canvas}

\begin{itemize}
\item \qic{class}{Canvas} is used to insert an element in which to paint
\item \qic{type}{onPaint} handler will contain the painting code
\item API somewhat similar to the old \qic{class}{QPainter} API
\item API compatible with the HTML5 Canvas API
  \begin{itemize}
  \item Need to request a \qic{class}{Context2D} instance first
  \end{itemize}
\item \qic{type}{requestPaint()} method to schedule repainting
\end{itemize}
\end{slide}

%----------------------------------------------------------------------

\begin{slide}{1801}\frametitle{Path Rendering}
\flushedImage{qml-graphics-effects/images/path-painting.png}

\inputqml{qml-graphics-effects/colorized/path-painting}

\demo{qml-graphics-effects/ex-canvas/path-painting.qml}
\end{slide}

%----------------------------------------------------------------------

\begin{slide}{1802}\frametitle{Scribble Area}
\inputqml{qml-graphics-effects/colorized/scribble-area}

\demo{qml-graphics-effects/ex-canvas/scribble-area.qml}
\end{slide}

%----------------------------------------------------------------------

\begin{slide}{1803}\frametitle{Lab: Screen Saver}
\centeredImage{qml-graphics-effects/images/screensaver.png}

Starting from the partial solution:

\begin{itemize}
\item Get the lines to be rendered
\item Have the points forming the lines animated
\end{itemize}

\lab{qml-graphics-effects/lab-screensaver/screensaver.qml}
\end{slide}


%%%%%%%%%%%%%%%%%%%%%%%%%%%%%%%%%%%%%%%%%%%%%%%%%%%%%%%%%%%%%%%%%%%%%%%%%%
%
% Copyright (c) 2008-2012, Nokia Corporation and/or its subsidiary(-ies).
% All rights reserved.
%
% This work, unless otherwise expressly stated, is licensed under a
% Creative Commons Attribution-ShareAlike 2.5.
%
% The full license document is available from
% http://creativecommons.org/licenses/by-sa/2.5/legalcode .
%
%%%%%%%%%%%%%%%%%%%%%%%%%%%%%%%%%%%%%%%%%%%%%%%%%%%%%%%%%%%%%%%%%%%%%%%%%%

\subsection{Particles}

%----------------------------------------------------------------------

\begin{slide}{1810}\frametitle{Particle System}

\begin{itemize}
\item A \qic{class}{ParticleSystem} requires
  \begin{itemize}
  \item at least a particles source
  \item the description of how particles look
  \end{itemize}
\vspace*{0.5em}
\item Particles sources are \qic{class}{Emitter} instances
  \begin{itemize}
  \item They emit the logical particles
  \item Provide initial attributes: \qic{type}{emitRate}, \qic{type}{lifeSpan}, \qic{type}{size}, \qic{type}{speed}, ...
  \item The flow can be controlled using \qic{type}{enabled}, \qic{type}{pulse()} and \qic{type}{burst()}
  \end{itemize}
\vspace*{0.5em}
\item Particle appearance is controlled by a \qic{class}{ParticlePainter} instance
  \begin{itemize}
  \item \qic{class}{ImageParticle} uses an image as \qic{type}{source}, it can be rotated, colorized, etc.
  \item \qic{class}{ItemParticle} uses an \qic{class}{Item} \qic{type}{delegate} to render particles
  \item \qic{class}{CustomParticle} uses shaders to render particles
  \end{itemize}
\end{itemize}

\end{slide}

%----------------------------------------------------------------------

\begin{slide}{1811}\frametitle{Christmas Lights}
\inputqml{qml-graphics-effects/colorized/christmas-lights}

\demo{qml-graphics-effects/ex-particles/christmas-lights.qml}

\vspace*{-20em}\hfill\image{qml-graphics-effects/images/christmas-lights.png}
\end{slide}

%----------------------------------------------------------------------

\begin{slide}{1812}\frametitle{Physics: Speed \& Acceleration}

\begin{itemize}
\item Both initial \qic{type}{speed} and \qic{type}{acceleration} are specified using a \qic{class}{Direction}
\vspace*{0.5em}
\item A \qic{class}{Direction} is a vector space of possible directions for a particle
  \begin{itemize}
  \item Value intervals are specified using \qic{type}{*Variation} properties
  \item Each particle gets a random vector of the vector space
  \end{itemize}
\vspace*{0.5em}
\item \qic{class}{Direction} is never used, it has subclasses
  \begin{itemize}
  \item \qic{class}{AngleDirection} for directions varying in \qic{type}{angle}
  \item \qic{class}{PointDirection} for directions varying in \qic{type}{x} and \qic{type}{y} components
  \item \qic{class}{TargetDirection} for directions toward a \qic{type}{targetItem}
  \item \qic{class}{CumulativeDirection} acts as a direction that sums the directions within it
  \end{itemize}
\end{itemize}

\end{slide}

%----------------------------------------------------------------------

\begin{slide}{1813}\frametitle{Explosion}
\inputqml{qml-graphics-effects/colorized/explosion}

\demo{qml-graphics-effects/ex-particles/explosion.qml}

\vspace*{-20em}\hfill\image{qml-graphics-effects/images/explosion.png}
\end{slide}

%----------------------------------------------------------------------

\begin{slide}{1814}\frametitle{Physics: Force fields}

\begin{itemize}
\item \qic{class}{Affector} instances can impact any attribute of the particles
  \begin{itemize}
  \item Behavior provided by \qic{type}{onAffectParticles}
  \item Gives access to all the \qic{type}{particles}
  \item \qic{type}{dt} is the time since last call (used to normalize to real time)
  \item Not recommended in high-volume particle systems (JS code being slower)
  \end{itemize}
\vspace*{0.5em}
\item Most common cases are covered with \qic{class}{Affector} subclasses
  \begin{itemize}
  \item \qic{class}{Attractor} attracts toward a point (magnetism, close object gravity)\linebreak\demo{qml-graphics-effects/ex-particles/explosion-implosion.qml}
  \item \qic{class}{Gravity} accelerates to a given vector (constant gravitational pull)
  \item \qic{class}{Friction} slows down proportionally to particles speed
  \item \qic{class}{Turbulence} provides fluid like forces based on a \qic{type}{noiseSource} image map
  \item \qic{class}{Wander} provides random particle perturbations
  \item ...
  \end{itemize}
\vspace*{0.5em}
\item The subclasses are C++ based and so much faster
\end{itemize}

\end{slide}

%----------------------------------------------------------------------

\begin{slide}{1815}\frametitle{Fountain}
\inputqml{qml-graphics-effects/colorized/fountain}

\demo{qml-graphics-effects/ex-particles/fountain.qml}

\vspace*{-10em}\hfill\image{qml-graphics-effects/images/fountain.png}
\end{slide}

%----------------------------------------------------------------------

\begin{slide}{1816}\frametitle{Lab: Screen Saver}
\centeredImage{qml-graphics-effects/images/makeitsnow.png}

Starting from the partial solution:

\begin{itemize}
\item Get the snow flakes to slowly fall
\item Make sure they're not all going exactly in the same direction/speed
\item Optionally: Get the snow flakes to rotate as they fall
\end{itemize}

\lab{qml-graphics-effects/lab-makeitsnow/makeitsnow.qml}
\end{slide}


%%%%%%%%%%%%%%%%%%%%%%%%%%%%%%%%%%%%%%%%%%%%%%%%%%%%%%%%%%%%%%%%%%%%%%%%%%
%
% Copyright (c) 2008-2012, Nokia Corporation and/or its subsidiary(-ies).
% All rights reserved.
%
% This work, unless otherwise expressly stated, is licensed under a
% Creative Commons Attribution-ShareAlike 2.5.
%
% The full license document is available from
% http://creativecommons.org/licenses/by-sa/2.5/legalcode .
%
%%%%%%%%%%%%%%%%%%%%%%%%%%%%%%%%%%%%%%%%%%%%%%%%%%%%%%%%%%%%%%%%%%%%%%%%%%

\subsection{Shaders}

%----------------------------------------------------------------------

\begin{slide}{1820}\frametitle{Shaders}

\begin{itemize}
\item A shader is a program used to calculate rendering effects on the GPU
\vspace*{1em}
\item Two types of shader are available in Qt Quick
  \begin{itemize}
  \item Fragment shaders
    \begin{itemize}
    \item Operate on each pixel
    \item Cannot be complex as it has no knowledge of the scene geometry
    \item Used for color manipulation, bump mapping, shadows, etc.
    \end{itemize}
  \item Vertex shaders
    \begin{itemize}
    \item Operate on each vertex
    \item Can change position, color and texture coordinate
    \item Cannot create new vertices
    \end{itemize}
  \end{itemize}
\item Their execution is heavily parallelized in the GPU pipeline
\item Extremely efficient
\item Written using OpenGL Shading Language (GLSL)
\end{itemize}

\end{slide}

%----------------------------------------------------------------------

\begin{slide}{1821}\frametitle{Fragment Shaders}

\begin{itemize}
\item \qic{class}{ShaderEffect} is a rectangle displaying the result of a shader program
\item The \qic{type}{fragmentShader} property is a string with the fragment shader code
\item Often such shaders use textures as inputs
\vspace*{1em}
\item \qic{class}{ShaderEffectSource} allows to render an item as a texture
\item \qic{type}{sourceItem} holds the \qic{class}{Item} to be rendered
\vspace*{1em}
\item Note that \qic{class}{ShaderEffectSource} is an invisible element aimed at consumption in \qic{class}{ShaderEffect} instances
\end{itemize}

\end{slide}

%----------------------------------------------------------------------

\begin{slide}{1822}\frametitle{Saturation Filter}
\inputqml{qml-graphics-effects/colorized/saturation-filter}

\demo{qml-graphics-effects/ex-shaders/saturation-filter.qml}

\vspace*{-20em}\hfill\image{qml-graphics-effects/images/saturation-filter.png}
\end{slide}

%----------------------------------------------------------------------

\begin{slide}{1823}\frametitle{Vertex Shaders}

\begin{itemize}
\item Works similarly to fragment shaders
\item Use the \qic{type}{vertexShader} property for the vertex shader code
\item Pay attention to the \qic{type}{mesh} property
  \begin{itemize}
  \item Specifies the number of vertices of the \qic{class}{ShaderEffect} element
  \item It must be fine enough to resolve the transformation
  \end{itemize}
\end{itemize}

\end{slide}

%----------------------------------------------------------------------

\begin{slide}{1824}\frametitle{Flag}
\inputqml{qml-graphics-effects/colorized/flag}

\demo{qml-graphics-effects/ex-shaders/flag.qml}

\vspace*{-20em}\hfill\image{qml-graphics-effects/images/flag.png}
\end{slide}

%----------------------------------------------------------------------

\begin{slide}{1825}\frametitle{Chaining Shaders}

\begin{itemize}
\item \qic{class}{ShaderEffectSource} can have any \qic{class}{Item} as \qic{type}{sourceItem}
\item Even a \qic{class}{ShaderEffect}!
\item Allows to create complex effects by chaining shader programs
\end{itemize}

\end{slide}

%----------------------------------------------------------------------

\begin{slide}{1826}\frametitle{Drop Shadow}
\centeredImage{qml-graphics-effects/images/drop-shadow.png}

\vspace*{2em}
A drop shadow is a combination of:

\begin{itemize}
\item A blur operation
\item A darkening of the result of the blur
\item A composition of the original on top of the created shadow with an offset
\end{itemize}

\demo{qml-graphics-effects/ex-shaders/drop-shadow.qml}
\end{slide}

%----------------------------------------------------------------------

\begin{slide}{1827}\frametitle{QtGraphicalEffects}

\begin{itemize}
\item New module providing reusable elements with premade shaders
\item Fairly extensive list of effects
  \begin{itemize}
  \item Colorize
  \item Gradients
  \item Blurs
  \item Gamma and levels adjustments
  \item Drop shadows
  \item Glows
  \item ...
  \end{itemize}
\end{itemize}

\demo{qml-graphics-effects/ex-shaders/drop-shadow2.qml}
\end{slide}

%----------------------------------------------------------------------

\begin{slide}{1828}\frametitle{Shaders and Particles}

\begin{itemize}
\item Everything about the particle systems is still valid
\item Allows for less CPU intensive particle rendering

\vspace*{1em}

\item Use \qic{class}{CustomParticle} instead of \qic{class}{ImageParticle}
\item Use the \qic{type}{vertexShader} and \qic{type}{fragmentShader} properties
\end{itemize}

\vspace*{2em}

\demo{qml-graphics-effects/ex-shaders/fountain-shaders.qml}
\end{slide}

%----------------------------------------------------------------------

\begin{slide}{1829}\frametitle{Lab: Dissolve Effect}
\centeredImage{qml-graphics-effects/images/dissolve.png}

Starting from the partial solution:

\begin{itemize}
\item Create an alpha gradient effect
\item Animate it so the item fades out from top to bottom and back in again
\end{itemize}

\lab{qml-graphics-effects/lab-dissolve/dissolve.qml}
\end{slide}



