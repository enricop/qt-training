%%%%%%%%%%%%%%%%%%%%%%%%%%%%%%%%%%%%%%%%%%%%%%%%%%%%%%%%%%%%%%%%%%%%%%%%%%
%
% Copyright (c) 2008-2012, Nokia Corporation and/or its subsidiary(-ies).
% All rights reserved.
%
% This work, unless otherwise expressly stated, is licensed under a
% Creative Commons Attribution-ShareAlike 2.5.
%
% The full license document is available from
% http://creativecommons.org/licenses/by-sa/2.5/legalcode .
%
%%%%%%%%%%%%%%%%%%%%%%%%%%%%%%%%%%%%%%%%%%%%%%%%%%%%%%%%%%%%%%%%%%%%%%%%%%

\subsection{Canvas}

%----------------------------------------------------------------------

\begin{slide}{1800}\frametitle{Canvas}

\begin{itemize}
\item \qic{class}{Canvas} is used to insert an element in which to paint
\item \qic{type}{onPaint} handler will contain the painting code
\item API somewhat similar to the old \qic{class}{QPainter} API
\item API compatible with the HTML5 Canvas API
  \begin{itemize}
  \item Need to request a \qic{class}{Context2D} instance first
  \end{itemize}
\item \qic{type}{requestPaint()} method to schedule repainting
\end{itemize}
\end{slide}

%----------------------------------------------------------------------

\begin{slide}{1801}\frametitle{Path Rendering}
\flushedImage{qml-graphics-effects/images/path-painting.png}

\inputqml{qml-graphics-effects/colorized/path-painting}

\demo{qml-graphics-effects/ex-canvas/path-painting.qml}
\end{slide}

%----------------------------------------------------------------------

\begin{slide}{1802}\frametitle{Scribble Area}
\inputqml{qml-graphics-effects/colorized/scribble-area}

\demo{qml-graphics-effects/ex-canvas/scribble-area.qml}
\end{slide}

%----------------------------------------------------------------------

\begin{slide}{1803}\frametitle{Lab: Screen Saver}
\centeredImage{qml-graphics-effects/images/screensaver.png}

Starting from the partial solution:

\begin{itemize}
\item Get the lines to be rendered
\item Have the points forming the lines animated
\end{itemize}

\lab{qml-graphics-effects/lab-screensaver/screensaver.qml}
\end{slide}

