%%%%%%%%%%%%%%%%%%%%%%%%%%%%%%%%%%%%%%%%%%%%%%%%%%%%%%%%%%%%%%%%%%%%%%%%%%
%
% Copyright (c) 2008-2012, Nokia Corporation and/or its subsidiary(-ies).
% All rights reserved.
%
% This work, unless otherwise expressly stated, is licensed under a
% Creative Commons Attribution-ShareAlike 2.5.
%
% The full license document is available from
% http://creativecommons.org/licenses/by-sa/2.5/legalcode .
%
%%%%%%%%%%%%%%%%%%%%%%%%%%%%%%%%%%%%%%%%%%%%%%%%%%%%%%%%%%%%%%%%%%%%%%%%%%

\subsection{Shaders}

%----------------------------------------------------------------------

\begin{slide}{1820}\frametitle{Shaders}

\begin{itemize}
\item A shader is a program used to calculate rendering effects on the GPU
\vspace*{1em}
\item Two types of shader are available in Qt Quick
  \begin{itemize}
  \item Fragment shaders
    \begin{itemize}
    \item Operate on each pixel
    \item Cannot be complex as it has no knowledge of the scene geometry
    \item Used for color manipulation, bump mapping, shadows, etc.
    \end{itemize}
  \item Vertex shaders
    \begin{itemize}
    \item Operate on each vertex
    \item Can change position, color and texture coordinate
    \item Cannot create new vertices
    \end{itemize}
  \end{itemize}
\item Their execution is heavily parallelized in the GPU pipeline
\item Extremely efficient
\item Written using OpenGL Shading Language (GLSL)
\end{itemize}

\end{slide}

%----------------------------------------------------------------------

\begin{slide}{1821}\frametitle{Fragment Shaders}

\begin{itemize}
\item \qic{class}{ShaderEffect} is a rectangle displaying the result of a shader program
\item The \qic{type}{fragmentShader} property is a string with the fragment shader code
\item Often such shaders use textures as inputs
\vspace*{1em}
\item \qic{class}{ShaderEffectSource} allows to render an item as a texture
\item \qic{type}{sourceItem} holds the \qic{class}{Item} to be rendered
\vspace*{1em}
\item Note that \qic{class}{ShaderEffectSource} is an invisible element aimed at consumption in \qic{class}{ShaderEffect} instances
\end{itemize}

\end{slide}

%----------------------------------------------------------------------

\begin{slide}{1822}\frametitle{Saturation Filter}
\inputqml{qml-graphics-effects/colorized/saturation-filter}

\demo{qml-graphics-effects/ex-shaders/saturation-filter.qml}

\vspace*{-20em}\hfill\image{qml-graphics-effects/images/saturation-filter.png}
\end{slide}

%----------------------------------------------------------------------

\begin{slide}{1823}\frametitle{Vertex Shaders}

\begin{itemize}
\item Works similarly to fragment shaders
\item Use the \qic{type}{vertexShader} property for the vertex shader code
\item Pay attention to the \qic{type}{mesh} property
  \begin{itemize}
  \item Specifies the number of vertices of the \qic{class}{ShaderEffect} element
  \item It must be fine enough to resolve the transformation
  \end{itemize}
\end{itemize}

\end{slide}

%----------------------------------------------------------------------

\begin{slide}{1824}\frametitle{Flag}
\inputqml{qml-graphics-effects/colorized/flag}

\demo{qml-graphics-effects/ex-shaders/flag.qml}

\vspace*{-20em}\hfill\image{qml-graphics-effects/images/flag.png}
\end{slide}

%----------------------------------------------------------------------

\begin{slide}{1825}\frametitle{Chaining Shaders}

\begin{itemize}
\item \qic{class}{ShaderEffectSource} can have any \qic{class}{Item} as \qic{type}{sourceItem}
\item Even a \qic{class}{ShaderEffect}!
\item Allows to create complex effects by chaining shader programs
\end{itemize}

\end{slide}

%----------------------------------------------------------------------

\begin{slide}{1826}\frametitle{Drop Shadow}
\centeredImage{qml-graphics-effects/images/drop-shadow.png}

\vspace*{2em}
A drop shadow is a combination of:

\begin{itemize}
\item A blur operation
\item A darkening of the result of the blur
\item A composition of the original on top of the created shadow with an offset
\end{itemize}

\demo{qml-graphics-effects/ex-shaders/drop-shadow.qml}
\end{slide}

%----------------------------------------------------------------------

\begin{slide}{1827}\frametitle{QtGraphicalEffects}

\begin{itemize}
\item New module providing reusable elements with premade shaders
\item Fairly extensive list of effects
  \begin{itemize}
  \item Colorize
  \item Gradients
  \item Blurs
  \item Gamma and levels adjustments
  \item Drop shadows
  \item Glows
  \item ...
  \end{itemize}
\end{itemize}

\demo{qml-graphics-effects/ex-shaders/drop-shadow2.qml}
\end{slide}

%----------------------------------------------------------------------

\begin{slide}{1828}\frametitle{Shaders and Particles}

\begin{itemize}
\item Everything about the particle systems is still valid
\item Allows for less CPU intensive particle rendering

\vspace*{1em}

\item Use \qic{class}{CustomParticle} instead of \qic{class}{ImageParticle}
\item Use the \qic{type}{vertexShader} and \qic{type}{fragmentShader} properties
\end{itemize}

\vspace*{2em}

\demo{qml-graphics-effects/ex-shaders/fountain-shaders.qml}
\end{slide}

%----------------------------------------------------------------------

\begin{slide}{1829}\frametitle{Lab: Dissolve Effect}
\centeredImage{qml-graphics-effects/images/dissolve.png}

Starting from the partial solution:

\begin{itemize}
\item Create an alpha gradient effect
\item Animate it so the item fades out from top to bottom and back in again
\end{itemize}

\lab{qml-graphics-effects/lab-dissolve/dissolve.qml}
\end{slide}

