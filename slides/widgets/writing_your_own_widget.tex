%%%%%%%%%%%%%%%%%%%%%%%%%%%%%%%%%%%%%%%%%%%%%%%%%%%%%%%%%%%%%%%%%%%%%%%%%%
%
% Copyright (c) 2008-2011, Nokia Corporation and/or its subsidiary(-ies).
% All rights reserved.
%
% This work, unless otherwise expressly stated, is licensed under a
% Creative Commons Attribution-ShareAlike 2.5.
%
% The full license document is available from
% http://creativecommons.org/licenses/by-sa/2.5/legalcode .
%
%%%%%%%%%%%%%%%%%%%%%%%%%%%%%%%%%%%%%%%%%%%%%%%%%%%%%%%%%%%%%%%%%%%%%%%%%%

\subsection{Guidelines for Custom Widgets}

%----------------------------------------------------------------------
\begin{slide}[fragile]{1328} \frametitle{Guidelines: Creating a Custom Widget}
  \begin{itemize}
    \item It's as easy as deriving from QWidget
    \item[] \begin{cpp}
class CustomWidget : public QWidget
{
public:
  explicit CustomWidget(QWidget* parent=0);
}
\end{cpp}           
  \item If you need custom signals, slots, or properties
  \begin{itemize}
    \item add Q\_OBJECT
  \end{itemize} 
  \item Use child widget members (composition)
  \item ...or paint the widget yourself (from scratch)
\end{itemize}
\end{slide}
%----------------------------------------------------------------------
\begin{slide}{0860}
  \frametitle{Guidelines: Base class and Event Handlers}
  \label{writing_your_own_widget}
  \begin{itemize}
  \item \textbf{Do not reinvent the wheel}
    \begin{itemize}
    \item \doc{gallery.html}{Widget Gallery}
    \end{itemize}
  \item \textbf{Decide on a base class}
    \begin{itemize}
    \item Often \iCls{QWidget} or \iCls{QFrame}
    \end{itemize}
  \item \textbf{Overload needed event handlers}
    \begin{itemize}
    \item Often:
      \begin{itemize}
      \item \iClsFn{QWidget}{mousePressEvent}, \\ \iClsFn{QWidget}{mouseReleaseEvent}
      \end{itemize}
    \item If widget accepts keyboard input
      \begin{itemize}
      \item \iClsFn{QWidget}{keyPressEvent}
      \end{itemize}
    \item If widget changes appearance on focus
      \begin{itemize}
      \item \iClsFn{QWidget}{focusInEvent}, \\ \iClsFn{QWidget}{focusOutEvent}
      \end{itemize}
  \end{itemize}
  \end{itemize}
\end{slide}
%----------------------------------------------------------------------
\begin{slide}{0861}\frametitle{Guidelines: Drawing a Widget}
\xConcept{Events}
\begin{itemize}
\item \textbf{Decide on composite or draw approach?}
  \begin{itemize}
  \item \textit{If composite}: Use layouts to arrange child widgets
  \item \textit{If draw}: implement the paint event
  \end{itemize}

\item \textbf{Reimplement \iClsFn{QWidget}{paintEvent} for drawing}
  \begin{itemize}
  \item To draw widget's visual appearance
  \item Drawing often depends on internal states
  \end{itemize}
\item \textbf{Decide which signals to emit}
  \begin{itemize}
  \item Usually from within event handlers
  \item Especially \hClsFn{QWidget}{mousePressEvent} or \hClsFn{QWidget}{mouseDoubleClickEvent}
  \end{itemize}
\item \textbf{Decide carefully on types of signal parameters}
  \begin{itemize}
  \item General types increase reusability
  \item Candidates are \texttt{bool}, \texttt{int} and \texttt{const QString\&}
  \end{itemize}
\end{itemize}
\end{slide}
%----------------------------------------------------------------------
\begin{slide}{0862}\frametitle{Guidelines: Internal States and Subclassing}
\xConcept{Interfaces}
\begin{itemize}
  \item \textbf{Decide on publishing internal states}
    \begin{itemize}
    \item Which internal states should be made publically accessible?
    \item Implement accessor methods
    \end{itemize}
 \item \textbf{Decide which setter methods should be slots}
   \begin{itemize}
   \item Candidates are methods with integral or common parameters
   \end{itemize}
  \item \textbf{Decide on allowing subclassing}
    \begin{itemize}
    \item If yes
      \begin{itemize}
      \item Decide which methods to make protected instead of private
      \item Which methods to make virtual
     \end{itemize}
    \end{itemize}
  \end{itemize}
\end{slide}
%----------------------------------------------------------------------
\begin{slide}[fragile]{0863}\frametitle{Guidelines: Widget Constructor}
\begin{itemize}
\item \textbf{Decide on parameters at construction time}
  \begin{itemize}
  \item Enrich the constructor as necessary
  \item Or implement more than one constructor
  \item If a parameter is needed for widget to work correctly
    \begin{itemize}
    \item User should be forced to pass it in the constructor
    \end{itemize}
  \end{itemize}
  \item \textbf{Keep the Qt convention with:}
\begin{lstlisting}
    explicit Constructor(..., QWidget *parent = 0)
\end{lstlisting}
  \end{itemize}
\end{slide}


%----------------------------------------------------------------------
\begin{slide}{1327}\frametitle{Lab: File Chooser}
 \begin{itemize}
  \item Create a reusable file chooser component
  \item 2 Modes
    \begin{itemize}
    \item Choose File
    \item Choose Directory
    \end{itemize}
  \item \textit{Think about the Custom Widget Guidelines!}
  \item \textit{Create a reusable API for a
      \texttt{FileChooser}?}
 \end{itemize}
 \centeredImageDoubleWidth{widgets/images/filechooser} \\
 \lab{widgets/lab-filechooser}
 \begin{itemize}
 \item After lab discuss your API
 \end{itemize}
\end{slide}

%----------------------------------------------------------------------
%\begin{slide}{0864}\frametitle{Lab: Compass Widget}
%  \label{writing_your_own_widget_exercize}
%  \begin{itemize}
%  \item Implement a ``compass widget'' and let user ...
%    \begin{itemize}
%    \item Select a direction
%      \begin{itemize}
%      \item north, west, south, east
%      \item and optionally none
%      \end{itemize}
%    \end{itemize}
%    \item Provide API to ...
%      \begin{itemize}
%      \item change direction programmatically
%      \item get informed when direction changes
%      \end{itemize}
%    \item Optional      
%    \begin{itemize}
%      \item Add direction None
%      \item Select direction with the keyboard
%    \end{itemize}
%  \end{itemize}
%  \flushedImage{widgets/images/compass}   
%  \lab{widgets/lab-compasswidget}
%\end{slide}

% Local Variables: 
% mode: latex 
% TeX-master: "course"
% End: 
