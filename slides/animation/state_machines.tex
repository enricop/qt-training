%%%%%%%%%%%%%%%%%%%%%%%%%%%%%%%%%%%%%%%%%%%%%%%%%%%%%%%%%%%%%%%%%%%%%%%%%%
%
% Copyright (c) 2008-2011, Nokia Corporation and/or its subsidiary(-ies).
% All rights reserved.
%
% This work, unless otherwise expressly stated, is licensed under a
% Creative Commons Attribution-ShareAlike 2.5.
%
% The full license document is available from
% http://creativecommons.org/licenses/by-sa/2.5/legalcode .
%
%%%%%%%%%%%%%%%%%%%%%%%%%%%%%%%%%%%%%%%%%%%%%%%%%%%%%%%%%%%%%%%%%%%%%%%%%%

\subsection{States and Animations}

\begin{slide}{46130}\frametitle{States and Animations}
\vspace*{1.5em}
Qt includes a state machine that can be used with animations:
\begin{itemize}
\item Includes features specifically for animation.
\item It can be used for other purposes.
\item Processing is handled by \iCls{QStateMachine} instances.
\item \iCls{QAbstractState} defines the state API.
\item \iCls{QAbstractTransition} defines the transition API.
\end{itemize}
\end{slide}

%----------------------------------------------------------------------
\begin{slide}{46135}\frametitle{Statecharts}
\vspace*{1.5em}
Each statechart contains all the states the machine can be in.

\begin{itemize}
\item Statecharts always contain an initial state.
\item There may be one or more final states.
\item States are connected by transitions.
\item Transitions define how the current state of the machine can change.
\end{itemize}

The state machine API is used to describe how states are connected.
\end{slide}

%----------------------------------------------------------------------
\begin{slide}{46136}\frametitle{Statecharts}
\textbf{Example:} A button that changes state when clicked.

\vspace*{1.5em}
\centeredImage{animation/diagrams/statemachine-twowaybutton.pdf}
\vspace*{1.5em}

\begin{itemize}
\item The black circle and arrow indicate the initial state.
\item When the button is clicked, there is a transition from one state to
the other.
\item There is no final state.
\end{itemize}
\end{slide}

%----------------------------------------------------------------------
\begin{slide}{46140}\frametitle{States}
\label{Animation-States-States}
\vspace*{1.0em}
The state chart contains all the states the machine can be in.

\begin{itemize}
\item Each state defines values for Qt properties.
\item One state is defined to be the initial state.
\item States can encapsulate other states (composite states).
\item States can be saved and restored (history states).
\item One or more states can be final states, where execution stops.
\item The main state class is \iCls{QState}.
\end{itemize}
\end{slide}

%----------------------------------------------------------------------
\begin{slide}[fragile]{46137}\frametitle{States and Properties}
\label{Animation-States-Properties}
\vspace*{1.0em}
States can be associated with object properties:

\vspace*{1.5em}
\centeredImage{animation/diagrams/statemachine-twowaybutton-properties.pdf}
\vspace*{1.0em}

\begin{itemize}
\item \iClsFn{QState}{assignProperty}
\begin{itemize}
\item Assigns a value to a property when the state is entered.
\item Used to drive a user interface or animation.
\end{itemize}
\end{itemize}
\end{slide}

%----------------------------------------------------------------------
\begin{slide}[fragile]{46138}\frametitle{States and Properties}
\vspace*{1.5em}
\small
% From examples/animation/states/twowaybutton/main.cpp
\begin{lstlisting}
QPushButton *button = new QPushButton();
QStateMachine *machine = new QStateMachine();

QState *offState = new QState();
offState->assignProperty(button, "text", "Off");
QState *onState = new QState();
onState->assignProperty(button, "text", "On");
\end{lstlisting}
\normalsize
\demo{animation/ex-twowaybutton}
\vfill
\end{slide}

%----------------------------------------------------------------------
\begin{slide}{46145}\frametitle{States and Transitions}
\label{Animation-States-Transitions}
Transitions describe how the machine changes from one state to another.

\vspace*{1.5em}
\centeredImage{animation/diagrams/statemachine-twowaybutton-transitions.pdf}
\vspace*{1.0em}

\begin{itemize}
\item \iClsFn{QState}{addTransition}
\begin{itemize}
\item Can define conditions or be unconditional.
\end{itemize}
\item Can apply conditions based on external input:
\begin{itemize}
\item From a signal, such as \iClsSig{QPushButton}{clicked}.
\item From an event, such as \iCls{QKeyEvent}.
\end{itemize}
\item Can also be linked to animations.
\end{itemize}
\end{slide}

%----------------------------------------------------------------------
\begin{slide}[fragile]{46238}\frametitle{States and Transitions}
\vspace*{1.5em}
\small
% From examples/animation/states/twowaybutton/main.cpp
\begin{lstlisting}
offState->addTransition(button, SIGNAL(clicked()), onState);
onState->addTransition(button, SIGNAL(clicked()), offState);

machine->addState(offState);
machine->addState(onState);
machine->setInitialState(offState);
machine->start();
\end{lstlisting}
\normalsize
\demo{animation/ex-twowaybutton}
\vfill
\end{slide}

%----------------------------------------------------------------------
\begin{slide}{46150}\frametitle{Final States}
\vspace*{1.0em}
A final state is a state at which the machine stops execution.

\vspace*{0.5em}
To make a state machine terminate:
\begin{itemize}
\item Define a top-level final state.
\item Make it the target of transitions from other states.
\item More than one final state can be defined.
\end{itemize}

\vspace*{0.5em}
When a top-level final state is entered:
\begin{itemize}
\item The state machine emits the \iClsSig{QStateMachine}{finished} signal.
\item The state machine halts.
\end{itemize}

\end{slide}

%----------------------------------------------------------------------
\begin{slide}{46338}\frametitle{Final States}
\textbf{Example:} Accepting input from the user.

\vspace*{1.25em}
\centeredImage{animation/diagrams/statemachine-initial-final.pdf}
\vspace*{1.5em}

\begin{itemize}
\item In the initial \textbf{questionState}, an input window is opened.
\item When the user enters a name, the window emits the \hClsSig{Window}{nameEntered}
signal.
\item The transition leads to the final state, \textbf{answeredState}, where
the machine emits the \hClsSig{QStateMachine}{finished} signal.
\end{itemize}
\end{slide}

%----------------------------------------------------------------------
\begin{slide}[fragile]{46139}\frametitle{Final States}
\vspace*{-0.5em}
\small
% From examples/animation/states/initial-final/main.cpp
\begin{lstlisting}
Window *window = new Window();
QStateMachine *machine = new QStateMachine();

QState *questionState = new QState();
QFinalState *answeredState = new QFinalState();
questionState->assignProperty(window, "visible", true);
questionState->addTransition(window, SIGNAL(nameEntered(QString)),
                             answeredState);

machine->addState(questionState);
machine->addState(answeredState);
machine->setInitialState(questionState);
QObject::connect(machine, SIGNAL(finished()), app, SLOT(quit()));
machine->start();
\end{lstlisting}
\normalsize
\demo{animation/ex-initial-final}
\end{slide}

%----------------------------------------------------------------------
\begin{slide}{46155}\frametitle{Composite and Child States}
\label{Animation-States-Composite}
\vspace*{1.0em}
Composite states:

\begin{itemize}
\item Are used to group states and transitions.
\item Can be used to encapsulate ``internal'' states.
\begin{itemize}
\item Example: the states of buttons in a dialog.
\end{itemize}
\end{itemize}

States in a group are known as child states.

\begin{itemize}
\item One child state must be made an initial state of its parent.
\item Child states are executed one at a time.
\item Child states inherit transitions from their parents.
\end{itemize}
\end{slide}

%----------------------------------------------------------------------
\begin{slide}{46156}\frametitle{Child States and Transitions}
\centeredImage{animation/diagrams/statemachine-inheritedtransitions.pdf}
\vfill

\begin{itemize}
\item Many states can share the same transition to another state.
\item The parent defines a common transition for all children.
\item When we leave a state, its internal state is not important.
\end{itemize}
\end{slide}

%----------------------------------------------------------------------
\begin{slide}{46157}\frametitle{Child States and Transitions}
\textbf{Example:} A set of traffic lights with a quit button.

\centeredImage{animation/diagrams/statemachine-trafficlights.pdf}

\vspace*{0.5em}
When the user clicks the quit button, it does not matter which child state
the machine was in.
\end{slide}

%----------------------------------------------------------------------
\begin{slide}[fragile]{46158}\frametitle{Child States and Transitions}
\vspace*{1.5em}
\small
% From examples/animation/states/trafficlights/main.cpp
\begin{lstlisting}
QState *runningState = new QState();
QState *stopState = new QState(runningState);
QState *waitState = new QState(runningState);
QState *driveState = new QState(runningState);
QState *slowState = new QState(runningState);
runningState->setInitialState(stopState);
\end{lstlisting}
\normalsize
\demo{animation/ex-trafficlights}
\vfill
\end{slide}

%----------------------------------------------------------------------
\begin{slide}[fragile]{46165}\frametitle{Final Child States}
Child states can contain final states

\vspace*{1.0em}
\centeredImage{animation/diagrams/statemachine-trafficlights-final-childstates.pdf}
\vfill
\small
% From examples/animation/states/trafficlights-final/main.cpp
\begin{lstlisting}
QState *runningState = new QState();
QState *stopState = new QState(runningState);
QState *waitState = new QState(runningState);
QState *driveState = new QState(runningState);
QFinalState *childFinalState = new QFinalState(runningState);
runningState->setInitialState(stopState);
\end{lstlisting}
\normalsize
\end{slide}

%----------------------------------------------------------------------
\begin{slide}{46166}\frametitle{Final Child States}
\centeredImage{animation/diagrams/statemachine-trafficlights-final.pdf}

When a final child state is entered:
\begin{itemize}
\item No more processing occurs in the state.
\item The state emits the \hClsSig{QState}{finished} signal.
\end{itemize}

We can use this signal to start a transition.
\end{slide}

%----------------------------------------------------------------------
\begin{slide}{46170}\frametitle{History States}
\centeredImage{animation/diagrams/statemachine-history.pdf}
\vfill
A history state contains information about a state and its children.

\begin{itemize}
\item It retains information when its parent state is left.
\item This enables execution of a state to be interrupted.
\item The history state is used to restore the state when it is returned to.
\end{itemize}
\end{slide}

%----------------------------------------------------------------------
\begin{slide}{1249}\frametitle{History States}
\centeredImage{animation/diagrams/statemachine-history.pdf}
\vfill
To record and restore a state:

\begin{itemize}
\item Make a history state as a child of the state.
\item Make a transition from the state to another state.
\item Make a transition back to the history state.
\end{itemize}
\end{slide}

%----------------------------------------------------------------------
\begin{slide}{46171}\frametitle{History States}
\textbf{Example:} A set of traffic lights widget with pause and quit buttons.

\vspace*{0.5em}
\centeredImage{animation/diagrams/statemachine-trafficlights-history.pdf}
\vfill
\end{slide}

%----------------------------------------------------------------------
\begin{slide}[fragile]{46172}\frametitle{History States}
\vspace*{-0.5em}
\centeredImage{animation/diagrams/statemachine-trafficlights-history-detail.pdf}
\vspace*{-1.0em}
\small
% From examples/animation/states/trafficlights-history/main.cpp
\begin{lstlisting}
QHistoryState *historyState = new QHistoryState(runningState);
QState *pausedState = new QState();
...
runningState->addTransition(pauseButton, SIGNAL(toggled(bool)),
                            pausedState);
pausedState->addTransition(pauseButton, SIGNAL(toggled(bool)),
                           historyState);
\end{lstlisting}
\normalsize
\smallskip
\demo{animation/ex-trafficlights}
\end{slide}

%----------------------------------------------------------------------
\begin{slide}{46180}\frametitle{Parallel States}
\label{Animation-States-Parallel}
\vspace*{1.5em}
Parallel child states are executed simultaneously.

\begin{itemize}
\item They can be used to represent
\begin{itemize}
\item independent components in a user interface,
\item independent properties of a component,
\item components that run in parallel.
\end{itemize}
\item Passing \iClsEnum{QState}{ParallelStates} to the \iCls{QState}
constructor makes a state parallel.
\item There is no initial child state~\textendash~all are executed
simultaneously.
\end{itemize}
\end{slide}

%----------------------------------------------------------------------
\begin{slide}{46181}\frametitle{Parallel States}
\textbf{Example:} Cursor key input widget.

\vspace*{0.5em}
\centeredImage{animation/diagrams/statemachine-parallel.pdf}

\end{slide}

%----------------------------------------------------------------------
\begin{slide}[fragile]{46182}\frametitle{Parallel States}

\vspace*{-0.5em}
\small
% From examples/animation/states/parallel/window.cpp
\begin{lstlisting}
QState *runningState = new QState(QState::ParallelStates);
QState *horizontalState = new QState(runningState);
QState *verticalState = new QState(runningState);

QState *hcenterState = new QState(horizontalState);
QState *leftState = new QState(horizontalState);
QState *rightState = new QState(horizontalState);
horizontalState->setInitialState(hcenterState);

QState *vcenterState = new QState(verticalState);
QState *upState = new QState(verticalState);
QState *downState = new QState(verticalState);
verticalState->setInitialState(vcenterState);
\end{lstlisting}
\normalsize
\demo{animation/ex-parallel-states}
\vfill
\end{slide}

%----------------------------------------------------------------------
\subsection{Events and Transitions}

\begin{slide}{46185}\frametitle{Events and Transitions}
\label{Animation-States-Events}
\iCls{QSignalTransition} causes a transition to occur when a signal
is emitted.

\begin{itemize}
\item We normally call \hClsFn{QState}{addTransition} with
a sender, signal and target state.
\item Create a subclass to check signal arguments.
\end{itemize}

\iCls{QEventTransition} causes a transition to occur when a Qt event
is received.

\begin{itemize}
\item There is no convenience function for this.
\item Create new instances and set them up by hand.
\item Create a subclass to check event properties.
\end{itemize}
\end{slide}

%----------------------------------------------------------------------

\begin{slide}[fragile]{46188}\frametitle{Key Event Transitions}
The \iCls{QKeyEventTransition} ties a transition to a key press or release:

\vspace*{0.5em}
\small
% From examples/animation/states/parallel-animations/main.cpp
\begin{lstlisting}
QKeyEventTransition *upTransition;
upTransition = new QKeyEventTransition(this,
                   QEvent::KeyPress, Qt::Key_Up);
upTransition->setTargetState(upState);
vcenterState->addTransition(upTransition);

QKeyEventTransition *upCenterTransition;
upCenterTransition = new QKeyEventTransition(this,
                         QEvent::KeyRelease, Qt::Key_Up);
upCenterTransition->setTargetState(vcenterState);
upState->addTransition(upCenterTransition);
\end{lstlisting}
\normalsize

\begin{itemize}
\item Pressing and releasing the up arrow causes transitions between the \texttt{vcenterState}
and \texttt{upState} states.
\end{itemize}
\end{slide}

%----------------------------------------------------------------------
\begin{slide}[fragile]{46190}\frametitle{Animated Transitions}
\label{Animation-Animated-Transitions}
Animated transitions blend the values of properties between states over time.

\begin{itemize}
\item Often used for cosmetic effects.
\item Animations do not block transitions; they run in parallel to them.
\item A default animation can be set for all transitions.
\end{itemize}

\small
% From examples/animation/states/parallel-animations/main.cpp
\begin{lstlisting}
QPropertyAnimation *upAnimation;
upAnimation = new QPropertyAnimation(upButton, "brightness", this);
upAnimation->setDuration(1000);
upTransition->addAnimation(upAnimation);
upCenterTransition->addAnimation(upAnimation);
\end{lstlisting}
\normalsize
\demo{animation/ex-parallel-animations}

\end{slide}

%----------------------------------------------------------------------
\begin{slide}{46199}\frametitle{State Machine Summary}

\vspace*{1.5em}
\begin{itemize}
\item \showHyperlink{Animation-States-States}{States} and
\showHyperlink{Animation-States-Transitions}{transitions} help
manage application logic.
\item \showHyperlink{Animation-States-Properties}{States} can be associated
with property values.
\item States can contain \showHyperlink{Animation-States-Composite}{child states}.
\item Child states can be executed sequentially or in parallel.
\item Transitions can be linked to \showHyperlink{Animation-States-Events}{events}.

\item State machines can be used with animations:
\begin{itemize}
\item Animations can be \showHyperlink{Animation-Animated-Transitions}{applied to transitions}.
\item States define the start and end points of animations.
\end{itemize}
\end{itemize}
\end{slide}

%----------------------------------------------------------------------
\begin{slide}{46220}\frametitle{Hints and Tips}

States and transitions:
\begin{itemize}
\item Ensure that there is an initial state for each composite
state~\textemdash~unless it is a parallel state.
\item Call \hClsFn{QObject}{setObjectName} on each state to set a readable
name for debugging.
\end{itemize}
\end{slide}

%----------------------------------------------------------------------
\begin{slide}[fragile]{46186}\frametitle{Custom Signal Transitions}

Custom signal transitions are created by subclassing \iCls{QSignalTransition}.

\vspace*{-0.5em}
\small
% From examples/animation/states/imageviewer/selecttransition.h
\begin{lstlisting}
class SelectTransition : public QSignalTransition
{
public:
    SelectTransition(QState *sourceState = 0);

protected:
    bool eventTest(QEvent *event);
};
\end{lstlisting}
\normalsize
\vfill
We check events by reimplementing the \hClsFn{QSignalTransition}{eventTest} function.
\end{slide}

%----------------------------------------------------------------------
\begin{slide}[fragile]{46187}\frametitle{Custom Signal Transitions}
Here, we make sure we only look at signal events.

\vspace*{-0.5em}
\small
% From examples/animation/states/imageviewer/selecttransition.cpp
\begin{lstlisting}
bool SelectTransition::eventTest(QEvent *event)
{
    if (!QSignalTransition::eventTest(event))
        return false;

    QStateMachine::SignalEvent *e = static_cast<QStateMachine::SignalEvent*>(event);

    QVariant argument = e->arguments()[0];
    QModelIndex index = qvariant_cast<QModelIndex>(argument);
    return index.isValid();
}
\end{lstlisting}
\normalsize

In this example, the transition is only allowed if the model index is valid.
\end{slide}
