%%%%%%%%%%%%%%%%%%%%%%%%%%%%%%%%%%%%%%%%%%%%%%%%%%%%%%%%%%%%%%%%%%%%%%%%%%
%
% Copyright (c) 2008-2011, Nokia Corporation and/or its subsidiary(-ies).
% All rights reserved.
%
% This work, unless otherwise expressly stated, is licensed under a
% Creative Commons Attribution-ShareAlike 2.5.
%
% The full license document is available from
% http://creativecommons.org/licenses/by-sa/2.5/legalcode .
%
%%%%%%%%%%%%%%%%%%%%%%%%%%%%%%%%%%%%%%%%%%%%%%%%%%%%%%%%%%%%%%%%%%%%%%%%%%

\subsection{Starting Animation}

%----------------------------------------------------------------------
\begin{slide}{46100}\frametitle{Overview}
\label{animations}
\xConcept{Animation}
\vspace*{1.5em}
Qt 4.6 and newer contains an animation API:
\begin{itemize}
\item Animations based on changing Qt properties.
\item Can be used with widgets and with Graphics View.
\item Easing curves for controlling the rate of animation.
\item Groups of animations, performed in series or parallel.
\item State-based animations.
\end{itemize}
\end{slide}

%----------------------------------------------------------------------

\begin{slide}{46101}\frametitle{Architecture}
\vspace*{1.5em}
\begin{itemize}
\item All animation classes are derived from \iCls{QAbstractAnimation}.
\item \iCls{QPropertyAnimation} class changes Qt properties to create animations.
\item \iCls{QAnimationGroup} and its subclasses group animations together.
\end{itemize}

\iCls{QPropertyAnimation} helps fit the animation framework into Qt:
\begin{itemize}
\item All \iCls{QObject}-based classes can be animated.
\item Graphics items with properties can be animated.
\end{itemize}

\end{slide}

%----------------------------------------------------------------------

\begin{slide}[fragile]{46102}\frametitle{Property Animations}
Animations can be created with the \iCls{QPropertyAnimation} class:
\small
% From examples/animation/widget/main.cpp
\begin{lstlisting}
QPushButton button("Animated Button");
button.show();

QPropertyAnimation animation(&button, "geometry");
animation.setDuration(10000);
animation.setStartValue(QRect(0, 0, 100, 30));
animation.setEndValue(QRect(250, 250, 100, 30));
\end{lstlisting}
\normalsize

\begin{itemize}
\item The animation interpolates between the start and end property values
(like \iCls{QTimeLine}).
\item By default, a linear interpolation between start and end values is
performed.
\end{itemize}

\end{slide}

%----------------------------------------------------------------------
\begin{slide}[fragile]{46103}\frametitle{Property Animations}
\vspace*{1.0em}
\textbf{Example:} The button moves smoothly from the start point to the end point.

\vspace*{1.5em}
\centeredImage{animation/diagrams/animation-widget.pdf}

\small
% From examples/animation/widgets-widget-figure/main.cpp
\begin{lstlisting}
    QPropertyAnimation animation(&button, "geometry");
    animation.setDuration(10000);
    animation.setStartValue(QRect(0, 0, 64, 30));
    animation.setEndValue(QRect(850, 0, 64, 30));
\end{lstlisting}
\normalsize
\demo{animation/ex-widget}

\end{slide}

%----------------------------------------------------------------------
\label{Animation-Key-Values}

\begin{slide}[fragile]{46104}\frametitle{Key Values}
Key values set the value of a property at different times in an animation.
\small
% From examples/animation/widget-keyvalues/main.cpp
\begin{lstlisting}
QPropertyAnimation animation(&button, "geometry");
animation.setDuration(10000);

animation.setKeyValueAt(0, QRect(0, 0, 100, 30));
animation.setKeyValueAt(0.8, QRect(250, 250, 100, 30));
animation.setKeyValueAt(1, QRect(0, 0, 100, 30));
\end{lstlisting}
\normalsize
\begin{itemize}
\item Property values are interpolated between key values.
\item A value of 0 corresponds to the start value.
\item A value of 1 corresponds to the end value.
\item They can be used to break an animation into stages...
\item ...or make the property change at a non-constant rate.
\end{itemize}

\end{slide}

%----------------------------------------------------------------------
\begin{slide}[fragile]{46105}\frametitle{Key Values}
\textbf{Example:} The label takes 80\% of the time to travel half the total
distance.

\vspace*{1.5em}
\centeredImage{animation/diagrams/animation-keyvalues.pdf}

\small
% From examples/animation/widgets-keyvalues-figure/main.cpp
\begin{lstlisting}
    QPropertyAnimation animation(label, "geometry");
    animation.setDuration(10000);
    animation.setKeyValueAt(0, QRect(0, 0, 64, 64));
    animation.setKeyValueAt(0.8, QRect(425, 0, 64, 64));
    animation.setKeyValueAt(1, QRect(850, 0, 64, 64));
\end{lstlisting}
\normalsize
\demo{animation/ex-widget-keyvalues-figures}
\end{slide}

%----------------------------------------------------------------------
\label{Animation-Easing-Curves}

\begin{slide}[fragile]{46106}\frametitle{Easing Curves}
Easing curves are used to create animations that can change speed and direction.

\centeredImage{animation/diagrams/animation-easing-bounce.pdf}
\vspace*{1.5em}

\begin{itemize}
\item Each curve transforms the animation parameter to another
value between 0 and 1.
\item Some curves change the speed of the animation: \texttt{InOutExpo}
\item Other curves change the direction: \texttt{OutBounce}
\end{itemize}

\end{slide}

%----------------------------------------------------------------------
\begin{slide}[fragile]{46107}\frametitle{Easing Curves}
\textbf{Example:} Using an \texttt{OutQuad} curve makes the start of the
animation run quickly and get slower towards the end.

\vspace*{1.5em}
\centeredImage{animation/diagrams/animation-easing.pdf}

\small
% From examples/animation/easingcurves-figure/main.cpp
\begin{lstlisting}
QPropertyAnimation animation(label, "geometry");
animation.setDuration(3000);
animation.setStartValue(QRect(0, 0, 64, 64));
animation.setEndValue(QRect(850, 0, 64, 64));
animation.setEasingCurve(QEasingCurve::OutQuad);
\end{lstlisting}
\normalsize
\demo{animation/ex-easingcurves-figure}
\end{slide}

%----------------------------------------------------------------------

\begin{slide}{46108}\frametitle{Looping, Changing Direction}
\vspace*{0.5em}
Animations can be looped:

\begin{itemize}
\item \hClsFn{QAbstractAnimation}{setLoopCount}
\begin{itemize}
\item Set the number of times an animation is played (default 1).
\end{itemize}
\item \hClsSig{QAbstractAnimation}{currentLoopChanged}
\begin{itemize}
\item Emitted with the current iteration when a new loop begins.
\end{itemize}
\end{itemize}

The playing direction can be changed:

\begin{itemize}
\item \hClsFn{QAbstractAnimation}{setDirection}
\begin{itemize}
\item Sets the animation direction.
\item Either \hClsEnum{QAbstractAnimation}{Forward} or \hClsEnum{QAbstractAnimation}{Backward}.
\end{itemize}
\item \hClsSig{QAbstractAnimation}{directionChanged}
\begin{itemize}
\item Emitted with the new direction when it changes.
\end{itemize}
\end{itemize}

\end{slide}

%----------------------------------------------------------------------

\begin{slide}{46109}\frametitle{Starting and Pausing}
\vspace*{1.5em}

Animations can be controlled using slots:

\begin{itemize}
\item \hClsFn{QAbstractAnimation}{start}
\begin{itemize}
\item Starts the animation from the beginning.
\end{itemize}
\item \hClsFn{QAbstractAnimation}{stop}
\begin{itemize}
\item Stops the animation completely.
\end{itemize}
\item \hClsFn{QAbstractAnimation}{pause}
\begin{itemize}
\item Pauses the animation, allowing it to continue later.
\end{itemize}
\item \hClsFn{QAbstractAnimation}{resume}
\begin{itemize}
\item Continues playing the animation from where it was paused.
\end{itemize}
\end{itemize}
\end{slide}

%----------------------------------------------------------------------

\begin{slide}{46110}\frametitle{Monitoring}
\vspace*{1.5em}

In general:
\begin{itemize}
\item \hClsSig{QAbstractAnimation}{stateChanged}
\begin{itemize}
\item Emitted with an animation's state when it changes.
\item The state is \hClsEnum{QAbstractAnimation}{Stopped},
\hClsEnum{QAbstractAnimation}{Paused} or \hClsEnum{QAbstractAnimation}{Running}.
\end{itemize}
\item \hClsSig{QAbstractAnimation}{finished}
\begin{itemize}
\item Emitted when the animation has finished playing.
\end{itemize}
\end{itemize}

For \iCls{QVariantAnimation} and \iCls{QPropertyAnimation}:
\begin{itemize}
\item \hClsSig{QVariantAnimation}{valueChanged}
\begin{itemize}
\item Emitted with the current value of the property.
\end{itemize}
\end{itemize}

\end{slide}

%----------------------------------------------------------------------

\begin{slide}[fragile]{46111}\frametitle{Animating Graphics Items}
\label{Animation-Graphics-Items}

Most \iCls{QGraphicsItem}s are not \iCls{QObject} subclasses.
\begin{itemize}
\item As a result, they do not have Qt properties.
\item Subclass \iCls{QGraphicsObject} to create new items.
\item Or subclass from \iCls{QObject} and another item class.
\end{itemize}
\small
% From examples/animation/graphicsitem/graphicspixmapitem.h
\begin{lstlisting}
class GraphicsPixmapItem : public QObject,
                           public QGraphicsPixmapItem
{
    Q_OBJECT
    Q_PROPERTY(QPointF pos READ pos WRITE setPos)

public:
    GraphicsPixmapItem(const QPixmap &pixmap,
                       QGraphicsItem *parent = 0);
};
\end{lstlisting}
\normalsize
\demo{animation/ex-graphicsitem}
\end{slide}

%----------------------------------------------------------------------
\subsection{Animation Groups}
\begin{slide}{46120}\frametitle{Animation Groups}
Animations can be placed in groups.

\begin{itemize}
\item Sequential groups run multiple animations one after the other.
\end{itemize}
\smallskip
\centeredImage{animation/diagrams/animation-sequential.pdf}
\smallskip
\begin{itemize}
\item Parallel groups run multiple animations at the same time.
\end{itemize}
\smallskip
\centeredImage{animation/diagrams/animation-parallel.pdf}
\smallskip
\begin{itemize}
\item Groups help you to organize animations.
\end{itemize}
\end{slide}

%----------------------------------------------------------------------

\begin{slide}[fragile]{46122}\frametitle{Sequential Animations}
\label{Animation-Sequential-Animations}

\small
% From examples/animation/sequential/main.cpp
\begin{lstlisting}
QPropertyAnimation *anim1 =
        new QPropertyAnimation(label1, "pos");
QPropertyAnimation *anim2 =
        new QPropertyAnimation(label2, "pos");
\end{lstlisting}
\smallskip
\centeredImage{animation/diagrams/animation-sequential.pdf}
\smallskip
\begin{lstlisting}
QSequentialAnimationGroup group;
group.addAnimation(anim1);
group.addAnimation(anim2);
group.start();
\end{lstlisting}
\normalsize
\medskip
\demo{animation/ex-sequential}
\end{slide}

%----------------------------------------------------------------------

\begin{slide}[fragile]{46124}\frametitle{Parallel Animations}
\label{Animation-Parallel-Animations}

\centeredImage{animation/diagrams/animation-parallel.pdf}
\vfill
\small
% From examples/animation/parallel/main.cpp
\begin{lstlisting}
QParallelAnimationGroup group;
group.addAnimation(anim1);
group.addAnimation(anim2);
group.start();
\end{lstlisting}
\normalsize
\vfill
\demo{animation/ex-parallel}
\end{slide}

%----------------------------------------------------------------------

\begin{slide}{46200}\frametitle{Animation Summary}

\vspace*{1.5em}
\begin{itemize}
\item Animations change the properties of \iCls{QObject}s over time.
\item Use \showHyperlink{Animation-Key-Values}{key values} to create variable
speed animations.
\item Use \showHyperlink{Animation-Easing-Curves}{easing curves} to implement
smooth flowing animations.
\item Groups of animations can be played
\showHyperlink{Animation-Sequential-Animations}{in sequence}
or \showHyperlink{Animation-Parallel-Animations}{in parallel}.
\item Animations can be use to \showHyperlink{Animation-Graphics-Items}{animate graphics items}.
\end{itemize}
\end{slide}

%----------------------------------------------------------------------
