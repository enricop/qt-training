%%%%%%%%%%%%%%%%%%%%%%%%%%%%%%%%%%%%%%%%%%%%%%%%%%%%%%%%%%%%%%%%%%%%%%%%%%
%
% Copyright (c) 2008-2011, Nokia Corporation and/or its subsidiary(-ies).
% All rights reserved.
%
% This work, unless otherwise expressly stated, is licensed under a
% Creative Commons Attribution-ShareAlike 2.5.
%
% The full license document is available from
% http://creativecommons.org/licenses/by-sa/2.5/legalcode .
%
%%%%%%%%%%%%%%%%%%%%%%%%%%%%%%%%%%%%%%%%%%%%%%%%%%%%%%%%%%%%%%%%%%%%%%%%%%

\subsection{Object Communication using Signals \& Slots}

%----------------------------------------------------------------------
\begin{slide}[fragile]{0795}\frametitle{Callbacks}
  \begin{block}{General Problem}
    How do you get from "the user clicks a button" to your business logic?
  \end{block}
  \begin{itemize}
 \item Possible solutions
    \begin{itemize}
    \item Callbacks
      \begin{itemize}
      \item Based on function pointers
      \item Not type-safe
     \end{itemize}
    \item Observer Pattern (Listener)
      \begin{itemize}
      \item Based on interface classes
      \item Needs listener registration
      \item Many interface classes
      \end{itemize}
    \end{itemize}
  \item Qt uses
    \begin{itemize}
    \item Signals and slots for high-level (semantic) callbacks
    \item Virtual methods for low-level (syntactic) events.
    \end{itemize}
  \end{itemize}
\end{slide}

%----------------------------------------------------------------------

\begin{slide}[fragile]{0797}\frametitle{Connecting Signals to Slots}
\begin{overprint}
  \onslide<1| handout:0> \centeredImageFullHeight{object-communication/images/signal-slot-cpp-step1}
  \onslide<2| handout:0> \centeredImageFullHeight{object-communication/images/signal-slot-cpp-step2}
  \onslide<3| handout:0> \centeredImageFullHeight{object-communication/images/signal-slot-cpp-step3}
  \onslide<4| handout:0> \centeredImageFullHeight{object-communication/images/signal-slot-cpp-step4}
  \onslide<5| handout:0> \centeredImageFullHeight{object-communication/images/signal-slot-cpp-step5}
  \onslide<6| handout:0> \centeredImageFullHeight{object-communication/images/signal-slot-cpp-step6}
  \onslide<7| handout:0> \centeredImageFullHeight{object-communication/images/signal-slot-cpp-step7}
  \onslide<8| handout:1> \centeredImageFullHeight{object-communication/images/signal-slot-cpp-all} \\ \demo{object-communication/ex-connect}

\end{overprint}
\end{slide}

%----------------------------------------------------------------------
\begin{slide}[fragile]{1032}\frametitle{Connection variants}
  \begin{itemize}
  \item Qt 4 style:\\
\begin{cpp}
connect( slider, SIGNAL(valueChanged(int)),
         spinbox, SLOT(setValue(int)));
\end{cpp}\bigskip

\item Using function pointers:\\
\begin{cpp}
connect( slider, &QSlider::valueChanged,
         spinbox, &QSpinBox::setValue );
\end{cpp}\bigskip

\item Using non-member function:\\
\begin{cpp}
static void printValue(int value) {...}
connect( slider, &QSignal::valueChanged, &printValue );
\end{cpp}\bigskip

\item Using C++11 lambda functions:\\
\begin{cpp}
connect( slider, &QSlider::valueChanged,
         [=] (int value) {...} );
\end{cpp}
  \end{itemize}
\end{slide}

\newcommand{\advantage}{\hspace{-5mm}\makebox[6mm]{\correct}}
\newcommand{\disadvantage}{\hspace{-5mm}\makebox[6mm]{\incorrect}}
%----------------------------------------------------------------------
\begin{slide}[fragile]{1030}\frametitle{connect - function pointers}
  \begin{itemize}
\item Qt 5 components
\begin{cpp}
connect( slider, &QSlider::valueChanged,
         spinbox, &QSpinBox::setValue );
\end{cpp}\medskip
\item Primary choice when connecting objects\medskip
\item[]\advantage Compile type errors
\item[]\advantage No special syntax for slots
\item[]\advantage \texttt{Q\_OBJECT} not need for slots
\item[]\disadvantage connecting to overloaded slots is hard
  \end{itemize}
\demo{object-communication/ex-connect-function-pointers}
\end{slide}


%----------------------------------------------------------------------
\begin{slide}[fragile]{1033}\frametitle{connect - Qt4 style}
  \begin{itemize}
  \item Qt 4 ported code:\\
\begin{cpp}
connect( slider, SIGNAL(valueChanged(int)),
         spinbox, SLOT(setValue(int)));
\end{cpp}\medskip
\hspace{-3mm}Receiving object:
\item[]\disadvantage need to declare the slot in a \textit{slots}
  section
\item[]\disadvantage need the Q\_OBJECT macro
\item[]\disadvantage need to have \texttt{moc} run on it\bigskip

\item[]\disadvantage Only run time errors
\item[]\advantage overloaded slot are easy
\item[]\advantage Existing Qt4 code do not need to be rewritten
\end{itemize}
\demo{object-communication/ex-connect}\bigskip\bigskip
\end{slide}


%----------------------------------------------------------------------
\begin{slide}[fragile]{0798}\frametitle{Custom Slots - Qt4 style}
 \label{codeForSlots1}
\xConcept{connect}
\xConcept{Q\_OBJECT}
\begin{itemize}
\item File: \textbf{myclass.h}
  \begin{cpp}
class MyClass : public QObject
{
  Q_OBJECT // marker for moc
  // ...
public slots:
  void setValue(int value); // a custom slot
};
  \end{cpp}\medskip
\item File: \textbf{myclass.cpp}
  \begin{cpp}
void MyClass::setValue(int value) {
  // slot implementation
}
  \end{cpp}
\end{itemize}
\demo{object-communication/ex-stop-watch}
\end{slide}

%----------------------------------------------------------------------
\begin{slide}[fragile]{1031}\frametitle{connect - non-member }
  \begin{itemize}
\item Using non-member functions:\\
\begin{cpp}
static void printValue(int value) {
  qDebug( "value = %d", value );
}

connect( slider, &QSignal::valueChanged, &printValue );
\end{cpp}\medskip

\item[] \advantage No slots syntax, no Q\_OBJECT, no moc
\item[] \advantage Compile time errors
\item[] \advantage Any function, e.g.\ the return value of std::bind
\end{itemize}
\demo{object-communication/ex-connect-non-member}\bigskip\bigskip
\end{slide}

%----------------------------------------------------------------------
\begin{slide}[fragile]{1031}\frametitle{connect - lambda functions}
  \begin{itemize}
\item Using C++11 lambda functions:\\
\begin{cpp}
connect( slider, &QSlider::valueChanged,
         [=] (int value) { qDebug("%d", value); } );
\end{cpp}\medskip
\item[] \advantage No slots syntax, no Q\_OBJECT, no moc
\item[] \advantage Type safe
\item[] \advantage No need for an extra function
  \end{itemize}
\demo{object-communication/ex-connect-lambda}
\end{slide}

%----------------------------------------------------------------------

\begin{slide}[fragile]{07981}\frametitle{Custom Signals}
\label{codeForSignals1}
\xConcept{Q\_OBJECT}
\begin{itemize}
\item File: \textbf{myclass.h}
  \begin{cpp}
class MyClass : public QObject
{
  Q_OBJECT // marker for moc
  // ...
signals:
  void valueChanged(int value); // a custom signal
};
  \end{cpp}
\item File: \textbf{myclass.cpp}
  \begin{cpp}
// No implementation for a signal

  \end{cpp}
\item Sending a signal
\begin{cpp}
emit valueChanged(value);
\end{cpp}
\end{itemize}

\demo{object-communication/ex-quotebutton}

\end{slide}


%----------------------------------------------------------------------


\begin{slide}[fragile]{0799}\frametitle{\texttt{Q\_OBJECT} - flag for MOC}
\begin{itemize}
\item \textbf{\texttt{Q\_OBJECT}}
  \begin{itemize}
  \item Enhances QObject with meta-object information
  \item Required for signals
  \item Required for slots when using the Qt4 way
  \end{itemize}\medskip

  \item \textbf{moc} creates meta-object information
    \begin{itemize}
    \item[] \begin{shell}
moc -o moc_myclass.cpp myclass.h
c++ -c myclass.cpp; c++ -c moc_myclass.cpp
c++ -o myapp moc_myclass.o myclass.o
    \end{shell}
 \end{itemize}\medskip

 \item \textbf{qmake} takes care of mocing files for you
\end{itemize}

\end{slide}

%----------------------------------------------------------------------
\begin{slide}[fragile]{0803}\frametitle{Variations of Signal/Slot
    Connections}
  \begin{center}
 \begin{tabular}{|c|c|c|}
\hline
\textbf{Signal(s)}                & \textbf{Connect to}   & \textbf{Slot(s)} \\\hline
\hline
one & \correct & many \\\hline
many & \correct & one \\\hline
one & \correct & another signal \\\hline
  \end{tabular}
\end{center}
\begin{itemize}\medskip

\item Signal to Signal connection
 \item[] \begin{cpp}
connect(bt, SIGNAL(clicked()), this, SIGNAL(okSignal()));
  \end{cpp}\medskip

\item \textbf{Not} allowed to name parameters
\begin{tabbing}
connect( \=m\_slider, \=SIGNAL( valueChanged( int \incorrecttext{value} ) )\\
         \>this,      \>SLOT( setValue( int \incorrecttext{newValue} ) ) )
  \end{tabbing}
\end{itemize}
\end{slide}

%----------------------------------------------------------------------
\begin{slide}{1957}
  \frametitle{Making the Connection}
  \begin{block}{Rule for Signal/Slot Connection}
    Can ignore arguments, but not create values from nothing
  \end{block}
  \begin{center}

 \begin{tabular}{r|c|l}
\textbf{Signal}                &    & \textbf{Slot} \\
\hline
\multirow{3}{*}{rangeChanged(int,int)} & \correct & setRange(int,int) \\
 & \correct & setValue(int) \\
 & \correct & update() \\
\hline

\multirow{3}{*}{valueChanged(int)}     & \correct & setValue(int) \\
 & \correct & update() \\
 & \incorrect & setRange(int,int) \\
 & \correct & setValue(float)$^\dagger$ \\
\hline

textChanged(QString) & \incorrect & setValue(int) \\
\hline

\multirow{2}{*}{clicked()}  & \correct & update() \\
& \incorrect & setValue(int) \\

  \end{tabular}
  \end{center}
{\hfill\tiny$\dagger$ Though not for Qt4 connection types}
\end{slide}

%----------------------------------------------------------------------
\begin{slide}[fragile]{0807}
\frametitle{Lab: Connect to Click}
\label{project_signal_slot}
\xProject{Color Tester}
\flushedImageDoubleWidth{object-communication/images/colorTester}
\begin{itemize}
\item \textbf{Create an application as shown here}
  \begin{itemize}
  \item Clicking on ``Select Color'' \\
    updates label with color's name.
  \end{itemize}
\item \textbf{Hints}
  \begin{itemize}
  \item \iClsFn{QColorDialog}{getColor} to fetch a color
  \item \iClsFn{QColor}{name} to get the color name
 \end{itemize}
\item \textbf{Optional}
  \begin{itemize}
  \item In \iCls{QColorDialog}, honor the user clicking
    ``cancel'', and provide it with the current color to start from.
  \item Set the selected color as the label's background
    \begin{itemize}
    \item Hint: see \iCls{QPalette}
    \item Hint: see \iClsFn{QWidget}{setAutoFillBackground}
    \end{itemize}

  \end{itemize}
\end{itemize}
\vfill
\lab{object-communication/lab-selectcolor}
\end{slide}

%----------------------------------------------------------------------
\begin{slide}{0808}
\frametitle{Lab: Source Compatibility}
\label{project_slider_with_indicator}
\flushedImageDoubleWidth{object-communication/images/SliderWithValue}
 \begin{itemize}
  \item \textbf{Implement custom slider}
    \begin{itemize}
    \item API compatible with \iCls{QSlider}
    \item Shows current value of slider
    \end{itemize}
 \item \textbf{To create custom slider}
    \begin{itemize}
    \item use \iCls{QSlider} and \iCls{QLabel}
    \end{itemize}
  \item \textbf{To test slider}
    \begin{itemize}
    \item \texttt{main.cpp} provides test code
    \item \iCls{QLCDNumber} is part of test code
    \end{itemize}
  \item \textbf{Optional:}
    \begin{itemize}
    \item  Discuss pros and cons of inheriting from
      QSlider instead of using an instance in a layout.
    \end{itemize}
  \end{itemize}
  \vfill
  \lab{object-communication/lab-slider}

\end{slide}
