%%%%%%%%%%%%%%%%%%%%%%%%%%%%%%%%%%%%%%%%%%%%%%%%%%%%%%%%%%%%%%%%%%%%%%%%%%
%
% Copyright (c) 2008-2011, Nokia Corporation and/or its subsidiary(-ies).
% All rights reserved.
%
% This work, unless otherwise expressly stated, is licensed under a
% Creative Commons Attribution-ShareAlike 2.5.
%
% The full license document is available from
% http://creativecommons.org/licenses/by-sa/2.5/legalcode .
%
%%%%%%%%%%%%%%%%%%%%%%%%%%%%%%%%%%%%%%%%%%%%%%%%%%%%%%%%%%%%%%%%%%%%%%%%%%

\subsection{QSignalMapper}

%----------------------------------------------------------------------
\begin{slide}[fragile]{0480}\frametitle{QSignalMapper}
\label{qsignalmapper}
\begin{itemize}
  \item Beginners often try to do the following:
\begin{verbatim}
connect( button, SIGNAL( clicked() ), 
         this, SLOT( action(7) ) );
\end{verbatim}
  \item THIS IS WRONG! \iConcept{connect} is only a tool for
    \textit{connecting}, it is not possible to pass arguments to the slot
    when connecting.
  \item The above can however be obtained using the class
    \iCls{QSignalMapper}.
  \end{itemize}
\end{slide}
%----------------------------------------------------------------------
\begin{slide}[fragile]{0481}\frametitle{QSignalMapper}
\small
\begin{alltt}
\iCls{QPushButton}* but1 = new QPushButton("Button 1", this);
QPushButton* but2 = new QPushButton("Button 2", this);
\iCls{QSignalMapper}* mapper = new QSignalMapper( this );
mapper->setMapping( but1, 1 );
mapper->setMapping( but2, 2 );
connect(but1, SIGNAL(clicked()), mapper, SLOT(map()));
connect(but2, SIGNAL(clicked()), mapper, SLOT(map()));
connect(mapper, SIGNAL(mapped(int)), this, SLOT(action(int)));
\end{alltt}
\end{slide}
