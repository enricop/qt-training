%%%%%%%%%%%%%%%%%%%%%%%%%%%%%%%%%%%%%%%%%%%%%%%%%%%%%%%%%%%%%%%%%%%%%%%%%%
%
% Copyright (c) 2008-2011, Nokia Corporation and/or its subsidiary(-ies).
% All rights reserved.
%
% This work, unless otherwise expressly stated, is licensed under a
% Creative Commons Attribution-ShareAlike 2.5.
%
% The full license document is available from
% http://creativecommons.org/licenses/by-sa/2.5/legalcode .
%
%%%%%%%%%%%%%%%%%%%%%%%%%%%%%%%%%%%%%%%%%%%%%%%%%%%%%%%%%%%%%%%%%%%%%%%%%%

\subsection{Hello World using Qt Creator}

%----------------------------------------------------------------------
\begin{slide}{1014}\frametitle{QtCreator IDE} \label{qtcreator}
\xConcept{Qt Creator}
\begin{itemize}
  \item Advanced C++ code editor
  \item Integrated GUI layout and form designer
  \item Project and build management tools
  \item Integrated, context-sensitive help system
  \item Visual debugger
  \item Rapid code navigation tools
  \flushedImage{fundamentals/images/qtcreator_screenshots}
  \item Supports multiple platforms
  \end{itemize}
\end{slide}


%----------------------------------------------------------------------
\begin{slide}{1606}
  \frametitle{Overview of Qt Creator Components}
\centeredImageFullWidth{fundamentals/images/qtcreator-breakdown}
\doccreator{creator-quick-tour.html}{Creator Quick Tour}
\end{slide}

%----------------------------------------------------------------------
\begin{slide}{1606}
  \frametitle{Creating a New Project}
 \begin{itemize}
 \item Qt Creator will provide you with minimal boilerplate to get started
 \item Many different project types to choose from
 \end{itemize}
\centeredImageFullWidth{fundamentals/images/qtcreator-new-qt-quick-project-wizard}
\doccreator{creator-project-creating.html}{Creating Projects}
\end{slide}

%----------------------------------------------------------------------
\begin{slide}{1605}
  \frametitle{Finding Code -Locator}
 \begin{itemize}
  \item Click on Locator or press Ctrl+K (Mac OS X: Cmd+K)
 \item Type in the file name
 \item Press Return
 \end{itemize}
 \centeredImage{fundamentals/images/qtcreator-locator}
 \newline
 Locator Prefixes
 \begin{itemize}
 \item \textbf{: <class name>} - Go to a symbol definition
 \item \textbf{l <line number> } - Go to a line in the current document
 \item \textbf{? <help topic>} - Go to a help topic
 \item \textbf{o <open document>} - Go to an opened document
 \end{itemize}
\end{slide}

%----------------------------------------------------------------------
\begin{slide}{1604}
  \frametitle{Debugging an Application}
  \begin{itemize}
  \item Debug $->$ Start Debugging (or \texttt{F5})
  \end{itemize}
  \centeredImage{fundamentals/images/qtcreator-debugging}
  \doccreator{creator-debugging.html}{Qt Creator and Debugging}
\end{slide}

\begin{slide}{1603}
  \frametitle{Qt Creator Demo "Hello World"}
  What we'll show:
  \begin{itemize}
  \flushedImage{fundamentals/images/qtcreator-helloworld}
  \item Creation of an empty Qt project
  \item Adding the \texttt{main.cpp} source file
  \item Writing of the Qt Hello World Code
    \begin{itemize}
    \item Showing Locator Features
    \end{itemize}
  \item Running the application
  \item Debugging the application
    \begin{itemize}
    \item Looking up the \texttt{text} property of our label
    \end{itemize}
  \end{itemize}
  \smallskip
  \begin{itemize}
  \item There is more to Qt Creator
    \begin{itemize}
    \item[] \doccreator{index.html}{Qt Creator Manual}
    \end{itemize}
  \end{itemize}
\end{slide}
