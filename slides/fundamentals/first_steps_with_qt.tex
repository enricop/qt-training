%%%%%%%%%%%%%%%%%%%%%%%%%%%%%%%%%%%%%%%%%%%%%%%%%%%%%%%%%%%%%%%%%%%%%%%%%%
%
% Copyright (c) 2008-2011, Nokia Corporation and/or its subsidiary(-ies).
% All rights reserved.
%
% This work, unless otherwise expressly stated, is licensed under a
% Creative Commons Attribution-ShareAlike 2.5.
%
% The full license document is available from
% http://creativecommons.org/licenses/by-sa/2.5/legalcode .
%
%%%%%%%%%%%%%%%%%%%%%%%%%%%%%%%%%%%%%%%%%%%%%%%%%%%%%%%%%%%%%%%%%%%%%%%%%%

\subsection{Developing a Hello World Application}

%----------------------------------------------------------------------
\begin{slide}[fragile]{1227}
  \frametitle{``Hello World'' in Qt}
  \flushedImageDoubleWidth{fundamentals/images/helloWorld}
  \medskip
  \begin{itemize}
  \item [] \begin{cpp}
// main.cpp
#include <QApplication>
#include <QPushButton>

int main(int argc, char *argv[])
{
  QApplication app(argc, argv);
  QPushButton button("Hello world");
  button.show();
  return app.exec();
}
  \end{cpp}
 \item[]
 \item Program consists of
    \begin{itemize}
    \item \texttt{main.cpp} - application code
    \item \texttt{helloworld.pro} - project file
    \end{itemize}
  \end{itemize}
\demo{fundamentals/ex-helloworld}
\end{slide}

%----------------------------------------------------------------------
\begin{slide}[fragile]{1226}
\frametitle{Project File - \texttt{helloworld.pro}}
\begin{itemize}
\item \texttt{helloworld.pro} file
  \begin{itemize}
  \item lists source and header files
  \item provides project configuration
 \end{itemize}
\begin{qmake}
# File: helloworld.pro
greaterThan(QT_MAJOR_VERSION, 4): QT += widgets
SOURCES  = main.cpp
HEADERS +=          # No headers used
\end{qmake}
\item Assignment to variables
  \begin{itemize}
  \item Possible operators \texttt{=, +=, -=}
  \end{itemize}
\end{itemize}
\doc{qmake-tutorial.html}{qmake tutorial}
\end{slide}

% ----------------------------------------------------------------------
\begin{slide}[fragile]{1225}
  \frametitle{Using qmake}
  \begin{itemize}
  \item \texttt{qmake} tool
    \begin{itemize}
    \item Creates cross-platform make-files
    \end{itemize}
  \item Build project using qmake
  \item[] \begin{shell}
cd helloworld
qmake helloworld.pro # creates Makefile
make                 # compiles and links application
./helloworld         # executes application
  \end{shell}
\item Tip: \texttt{qmake -project}
  \begin{itemize}
  \item Creates default project file based on directory content
  \end{itemize}
\end{itemize}
\doc{qmake-manual.html}{qmake Manual}
\begin{center}\emph{Qt Creator does it all for you}\end{center}
\end{slide}

