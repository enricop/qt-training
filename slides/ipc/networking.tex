\subsection{Network Programming}

%----------------------------------------------------------------------
\begin{slide}{1744}
\frametitle{Network Programming Overview}

\begin{itemize}

\item \textbf{High level access:} Network operations to handle data.
\item \textbf{Bearer Management:} Network interfaces, online/offline status changes, roaming
\item \textbf{Sockets:} Low level network programming.
\item \textbf{Secure communication:} Using SSL and certificates.
\item \textbf{Name lookup:} Using DNS to look up host names.
\item \textbf{Proxying:} Networking via a proxy.
\end{itemize}

\end{slide}

%----------------------------------------------------------------------
\begin{slide}[fragile]{0744}
\frametitle{Networking Classes} \label{network_programming}
\xConcept{Network programming}
  \begin{itemize}
  \item Qt has a number of network programming tools:
    \begin{itemize}
          \item Deprecated APIs: \iCls{QHttp} and \iCls{QFtp}
          \item \contentHyperlink{\iCls{QNetworkAccessManager},
        \iCls{QNetworkRequest}, and \iCls{QNetworkReply}}{QNetworkAccess}.
        \item \contentHyperlink{\iCls{QNetworkConfiguration} - querying network configuration}{QNetworkConfiguration}. 
        
        \item \contentHyperlink{TCP and
        UDP sockets by means of \iCls{QTcpSocket}, \iCls{QTcpServer},
        \iCls{QSslSocket}, and \iCls{QUdpSocket}}{tcp-sockets}.
    \item \contentHyperlink{\iCls{QHostInfo} - resolving host names}{QHostInfo}.
    \item \contentHyperlink{\iCls{QNetworkProxy}}{QNetworkProxy}.
    \end{itemize}
    \item In order to use the network support, add \verb!QT += network! to
      your QMake files.
  \end{itemize}
\end{slide}

\subsection{QNetworkAccessManager Request / Reply}

%----------------------------------------------------------------------
\begin{slide}{0510}
\frametitle{QNetworkAccessManager} \label{QNetworkAccess}
\begin{itemize}
\item New in Qt 4.4 is a set of classes for handling http and ftp requests.
\item The core class is \iCls{QNetworkAccessManager}, which offers:\\
  \begin{itemize}
  \item \hClsFn{QNetworkAccessManager}{get} - download.
  \item \hClsFn{QNetworkAccessManager}{put} - upload.
  \item \hClsFn{QNetworkAccessManager}{head} - fetch the header.
  \item \hClsFn{QNetworkAccessManager}{post} - post a HTTP(S) request.
\end{itemize}
\item In addition \hClsFn{QNetworkAccessManager}{setProxy} and
  \hClsFn{QNetworkAccessManager}{setCookieJar} configure proxies and cookie
  handling. % FIXME(till) where does one get the cookiejar from?
\end{itemize}
\centeredImage{ipc/images/qnetworkaccessmanager}
\end{slide}

%----------------------------------------------------------------------
\begin{slide}{0511}
\frametitle{QNetworkRequest}
\begin{itemize}
\item The argument to the methods of \iCls{QNetworkAccessManager} are
  instances of \iCls{QNetworkRequest}.
\item In the simplest setup, \iCls{QNetworkRequest} is created with a
  \iCls{QUrl} as argument.
\item \iConcept{SSL} is configured using
  \hClsFn{QNetworkRequest}{setSSLConfiguration}.
\item Raw headers may be configured using:
  \begin{itemize}
  \item \hClsFnPar{QNetworkRequest}{setHeader}{KnownHeaders headerName,\\
      \hspace{12ex}QVariant headerValue}
  \item \hClsFnPar{QNetworkRequest}{setRawHeader}{QByteArray headerName,\\
      \hspace{16ex}QByteArray headerValue}
  \end{itemize}
\end{itemize}
\centeredImage{ipc/images/qnetworkrequest}
\end{slide}

%----------------------------------------------------------------------
\begin{slide}{0512}
\frametitle{QNetworkReply}
\begin{itemize}
\item The methods of \iCls{QNetworkAccessManager} are all asynchronous.
\item The result of the calls are instances of a \iCls{QNetworkReply}.
\item The signals \iClsFn{QNetworkReply}{finished} and
  \iClsFnPar{QNetworkAccessManager}{finished}{QNetworkReply*} tells you
  when the operation is done.
\item The signal \hClsFnPar{QNetworkReply}{downloadProgress}{qint64,
    qint64} respectively \hClsFn{QNetworkReply}{uploadProgress} informs
  you about progress.
\item Errors are signaled with
  \hClsFnPar{QNetworkReply}{error}{NetworkError} - a printable string may
  be obtained from \hClsFn{QIODevice}{errorString}
\end{itemize}
\centeredImage{ipc/images/qnetworkreply}
\end{slide}

%----------------------------------------------------------------------
\begin{slide}{0513}
\frametitle{QNetworkReply}
\begin{itemize}
\item \iCls{QNetworkReply} is a subclass of \iCls{QIODevice} which means
  you may use it in similar ways to a \iCls{QFile}.
\item \pleaseNote: It is your responsibility to delete the
  \iCls{QNetworkReply} pointer. (the template shared\_ptr from \iNs{std::tr1} or
  \iNs{boost} may be worth considering)
\end{itemize}
\demo{network-programming/ex-multi-download}
\centeredImage{ipc/images/qnetworkreply}
\end{slide}

\subsection{Bearer Management API}

% ----------------------------------------------------------------------
\begin{slide}{1653}
\frametitle{Bearer Management}
Manages access to the network
\begin{itemize}
\item Starting and stopping interfaces
\item Querying information about interfaces
\item Setting priorities for different access methods
\item Track connectivity state and status
\item Transparent transition (roaming) between interfaces
\item Desktop and mobile systems supported
\end{itemize}
\end{slide}

% ----------------------------------------------------------------------
\begin{slide}{1652}
  \frametitle{Bearer Management API}
  \centeredImage{ipc/images/bearermonitor-screenshot}
  \begin{itemize}
    \item[] \qtdemo{examples/network/bearermonitor}
  \end{itemize}
\end{slide}


% ----------------------------------------------------------------------
\begin{slide}{1651}
\frametitle{What does it mean to roam?}
\begin{itemize}
\item Switching to another network configuration in response to an event
    \begin{itemize}
    \item Use ethernet when available.
    \item Switch to WIFI when it is not.
    \item Switch to cellular when both are unavailable
    \end{itemize}
\item Roaming requires some capabilities of configuration manager:
    \begin{itemize}
    \item CanStartAndStopInterfaces, SystemSessionSupport,  
    \item ApplicationLevelRoaming, ForcedRoaming
    \item DirectConnectionRouting
    \end{itemize}
\item API can also be used read-only to get configuration info
\end{itemize} 
\end{slide}

% ----------------------------------------------------------------------
\begin{slide}{1650} 
\frametitle{Access Points and Network Interfaces}
\label{QNetworkInterface}
\label{QNetworkConfiguration}
An access point maps to a network interface
\begin{itemize}
\item Multiple access points can map to the same network interface
\end{itemize}
\textbf{QNetworkConfiguration}
\begin{itemize}
\item Provides state information about access point configurations
\item Groups common access point service types
\end{itemize}
\textbf{QNetworkInterface}
\begin{itemize}
\item Can query for hardware address, capabilities, and human readable name. 
\end{itemize}
\end{slide}


\begin{slide}
\frametitle{Network Configurations and Sessions}

\textbf{QNetworkConfigurationManager}
\begin{itemize}
\item Manages all network configurations
\item \texttt{defaultConfiguration()} - identifies default
\item Emits signals when configurations are added/removed/changed
\item Keeps track of network capabilities
\end{itemize}
\textbf{QNetworkSession}
\begin{itemize}
\item Enables and activates access points
\item Allows sharing of access points between applications
\item Provides access to \texttt{QNetworkInterface}
\end{itemize}
\end{slide}


\subsection{TCP/UDP Sockets}
%----------------------------------------------------------------------
\begin{slide}[fragile]{0745}
\frametitle{\iConcept{TCP Sockets}}\label{tcp-sockets}
  \begin{itemize}
  \item TCP connections of protocols other than HTTP and FTP can
        easily be implemented using \iCls{QTcpSocket} (or
        \iCls{QTcpServer}), even for the HTTP and FTP.
  \item UDP connections are implemented using \iCls{QUdpSocket}.
  \item TCP and UDP connections can be written in a non-blocking and a
        blocking fashion.
  \item \iCls{QTcpSocket} and \iCls{QUdpSocket} are (indirect)
        subclasses of \iCls{QIODevice}. You can therefore read and
        write data using \iCls{QTextStream} and \iCls{QDataStream}
        (and the raw byte-oriented methods, of course).
  \end{itemize}
\end{slide}
%----------------------------------------------------------------------
\begin{slide}{0746}
\frametitle{\iConcept{TCP Client} (Non-Blocking)}
  \begin{itemize}
  \item Create an instance of \iCls{QTcpSocket}.
  \item Call \iClsFn{QAbstractSocket}{connectToHost}
  \item Create a stream from the socket, and write data to the stream.
  \item The signal \hClsFn{QTcpSocket}{readyRead} is emitted whenever data is
    ready to be read on the socket.
  \end{itemize}
  \demo{network-programming/ex-sockets}\xExample{TCP client}
\end{slide}

%----------------------------------------------------------------------
\begin{slide}[fragile]{0747}
\frametitle{TCP Client (Blocking)}
  \begin{itemize}
        \item The socket classes also have a blocking mode. In this
        case, a local event loop is provided, you do not need a global
        one.
        \item Do not use the blocking mode from the GUI thread,
        otherwise your UI will freeze. Use it from a separate thread
          (see the \emph{Multithreading} section on
          \pageHyperlink{multithreading} for more details),
        or in non-GUI applications (e.g., command-line tools like
        \textit{wget}).
        \item For a blocking connect, call
         \iClsFn{QAbstractSocket}{waitForConnected} \emph{after} the
        call to \hClsFn{QAbstractSocket}{connectToHost}.
        \item Then call either \hClsFn{QIODevice}{waitForReadyRead} or
        \hClsFn{QIODevice}{waitForBytesWritten}, and read or write as usual.
        \item Call \hClsFn{QAbstractSocket}{disconnectFromHost} and
        \hClsFn{QAbstractSocket}{waitForDisconnected} at the end.
  \end{itemize}
\end{slide}

%----------------------------------------------------------------------
\begin{slide}{0748}
\frametitle{\iConcept{TCP Server} (Non-Blocking)}
  \begin{itemize}
  \item Non-blocking TCP servers need an event loop (like non-blocking
        TCP clients).
  \item Create an object of class \iCls{QTcpServer}.
  \item Call \hClsFn{QTcpServer}{listen} on that object. You can either specify
        the port to listen to or let \iCls{QTcpServer} pick a free
        one. \hClsFn{QTcpServer}{serverPort} will tell you the one it is using.
  \item When a connection is made, the \hClsFn{QTcpServer}{newConnection} signal
        is emitted. Upon this, call \hClsFn{QTcpServer}{nextPendingConnection}
        to get a \iCls{QTcpSocket} that is already connected to the
        client, and that you can then use for communication.
  \end{itemize}
\end{slide}

%----------------------------------------------------------------------
\begin{slide}{0749}
\frametitle{TCP Server (Blocking)}
  \begin{itemize}
  \item Like blocking TCP clients, blocking TCP servers should not be
        run in the GUI thread, as that would freeze the UI. Use a
        separate thread or use this in command-line tools.
  \item Start like in the previous example, but call
        \hClsFn{QTcpServer}{waitForNewConnection} after the \hClsFn{QTcpServer}{listen}
        call. When this method returns, you can call
        \hClsFn{QTcpServer}{nextPendingConnection} and proceed as with the
        non-blocking server. 
  \end{itemize}
\end{slide}

%----------------------------------------------------------------------
\begin{slide}{0750}
\frametitle{Local TCP Servers}
\begin{itemize}
\item The class \iCls{QLocalSocket} and \iCls{QLocalServer} are similar to
  \iCls{QTcpSocket} respectively \iCls{QTcpServer}, with the exception
  that they are accessible locally only.
\item On Windows, they are implemented using a \iConcept{named pipe}
\item On Unix (incl. MacOSX), they are implemented using a \iCls{local domain socket}
\item Both classes allow blocking and non-blocking behavior.
\item \iClsFn{QTcpSocket}{connectToHost} is replaced with
  \iClsFnPar{QLocalSocket}{connectToServer}{QString serviceName, \ldots}
\item \iClsFn{QLocalServer}{listen} takes a serviceName instead of a
  host address, too.
\end{itemize}
\end{slide}

%----------------------------------------------------------------------
\begin{slide}{0751}
\frametitle{\iConcept{UDP Connection}}
\xConcept{UDP socket}
  \begin{itemize}
  \item Create a \iCls{QUdpSocket} object and send data with
        \hClsFn{QUdpSocket}{writeDatagram}. You need to specify the destination host and
        port each time.
  \item You can use \hClsFn{QAbstractSocket}{connectToHost} and then just use
        \hClsFn{QIODevice}{read} and \hClsFn{QIODevice}{write}, but this is still not
        connection-oriented, it just sets the values for host and port
        for the subsequent calls.
  \item If you want to read data as well, you need to call
        \hClsFn{QUdpSocket}{bind}, specifying the port and host address. Like
        \iClsFn{QTcpServer}{listen}, \iClsFn{QUdpSocket}{bind} can pick a
        free port for you.
  \item The signal \hClsFn{QIODevice}{readyRead} is emitted when data is
        available for reading.
  \end{itemize}
\end{slide}

\subsection{SSL Sockets}
%----------------------------------------------------------------------
\begin{slide}{0752}
\frametitle{\iConcept{Secure Network Programming} using QSslSocket}
\begin{itemize}
\item The class \iCls{QSslSocket} supports secure network access using
  either the \iConcept{SSLv3} protocol  or the \iConcept{TLSv1} protocol.
\item \iCls{QSslSocket} inherits from \iCls{QTcpSocket}, and, after setup,
  the communication is just like with a \iCls{QTcpSocket}.
\end{itemize}
\end{slide}

%----------------------------------------------------------------------
\begin{slide}{0753}
\frametitle{QSslSocket clients}
\begin{itemize}
\item The common way for clients is to call
  \iClsFn{QSslSocket}{connectToHostEncrypted}, which is similar to
  \iClsFn{QTcpSocket}{connectToHost}, except that it will set up a secure
  connection. 
\item Next clients should either call \hClsFn{QSslSocket}{waitForEncrypted} or
  connect to the \hClsFn{QSslSocket}{encrypted} signal.
\end{itemize}
\end{slide}

%----------------------------------------------------------------------
\begin{slide}[fragile]{0754}
\frametitle{QSslSocket servers}
\begin{itemize}
\item The easiest way to implement a \iConcept{SSL server} is to inherit
  from \iCls{QTcpServer}, and override
  \hClsFnPar{QTcpServer}{incomingConnection}{int socketDescriptor}.
\item A \iCls{QSslSocket} is then constructed based on the socket
  descriptor.
\item Once this is set up, handshaking is started using
  \hClsFn{QSslSocket}{startServerEncryption}.
\end{itemize}
\end{slide}

%----------------------------------------------------------------------
\begin{slide}[fragile]{0755}
\frametitle{QSslSocket servers}
\begin{verbatim}
void SslServer::incomingConnection(int socketDescriptor)
{
  QSslSocket *serverSocket = new QSslSocket;
  if (serverSocket->setSocketDescriptor(socketDescriptor)) 
  {
    connect(serverSocket, SIGNAL(encrypted()), 
            this, SLOT(ready()));
    serverSocket->startServerEncryption();
  }
}
\end{verbatim}
\end{slide}

%----------------------------------------------------------------------
\begin{slide}{0756}
\frametitle{Certificates, keys, and ciphers}
\begin{itemize}
\item Ciphers, the socket's local certificate, and the private keys must be set
  before the handshake starts.
\item Ciphers are set using \hClsFn{QSslSocket}{setCiphers} and
  \hClsFn{QSslSocket}{setDefaultCiphers}.
\item The private key and certificate are set using
  \hClsFn{QSslSocket}{setPrivateKey} and
  \hClsFn{QSslSocket}{setLocalCertificate}
\item The CA certificate database is manipulated with
  \hClsFn{QSslSocket}{addCaCertificate},
  \hClsFn{QSslSocket}{addCaCertificates},
  \hClsFn{QSslSocket}{setCaCertificates},
  \hClsFn{QSslSocket}{addDefaultCaCertificate},
  \hClsFn{QSslSocket}{addDefaultCaCertificates}, and
  \hClsFn{QSslSocket}{setDefaultCaCertificates}.
\item \qtdemo{examples/network/securesocketclient}
  \xExample{QSslSocket}
\end{itemize}
\end{slide}


\subsection{DNS and Proxies}
%----------------------------------------------------------------------
\begin{slide}[fragile]{0757}
\frametitle{\iConcept{Resolving Hostnames}}\label{QHostInfo}
  \begin{itemize}
    \item \iCls{QHostInfo} serves for resolving host names.
    \item \iCls{QHostInfo} uses the underlying
        \hClsFn{QHostInfo}{gethostbyname} call and therefore supports all
        lookup schemes that the operating system supports (this could
        e.g. be NIS).
    \item In order to do a blocking lookup, call the static method
        \iClsFn{QHostInfo}{fromName} and use the
        \hClsFn{QHostInfo}{addresses} method to get all addresses for the named
        host:
\begin{alltt}\small
QHostInfo info = QHostInfo::fromName( "www.trolltech.com" );
QList<QHostAddress> addresses = info.addresses();
if( addresses.count() > 0 )
  qDebug() << "First address: " 
           << addresses.first().toString();
\end{alltt}
  \end{itemize}
\end{slide}

%----------------------------------------------------------------------
\begin{slide}{0758}
\frametitle{Resolving Hostnames}
  \begin{itemize}
     \item Do not do this in the GUI thread, or your UI will
        freeze.
     \item For a non-blocking lookup, call
        \iClsFn{QHostInfo}{lookupHost}, providing the receiver object and
        slot to be called when the lookup is finished (see the example
        on the next slide).
    \item You can abort a host lookup using \hClsFn{QHostInfo}{abortHostLookup}.
  \end{itemize}
\end{slide}

%----------------------------------------------------------------------
\begin{slide}[fragile]{0759}
\frametitle{Resolving Hostnames: Example}
\begin{lstlisting}
MyClass::startLookup()
{
    QHostInfo::lookupHost( "qt.nokia.com", this,
            SLOT( slotLookupDone( const QHostInfo& ) ) );
}

MyClass::slotLookupDone( const QHostInfo& info )
{
    QList<QHostAddress> addresses = info.addresses();
    if( addresses.count() > 0 )
        qDebug() << "First address: "
                 << addresses.first().toString();
}
\end{lstlisting}
\end{slide}

%----------------------------------------------------------------------
\begin{slide}{0761}
\frametitle{\iConcept{Proxies}}\label{QNetworkProxy}
Proxies can be set up with the class \iCls{QNetworkProxy}.
\begin{itemize}
\item \iCls{QNetworkProxy} is used to identify HTTP, FTP and SOCKS5 proxies.
\item HTTP and FTP proxies can perform caching.
\end{itemize}
To use a proxy:
\begin{itemize}
\item Create a \iCls{QNetworkProxy} object
and populate it with hostname, port, etc.
\item Assign the proxy globally with the static method
\iClsFn{QNetworkProxy}{setApplicationProxy} or...
\item ...just on one socket using \hClsFn{QAbstractSocket}{setProxy}.
\end{itemize}
\end{slide}

%----------------------------------------------------------------------
\begin{slide}{45761}
\frametitle{Customizing Proxies}
Proxy factories are used to create policies for proxy use.
\begin{itemize}
  \item \iCls{QNetworkProxyFactory} supplies proxies based on queries for
  specific proxy types.
  \item Queries are encoded in \iCls{QNetworkProxyQuery} objects.
  \item \hClsFn{QNetworkProxyFactory}{proxyForQuery} is used to query the factory
  directly. 
  \item To change the behavior, reimplement
  \hClsFn{QNetworkProxyFactory}{queryProxy}.
  \item To implement an application-wide policy with the factory, call
  \hClsFn{QNetworkProxyFactory}{setApplicationProxyFactory}.
  \begin{itemize}
    \item This overrides any proxy set with
          \iClsFn{QNetworkProxy}{setApplicationProxy}.
    \item Querying \iClsFn{QNetworkProxy}{applicationProxy} causes the
          factory to be queried.
  \end{itemize}
\end{itemize}
\end{slide}

%----------------------------------------------------------------------
\begin{slide}{45762}
\frametitle{Proxy Queries}
Queries enable proxies to be selected based on key criteria:
\begin{itemize}
  \item The purpose of the proxy: TCP, UDP, TCP server, URL request
  \item Local port, remote host and port
  \item The protocol in use: such as HTTP or FTP
  \item The URL being requested
\end{itemize}
\end{slide}

%----------------------------------------------------------------------
\begin{slide}[fragile]{0762}
\frametitle{Error Handling}
  \begin{itemize}
      \item \iCls{QHttp} and \iCls{QFtp} indicate errors by means of a
        boolean parameter in the \hClsFn{QFtp}{done}\xClsFn{QHttp}{done},
        \iClsFn{QHttp}{requestFinished}, and \iClsFn{QFtp}{commandFinished}
        signals.

  \item \iCls{QTcpSocket} and \iCls{QUdpSocket} indicate errors
        by means of the \hClsFn{QAbstractSocket}{error} signal.
    \item \iCls{QTcpServer} and \iCls{QHostInfo} have no notification mechanism for
        errors.
    \item In any of these classes, you can then request an error code
        with \hClsFn{QAbstractSocket}{error} (\hClsFn{QTcpServer}{serverError} for
        \iCls{QTcpServer}).
    \item \texttt{errorString()} provides (for all
        classes) a human-readable error description.
      \item \iCls{QSslSocket} emits an \hClsFn{QSslSocket}{sslErrors}
        signal. This can be ignored using
        \hClsFn{QSslSocket}{ignoreSslErrors}. If not ignored, the
        connection is torn down.
  \end{itemize}
\end{slide}

