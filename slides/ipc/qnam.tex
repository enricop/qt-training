\subsection{QNetworkAccessManager Request / Reply}

%----------------------------------------------------------------------
\begin{slide}{0510}
\frametitle{QNetworkAccessManager} \label{QNetworkAccess}
\begin{itemize}
\item New in Qt 4.4 is a set of classes for handling http and ftp requests.
\item The core class is \iCls{QNetworkAccessManager}, which offers:\\
  \begin{itemize}
  \item \hClsFn{QNetworkAccessManager}{get} - download.
  \item \hClsFn{QNetworkAccessManager}{put} - upload.
  \item \hClsFn{QNetworkAccessManager}{head} - fetch the header.
  \item \hClsFn{QNetworkAccessManager}{post} - post a HTTP(S) request.
\end{itemize}
\item In addition \hClsFn{QNetworkAccessManager}{setProxy} and
  \hClsFn{QNetworkAccessManager}{setCookieJar} configure proxies and cookie
  handling. % FIXME(till) where does one get the cookiejar from?
\end{itemize}
\centeredImage{ipc/images/qnetworkaccessmanager}
\end{slide}

%----------------------------------------------------------------------
\begin{slide}{0511}
\frametitle{QNetworkRequest}
\begin{itemize}
\item The argument to the methods of \iCls{QNetworkAccessManager} are
  instances of \iCls{QNetworkRequest}.
\item In the simplest setup, \iCls{QNetworkRequest} is created with a
  \iCls{QUrl} as argument.
\item \iConcept{SSL} is configured using
  \hClsFn{QNetworkRequest}{setSSLConfiguration}.
\item Raw headers may be configured using:
  \begin{itemize}
  \item \hClsFnPar{QNetworkRequest}{setHeader}{KnownHeaders headerName,\\
      \hspace{12ex}QVariant headerValue}
  \item \hClsFnPar{QNetworkRequest}{setRawHeader}{QByteArray headerName,\\
      \hspace{16ex}QByteArray headerValue}
  \end{itemize}
\end{itemize}
\centeredImage{ipc/images/qnetworkrequest}
\end{slide}

%----------------------------------------------------------------------
\begin{slide}{0512}
\frametitle{QNetworkReply}
\begin{itemize}
\item The methods of \iCls{QNetworkAccessManager} are all asynchronous.
\item The result of the calls are instances of a \iCls{QNetworkReply}.
\item The signals \iClsFn{QNetworkReply}{finished} and
  \iClsFnPar{QNetworkAccessManager}{finished}{QNetworkReply*} tells you
  when the operation is done.
\item The signal \hClsFnPar{QNetworkReply}{downloadProgress}{qint64,
    qint64} respectively \hClsFn{QNetworkReply}{uploadProgress} informs
  you about progress.
\item Errors are signaled with
  \hClsFnPar{QNetworkReply}{error}{NetworkError} - a printable string may
  be obtained from \hClsFn{QIODevice}{errorString}
\end{itemize}
\centeredImage{ipc/images/qnetworkreply}
\end{slide}

%----------------------------------------------------------------------
\begin{slide}{0513}
\frametitle{QNetworkReply}
\begin{itemize}
\item \iCls{QNetworkReply} is a subclass of \iCls{QIODevice} which means
  you may use it in similar ways to a \iCls{QFile}.
\item \pleaseNote: It is your responsibility to delete the
  \iCls{QNetworkReply} pointer. (the template shared\_ptr from \iNs{std::tr1} or
  \iNs{boost} may be worth considering)
\end{itemize}
\demo{network-programming/ex-multi-download}
\centeredImage{ipc/images/qnetworkreply}
\end{slide}
