\subsection{Shared Memory with Qt}\label{sharedMemory}

%----------------------------------------------------------------------
\begin{slide}{0255}
\frametitle{Shared Memory Basics} \label{sharedmemory}
\begin{itemize}
  \item \iConcept{Shared memory} is a feature offered by modern OS'es for:
  \begin{itemize}
  \item \iConcept{Interprocess communication}, by allowing \iConcept{multiple processes} to share data efficiently.
  \item Conserving memory, by allowing big chunks of data to be shared between processes while only keeping one copy in memory.
  \end{itemize}
  \end{itemize}
\centeredImage{ipc/images/shared-memory-overview}
\end{slide}

%----------------------------------------------------------------------
\begin{slide}{0256}
\frametitle{Shared Memory with Qt}
\xConcept{SharedMemory}
\begin{itemize}
  \item Qt provides access to OS-level shared memory functionality since version 4.4.
  \item The central class is \iCls{QSharedMemory}.
  \item Besides managing the shared memory itself, \iCls{QSharedMemory} provides synchronisation through an internal \iCls{QSystemSemaphore}.
  \item The layer provided by Qt is fairly thin, so the programmer needs to be aware of the different ways shared memory works on different OS'es.
  \item Think of \iCls{QSharedMemory} as a way to access shared memory, not as shared memory itself.
  \end{itemize}
\end{slide}

%----------------------------------------------------------------------
\begin{slide}{0257}
\frametitle{Shared Memory with Qt}
\xConcept{SharedMemory}
\begin{itemize}
  \item A shared memory segment is identified by a QString key. 
  \item Set in the constructor or via \iClsFn{QSharedMemory}{setKey}.
  \item An attempt to \hClsFn{QSharedMemory}{create} a shm segment that
    already exists, or to \hClsFn{QSharedMemory}{attach} to a segment that
    does not exist will result in an error.
  \item Error in this context means that the methods in question return \texttt{false}.
  \item The only way to check if a memory segment with a given key exists is to try to attach to it.
  \end{itemize}
\end{slide}

%----------------------------------------------------------------------
\begin{slide}{0258}
\frametitle{OS differences, Linux/UNIX}
\xConcept{SharedMemory}
  \begin{itemize}
  \item The shared memory segment is not owned by any process in particular, so even if all processes using a shm segment are destroyed, the shm segment memory is still allocated.
  \item \iCls{QSharedMemory} takes ownership of the shared memory segment, and the last \iCls{QSharedMemory} instance attached to a shm segment will destroy it in its destructor.
  \item This of course still doesn't prevent the process from leaking shm segments in case of a crash.
  \item Use the command line tool \texttt{ipcs} to examine and destroy shm segments.
  \end{itemize}
\end{slide}


%----------------------------------------------------------------------
\begin{slide}{0259}
\frametitle{OS differences, Windows}
\xConcept{SharedMemory}
  \begin{itemize}
  \item On Windows, a shared memory segment is owned collectively by all processes attached to it. The operating system deletes a segment when the last process attached to it exits.
  \item \iCls{QSharedMemory} does not destroy anything in its destructor.
  \item So, no resources can be leaked, but on the other hand, it is not possible to ``keep shared memory around'' in applications.
  \end{itemize}
\end{slide}

%----------------------------------------------------------------------
\begin{slide}[fragile]{0261}
\frametitle{Usage}
\xConcept{SharedMemory}
  \begin{itemize}
  \item Creation:
  \begin{center}
\xExample{SharedMemory!create}
\tiny
\begin{alltt}
QSharedMemory shm;
shm.setKey("MyShm");
if( !shm.create( 4242 /*bytes*/,
                 QSharedMemory::ReadWrite ) ) \{
    /* failure, error details available in
       shm.error()/shm.errorString() */
\}
/* create() also attaches to memory, so we can
use it right away */
shm.lock();
void* data = shm.data();
/* ... do interesting stuff with data... */
shm.unlock();
\end{alltt}
  \end{center}
\item Destruction:
  \begin{center}
\xExample{SharedMemory!destroy}
\tiny
\begin{alltt}
/* Detach from memory, if we're the last process to do so,
the memory is released */
shm.detach();
\end{alltt}
  \end{center}
\item The \iCls{QSharedMemory} instance should be kept around as long as we want to keep the memory.
  \end{itemize}
\end{slide}

%----------------------------------------------------------------------
\begin{slide}[fragile]{0262}
\frametitle{Usage}
\xConcept{SharedMemory}
  \begin{itemize}
  \item Attach to existing shared memory segment:
  \begin{center}
\xExample{SharedMemory!attach}
\tiny
\begin{alltt}
QSharedMemory shm;
shm.setKey("MyShm");
if( !shm.attach( QSharedMemory::ReadWrite ) ) \{
    /* failure, error details available in
       shm.error()/shm.errorString() */
\}
shm.lock();
void* data = shm.data();
/* ... do interesting stuff with data... */
shm.unlock();
shm.detach();
\end{alltt}
  \end{center}
  \end{itemize}
\demo{inter-process-communication/ex-renderer}
\demo{inter-process-communication/ex-viewer}\\
(run one renderer and as many viewers as you like)
\end{slide}

%----------------------------------------------------------------------
\begin{slide}{0263}
\frametitle{QSystemSemaphore}
\begin{itemize}
\item When sharing memory, it may be needed to lock the memory while doing
  an operation on it.
\item For that purpose the class \iCls{QSystemSemaphore} is a cross-process
  semaphore.
\item The \iCls{QSystemSemaphore} instance is created with a key that all
  instances use to share the semaphore. Additionally, the available
  semaphores can be specified as the second argument.
\item \iClsFn{QSystemSemaphore}{acquire} and
  \iClsFn{QSystemSemaphore}{release} are the main operations.
\end{itemize}
\end{slide}


%----------------------------------------------------------------------
\begin{slide}{0264}
\frametitle{QSystemSemaphore}
\begin{itemize}
\item Windows owns the semaphore, Unix does not!
\item On Windows, the first process to access the semaphore (e.g. the server) should
  specify \iClsEnum{QSystemSemaphore}{Create}, as the optional argument to
  the constructor, in order to clean up stale semaphores.
\end{itemize}
\end{slide}
