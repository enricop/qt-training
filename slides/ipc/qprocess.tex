\subsection{Running Processes}\label{qprocess}

%----------------------------------------------------------------------
\begin{slide}[fragile]{0591}
\frametitle{QProcess}
\begin{itemize}
\item \iCls{QProcess} executes external processes asynchronously
\item Also offers a synchronous API
\item Signals are emitted when
\begin{itemize}
\item data arrives on stdout
\item data arrives on stderr
\item process finishes
\item process changes state
\end{itemize}
\item \hClsFn{QProcess}{start} to start it
\item A \iCls{QProcess} can be used to start same process several times.
\end{itemize}
\end{slide}

%----------------------------------------------------------------------
\begin{slide}{0592}
\frametitle{Setting up the Process Call}
\begin{itemize}
\item The command to invoke as well as the arguments are specified in
  the \hClsFn{QProcess}{start} call.
\item Alternatively, you can specify the arguments (but not the
  command) using the \texttt{<<} streaming operator.
\item No escaping is required for the arguments.
\item Set environment variables with \hClsFn{QProcess}{setEnvironment}.
\item \hClsFn{QProcess}{setWorkingDirectory} to change working dir.
\item If not specified, the environment and working dir of your process is used.

\end{itemize}
\end{slide}

%----------------------------------------------------------------------
\begin{slide}{0593}
\frametitle{Process states}
A process passes the following states during its lifetime; the current
state can be queried with \hClsFn{QProcess}{state}:

\centeredImageDoubleWidth{ipc/images/qprocessStates}
\end{slide}

%----------------------------------------------------------------------
\begin{slide}[fragile]{0594}
\frametitle{Process Input}
\xConcept{QProcess!Input}
\begin{itemize}
\item \iCls{QProcess} inherits \iCls{QIODevice}.
\item \iCls{QTextStream} and \iCls{QDataStream} can therefore be used
  (\iCls{QDataStream} is really only useful if the child process uses
  \iCls{QDataStream} for its output as well.)
\item You can also use \hClsFn{QIODevice}{write}, \hClsFn{QIODevice}{read},
  \hClsFn{QIODevice}{readLine}, and \hClsFn{QIODevice}{getChar} directly.
\item In order to provide \emph{input} to the external process, just
  \hClsFn{QIODevice}{write} to it. (Writing \emph{from} the current process means  providing \emph{input} to the child process.)
\end{itemize}
\end{slide}

%----------------------------------------------------------------------
\begin{slide}{0595}
\frametitle{Process Output}
\xConcept{QProcess!Output}
\begin{itemize}
\item The \emph{output} of the external process is the \emph{input} to
  the current process and is therefore \emph{read}, using
  \hClsFn{QIODevice}{read}, \hClsFn{QIODevice}{readLine}, or similar methods.
\item Since the process output can be on either of the two channels
  \emph{stdout} and \emph{stderr}, you need to select the one you want
  using \hClsFn{QProcess}{setReadChannel}.
\item Alternatively, merge the output with\\
  \hClsFnPar{QProcess}{setReadChannelMode}{QProcess::MergedChannels}\\
  (careful that you are not mixing up information then!).
\end{itemize}
\end{slide}

%----------------------------------------------------------------------
\begin{slide}{0596}
\frametitle{Process Output}
\begin{itemize}
\item You can also read everything from one channel with
\hClsFn{QProcess}{readAllStandardOutput} and \hClsFn{QProcess}{readAllStandardError}.
\item The signals \hClsFn{QProcess}{readyReadStandardOutput} and
\hClsFn{QProcess}{readyReadStandardError} signal availability of data to be read.
\end{itemize}
\end{slide}

%----------------------------------------------------------------------
\begin{slide}[fragile]{0597}
\frametitle{Process Output, Example}
Example: Reading all of standard output as a string after process
  \begin{itemize}
  \item Example: Reading all of standard output as a string after process
termination:
\begin{alltt}
QProcess process;
process.start( "command"  );
process.\hClsFn{QProcess}{waitForFinished};
\iCls{QTextStream} stream( \&process );
QString output = stream.readAll();
\end{alltt}
\demo{inter-process-communication/ex-findTool}
\end{itemize}
\end{slide}

%----------------------------------------------------------------------
\begin{slide}[fragile]{0598}
\frametitle{Controlling the Process}
\xConcept{QProcess!Controlling the process}
\begin{itemize}
\item The signal \hClsFnPar{QPorcess}{finished}{int} is emitted when the program
terminates, passing the exit code. \hClsFn{QProcess}{exitCode} also provides
the exit code.
\item You can ask the external process to terminate with
\hClsFn{QProcess}{terminate}, \hClsFn{QProcess}{kill} does the same, but more
forceful, without leaving the process a chance to resist.
\item \hClsFn{QProcess}{error} tells you which error occurred last (only useful
if \hClsFn{QProcess}{state} returns \texttt{QProcess::NotRunning}).
\end{itemize}
\end{slide}

%----------------------------------------------------------------------
\begin{slide}{0599}
\frametitle{Synchronous Process Invocation}
\xConcept{QProcess!Synchronous invocation}
\begin{itemize}
\item \iCls{QProcess} has blocking methods to wait for a certain
event in the lifecycle of a process:
\begin{itemize}
        \item \hClsFn[\textbf]{QProcess}{waitForStarted} returns when the
        \hClsFn{QProcess}{started} signal has been emitted, i.e., the process has
                started successfully (or has failed to start).
        \item \hClsFn[\textbf]{QProcess}{waitForReadyRead} returns when data is
                available for reading on the current channel (careful!).
        \item \hClsFn[\textbf]{QProcess}{waitForBytesWritten} returns when certain
                (unspecified) amount of data has been written.
        \item \hClsFn[\textbf]{QProcess}{waitForFinished} returns when the
                \hClsFn{QProcess}{finished} signal has been emitted, i.e., when the process has
                finished its execution.
\end{itemize}
\item Remember that calling any of these methods in the main thread
will freeze your user interface; use multithreading.
\end{itemize}
\end{slide}

\begin{slide}[fragile]{0600}
\frametitle{Simplified Synchronous Process Invocation}
  \begin{itemize}
\item If you are not interested in processing the output of the
external process, nor in supplying input to it, use the static
convenience method \iClsFn{QProcess}{execute} that starts the
process, waits for its termination, and returns:
\begin{alltt}
\iCls{QStringList} arguments;
arguments << "Argument1" << "Argument2";
\iClsFnPar{QProcess}{execute}{"do_it_now", arguments};
// won't get here until do_it_now terminates
\end{alltt}
\item Use \iClsFn{QProcess}{startDetached} instead if you want the child
process to detach from the current one. This method will not wait for
termination, and the child process will not be terminated
when the current process terminates (``fire and forget'').
  \end{itemize}
\end{slide}
