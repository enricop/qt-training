%%%%%%%%%%%%%%%%%%%%%%%%%%%%%%%%%%%%%%%%%%%%%%%%%%%%%%%%%%%%%%%%%%%%%%%%%%
%
% Copyright (c) 2008-2011, Nokia Corporation and/or its subsidiary(-ies).
% All rights reserved.
%
% This work, unless otherwise expressly stated, is licensed under a
% Creative Commons Attribution-ShareAlike 2.5.
%
% The full license document is available from
% http://creativecommons.org/licenses/by-sa/2.5/legalcode .
%
%%%%%%%%%%%%%%%%%%%%%%%%%%%%%%%%%%%%%%%%%%%%%%%%%%%%%%%%%%%%%%%%%%%%%%%%%%

\subsection{Modules}
\begin{slide}{1617}\frametitle{Modules}

Modules hold collections of elements:

\begin{itemize}
\item Contain definitions of new elements
\item Allow and promote re-use of elements and higher level components
\item Versioned
  \begin{itemize}
  \item allows specific versions of modules to be chosen
  \item guarantees certain features/behavior
  \end{itemize}
\item Import a directory name to import all modules within it
\end{itemize}

%%% Module != "module"
%%% qmldir files in directories or in C++ library
%%% qmldir contains a mapping between .qml files and versions
%%% Version guarantees behavior
%%% Version number is effectively maximum version guarantee
%%% Use "as" to import two versions of a module in order to access
%%% import "myfile.qml"
%%% compiled components
%%% import "myfile.js" as Script

\doc{qdeclarativemodules.html}{QML Modules}
\end{slide}

%----------------------------------------------------------------------

%%% Two cases: built-in, versioned modules and custom items
%%% Imports only introduce objects into the namespace of the module
%%% that imports them. Cannot import Qt indirectly using a custom item.

%----------------------------------------------------------------------

\begin{slide}{1616}\frametitle{Custom Item Revisited}

% declarative-uis/modules-components/lineedit/use-lineedit.qml
\inputqml{qml-structures/colorized/use-lineedit}

\vspace*{0.5em}
\begin{itemize}
\item Recall this example from earlier
\item \qtt{LineEdit.qml} is in the same directory
  \begin{itemize}
  \item item within the file automatically available as \qtt{LineEdit}
  \end{itemize}
\item We would like to make different versions of this item
  \begin{itemize}
  \item to do this, we need to learn about collections of items
  \end{itemize}
\end{itemize}

\demo{qml-modules-components/ex-modules-components/lineedit/use-lineedit.qml}
\end{slide}

%----------------------------------------------------------------------

\begin{slide}{1615}\frametitle{Collections of Items}

% declarative-uis/modules-components/use-collection-of-items.qml
\inputqml{qml-structures/colorized/use-collection-of-items}

\begin{itemize}
\item Importing \qic{string}{"items"}
  \begin{itemize}
  \item imports all files in \texttt{items} directory
  \item including file \texttt{items/CheckBox.qml}
  \item the item within is available as \qtt{CheckBox}
  \end{itemize}
\item Useful for splitting up an application
\item Provides the mechanism for versioning of modules
\end{itemize}

\demo{qml-modules-components/ex-modules-components/use-collection-of-items.qml}

\end{slide}

%----------------------------------------------------------------------

\begin{slide}{1614}\frametitle{Versioning Modules}

\flushedImageDoubleWidth{qml-structures/images/module-structure.pdf}

\begin{itemize}
\item Create a directory called \qtt{LineEdit}\\
      containing
  \begin{itemize}
  \item \qtt{1.0.qml}~\textendash~implementation of the custom item
  \item \qtt{qmldir}~\textendash~version information for the module
  \end{itemize}

\item The \qtt{qmldir} file contains a single line:

\begin{alltt}
LineEdit 1.0 LineEdit-1.0.qml
\end{alltt}

\item Describes the name of the item exported by the module
\item Relates a version number to the file containing the implementation
\end{itemize}

\end{slide}

%----------------------------------------------------------------------

\begin{slide}{1613}\frametitle{Using a Versioned Module}

% declarative-uis/modules-components/versioned/use-lineedit-version.qml
\inputqml{qml-structures/colorized/use-lineedit-version}

\vspace*{0.5em}
\begin{itemize}
\item Now explicitly import the \qic{class}{LineEdit}
  \begin{itemize}
  \item using a relative path
  \item and a version number
  \end{itemize}
\end{itemize}

\demo{qml-modules-components/ex-modules-components/versioned/use-lineedit-version.qml}


\end{slide}

%----------------------------------------------------------------------

\begin{slide}{1612}\frametitle{Running the Example}

\begin{itemize}
\item Locate the \qtt{declarative-uis/modules-components} directory
\item Launch the example:\\
\consolecode{\footnotesize qmlscene -I versioned versioned/use-lineedit-version.qml}
\end{itemize}

\begin{itemize}
\item Normally, the module would be installed on the system
  \begin{itemize}
  \item within the Qt installation's \qtt{imports} directory
  \item so the \qtt{-I} option would not be needed for \qtt{qmlscene}
  \end{itemize}
\end{itemize}

\end{slide}

%----------------------------------------------------------------------

\begin{slide}{1611}\frametitle{Supporting Multiple Versions}

\begin{itemize}
\item Imagine that we release version 1.1 of \qtt{LineEdit}
\item The example imports version 1.0
  \begin{itemize}
  \item only the features from that version are required
  \item we need to ensure backward compatibility
  \end{itemize}
\item \qtt{LineEdit} needs to include support for multiple versions
\item Version handling is done in the \qtt{qmldir} file

\begin{alltt}
LineEdit 1.1 LineEdit-1.1.qml\\
LineEdit 1.0 LineEdit-1.0.qml
\end{alltt}

\item Each implementation file is declared
  \begin{itemize}
  \item with its version
  \item in decreasing version order (newer versions first)
  \end{itemize}
\end{itemize}

\end{slide}

%----------------------------------------------------------------------

\begin{slide}{1610}\frametitle{Importing into a Namespace}

% declarative-uis/modules-components/use-namespace-module.qml
\inputqml{qml-structures/colorized/use-namespace-module}

\begin{itemize}
\item \qic{keyword}{import} ... \qic{keyword}{as} ...
  \begin{itemize}
  \item all items in the \qtt{Qt} module are imported
  \item accessed via the \qtt{MyQt} namespace
  \end{itemize}
\item Allows multiple versions of modules to be imported
\end{itemize}

\demo{qml-modules-components/ex-modules-components/use-namespace-module.qml}

\end{slide}

%----------------------------------------------------------------------

\begin{slide}{1609}\frametitle{Importing into a Namespace}

% declarative-uis/modules-components/use-namespace.qml
\inputqml{qml-structures/colorized/use-namespace}

\begin{itemize}
\item Importing a collection of items from a path
\item Avoids potential naming clashes with items from other collections
      and modules
\end{itemize}

\demo{qml-modules-components/ex-modules-components/use-namespace.qml}

\end{slide}

