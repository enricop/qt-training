%%%%%%%%%%%%%%%%%%%%%%%%%%%%%%%%%%%%%%%%%%%%%%%%%%%%%%%%%%%%%%%%%%%%%%%%%%
%
% Copyright (c) 2008-2011, Nokia Corporation and/or its subsidiary(-ies).
% All rights reserved.
%
% This work, unless otherwise expressly stated, is licensed under a
% Creative Commons Attribution-ShareAlike 2.5.
%
% The full license document is available from
% http://creativecommons.org/licenses/by-sa/2.5/legalcode .
%
%%%%%%%%%%%%%%%%%%%%%%%%%%%%%%%%%%%%%%%%%%%%%%%%%%%%%%%%%%%%%%%%%%%%%%%%%%

\section{Introduction to Qt Quick}

%----------------------------------------------------------------------
\begin{slide}{2024}\frametitle{Objectives}

\begin{itemize}
\item Overview of the Qt library
  \begin{itemize}
  \item Qt framework presentation
  \item Qt Quick inside the Qt framework
  \end{itemize}\medskip
\item Understanding of QML syntax and concepts
  \begin{itemize}
  \item elements and identities
  \item properties and property binding
  \end{itemize}\medskip
\item Basic user interface composition skills
  \begin{itemize}
  \item familiarity with common elements
  \item understanding of anchors and their uses
  \item ability to reproduce a design
  \end{itemize}
\end{itemize}

\end{slide}

%%%%%%%%%%%%%%%%%%%%%%%%%%%%%%%%%%%%%%%%%%%%%%%%%%%%%%%%%%%%%%%%%%%%%%%%%%%
%
% Copyright (c) 2008-2011, Nokia Corporation and/or its subsidiary(-ies).
% All rights reserved.
%
% This work, unless otherwise expressly stated, is licensed under a
% Creative Commons Attribution-ShareAlike 2.5.
%
% The full license document is available from
% http://creativecommons.org/licenses/by-sa/2.5/legalcode .
%
%%%%%%%%%%%%%%%%%%%%%%%%%%%%%%%%%%%%%%%%%%%%%%%%%%%%%%%%%%%%%%%%%%%%%%%%%%

%----------------------------------------------------------------------
\subsection{Meet Qt}

%----------------------------------------------------------------------
\begin{slide}{2003}\frametitle{The Qt SDK}
\vspace*{0.5em}
\centeredImageFullWidth{qml-intro/images/qt_architecture.png}

\end{slide}

%----------------------------------------------------------------------
\begin{slide}{2002}\frametitle{Qt modules}
\vspace*{1.5em}
The Qt framework is split in modules:
\begin{itemize}
\item Each module has is own repository
\item QtBase is the only mandatory module for all Qt application
\item QtTools is a special module containing Qt applications
\item examples: QtDeclarative, QtWebKit, QtMultimedia...
\end{itemize}\medskip
Modules are themselves split into one or more libraries
\begin{itemize}
\item libraries are linked to your applications
\item they group a set of common features (xml, dbus, network...)
\item QtCore (part of QtBase) is mandatory for all Qt applications
\item examples: QtNetwork, QtGui, QtDeclrative
\end{itemize}

\end{slide}

%----------------------------------------------------------------------

\begin{slide}{2001}\frametitle{Creating user interface}
\vspace*{1.5em}
Using widgets (C++)
\begin{itemize}
\item fixed interface
\item mostly targeting desktop application
\end{itemize}
\centeredImageFullWidth{qml-intro/images/ui-widget.png}

\end{slide}

%----------------------------------------------------------------------

\begin{slide}{2001}\frametitle{Creating user interface}
\vspace*{1.5em}
Using graphics view (C++)
\begin{itemize}
\item 2D canvas (with z axis)
\item handle a high number of 2D items
\item transformations and animations of items
\end{itemize}
\centeredImageFullWidth{qml-intro/images/ui-graphicsview.png}

\end{slide}

%----------------------------------------------------------------------

\begin{slide}{2001}\frametitle{Creating user interface}
\vspace*{1.5em}
Using openGL (C++/OpenGL)
\begin{itemize}
\item full strength of OpenGL
\item user interaction (mouse and keybboard) using Qt API
\item all classes from Qt are usable
\end{itemize}
%\centeredImageFullWidth{qml-intro/images/ui-opengl.png}

\end{slide}

%----------------------------------------------------------------------

\begin{slide}{2001}\frametitle{Creating user interface}
\vspace*{1.5em}
Using QtQuick (QML/Javascript/C++/OpenGL)
\begin{itemize}
\item declarative langage using the QML langage
\item provide smooth animation
\item easily linked to C++
\end{itemize}
\centeredImageFullWidth{qml-intro/images/ui-quick.png}

\end{slide}

%----------------------------------------------------------------------

\begin{slide}{2000}\frametitle{Qt Quick requirement}
\vspace*{1.5em}
\begin{itemize}
\item Platform must support OpenGL ES2
\item Needs at least QtBase, QtJsBackend and QtDeclarative module
\item Other module can be used to add new features:
\begin{itemize}
\item QtGraphicalEffects: add effects like blur, drop shadow...
\item Qt3D: 3D programming in QML
\item QtMultimedia: audio and video items
\item ...
\end{itemize}
\end{itemize}

\end{slide}

%----------------------------------------------------------------------

%%%%%%%%%%%%%%%%%%%%%%%%%%%%%%%%%%%%%%%%%%%%%%%%%%%%%%%%%%%%%%%%%%%%%%%%%%
%
% Copyright (c) 2008-2011, Nokia Corporation and/or its subsidiary(-ies).
% All rights reserved.
%
% This work, unless otherwise expressly stated, is licensed under a
% Creative Commons Attribution-ShareAlike 2.5.
%
% The full license document is available from
% http://creativecommons.org/licenses/by-sa/2.5/legalcode .
%
%%%%%%%%%%%%%%%%%%%%%%%%%%%%%%%%%%%%%%%%%%%%%%%%%%%%%%%%%%%%%%%%%%%%%%%%%%

%----------------------------------------------------------------------
\subsection{Meet Qt}

%----------------------------------------------------------------------
\begin{slide}{2003}\frametitle{The Qt SDK}
\vspace*{0.5em}
\centeredImageFullWidth{qml-intro/images/qt_architecture.png}

\end{slide}

%----------------------------------------------------------------------
\begin{slide}{2002}\frametitle{Qt modules}
\vspace*{1.5em}
The Qt framework is split in modules:
\begin{itemize}
\item Each module has is own repository
\item QtBase is the only mandatory module for all Qt application
\item QtTools is a special module containing Qt applications
\item examples: QtDeclarative, QtWebKit, QtMultimedia...
\end{itemize}\medskip
Modules are themselves split into one or more libraries
\begin{itemize}
\item libraries are linked to your applications
\item they group a set of common features (xml, dbus, network...)
\item QtCore (part of QtBase) is mandatory for all Qt applications
\item examples: QtNetwork, QtGui, QtDeclrative
\end{itemize}

\end{slide}

%----------------------------------------------------------------------

\begin{slide}{2001}\frametitle{Creating user interface}
\vspace*{1.5em}
Using widgets (C++)
\begin{itemize}
\item fixed interface
\item mostly targeting desktop application
\end{itemize}
\centeredImageFullWidth{qml-intro/images/ui-widget.png}

\end{slide}

%----------------------------------------------------------------------

\begin{slide}{2001}\frametitle{Creating user interface}
\vspace*{1.5em}
Using graphics view (C++)
\begin{itemize}
\item 2D canvas (with z axis)
\item handle a high number of 2D items
\item transformations and animations of items
\end{itemize}
\centeredImageFullWidth{qml-intro/images/ui-graphicsview.png}

\end{slide}

%----------------------------------------------------------------------

\begin{slide}{2001}\frametitle{Creating user interface}
\vspace*{1.5em}
Using openGL (C++/OpenGL)
\begin{itemize}
\item full strength of OpenGL
\item user interaction (mouse and keybboard) using Qt API
\item all classes from Qt are usable
\end{itemize}
%\centeredImageFullWidth{qml-intro/images/ui-opengl.png}

\end{slide}

%----------------------------------------------------------------------

\begin{slide}{2001}\frametitle{Creating user interface}
\vspace*{1.5em}
Using QtQuick (QML/Javascript/C++/OpenGL)
\begin{itemize}
\item declarative langage using the QML langage
\item provide smooth animation
\item easily linked to C++
\end{itemize}
\centeredImageFullWidth{qml-intro/images/ui-quick.png}

\end{slide}

%----------------------------------------------------------------------

\begin{slide}{2000}\frametitle{Qt Quick requirement}
\vspace*{1.5em}
\begin{itemize}
\item Platform must support OpenGL ES2
\item Needs at least QtBase, QtJsBackend and QtDeclarative module
\item Other module can be used to add new features:
\begin{itemize}
\item QtGraphicalEffects: add effects like blur, drop shadow...
\item Qt3D: 3D programming in QML
\item QtMultimedia: audio and video items
\item ...
\end{itemize}
\end{itemize}

\end{slide}

%----------------------------------------------------------------------

%%%%%%%%%%%%%%%%%%%%%%%%%%%%%%%%%%%%%%%%%%%%%%%%%%%%%%%%%%%%%%%%%%%%%%%%%%
%
% Copyright (c) 2008-2011, Nokia Corporation and/or its subsidiary(-ies).
% All rights reserved.
%
% This work, unless otherwise expressly stated, is licensed under a
% Creative Commons Attribution-ShareAlike 2.5.
%
% The full license document is available from
% http://creativecommons.org/licenses/by-sa/2.5/legalcode .
%
%%%%%%%%%%%%%%%%%%%%%%%%%%%%%%%%%%%%%%%%%%%%%%%%%%%%%%%%%%%%%%%%%%%%%%%%%%

\subsection{Developing a Hello World Application}

%----------------------------------------------------------------------
\begin{slide}[fragile]{1227}
  \frametitle{``Hello World'' in Qt}
  \flushedImageDoubleWidth{fundamentals/images/helloWorld}
  \medskip
  \begin{itemize}
  \item [] \begin{cpp}
// main.cpp

#include <QtGui>

int main(int argc, char *argv[])
{
  QApplication app(argc, argv);
  QPushButton button("Hello world");
  button.show();
  return app.exec();
}
  \end{cpp}
 \item[]
 \item Program consists of
    \begin{itemize}
    \item \texttt{main.cpp} - application code
    \item \texttt{helloworld.pro} - project file
    \end{itemize}
  \end{itemize}
\demo{fundamentals/ex-helloworld}
\end{slide}

%----------------------------------------------------------------------
\begin{slide}[fragile]{1226}
\frametitle{Project File - \texttt{helloworld.pro}}
\begin{itemize}
\item \texttt{helloworld.pro} file
  \begin{itemize}
  \item lists source and header files
  \item provides project configuration
 \end{itemize}
\begin{qmake}
# File: helloworld.pro
SOURCES  = main.cpp
HEADERS +=          # No headers used
\end{qmake}
\item Assignment to variables
  \begin{itemize}
  \item Possible operators \texttt{=, +=, -=}
  \end{itemize}
\end{itemize}
\doc{qmake-tutorial.html}{qmake tutorial}
\end{slide}

% ----------------------------------------------------------------------
\begin{slide}[fragile]{1225}
  \frametitle{Using qmake}
  \begin{itemize}
  \item \texttt{qmake} tool
    \begin{itemize}
    \item Creates cross-platform make-files
    \end{itemize}
  \item Build project using qmake
  \item[] \begin{shell}
cd helloworld
qmake helloworld.pro # creates Makefile
make                 # compiles and links application
./helloworld         # executes application
  \end{shell}
\item Tip: \texttt{qmake -project}
  \begin{itemize}
  \item Creates default project file based on directory content
  \end{itemize}
\end{itemize}
\doc{qmake-manual.html}{qmake Manual}
\begin{center}\emph{Qt Creator does it all for you}\end{center}
\end{slide}


%%%%%%%%%%%%%%%%%%%%%%%%%%%%%%%%%%%%%%%%%%%%%%%%%%%%%%%%%%%%%%%%%%%%%%%%%%
%
% Copyright (c) 2008-2011, Nokia Corporation and/or its subsidiary(-ies).
% All rights reserved.
%
% This work, unless otherwise expressly stated, is licensed under a
% Creative Commons Attribution-ShareAlike 2.5.
%
% The full license document is available from
% http://creativecommons.org/licenses/by-sa/2.5/legalcode .
%
%%%%%%%%%%%%%%%%%%%%%%%%%%%%%%%%%%%%%%%%%%%%%%%%%%%%%%%%%%%%%%%%%%%%%%%%%%

\subsection{Hello World using Qt Creator}

%----------------------------------------------------------------------
\begin{slide}{1014}\frametitle{QtCreator IDE} \label{qtcreator}
\xConcept{Qt Creator}
\begin{itemize}
  \item Advanced C++ code editor
  \item Integrated GUI layout and forms designer
  \item Project and build management tools
  \item Integrated, context-sensitive help system
  \item Visual debugger
  \item Rapid code navigation tools
  \flushedImage{fundamentals/images/qtcreator_screenshots}
  \item Supports multiple platforms
  \end{itemize}
\end{slide}


%----------------------------------------------------------------------
\begin{slide}{1606}
  \frametitle{Overview of Qt Creator Components}
\centeredImageFullWidth{fundamentals/images/qtcreator-breakdown}
\doccreator{creator-quick-tour.html}{Creator Quick Tour}
\end{slide}

%----------------------------------------------------------------------
\begin{slide}{1605}
  \frametitle{Finding Code -Locator}
 \begin{itemize}
  \item Click on Locator or press Ctrl+K (Mac OS X: Cmd+K)
 \item Type in the file name
 \item Press Return
 \end{itemize}
 \centeredImage{fundamentals/images/qtcreator-locator}
 \newline
 Locator Prefixes
 \begin{itemize}
 \item \textbf{: <class name>} - Go to a symbol definition
 \item \textbf{l <line number> } - Go to a line in the current document
 \item \textbf{? <help topic>} - Go to a help topic
 \item \textbf{o <open document>} - Go to an opened document
 \end{itemize}
\end{slide}

%----------------------------------------------------------------------
\begin{slide}{1604}
  \frametitle{Debugging an Application}
  \begin{itemize}
  \item Debug $->$ Start Debugging (or \texttt{F5})
  \end{itemize}
  \centeredImage{fundamentals/images/qtcreator-debugging}
  \doccreator{creator-debugging.html}{Qt Creator and Debugging}
\end{slide}

\begin{slide}{1603}
  \frametitle{Qt Creator Demo "Hello World"}
  What we'll show:
  \begin{itemize}
  \flushedImage{fundamentals/images/qtcreator-helloworld}
  \item Creation of an empty Qt project
  \item Adding the \texttt{main.cpp} source file
  \item Writing of the Qt Hello World Code
    \begin{itemize}
    \item Showing Locator Features
    \end{itemize}
  \item Running the application
  \item Debugging the application
    \begin{itemize}
    \item Looking up the \texttt{text} property of our button
    \end{itemize}
  \end{itemize}
  \smallskip
  \begin{itemize}
  \item There is more to Qt Creator
    \begin{itemize}
    \item[] \doccreator{index.html}{Qt Creator Manual}
    \end{itemize}
  \end{itemize}
\end{slide}

\subsection{Practical Tips for Developers}

% ----------------------------------------------------------------------
\begin{slide}{0557}
  \frametitle{How much C++ do you need to know?} \label{cpp_needed}
  \begin{itemize}
  \item Objects and classes
    \begin{itemize}
    \item Declaring a class, inheritance, calling member functions etc.
    \end{itemize}
  \item \iConcept{Polymorphism}
    \begin{itemize}
    \item That is virtual methods
    \end{itemize}
  \item \iConcept{Operator overloading}
  \item \iConcept{Templates}
    \begin{itemize}
    \item For the container classes only
    \end{itemize}
  \item No ...
    \begin{itemize}
    \item ... RTTI
    \item ... sophisticated templates
    \item ... exceptions thrown
    \item ...
    \end{itemize}

  \end{itemize}
\end{slide}

% ----------------------------------------------------------------------
\begin{slide}{0558} \frametitle{Qt Documentation} 
  \begin{itemize}
    \item Reference Documentation
    \begin{itemize}
      \item All classes documented
      \item Contains tons of examples
    \end{itemize}
    \item Collection of Howto's and Overviews
    \item A set of Tutorials for Learners
  \end{itemize}
  \vspace{5mm}
  \centeredImage{fundamentals/images/assistant}      
  %%%
\end{slide}

% ----------------------------------------------------------------------
\begin{slide}[fragile]{0559}
  \frametitle{Finding the Answers}\label{findingTheAnswer}
  \xConcept{Answers!Searching for}
  \xConcept{Source code}
  \begin{itemize}
  \item Documentation in Qt Assistant (or QtCreator)
  \item Qt's examples:  \texttt{\$QTDIR/examples}
    \item Qt developer network: \texttt{http://developer.qt.nokia.com/}
 \item Qt Centre Forum: \textit{\texttt{http://www.qtcentre.org/}}
  \item KDE project source code
    \begin{itemize}
    \item \texttt{http://lxr.kde.org/} \textit{(cross-referenced)}.
    \end{itemize}
  \item Mailing lists see \texttt{http://qt.nokia.com/lists}
  \item IRC: 
    \begin{itemize}
    \item On \textit{irc.freenode.org} channel: \texttt{\#qt}
    \end{itemize}
  \end{itemize}
   \begin{block}{Use the source!}
    \textit{Qt's source code is easy to read, and can answer
      questions the reference manual cannot answer!} 
   \end{block}
\end{slide}
% ----------------------------------------------------------------------
\begin{slide}[fragile]{0561}
  \frametitle{Modules and Include files}
  \xConcept{QMake!Modules}
  \begin{itemize}
  \item Qt Modules
    \begin{itemize}
    \item  QtCore, QtGui, QtXml, QtSql, QtNetwork, QtTest ...
    \item[] \doc{modules.html}{Qt Modules}
    \end{itemize}
  \item Enable Qt Module in qmake \texttt{.pro} file: 
    \begin{itemize}
    \item  \verb!QT += network!
    \end{itemize}
  \item Default: qmake projects use QtCore and QtGui
    \begin{itemize}
    \item Any Qt class has a header file.
  \begin{cpp}
#include <QLabel>
#include <QtGui/QLabel>
  \end{cpp}
    \item Any Qt Module has a header file.
  \begin{cpp}
#include <QtGui>
  \end{cpp}
    \end{itemize} 
  \end{itemize}
\end{slide}

% ----------------------------------------------------------------------
\begin{slide}[fragile]{0562}\frametitle{Includes and Compilation Time} 
  \xConcept{Include files}
  \textbf{Module includes}
  \begin{itemize}
  \begin{cpp}
#include <QtGui>
  \end{cpp}
\item Precompiled header and the compiler
  \begin{itemize}
  \item If \textbf{not} supported may add extra compile time
  \item If supported may speed up compilation
  \item Supported on: Windows, Mac OS X, Unix \\
    \doc{qmake-precompiledheaders.html}{qmake precompiled headers}
  \end{itemize}
 \end{itemize}
\textbf{Class includes}
\begin{itemize}
 \begin{cpp}
#include <QLabel>
  \end{cpp}
   \item Reduce compilation time
     \begin{itemize}
     \item Use class includes (\#include <QLabel>)
     \item Forward declarations (\texttt{class QLabel;})
     \end{itemize}
   \end{itemize}
   \vspace{3mm}
   \textit{Place module includes before other includes.}
\end{slide}

%%%%%%%%%%%%%%%%%%%%%%%%%%%%%%%%%%%%%%%%%%%%%%%%%%%%%%%%%%%%%%%%%%%%%%%%%%
%
% Copyright (c) 2008-2011, Nokia Corporation and/or its subsidiary(-ies).
% All rights reserved.
%
% This work, unless otherwise expressly stated, is licensed under a
% Creative Commons Attribution-ShareAlike 2.5.
%
% The full license document is available from
% http://creativecommons.org/licenses/by-sa/2.5/legalcode .
%
%%%%%%%%%%%%%%%%%%%%%%%%%%%%%%%%%%%%%%%%%%%%%%%%%%%%%%%%%%%%%%%%%%%%%%%%%%

%----------------------------------------------------------------------
\subsection{Meet Qt Quick}

%----------------------------------------------------------------------
\begin{slide}{2003}\frametitle{What is Qt Quick?}
\vspace*{1.5em}

A set of technologies including:
\begin{itemize}
\item Declarative markup language: QML
\item Language runtime integrated with Qt
\item Qt Creator IDE support for the QML language
\item Graphical design tool
\item C++ API for integration with Qt applications
\end{itemize}
\end{slide}

%----------------------------------------------------------------------
\begin{slide}{2002}\frametitle{Philosophy of Qt Quick}
\vspace*{1.5em}

\begin{itemize}
\item Intuitive User Interfaces
\item Design-Oriented
\item Rapid Prototyping and Production
\item Easy Deployment
\end{itemize}
\end{slide}

%----------------------------------------------------------------------

\begin{slide}{2001}\frametitle{Design Criteria: Intuitive User Interfaces}

% Pictures of new-style user interfaces


\begin{itemize}
\item More natural ways to interact with applications
\item Easier to predict what elements will do
\item Smooth motion like real-world objects
\end{itemize}

\end{slide}

%----------------------------------------------------------------------

\begin{slide}{2000}\frametitle{Design Criteria: Design-Oriented}

% Pictures of Qt Creator QML integration

Traditional Qt way:
\begin{itemize}
\item Design tools for developers
\item Rich, platform-native widgets
\end{itemize}

% Qt way to QML way (rich widgets -> poor elements)

The Qt Quick way:
\begin{itemize}
\item Development tools and processes that are accessible to designers
\item Lightweight, customizable elements
\end{itemize}

\end{slide}

%----------------------------------------------------------------------

\begin{slide}{1999}\frametitle{Design Criteria: Rapid Prototyping and Production}

\begin{itemize}
    \item No compilation step
    \item JavaScript used as scripting language
    \begin{itemize}
       \item See \emph{JavaScript: The Definitive Guide} 
       \item See \externalLink{https://developer.mozilla.org/en/JavaScript}
    \end{itemize}
    \item Helpful to understand:
    \begin{itemize}
       \item HTML, CSS 
       \item ... but not required  
    \end{itemize} 
    \item Requires little or no programming experience
\end{itemize}

% JavaScript is a prototype-based programming language
% - slightly different paradigm to object-oriented programming
% http://www.ecma-international.org/publications/standards/Ecma-262.htm

\end{slide}

%----------------------------------------------------------------------

\begin{slide}{1998}\frametitle{Design Criteria: Easy Deployment}

\begin{itemize}
\item Self-contained packages
\item Installation is optional
\item Network transparent
  \begin{itemize}
  \item allows deployment over a network
  \item rich client front end to online services
  \end{itemize}
\end{itemize}

\end{slide}

%----------------------------------------------------------------------

%%%%%%%%%%%%%%%%%%%%%%%%%%%%%%%%%%%%%%%%%%%%%%%%%%%%%%%%%%%%%%%%%%%%%%%%%%
%
% Copyright (c) 2008-2011, Nokia Corporation and/or its subsidiary(-ies).
% All rights reserved.
%
% This work, unless otherwise expressly stated, is licensed under a
% Creative Commons Attribution-ShareAlike 2.5.
%
% The full license document is available from
% http://creativecommons.org/licenses/by-sa/2.5/legalcode .
%
%%%%%%%%%%%%%%%%%%%%%%%%%%%%%%%%%%%%%%%%%%%%%%%%%%%%%%%%%%%%%%%%%%%%%%%%%%

%----------------------------------------------------------------------

\subsection{Concepts}
\begin{slide}{2023}\frametitle{What is QML?}
\vspace*{0.5em}

% Basic concepts - what is Qt Quick, declarative UI?
Declarative language for User Interface elements:

\vspace*{0.5em}
\begin{itemize}
    \item Describes the user interface
    \begin{itemize}
    \item What elements look like
    \item How elements behave
    \end{itemize}
\item UI specified as tree of elements with properties     
\end{itemize}

\end{slide}

%----------------------------------------------------------------------

\begin{slide}{2022}\frametitle{A Tree of Elements}

% Insert tree <-> UI comparison image here.

% Other than the fundamental difference inheritance/parent-child trees that
% needs to be communicated, is there some fundamental difference between
% nested elements and elements listed using the children property.
% (Probably an implementation detail we don't want to address.)
                               
\centeredImageDoubleWidth{qml-intro/images/tree-of-elements.pdf}

\vspace*{1em}
Let's start with an example...

\end{slide}

%----------------------------------------------------------------------

\begin{slide}[fragile]{2021}\frametitle{Viewing an Example}

% declarative-uis/concepts/rectangle.qml
\inputqml{qml-intro/colorized/rectangle}

% Launching an example - the QML viewer, some simple markup
\vspace*{1em}
\begin{itemize}
\item Locate the example: \texttt{rectangle.qml}
\item Launch the QML viewer:\\
\consolecode{qmlscene~rectangle.qml}

% qml(viewer) shows the top level element at the correct size initially, but
% it will be resized with the window.
%
% Default values for x and y.

%%% Explain QML viewer                  

\demo{qml-intro/ex-concepts/rectangle.qml}

\end{itemize}

\end{slide}

%----------------------------------------------------------------------

\begin{slide}{2020}\frametitle{Example Concepts}

\begin{qml}
\qtt{\qc{keyword}{import}~\qc{class}{QtQuick}~\qc{number}{2.0}}\\
\vspace*{0.5em}
\qtt{\qc{lightgray}{// Define a light blue square}}\\
\vspace*{0.5em}
\qtt{\qc{lightgray}{Rectangle~\{}}\\
\qtt{~~~~\qc{lightgray}{width:~400;~height:~400}}\\
\qtt{~~~~\qc{lightgray}{color:~"lightblue"}}\\
\qtt{\qc{lightgray}{\}}}
\end{qml}

\vspace*{1em}
\begin{itemize}
\item To use Qt's features, \qic{keyword}{import} the \qic{class}{Qt} module
\item Specify a version to get the features you want
  \begin{itemize}
  \item only imports the features from that version
  \item will not use features from later versions, even if available
  \item will not fall back to features from an earlier version
  \end{itemize}
\item Guarantees that the behavior of the code will not change
  \begin{itemize}
  \item modules include support for multiple versions
  \item an upgraded module retains support for older versions of itself
  \end{itemize}
\end{itemize}

%%% Imports of other versions

\end{slide}

%----------------------------------------------------------------------

\begin{slide}{2019}\frametitle{Example Concepts}

\begin{qml}
\qtt{\qc{keyword}{import}~\qc{class}{QtQuick}~\qc{number}{2.0}}\\
\vspace*{0.5em}
\qtt{\qc{comment}{// Define a light blue square}}\\
\vspace*{0.5em}
\qtt{\qc{lightgray}{Rectangle~\{}}\\
\qtt{~~~~\qc{lightgray}{width:~400;~height:~400}}\\
\qtt{~~~~\qc{lightgray}{color:~"lightblue"}}\\
\qtt{\qc{lightgray}{\}}}
\end{qml}

\vspace*{1em}
\begin{itemize}
\item Use \qic{comment}{//} to add single line comments
\item Put multi-line comments inside \qic{comment}{/*} and \qic{comment}{*/}
\end{itemize}

\end{slide}

%----------------------------------------------------------------------

\begin{slide}{2018}\frametitle{Example Concepts}

\begin{qml}
\qtt{\qc{lightgray}{import~QtQuick~2.0}}\\
\vspace*{0.5em}
\qtt{\qc{lightgray}{// Define a light blue square}}\\
\vspace*{0.5em}
\qtt{\qc{class}{Rectangle}~\{}\\
\qtt{~~~~\qc{lightgray}{width:~400;~height:~400}}\\
\qtt{~~~~\qc{lightgray}{color:~"lightblue"}}\\
\qtt{\}}
\end{qml}

\vspace*{1em}
\begin{itemize}
\item Declare the elements you want to use
\item Each element has a body between \texttt{\{} and \texttt{\}}
\item A set of default elements are included in the \texttt{\qc{class}{Qt}} module
\end{itemize}

\end{slide}

%----------------------------------------------------------------------

\begin{slide}{2017}\frametitle{Example Concepts}

\begin{qml}
\qtt{\qc{lightgray}{import~QtQuick~2.0}}\\
\vspace*{0.5em}
\qtt{\qc{lightgray}{// Define a light blue square}}\\
\vspace*{0.5em}
\qtt{\qc{lightgray}{Rectangle~\{}}\\
\qtt{~~~~\qc{type}{width}:~\qc{number}{400};~\qc{type}{height}:~\qc{number}{400}}\\
\qtt{~~~~\qc{type}{color}:~\qc{string}{"lightblue"}}\\
\qtt{\qc{lightgray}{\}}}
\end{qml}

\vspace*{1em}
\begin{itemize}
\item Elements contain properties
\item Each property is defined using its name and a value
\item \texttt{\qc{type}{name}~:~value}
\end{itemize}

\end{slide}

%----------------------------------------------------------------------

\begin{slide}{2016}\frametitle{Example Summary}

\flushedImage{qml-intro/images/rectangle.pdf}
\begin{qml}
\qtt{\qc{keyword}{import}~\qc{class}{QtQuick}~\qc{number}{2.0}}\\
\vspace*{0.5em}
\qtt{\qc{comment}{// Define a light blue square}}\\
\vspace*{0.5em}
\qtt{\qc{class}{Rectangle}~\{}\\
\qtt{~~~~\qc{type}{width}:~\qc{number}{400};~\qc{type}{height}:~\qc{number}{400}}\\
\qtt{~~~~\qc{type}{color}:~\qc{string}{"lightblue"}}\\
\qtt{\}}
\end{qml}

\vspace*{1em}
\begin{itemize}
\item A \texttt{\textcolor{class}{Rectangle}} element
  \begin{itemize}
  \item with a body: \texttt{\{ ... \}}
  \item containing \qic{type}{width}, \qic{type}{height} and \qic{type}{color}
  properties
  \item separated by line breaks or semicolons
  \end{itemize}
\item Running the example in the \texttt{qml} viewer
  \begin{itemize}
  \item the viewer window will be 400 by 400
  \end{itemize}
\end{itemize}

\end{slide}

%----------------------------------------------------------------------

\begin{slide}{2015}\frametitle{Elements}

\begin{itemize}
\item Elements are structures in the markup language
  \begin{itemize}
  \item represent visible and non-visible parts
  \end{itemize}
\item \qic{class}{Item} is the base type of visible elements
  \begin{itemize}
  \item not visible itself
  \item has a position, dimensions
  \item usually used to group visual elements
  \item often used as the top-level element
  \item \qic{class}{Rectangle}, \qic{class}{Text}, \qic{class}{TextInput}, ...
  \end{itemize}
\item Non-visible elements:
  \begin{itemize}
  \item states, transitions, ...
  \item models, paths, ...
  \item gradients, timers, etc.
  \end{itemize}
\item Elements contain properties
  \begin{itemize}
  \item can also be extended with custom properties
  \end{itemize}
\end{itemize}

% A full list of elements and items is available in the documentation.
% Alternatively, you can look in the Qt Creator sources for the XML
% file that defines the basic types.

%%% Items based on QGraphicsItems
%%% qml viewer uses QGraphicsView
%%% Element hierarchy

\doc{qdeclarativeelements.html}{QML Elements}
\end{slide}

%----------------------------------------------------------------------

\begin{slide}{2014}\frametitle{Properties}

Elements are described by properties:

\begin{itemize}
\item Simple name-value definitions
  \begin{itemize}
  \item \qic{type}{width}, \qic{type}{height}, \qic{type}{color}, ...
  \item with default values
  \item each has a well-defined type
  \item separated by semicolons or line breaks
  \end{itemize}
  \vspace*{0.5em}
\item Used for
  \begin{itemize}
  \item identifying elements (\qic{type}{id} property)
  \item customizing their appearance
  \item changing their behavior
  \end{itemize}
  \vspace*{0.5em}
\end{itemize}

\end{slide}

%----------------------------------------------------------------------

\begin{slide}{2013}\frametitle{Property Examples}

\begin{itemize}
\item \textbf{Standard properties} can be given values:

\vspace*{0.25em}
\inputqml{qml-intro/colorized/standard-properties}

\vspace*{0.5em}
\item \textbf{Grouped properties} keep related properties together:

\vspace*{0.25em}
\inputqml{qml-intro/colorized/grouped-properties}

\vspace*{0.5em}
\item \textbf{Identity property} gives the element a name:

\vspace*{0.25em}
\inputqml{qml-intro/colorized/id-property}
\end{itemize}

\end{slide}

%----------------------------------------------------------------------

\begin{slide}{2012}\frametitle{Property Examples}

\begin{itemize}
\item \textbf{Attached properties} are applied to elements:

\vspace*{0.25em}
\inputqml{qml-intro/colorized/attached-properties}

\begin{itemize}
\item \qic{type}{KeyNagivation.tab} is not a standard property of \qic{class}{TextInput}
\item is a standard property that is attached to elements
\end{itemize}

\vspace*{0.5em}
\item \textbf{Custom properties} can be added to any element:

\vspace*{0.25em}
\inputqml{qml-intro/colorized/custom-property}
\end{itemize}

\end{slide}

%%% Add examples of properties with some standard elements
%%% Mention attached properties, but explain them when showing key input
%%% Highlight custom properties, but explain them 

%----------------------------------------------------------------------

\begin{slide}{2011}\frametitle{Binding Properties}

\flushedImage{qml-intro/images/expressions.png}
% declarative-uis/concepts/expressions.qml
\inputqml{qml-intro/colorized/expressions}

\demo{qml-intro/ex-concepts/expressions.qml}

\begin{itemize}
\item Properties can contain expressions
  \begin{itemize}
  \item see above: \qic{type}{width} is twice the \qic{type}{height}
  \end{itemize}
\item Not just initial assignments
\item Expressions are evaluated when needed
\end{itemize}

\doc{propertybinding.html}{Property Binding}
\end{slide}

%----------------------------------------------------------------------

\begin{slide}{2010}\frametitle{Identifying Elements}

% id is an especially important property because it provides a name
% for an element
% Coding convention - always first property?

The \qic{type}{id} property defines an identity for an element

\begin{itemize}
\item Lets other elements refer to it
  \begin{itemize}
  \item for relative alignment and positioning
  \item to use its properties
  \item to change its properties (e.g., for animation)
  \item for re-use of common elements (e.g., gradients, images)
  \end{itemize}
\item Used to \textit{create relationships} between elements
\end{itemize}

%%% Show Qt Creator id highlighting feature

\end{slide}

%----------------------------------------------------------------------

\begin{slide}{2009}\frametitle{Using Identities}

\flushedImageDoubleWidth{qml-intro/images/identity.png}
% declarative-uis/concepts/identity.qml
\inputqml{qml-intro/colorized/identity} 

\demo{qml-intro/ex-concepts/identity.qml}

\end{slide}

%----------------------------------------------------------------------

\begin{slide}{2008}\frametitle{Using Identities}

\flushedImageDoubleWidth{qml-intro/images/identity.png}
% declarative-uis/concepts/identity.qml
\begin{qml}
\qtt{\qc{class}{Text}~\{}\\
\qtt{~~~~\qc{type}{id}:~textElement}\\
\qtt{~~~~\qc{lightgray}{x:~50;~y:~25}}\\
\qtt{~~~~\qc{lightgray}{text:~"Qt~Quick"}}\\
\qtt{~~~~\qc{lightgray}{font.family:~"Helvetica";~font.pixelSize:~50}}\\
\qtt{\}}\\
\vspace*{0.5em}
\qtt{\qc{class}{Rectangle}~\{}\\
\qtt{~~~~\qc{lightgray}{x:~50;~y:~75;~height:~5}}\\
\qtt{~~~~\qc{type}{width}:~textElement.width}\\
\qtt{~~~~\qc{lightgray}{color:~"green"}}\\
\qtt{\}}
\end{qml}      

\vspace*{0.5em}
\begin{itemize}
\item \qic{class}{Text} element has the identity, \qtt{textElement}
\item \qic{type}{width} of \qic{class}{Rectangle} bound to \qic{type}{width} of
      \qtt{textElement}
\item Try using \qic{class}{TextInput} instead of \qic{class}{Text}
\end{itemize}

%%% Can identities be changed/reassigned?
%%% Restrictions on IDs? Can we use "parent"?

\end{slide}

%----------------------------------------------------------------------
\begin{slide}{2007}\frametitle{Methods}

\begin{itemize}
\item Most features are accessed via properties
\item Some actions cannot be exposed as properties
\item Elements have methods to perform actions:
  \begin{itemize}
  \item \qic{class}{TextInput} has a \qic{type}{selectAll()} method
  \item \qic{class}{Timer} has \qic{type}{start()}, \qic{type}{stop()}
        and \qic{type}{restart()} methods
  \item \qic{class}{Particles} has a \qic{type}{burst()} method
  \end{itemize}
\item All methods are public in QML
\item Other methods are used to convert values between types:
  \begin{itemize}
  \item \qic{class}{Qt}.\qic{type}{formatDateTime(datetime, format)}
  \item \qic{class}{Qt}.\qic{type}{md5(data)}
  \item \qic{class}{Qt}.\qic{type}{tint(baseColor, tintColor)}
  \end{itemize}
\end{itemize}

\end{slide}

%----------------------------------------------------------------------

\begin{slide}{2006}\frametitle{Basic Types}
Property values can have different types:
\begin{itemize}
\item Numbers (int and real): \qic{number}{400} and \qic{number}{1.5}
\item Boolean values: \qic{number}{true} and \qic{number}{false}
\item Strings: \qic{string}{"Hello Qt"}
\item Constants: \qic{class}{AlignLeft}
\vspace*{0.25em}
\item Lists: \qic{black}{[ ... ]}
  \begin{itemize}
  \item lists with one item can be written as just the item itself
  \end{itemize}
  \vspace*{0.25em}
\item Scripts:
  \begin{itemize}
  \item included directly in property definitions
%  \item can be written inside \qic{black}{\{ ... \}} for clarity
  \end{itemize}
  \vspace*{0.25em}
\item Other types:
  \begin{itemize}
  \item colors, dates, times, rects, points, sizes, 3D vectors, ...
  \item usually created using constructors
  \end{itemize}
\end{itemize}

%%% None/null/nil type?

\doc{qdeclarativeglobalobject.html\#types}{QML Types}
\end{slide}

%----------------------------------------------------------------------
\begin{slide}{2005}\frametitle{Summary}

\begin{itemize}
\item QML defines user interfaces using elements and properties
  \begin{itemize}
  \item elements are the structures in QML source code
  \item items are visible elements
  \end{itemize}
\item Standard elements contain properties and methods
  \begin{itemize}
  \item properties can be changed from their default values
  \item property values can be expressions
  \item \qic{type}{id} properties give identities to elements
  \end{itemize}
\item Properties are bound together
  \begin{itemize}
  \item when a property changes, the properties that reference it are updated
  \end{itemize}
\item Some standard elements define methods
\item A range of built-in types is provided
\end{itemize}

%%% Show property, method syntax

\end{slide}

%%% Scopes - lexical vs. dynamic
                                   

%----------------------------------------------------------------------
\begin{slide}{2004}\frametitle{Exercise}

\begin{enumerate}
\item How do you request features from a certain version of Qt?
% Import Qt with the relevant version.
\item What is the difference between \qic{class}{Rectangle} and \qic{type}{width}?
% Rectangle is an element; width is a property.
\item How would you create an element with an identity?
% Create an id property with a suitable name
\item What syntax do you use to refer to a property of another element?
% identity.property
\end{enumerate}

\end{slide}

%----------------------------------------------------------------------
\begin{slide}{1664}\frametitle{Exercise~\textendash~Items}

% Something about nesting items and relative positions

\flushedImage{qml-intro/images/nested-elements.png}
The image on the right shows two items and two child items inside
a 400 $\times$ 400 rectangle.\\

\begin{enumerate}
\item Recreate the scene using \qic{class}{Rectangle} items.
\item Can items overlap?\\
{\small Experiment by moving the light blue or green\\
rectangles.}
\item Can child items be displayed outside their parents?\\
{\small Experiment by giving one of the child items negative coordinates.}
\end{enumerate}

\end{slide}

