%%%%%%%%%%%%%%%%%%%%%%%%%%%%%%%%%%%%%%%%%%%%%%%%%%%%%%%%%%%%%%%%%%%%%%%%%%
%
% Copyright (c) 2008-2011, Nokia Corporation and/or its subsidiary(-ies).
% All rights reserved.
%
% This work, unless otherwise expressly stated, is licensed under a
% Creative Commons Attribution-ShareAlike 2.5.
%
% The full license document is available from
% http://creativecommons.org/licenses/by-sa/2.5/legalcode .
%
%%%%%%%%%%%%%%%%%%%%%%%%%%%%%%%%%%%%%%%%%%%%%%%%%%%%%%%%%%%%%%%%%%%%%%%%%%

%----------------------------------------------------------------------

\subsection{Concepts}
\begin{slide}{2023}\frametitle{What is QML?}
\vspace*{0.5em}

% Basic concepts - what is Qt Quick, declarative UI?
Declarative language for User Interface elements:

\vspace*{0.5em}
\begin{itemize}
    \item Describes the user interface
    \begin{itemize}
    \item What elements look like
    \item How elements behave
    \end{itemize}
\item UI specified as tree of elements with properties     
\end{itemize}

\end{slide}

%----------------------------------------------------------------------

\begin{slide}{2022}\frametitle{A Tree of Elements}

% Insert tree <-> UI comparison image here.

% Other than the fundamental difference inheritance/parent-child trees that
% needs to be communicated, is there some fundamental difference between
% nested elements and elements listed using the children property.
% (Probably an implementation detail we don't want to address.)
                               
\centeredImageDoubleWidth{qml-intro/images/tree-of-elements.pdf}

\vspace*{1em}
Let's start with an example...

\end{slide}

%----------------------------------------------------------------------

\begin{slide}[fragile]{2021}\frametitle{Viewing an Example}

% declarative-uis/concepts/rectangle.qml
\inputqml{qml-intro/colorized/rectangle}

% Launching an example - the QML viewer, some simple markup
\vspace*{1em}
\begin{itemize}
\item Locate the example: \texttt{rectangle.qml}
\item Launch the QML viewer:\\
\consolecode{qmlscene~rectangle.qml}

% qml(viewer) shows the top level element at the correct size initially, but
% it will be resized with the window.
%
% Default values for x and y.

%%% Explain QML viewer                  

\demo{qml-intro/ex-concepts/rectangle.qml}

\end{itemize}

\end{slide}

%----------------------------------------------------------------------

\begin{slide}{2020}\frametitle{Example Concepts}

\begin{qml}
\qtt{\qc{keyword}{import}~\qc{class}{QtQuick}~\qc{number}{2.0}}\\
\vspace*{0.5em}
\qtt{\qc{lightgray}{// Define a light blue square}}\\
\vspace*{0.5em}
\qtt{\qc{lightgray}{Rectangle~\{}}\\
\qtt{~~~~\qc{lightgray}{width:~400;~height:~400}}\\
\qtt{~~~~\qc{lightgray}{color:~"lightblue"}}\\
\qtt{\qc{lightgray}{\}}}
\end{qml}

\vspace*{1em}
\begin{itemize}
\item To use Qt's features, \qic{keyword}{import} the \qic{class}{Qt} module
\item Specify a version to get the features you want
  \begin{itemize}
  \item only imports the features from that version
  \item will not use features from later versions, even if available
  \item will not fall back to features from an earlier version
  \end{itemize}
\item Guarantees that the behavior of the code will not change
  \begin{itemize}
  \item modules include support for multiple versions
  \item an upgraded module retains support for older versions of itself
  \end{itemize}
\end{itemize}

%%% Imports of other versions

\end{slide}

%----------------------------------------------------------------------

\begin{slide}{2019}\frametitle{Example Concepts}

\begin{qml}
\qtt{\qc{keyword}{import}~\qc{class}{QtQuick}~\qc{number}{2.0}}\\
\vspace*{0.5em}
\qtt{\qc{comment}{// Define a light blue square}}\\
\vspace*{0.5em}
\qtt{\qc{lightgray}{Rectangle~\{}}\\
\qtt{~~~~\qc{lightgray}{width:~400;~height:~400}}\\
\qtt{~~~~\qc{lightgray}{color:~"lightblue"}}\\
\qtt{\qc{lightgray}{\}}}
\end{qml}

\vspace*{1em}
\begin{itemize}
\item Use \qic{comment}{//} to add single line comments
\item Put multi-line comments inside \qic{comment}{/*} and \qic{comment}{*/}
\end{itemize}

\end{slide}

%----------------------------------------------------------------------

\begin{slide}{2018}\frametitle{Example Concepts}

\begin{qml}
\qtt{\qc{lightgray}{import~QtQuick~2.0}}\\
\vspace*{0.5em}
\qtt{\qc{lightgray}{// Define a light blue square}}\\
\vspace*{0.5em}
\qtt{\qc{class}{Rectangle}~\{}\\
\qtt{~~~~\qc{lightgray}{width:~400;~height:~400}}\\
\qtt{~~~~\qc{lightgray}{color:~"lightblue"}}\\
\qtt{\}}
\end{qml}

\vspace*{1em}
\begin{itemize}
\item Declare the elements you want to use
\item Each element has a body between \texttt{\{} and \texttt{\}}
\item A set of default elements are included in the \texttt{\qc{class}{Qt}} module
\end{itemize}

\end{slide}

%----------------------------------------------------------------------

\begin{slide}{2017}\frametitle{Example Concepts}

\begin{qml}
\qtt{\qc{lightgray}{import~QtQuick~2.0}}\\
\vspace*{0.5em}
\qtt{\qc{lightgray}{// Define a light blue square}}\\
\vspace*{0.5em}
\qtt{\qc{lightgray}{Rectangle~\{}}\\
\qtt{~~~~\qc{type}{width}:~\qc{number}{400};~\qc{type}{height}:~\qc{number}{400}}\\
\qtt{~~~~\qc{type}{color}:~\qc{string}{"lightblue"}}\\
\qtt{\qc{lightgray}{\}}}
\end{qml}

\vspace*{1em}
\begin{itemize}
\item Elements contain properties
\item Each property is defined using its name and a value
\item \texttt{\qc{type}{name}~:~value}
\end{itemize}

\end{slide}

%----------------------------------------------------------------------

\begin{slide}{2016}\frametitle{Example Summary}

\flushedImage{qml-intro/images/rectangle.pdf}
\begin{qml}
\qtt{\qc{keyword}{import}~\qc{class}{QtQuick}~\qc{number}{2.0}}\\
\vspace*{0.5em}
\qtt{\qc{comment}{// Define a light blue square}}\\
\vspace*{0.5em}
\qtt{\qc{class}{Rectangle}~\{}\\
\qtt{~~~~\qc{type}{width}:~\qc{number}{400};~\qc{type}{height}:~\qc{number}{400}}\\
\qtt{~~~~\qc{type}{color}:~\qc{string}{"lightblue"}}\\
\qtt{\}}
\end{qml}

\vspace*{1em}
\begin{itemize}
\item A \texttt{\textcolor{class}{Rectangle}} element
  \begin{itemize}
  \item with a body: \texttt{\{ ... \}}
  \item containing \qic{type}{width}, \qic{type}{height} and \qic{type}{color}
  properties
  \item separated by line breaks or semicolons
  \end{itemize}
\item Running the example in the \texttt{qml} viewer
  \begin{itemize}
  \item the viewer window will be 400 by 400
  \end{itemize}
\end{itemize}

\end{slide}

%----------------------------------------------------------------------

\begin{slide}{2015}\frametitle{Elements}

\begin{itemize}
\item Elements are structures in the markup language
  \begin{itemize}
  \item represent visible and non-visible parts
  \end{itemize}
\item \qic{class}{Item} is the base type of visible elements
  \begin{itemize}
  \item not visible itself
  \item has a position, dimensions
  \item usually used to group visual elements
  \item often used as the top-level element
  \item \qic{class}{Rectangle}, \qic{class}{Text}, \qic{class}{TextInput}, ...
  \end{itemize}
\item Non-visible elements:
  \begin{itemize}
  \item states, transitions, ...
  \item models, paths, ...
  \item gradients, timers, etc.
  \end{itemize}
\item Elements contain properties
  \begin{itemize}
  \item can also be extended with custom properties
  \end{itemize}
\end{itemize}

% A full list of elements and items is available in the documentation.
% Alternatively, you can look in the Qt Creator sources for the XML
% file that defines the basic types.

%%% Items based on QGraphicsItems
%%% qml viewer uses QGraphicsView
%%% Element hierarchy

\doc{qdeclarativeelements.html}{QML Elements}
\end{slide}

%----------------------------------------------------------------------

\begin{slide}{2014}\frametitle{Properties}

Elements are described by properties:

\begin{itemize}
\item Simple name-value definitions
  \begin{itemize}
  \item \qic{type}{width}, \qic{type}{height}, \qic{type}{color}, ...
  \item with default values
  \item each has a well-defined type
  \item separated by semicolons or line breaks
  \end{itemize}
  \vspace*{0.5em}
\item Used for
  \begin{itemize}
  \item identifying elements (\qic{type}{id} property)
  \item customizing their appearance
  \item changing their behavior
  \end{itemize}
  \vspace*{0.5em}
\end{itemize}

\end{slide}

%----------------------------------------------------------------------

\begin{slide}{2013}\frametitle{Property Examples}

\begin{itemize}
\item \textbf{Standard properties} can be given values:

\vspace*{0.25em}
\inputqml{qml-intro/colorized/standard-properties}

\vspace*{0.5em}
\item \textbf{Grouped properties} keep related properties together:

\vspace*{0.25em}
\inputqml{qml-intro/colorized/grouped-properties}

\vspace*{0.5em}
\item \textbf{Identity property} gives the element a name:

\vspace*{0.25em}
\inputqml{qml-intro/colorized/id-property}
\end{itemize}

\end{slide}

%----------------------------------------------------------------------

\begin{slide}{2012}\frametitle{Property Examples}

\begin{itemize}
\item \textbf{Attached properties} are applied to elements:

\vspace*{0.25em}
\inputqml{qml-intro/colorized/attached-properties}

\begin{itemize}
\item \qic{type}{KeyNagivation.tab} is not a standard property of \qic{class}{TextInput}
\item is a standard property that is attached to elements
\end{itemize}

\vspace*{0.5em}
\item \textbf{Custom properties} can be added to any element:

\vspace*{0.25em}
\inputqml{qml-intro/colorized/custom-property}
\end{itemize}

\end{slide}

%%% Add examples of properties with some standard elements
%%% Mention attached properties, but explain them when showing key input
%%% Highlight custom properties, but explain them 

%----------------------------------------------------------------------

\begin{slide}{2011}\frametitle{Binding Properties}

\flushedImage{qml-intro/images/expressions.png}
% declarative-uis/concepts/expressions.qml
\inputqml{qml-intro/colorized/expressions}

\demo{qml-intro/ex-concepts/expressions.qml}

\begin{itemize}
\item Properties can contain expressions
  \begin{itemize}
  \item see above: \qic{type}{width} is twice the \qic{type}{height}
  \end{itemize}
\item Not just initial assignments
\item Expressions are evaluated when needed
\end{itemize}

\doc{propertybinding.html}{Property Binding}
\end{slide}

%----------------------------------------------------------------------

\begin{slide}{2010}\frametitle{Identifying Elements}

% id is an especially important property because it provides a name
% for an element
% Coding convention - always first property?

The \qic{type}{id} property defines an identity for an element

\begin{itemize}
\item Lets other elements refer to it
  \begin{itemize}
  \item for relative alignment and positioning
  \item to use its properties
  \item to change its properties (e.g., for animation)
  \item for re-use of common elements (e.g., gradients, images)
  \end{itemize}
\item Used to \textit{create relationships} between elements
\end{itemize}

%%% Show Qt Creator id highlighting feature

\end{slide}

%----------------------------------------------------------------------

\begin{slide}{2009}\frametitle{Using Identities}

\flushedImageDoubleWidth{qml-intro/images/identity.png}
% declarative-uis/concepts/identity.qml
\inputqml{qml-intro/colorized/identity} 

\demo{qml-intro/ex-concepts/identity.qml}

\end{slide}

%----------------------------------------------------------------------

\begin{slide}{2008}\frametitle{Using Identities}

\flushedImageDoubleWidth{qml-intro/images/identity.png}
% declarative-uis/concepts/identity.qml
\begin{qml}
\qtt{\qc{class}{Text}~\{}\\
\qtt{~~~~\qc{type}{id}:~textElement}\\
\qtt{~~~~\qc{lightgray}{x:~50;~y:~25}}\\
\qtt{~~~~\qc{lightgray}{text:~"Qt~Quick"}}\\
\qtt{~~~~\qc{lightgray}{font.family:~"Helvetica";~font.pixelSize:~50}}\\
\qtt{\}}\\
\vspace*{0.5em}
\qtt{\qc{class}{Rectangle}~\{}\\
\qtt{~~~~\qc{lightgray}{x:~50;~y:~75;~height:~5}}\\
\qtt{~~~~\qc{type}{width}:~textElement.width}\\
\qtt{~~~~\qc{lightgray}{color:~"green"}}\\
\qtt{\}}
\end{qml}      

\vspace*{0.5em}
\begin{itemize}
\item \qic{class}{Text} element has the identity, \qtt{textElement}
\item \qic{type}{width} of \qic{class}{Rectangle} bound to \qic{type}{width} of
      \qtt{textElement}
\item Try using \qic{class}{TextInput} instead of \qic{class}{Text}
\end{itemize}

%%% Can identities be changed/reassigned?
%%% Restrictions on IDs? Can we use "parent"?

\end{slide}

%----------------------------------------------------------------------
\begin{slide}{2007}\frametitle{Methods}

\begin{itemize}
\item Most features are accessed via properties
\item Some actions cannot be exposed as properties
\item Elements have methods to perform actions:
  \begin{itemize}
  \item \qic{class}{TextInput} has a \qic{type}{selectAll()} method
  \item \qic{class}{Timer} has \qic{type}{start()}, \qic{type}{stop()}
        and \qic{type}{restart()} methods
  \item \qic{class}{Particles} has a \qic{type}{burst()} method
  \end{itemize}
\item All methods are public in QML
\item Other methods are used to convert values between types:
  \begin{itemize}
  \item \qic{class}{Qt}.\qic{type}{formatDateTime(datetime, format)}
  \item \qic{class}{Qt}.\qic{type}{md5(data)}
  \item \qic{class}{Qt}.\qic{type}{tint(baseColor, tintColor)}
  \end{itemize}
\end{itemize}

\end{slide}

%----------------------------------------------------------------------

\begin{slide}{2006}\frametitle{Basic Types}
Property values can have different types:
\begin{itemize}
\item Numbers (int and real): \qic{number}{400} and \qic{number}{1.5}
\item Boolean values: \qic{number}{true} and \qic{number}{false}
\item Strings: \qic{string}{"Hello Qt"}
\item Constants: \qic{class}{AlignLeft}
\vspace*{0.25em}
\item Lists: \qic{black}{[ ... ]}
  \begin{itemize}
  \item lists with one item can be written as just the item itself
  \end{itemize}
  \vspace*{0.25em}
\item Scripts:
  \begin{itemize}
  \item included directly in property definitions
%  \item can be written inside \qic{black}{\{ ... \}} for clarity
  \end{itemize}
  \vspace*{0.25em}
\item Other types:
  \begin{itemize}
  \item colors, dates, times, rects, points, sizes, 3D vectors, ...
  \item usually created using constructors
  \end{itemize}
\end{itemize}

%%% None/null/nil type?

\doc{qdeclarativeglobalobject.html\#types}{QML Types}
\end{slide}

%----------------------------------------------------------------------
\begin{slide}{2005}\frametitle{Summary}

\begin{itemize}
\item QML defines user interfaces using elements and properties
  \begin{itemize}
  \item elements are the structures in QML source code
  \item items are visible elements
  \end{itemize}
\item Standard elements contain properties and methods
  \begin{itemize}
  \item properties can be changed from their default values
  \item property values can be expressions
  \item \qic{type}{id} properties give identities to elements
  \end{itemize}
\item Properties are bound together
  \begin{itemize}
  \item when a property changes, the properties that reference it are updated
  \end{itemize}
\item Some standard elements define methods
\item A range of built-in types is provided
\end{itemize}

%%% Show property, method syntax

\end{slide}

%%% Scopes - lexical vs. dynamic
                                   

%----------------------------------------------------------------------
\begin{slide}{2004}\frametitle{Exercise}

\begin{enumerate}
\item How do you request features from a certain version of Qt?
% Import Qt with the relevant version.
\item What is the difference between \qic{class}{Rectangle} and \qic{type}{width}?
% Rectangle is an element; width is a property.
\item How would you create an element with an identity?
% Create an id property with a suitable name
\item What syntax do you use to refer to a property of another element?
% identity.property
\end{enumerate}

\end{slide}

