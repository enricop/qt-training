%%%%%%%%%%%%%%%%%%%%%%%%%%%%%%%%%%%%%%%%%%%%%%%%%%%%%%%%%%%%%%%%%%%%%%%%%%
%
% Copyright (c) 2008-2011, Nokia Corporation and/or its subsidiary(-ies).
% All rights reserved.
%
% This work, unless otherwise expressly stated, is licensed under a
% Creative Commons Attribution-ShareAlike 2.5.
%
% The full license document is available from
% http://creativecommons.org/licenses/by-sa/2.5/legalcode .
%
%%%%%%%%%%%%%%%%%%%%%%%%%%%%%%%%%%%%%%%%%%%%%%%%%%%%%%%%%%%%%%%%%%%%%%%%%%

%----------------------------------------------------------------------
\subsection{Meet Qt}

%----------------------------------------------------------------------
\begin{slide}{2003}\frametitle{The Qt SDK}
\vspace*{0.5em}
\centeredImageFullWidth{qml-intro/images/qt_architecture.png}

\end{slide}

%----------------------------------------------------------------------
\begin{slide}{2002}\frametitle{Qt modules}
\vspace*{1.5em}
The Qt framework is split in modules:
\begin{itemize}
\item Each module has is own repository
\item QtBase is the only mandatory module for all Qt application
\item QtTools is a special module containing Qt applications
\item examples: QtDeclarative, QtWebKit, QtMultimedia...
\end{itemize}\medskip
Modules are themselves split into one or more libraries
\begin{itemize}
\item libraries are linked to your applications
\item they group a set of common features (xml, dbus, network...)
\item QtCore (part of QtBase) is mandatory for all Qt applications
\item examples: QtNetwork, QtGui, QtDeclrative
\end{itemize}

\end{slide}

%----------------------------------------------------------------------

\begin{slide}{1023}\frametitle{Creating user interface}
\vspace*{1.5em}
The four Qt worlds
\begin{itemize}
\item Widgets
\item Graphcis View
\item Open GL
\item Qt Quick
\end{itemize}

\end{slide}

%----------------------------------------------------------------------

\begin{slide}{2001}\frametitle{The Widget World}
\centeredImageFullWidth{qml-intro/images/ui-widget.png}

\end{slide}

%----------------------------------------------------------------------

\begin{slide}{2001}\frametitle{The Graphics View World}
\centeredImageFullWidth{qml-intro/images/ui-graphicsview.png}

\end{slide}

%----------------------------------------------------------------------

\begin{slide}{2001}\frametitle{The Open GL world}
\vspace*{1.5em}
Using openGL (C++/OpenGL)
\begin{itemize}
\item full strength of OpenGL
\item user interaction (mouse and keybboard) using Qt API
\item all classes from Qt are usable
\end{itemize}
%\centeredImageFullWidth{qml-intro/images/ui-opengl.png}

\end{slide}

%----------------------------------------------------------------------

\begin{slide}{2001}\frametitle{The Qt Quick World}
\vspace*{15mm}
\centeredImageFullWidth{qml-intro/images/ui-quick.png}

\end{slide}

%----------------------------------------------------------------------

\begin{slide}{2000}\frametitle{Qt Quick requirement}
\vspace*{1.5em}
\begin{itemize}
\item Platform must support OpenGL ES2
\item Needs at least QtBase, QtJsBackend and QtDeclarative module
\item Other module can be used to add new features:
\begin{itemize}
\item QtGraphicalEffects: add effects like blur, drop shadow...
\item Qt3D: 3D programming in QML
\item QtMultimedia: audio and video items
\item ...
\end{itemize}
\end{itemize}

\end{slide}

%----------------------------------------------------------------------
