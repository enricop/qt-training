%%%%%%%%%%%%%%%%%%%%%%%%%%%%%%%%%%%%%%%%%%%%%%%%%%%%%%%%%%%%%%%%%%%%%%%%%%
%
% Copyright (c) 2008-2011, Nokia Corporation and/or its subsidiary(-ies).
% All rights reserved.
%
% This work, unless otherwise expressly stated, is licensed under a
% Creative Commons Attribution-ShareAlike 2.5.
%
% The full license document is available from
% http://creativecommons.org/licenses/by-sa/2.5/legalcode .
%
%%%%%%%%%%%%%%%%%%%%%%%%%%%%%%%%%%%%%%%%%%%%%%%%%%%%%%%%%%%%%%%%%%%%%%%%%%

% ----------------------------------------------------------------------
\subsection{Creating Custom Items}
% ----------------------------------------------------------------------
\begin{slide}{1749}

\frametitle{Extending QGraphicsItem}
\iCls{QGraphicsItem} pure virtual methods (required overrides):

\begin{itemize}
\item \texttt{void }\hClsFn{QGraphicsItem}{paint}
    \begin{itemize}
    \item Paints contents of item in local coordinates
    \end{itemize}
\item \iCls{QRectF} \hClsFn{QGraphicsItem}{boundingRect}
    \begin{itemize}
    \item Returns outer bounds of item as a rectangle 
    \item Called by \iCls{QGraphicsView} to determine what regions need to be redrawn

    \end{itemize}
      
\item \iCls{QPainterPath} \hClsFn{QGraphicsItem}{shape} - shape of item
\begin{itemize}   
    \item Used by \hClsFn{QGraphicsItem}{contains} and \hClsFn{QGraphicsItem}{collidesWithPath} for collision detection
    \item Defaults to \hClsFn{QGraphicsItem}{boundingRect} if not implemented
    \end{itemize}
\end{itemize}
\end{slide}


% ----------------------------------------------------------------------

\begin{slide}{1748}

\frametitle{Boundaries}
\centeredImageFullWidth{graphicsview/images/boundaries}
\end{slide}


% ----------------------------------------------------------------------

\begin{slide}{1747}

\frametitle{Painting Items}
\begin{itemize}
\item Item is in complete control of drawing itself
\item Use standard \iCls{QPainter} drawing methods

    \begin{itemize}
    \item \iCls{QPen}, \iCls{QBrush}, pixmaps, gradients, text, etc. 

    \end{itemize}
\item No background to draw

\item Dynamic boundary and arbitrary shape
    \begin{itemize}
    \item Polygon, curved, non-contiguous, etc. 
    \end{itemize}
\end{itemize}

\end{slide}


% ----------------------------------------------------------------------


\begin{slide}[fragile]{1746}


\frametitle{Custom Item example}
\begin{cpp}

class PendulumItem : public QGraphicsItem {
public:
    QRectF boundingRect() const;
    void paint(QPainter* painter,
               const QStyleOptionGraphicsItem* option,
               QWidget* widget);
};

\end{cpp}
\begin{columns}[t]
\begin{column}{10mm}
\centeredImage{graphicsview/images/pendulum}
\end{column}
\begin{column}{10mm}
\centeredImage{graphicsview/images/pendulum-schema} 
\end{column}
\end{columns}
\end{slide}

% ----------------------------------------------------------------------
\begin{slide}[fragile]{1745}
\frametitle{paint() and boundingRect()}
\begin{itemize}
\item \texttt{boundingRect()} must take the pen width into consideration
\end{itemize}
\begin{cpp}
QRectF PendulumItem::boundingRect() const {
    return QRectF(-20.0 - PENWIDTH/2.0, -PENWIDTH/2.0,
                  40.0 + PENWIDTH, 140.0 + PENWIDTH );
}

void PendulumItem::paint( QPainter* painter,
    const QStyleOptionGraphicsItem*, QWidget*) {
    painter->setPen( QPen( Qt::black, PENWIDTH ) );
    painter->drawLine(0,0,0,100);
    QRadialGradient g( 0, 120, 20, -10, 110 );
    g.setColorAt( 0.0, Qt::white );
    g.setColorAt( 0.5, Qt::yellow );
    g.setColorAt( 1.0, Qt::black );
    painter->setBrush(g);
    painter->drawEllipse(-20, 100, 40, 40);
}
\end{cpp}
\end{slide}

% ----------------------------------------------------------------------
\begin{slide}{1421}
\frametitle{Choosing a \texttt{boundingRect()}}
\begin{itemize}
\item \texttt{boundingRect()} 
    \begin{itemize}
    \item Influences drawing code
    \item Influences "origin" of item transforms
    \end{itemize}
\item i.e. for Pendulum that swings:
    \begin{itemize}
    \item Good origin is non-weighted end of line
    \item Can rotate around (0,0) without translation 
    \end{itemize}
\end{itemize}
\centeredImage{graphicsview/images/boundingRect-implications}

\end{slide}

% ----------------------------------------------------------------------
\begin{slide}[fragile]{1743}
\frametitle{QGraphicsItemGroup}
\begin{itemize}
\item Easier approach to making a Pendulum:
    \begin{itemize}
    \item Extend \iCls{QGraphicsItemGroup}

    \item Use other concrete items as elements, add as children
    \item No need to override \texttt{paint()} or \texttt{shape()}
    \end{itemize} 
\end{itemize}
\begin{cpp}
PendulumItem::PendulumItem(QGraphicsItem* parent)
  : QGraphicsItemGroup(parent) { 
  m_line = new QGraphicsLineItem( 0,0,0,100, this);
  m_line->setPen( QPen( Qt::black, 3 ) );
  m_circle = new QGraphicsEllipseItem( -20, 100, 40, 40, this );
  m_circle->setPen( QPen(Qt::black, 3 ));
  QRadialGradient g( 0, 120, 20, -10, 110 );
  g.setColorAt( 0.0, Qt::white );
  g.setColorAt( 0.5, Qt::yellow );
  g.setColorAt( 1.0, Qt::black );
  m_circle->setBrush(g);  
}
\end{cpp}
\demo{graphicsview/ex-pendulum}
\end{slide}


% ----------------------------------------------------------------------

\begin{slide}{1741}

\frametitle{Event Handling}
\begin{itemize}
\item \texttt{QGraphicsItem::sceneEvent(QEvent*)}
    \begin{itemize}
    \item Receives all events for an item
    \item Similar to \iClsFn{QWidget}{event}

    \end{itemize} 
\item Specific typed event handlers:
    \begin{itemize}
    \item \texttt{keyPressEvent(QKeyEvent*)}
    \item \texttt{mouseMoveEvent(QGraphicsSceneMouseEvent*)}
    \item \hClsFnPar{QGraphicsItem}{wheelEvent}{QGraphicsSceneWheelEvent*}
    \item \texttt{mousePressEvent(QGraphicsSceneMouseEvent*)}
    \item \texttt{contextMenuEvent(QGraphicsSceneContextMenuEvent*)}
    \item \texttt{dragEnterEvent(QGraphicsSceneDragDropEvent*)}
    \item \texttt{focusInEvent(QFocusEvent*)}
    \item \texttt{hoverEnterEvent(QGraphicsSceneHoverEvent*)}
    \end{itemize}

\begin{block}{When overriding mouse event handlers:}
   Make sure to call base-class versions, too. Without this, the item select, focus, move behavior will not work as expected.
\end{block}
\end{itemize}
\end{slide}


% ----------------------------------------------------------------------
\begin{slide}[fragile]{1740}


\frametitle{Event Handler examples}
\begin{cpp}
void MyView::wheelEvent(QWheelEvent *event) {
    double factor =
        1.0 + (0.2 * qAbs(event->delta()) / 120.0);
    if (event->delta() > 0) zoom(factor);
    else                    zoom(1.0/factor);
}
void MyView::keyPressEvent(QKeyEvent *event) {
    switch (event->key()) {
      case Qt::Key_Plus:
          zoom(1.2);
          break;
      case Qt::Key_Minus:
          zoom(1.0/1.2);
          break;
      default:
          QGraphicsView::keyPressEvent(event);
    }
}
\end{cpp}

\end{slide}


% ----------------------------------------------------------------------
\begin{slide}{1739}

\frametitle{Collision Detection}
\begin{itemize}
\item Determines when items' shapes intersect
\item Two methods for collision detection:
    \begin{itemize}
    \item \texttt{collidesWithItem(QGraphicsItem* other)}
    \item \texttt{collidingItems(Qt::ItemSelectionMode)}
    \end{itemize}
\item \texttt{shape()}
    \begin{itemize}
    \item Returns \iCls{QPainterPath} used for collision detection

    \item Must be overridden properly
    \end{itemize}
\item \texttt{items()}
    \begin{itemize}
    \item Overloaded forms take \iCls{QRectF}, \iCls{QPolygonF}, \iCls{QPainterPath}

    \item Return items found in rect/polygon/shape
    \end{itemize}
\end{itemize}

\end{slide}

% ----------------------------------------------------------------------
\begin{slide}{1738}
\frametitle{Lab: Corner drag button}
\begin{itemize}
\item Define a \texttt{QGraphicsItem} which can display an image, and has at least 1 child item, that is a "corner drag" button, permitting the user to click and drag the button, to resize or rotate the image. 
\item Start with the handout provided in \texttt{graphicsview/lab-cornerdrag}
\flushedImage{graphicsview/images/cornerdrag}
\item Further details are in the \texttt{readme.txt} in the same directory.
\end{itemize}
\end{slide}


