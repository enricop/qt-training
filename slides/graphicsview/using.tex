\subsection{Using GraphicsView Classes}
% ----------------------------------------------------------------------
\begin{slide}{7982}
\frametitle{GraphicsView Framework}
\begin{itemize}
\item Provides:
    \begin{itemize}
    \item A surface for managing interactive 2D graphical items
    \item A view widget for visualizing the items
    \end{itemize}
\item Uses MVC paradigm
\item Resolution Independent
\item Animation Support
\item Fast item discovery, hit tests, collision detection
    \begin{itemize}
    \item Using Binary Space Paritioning (BSP) tree indexes
    \end{itemize}
\item Can manage large numbers of items (tens of thousands)
\item Supports zooming, printing and rendering

\end{itemize}
\end{slide}

% ----------------------------------------------------------------------

\begin{slide}[fragile]
\frametitle{Hello World}

\begin{cpp}
#include <QtGui>
int main(int argc, char **argv) {
  QApplication app(argc, argv);
  QGraphicsView view;
  QGraphicsScene *scene = new QGraphicsScene(&view);
  view.setScene(scene);
  QGraphicsRectItem *rect = 
      new QGraphicsRectItem(-10, -10, 120, 50);
  scene->addItem(rect);
  QGraphicsTextItem *text = scene->addText("Hello World!");
  view.show();
  return app.exec();
}
\end{cpp}
\centeredImageDoubleWidth{graphicsview/images/helloworld} \\
\demo{graphicsview/ex-helloworld}

\end{slide}

% ----------------------------------------------------------------------

\begin{slide}{1735}
\frametitle{UML relationship}
\begin{itemize}

\item \iCls{QGraphicsScene} is:
    \begin{itemize}
    \item a "model" for \iCls{QGraphicsView}
    \item a "container" for \iCls{QGraphicsItem}s
    \end{itemize}

\end{itemize}
\centeredImageFullWidth{graphicsview/images/graphicsview-uml}

\doc{qgraphicsview.html}{QGraphicsView}

\end{slide}
% ----------------------------------------------------------------------


\begin{slide}{1734}
\frametitle{QGraphicsScene}
\begin{itemize}
\item Container for Graphic Items
    \begin{itemize}
    \item Items can exist in only one scene at a time
    \end{itemize}
\item Propagates events to items
    \begin{itemize}
    \item Manages Collision Detection
    \item Supports fast item indexing
    \item Manages item selection and focus
    \end{itemize}
\item Renders scene onto view
    \begin{itemize}
    \item z-order determines which items show up in front of others
    \end{itemize}
\end{itemize}
\end{slide}
% ----------------------------------------------------------------------

\begin{slide}[fragile]{1733}
\frametitle{QGraphicsScene important methods}

\begin{itemize}
\item \texttt{addItem()}
    \begin{itemize}
    \item Add an item to the scene
        \begin{itemize}
        \item (remove from previous scene if necessary)
        \end{itemize}
    \item Also \texttt{addEllipse()}, \texttt{addPolygon()}, \texttt{addText()}, etc 
    \end{itemize}
\begin{cpp}

    QGraphicsEllipseItem *ellipse = 
        scene->addEllipse(-10, -10, 120, 50);
    QGraphicsTextItem *text = 
        scene->addText("Hello World!");
\end{cpp}    
\item \texttt{items()}
    \begin{itemize}
    \item returns items intersecting a particular point or region
    \end{itemize}

\item \texttt{selectedItems()}
    \begin{itemize}
    \item returns list of selected items
    \end{itemize}
\item \texttt{sceneRect()}
    \begin{itemize}
    \item bounding rectangle for the entire scene
    \end{itemize}
\end{itemize}


\end{slide}

% ----------------------------------------------------------------------


\begin{slide}{1732}
\frametitle{QGraphicsView}
\begin{itemize}
\item Scrollable widget viewport onto the scene
    \begin{itemize}
    \item Zooming, rotation, and other transformations
    \item Translates input events (from the View) into \iCls{QGraphicsSceneEvent}s
    \item Maps coordinates between scene and viewport
    \item Provides "level of detail" information to items
    \item Supports OpenGL
    \end{itemize}
\end{itemize}
\xCls{QGraphicsView}
\end{slide}
% ----------------------------------------------------------------------
\begin{slide}{1731}
\frametitle{QGraphicsView important methods}

\begin{itemize}
\item \texttt{setScene()}
    \begin{itemize}
    \item sets the QGraphicsScene to use
    \end{itemize}
\item \texttt{setRenderHints()}
    \begin{itemize}
    \item antialiasing, smooth pixmap transformations, etc
    \end{itemize}
\item \texttt{centerOn()}
    \begin{itemize}
    \item takes a QPoint or a QGraphicsItem as argument
    \item ensures point/item is centered in View
    \end{itemize}
\item \texttt{mapFromScene()}, \texttt{mapToScene()}
    \begin{itemize}
    \item map to/from scene coordinates
    \end{itemize}
\item \texttt{scale()}, \texttt{rotate()}, \texttt{translate()}, \texttt{matrix()}
    \begin{itemize}
    \item transformations
    \end{itemize}
\end{itemize}
\end{slide}

%TODO show example using mapfromscene/maptoscene

% ----------------------------------------------------------------------
\begin{slide}{1730}
\frametitle{QGraphicsItem}
\begin{itemize}

\item Abstract base class: basic canvas element
    \begin{itemize}
    \item Supports parent/child hierarchy
    \end{itemize}
    
\item Easy to extend or customize concrete items:
    \begin{itemize}
    \item \iCls{QGraphicsRectItem}, \iCls{QGraphicsPolygonItem}, \iCls{QGraphicsPixmapItem}, \iCls{QGraphicsTextItem}, etc.
    \item SVG Drawings, other widgets
    \end{itemize}

\item Items can be transformed:
    \begin{itemize}
    \item move, scale, rotate
    \item using local coordinate systems
    \end{itemize}
    
\item Supports Drag and Drop similar to QWidget
\end{itemize}
\xCls{QGraphicsItem}
\end{slide}

% ----------------------------------------------------------------------

\begin{slide}{1729}
\frametitle{Concrete QGraphicsItem Types}
\centeredImageDoubleWidth{graphicsview/images/concrete_items}
\demo{graphicsview/ex-concreteitems}
\end{slide}

% ----------------------------------------------------------------------
\begin{slide}{1728}
\frametitle{QGraphicsItem important methods}
\begin{itemize}

\item \texttt{pos()}
    \begin{itemize}
    \item get the item's position in scene
    \end{itemize}
\item \texttt{moveBy()}
    \begin{itemize}
    \item moves an item relative to its own position. 
    \end{itemize}
\item \texttt{zValue()}
    \begin{itemize}
    \item get a Z order for item in scene
    \end{itemize}
           
\item \texttt{show()}, \texttt{hide()} - set visibility
\item \texttt{setEnabled(bool)} - disabled items can not take focus or receive events
\item \texttt{setFocus(Qt::FocusReason)} - sets input focus.
\item \texttt{setSelected(bool)}
    \begin{itemize}
    \item select/deselect an item
    \item typically called from \iClsFn{QGraphicsScene}{setSelectionArea}
    \end{itemize}
\end{itemize}
\end{slide} 
% ----------------------------------------------------------------------
\begin{slide}[fragile]{1727}
\frametitle{Select, Focus, Move}
\begin{itemize}
\item \iClsFn{QGraphicsItem}{setFlags}
    \begin{itemize}
    \item Determines which operations are supported on an item
      
    \end{itemize}
\item \iCls{QGraphicsItemFlags}
    \begin{itemize}
    \item \iClsEnum{QGraphicsItem}{ItemIsMovable}
    \item \iClsEnum{QGraphicsItem}{ItemIsSelectable}
    \item \iClsEnum{QGraphicsItem}{ItemIsFocusable}
    \end{itemize}   
\begin{cpp}
item->setFlags(
  QGraphicsItem::ItemIsMovable|QGraphicsItem::ItemIsSelectable);
\end{cpp}
\end{itemize}

\end{slide}

% ----------------------------------------------------------------------
\begin{slide}{1726}
\frametitle{Groups of Items}
\begin{itemize}

\item Any \iCls{QGraphicsItem} can have children 
\item \iCls{QGraphicsItemGroup} is an invisible item for grouping child items
\item To group child items in a box with an outline (for example), use a \iCls{QGraphicsRectItem} 
\item Try dragging boxes in demo: 

\end{itemize}
\centeredImage{graphicsview/images/concrete-items-dragged}
\\
\medskip
\demo{graphicsview/ex-concreteitems} 
\end{slide}

\begin{slide}{1725}
\frametitle{Parents and Children}
\begin{itemize}
\item Parent propagates values to child items:
    \begin{itemize}
    \item \texttt{setEnabled()}
    \item \texttt{setFlags()}
    \item \texttt{setPos()}
    \item \texttt{setOpacity()}
    \item etc...
    \end{itemize}
\item Enables composition of items.
\end{itemize}
\end{slide}
