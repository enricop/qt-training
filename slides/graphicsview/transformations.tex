%%%%%%%%%%%%%%%%%%%%%%%%%%%%%%%%%%%%%%%%%%%%%%%%%%%%%%%%%%%%%%%%%%%%%%%%%%
%
% Copyright (c) 2008-2011, Nokia Corporation and/or its subsidiary(-ies).
% All rights reserved.
%
% This work, unless otherwise expressly stated, is licensed under a
% Creative Commons Attribution-ShareAlike 2.5.
%
% The full license document is available from
% http://creativecommons.org/licenses/by-sa/2.5/legalcode .
%
%%%%%%%%%%%%%%%%%%%%%%%%%%%%%%%%%%%%%%%%%%%%%%%%%%%%%%%%%%%%%%%%%%%%%%%%%%

% ----------------------------------------------------------------------

\subsection{Coordinate Systems and Transformations}


%----------------------------------------------------------------------
\begin{slide}{1629}\frametitle{Coordinate Systems}

\begin{itemize}
\item Each View and Item has its own local coordinate system
\item "Viewport" and "Scene" coordinates refer to the same thing:  \iCls{QGraphicsView}'s coordinate system.

\centeredImageFullWidth{graphicsview/images/qgv-coordinate-systems}
\end{itemize}
\end{slide}


%----------------------------------------------------------------------

\begin{slide}{1757}

\frametitle{Coordinates}
\begin{itemize}
\item Coordinates are local to an item
    \begin{itemize}
    \item Logical coordinates, not pixels
    \item Floating point, not integer
    \item Without transformations, 1 logical coordinate = 1 pixel.
    \end{itemize}   
\item Items inherit position and transform from parent
\item zValue is relative to parent
\item Item transformation does not affect its local coordinate system
\item Items are painted recursively
    \begin{itemize}
    \item From parent to children
    \item in increasing zValue order
    \end{itemize}
\end{itemize}
\end{slide}


%----------------------------------------------------------------------

\begin{slide}[fragile]{1756}


\frametitle{QTransform}
\begin{itemize}
\item Coordinate systems can be transformed using \texttt{QTransform}
\item \texttt{QTransform} is a $3x3$ matrix describing a linear transformation from (x,y) to (xt, yt)
\end{itemize}
\begin{columns}[t]
\begin{column}{50mm}

\begin{tabular}{c|c|c}
m11 & m12 & m13 \\ \hline
m21 & m22 & m23 \\ \hline
m31 & m32 & m33 \\
\end{tabular}

\end{column}
\begin{column}{50mm}

\begin{lstlisting}[language=Python]
xt = m11*x + m21*y + m31 
yt = m22*y + m12*x + m32 
if projected: 
    wt = m13*x + m23*y + m33  
    xt /= wt 
    yt /= wt 
\end{lstlisting}
\end{column}
\end{columns}
\begin{itemize}
\item $m_{13}$ and $m_{23}$
    \begin{itemize}
    \item Control perspective transformations
    \end{itemize}
\item \wiki{Affine_transformation}{Affine Transformations}
\end{itemize}

\end{slide}


%----------------------------------------------------------------------
\begin{slide}{1755}

\frametitle{Common Transformations}
\begin{itemize}

\item Commonly-used convenience functions:
    \begin{itemize}
    \item \texttt{scale()}
    \item \texttt{rotate()}
    \item \texttt{shear()}
    \item \texttt{translate()}
    \end{itemize}
\item Saves you the trouble of defining transformation matrices 
\item \texttt{rotate()} takes optional 2nd argument: axis of rotation.
    \begin{itemize}
    \item Z axis is "simple 2D rotation"
    \item Non-Z axis rotations are "perspective" projections. 
    \end{itemize}
\centeredImage{graphicsview/images/xyzaxis}
\end{itemize}
\end{slide}


%----------------------------------------------------------------------
\begin{slide}[fragile]{1754}\frametitle{View transformations}


\begin{itemize}
\item[] \begin{cpp}

    t = QTransform();        // identity matrix
    t.rotate(45, Qt::ZAxis); // simple rotate
    t.scale(1.5, 1.5)        // scale by 150%
    view->setTransform(t);   // apply transform to entire view
\end{cpp}

\item \hClsFn{QGraphicsView}{setTransformationAnchor}
    \begin{itemize}
    \item An \textbf{anchor} is a point that remains fixed before/after the transform.
    \item \texttt{AnchorViewCenter}: (Default) The \textit{center point} remains the same 
    \item \texttt{AnchorUnderMouse}: The \textit{point under the mouse} remains the same
    \item \texttt{NoAnchor}: Scrollbars remain unchanged.
    \end{itemize}
\end{itemize}
\end{slide}


%----------------------------------------------------------------------
\begin{slide}{1632}\frametitle{Item Transformations}
\begin{itemize}
\item \iCls{QGraphicsItem} supports same transform operations:

    \begin{itemize}
    \item \texttt{setTransform()}, \texttt{transform()}
    \item \texttt{rotate()}, \texttt{scale()}, \texttt{shear()}, \texttt{translate()} 
    \end{itemize}
\end{itemize}    

\begin{block}{An item's effective transformation:}
The product of its own and all its ancestors' transformations
\end{block}

{\small TIP: When managing the transformation of items, store the desired rotation, scaling etc. in member variables and build a \iCls{QTransform} from the identity transformation when they change. Don't try to deduce values from the current transformation and/or try to use it as the base for further changes.}


\end{slide}


% ----------------------------------------------------------------------
\begin{slide}[fragile]

\frametitle{Zooming}
\begin{itemize}
\item Zooming is done with \texttt{view->scale()}
\begin{cpp}
void MyView::zoom(double factor)
{
    double width =
        matrix().mapRect(QRectF(0, 0, 1, 1)).width();
    width *= factor;
    if ((width < 0.05) || (width > 10)) return;

    scale(factor, factor);
}
\end{cpp}
\end{itemize}
\end{slide}



%----------------------------------------------------------------------
\begin{slide}{1631}\frametitle{Mapping between Coordinate Systems}
%TODO - show code snippet here
\begin{itemize}
\item Mapping methods are overloaded for \iCls{QPolygonF}, \iCls{QPainterPath} etc
\begin{itemize}
\item \hClsFnPar{QGraphicsItem}{mapFromScene}{const QPointF\&}:
    \begin{itemize}
    \item Maps a point from viewport coordinates to item coordinates. Inverse:  \hClsFnPar{QGraphicsItem}{mapToScene}{const QPointF\&}
    \end{itemize}

\item \hClsFnPar{QGraphicsItem}{mapFromItem}{const QGraphicsItem*, const QPointF\&}
    \begin{itemize}
    \item Maps a point from another item's coordinate system to this item's. Inverse:  \hClsFnPar{QGraphicsItem}{mapToItem}{const QGraphicsItem*, const QPointF\&}.
    \end{itemize}
\item Special case: \hClsFnPar{QGraphicsItem}{mapFromParent}{const QPointF\&}.
\end{itemize}

\end{itemize}
\end{slide}


%----------------------------------------------------------------------
\begin{slide}{1630}\frametitle{Ignoring Transformations}
\begin{itemize}
\item Sometimes we don't want particular items to be transformed before display.
\item View transformation can be disabled for individual items.
\item Used for text labels in a graph that should not change size when the graph is zoomed.
\end{itemize}
\small{
\texttt{item->setFlag( QGraphicsItem::ItemIgnoresTransformations);}
}


\end{slide}


\begin{slide}{1750}

\frametitle{Transforms Demo}
\imageFullWidth{graphicsview/images/transformination-screenshot}

\demo{graphicsview/ex-transformination}

\end{slide}


