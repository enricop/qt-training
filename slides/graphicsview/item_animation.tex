%%%%%%%%%%%%%%%%%%%%%%%%%%%%%%%%%%%%%%%%%%%%%%%%%%%%%%%%%%%%%%%%%%%%%%%%%%
%
% Copyright (c) 2008-2011, Nokia Corporation and/or its subsidiary(-ies).
% All rights reserved.
%
% This work, unless otherwise expressly stated, is licensed under a
% Creative Commons Attribution-ShareAlike 2.5.
%
% The full license document is available from
% http://creativecommons.org/licenses/by-sa/2.5/legalcode .
%
%%%%%%%%%%%%%%%%%%%%%%%%%%%%%%%%%%%%%%%%%%%%%%%%%%%%%%%%%%%%%%%%%%%%%%%%%%

\subsection{Item Animation}
%----------------------------------------------------------------------
\begin{slide}{0994}\frametitle{Animation - Introduction}
\begin{itemize}
\item Animation is supported both for individual items and at the scene level.
\item Individual item animation works best when items dont interact.
\item Scene level animation is suitable for interacting items (for example colliding items).
\end{itemize}
\end{slide}

%----------------------------------------------------------------------
\begin{slide}[fragile]{0995}\frametitle{Animation - Single Items}
\begin{itemize}
\item Single items can be animated by using a \iCls{QGraphicsItemAnimation}.
\item \iCls{QGraphicsItemAnimation} can animate an item by interpolating the items position and transformation over a timeline.
\begin{alltt}
\footnotesize
QGraphicsItem\pointerStar myitem = ...;
QTimeLine\pointerStar timeline = new QTimeLine(500);
timeline\pointerDeref{}setFrameRange(0,100);
QGraphicsItemAnimation\pointerStar animation = new QGraphicsItemAnimation;
animation\pointerDeref{}setItem(myitem);
animation\pointerDeref{}setTimeLine(timeline);
animation\pointerDeref{}setPosAt(0.0,QPointF(0.0,0.0));
animation\pointerDeref{}setPosAt(0.5,QPointF(100.0,100.0));
animation\pointerDeref{}setPosAt(1.0,QPointF(0.0,50.0));
timeline\pointerDeref{}start();
\end{alltt}
\end{itemize}
\end{slide}

%----------------------------------------------------------------------
\begin{slide}{0996}\frametitle{Animation - Whole Scene}
\begin{itemize}
\item The main entrypoint for whole scene animation is \iClsFnPar{QGraphicsScene}{advance}{}.
\item \iClsFnPar{QGraphicsScene}{advance}{} calls the virtual member function 
\hClsFnPar{QGraphicsItem}{advance}{int phase} \emph{twice} on each item in the scene. First
with \texttt{phase=0} for each item and then with \texttt{phase=1} for each item.
\item \iClsFnPar{QGraphicsScene}{advance}{} should be called at regular intervals, for example by using a \iCls{QTimer}.
\end{itemize}
\end{slide}

%----------------------------------------------------------------------
\begin{slide}{0997}\frametitle{QGraphicsItem::advance(int phase)}
\begin{itemize}
\item Override \texttt{advance} to provide the logic for moving the item.
\item The method is called in two steps, specified using the integer
  argument \texttt{phase}.
\item Phase 0 is the \emph{calculation phase}, no items should move in this
  phase, instead they should calculate their movement, optionally on the basis of
  placement of other items.
\item Phase 1 is the \emph{movement phase}. Here, items should move
  unconditionally on the basis of the calculation in phase 0.
\end{itemize}
\end{slide}

%----------------------------------------------------------------------
\begin{slide}{0998}\frametitle{Collision Detection}
\begin{itemize}
\item \iCls{QGraphicsItem} offers two methods for detecting collisions with other items:
{\footnotesize
\begin{itemize}
\item \texttt{bool} \hClsFnPar{QGraphicsItem}{collidesWithItem}{const QGraphicsItem\pointerStar other, Qt\classsep{}ItemSelectionMode mode = Qt\classsep{}IntersectsItemShape}
\item \texttt{QList<QGraphicsItem\pointerStar{}>} \hClsFnPar{QGraphicsItem}{collidingItems}{Qt\classsep{}ItemSelectionMode mode = Qt\classsep{}IntersectsItemShape}
\end{itemize}
}
\item \hClsFn{QGraphicsItem}{shape} returns the QPainterPath used for collision detection.
\item Those can be used for example in \emph{phase 0} in \hClsFn{QGraphicsItem}{advance} to determine how the items should move around.
\item See example \iExample{QGraphicsView/breakout}
\end{itemize}
\end{slide}
