%%%%%%%%%%%%%%%%%%%%%%%%%%%%%%%%%%%%%%%%%%%%%%%%%%%%%%%%%%%%%%%%%%%%%%%%%%
%
% Copyright (c) 2008-2011, Nokia Corporation and/or its subsidiary(-ies).
% All rights reserved.
%
% This work, unless otherwise expressly stated, is licensed under a
% Creative Commons Attribution-ShareAlike 2.5.
%
% The full license document is available from
% http://creativecommons.org/licenses/by-sa/2.5/legalcode .
%
%%%%%%%%%%%%%%%%%%%%%%%%%%%%%%%%%%%%%%%%%%%%%%%%%%%%%%%%%%%%%%%%%%%%%%%%%%

%----------------------------------------------------------------------   

\subsection{Graphical Elements}

\begin{slide}{1693}\frametitle{Colors}

Specifying colors
\begin{itemize}
\item Named colors (using SVG names): \qic{string}{"red"}, \qic{string}{"green"}, \qic{string}{"blue"}, ...
\item HTML style color components: \qic{string}{"\#ff0000"}, \qic{string}{"\#008000"}, \qic{string}{"\#0000ff"}, ...
\item Built-in function: \qtt{\qc{class}{Qt}.rgba(\qc{number}{0},\qc{number}{0.5},\qc{number}{0},\qc{number}{1})}
\end{itemize}
\vspace*{0.5em}
Changing items opacity:
\begin{itemize}
\item using the \qic{type}{opacity} property
\item values from \qic{number}{0.0} (transparent) to \qic{number}{1.0} (opaque)
\end{itemize}

\doc{qml-color.html}{QML Basic Type: color}
\end{slide}

%----------------------------------------------------------------------

\begin{slide}{1692}\frametitle{Colors}

\flushedImage{qml-composing-uis/images/colors.png}
% declarative-uis/composing-uis/colors.qml
\inputqml{qml-composing-uis/colorized/colors}

\demo{qml-composing-uis/ex-elements/colors.qml}

\end{slide}

%----------------------------------------------------------------------

\begin{slide}{1691}\frametitle{Images}

\begin{itemize}
\item Represented by the \qic{class}{Image} element
\item Refer to image files with the \qic{type}{source} property
  \begin{itemize}
  \item using absolute URLs
  \item or relative to the QML file
  \end{itemize}
\item Can be transformed
  \begin{itemize}
  \item scaled, rotated
  \item about an axis or central point
  \end{itemize}
\end{itemize}

%%% BorderImage

\end{slide}

%----------------------------------------------------------------------

\begin{slide}{1690}\frametitle{Images}

\flushedImage{qml-composing-uis/images/images.png}
% declarative-uis/composing-uis/images.qml
\inputqml{qml-composing-uis/colorized/images}

\begin{itemize}
\item \qic{type}{source} contains a relative path
\item \qic{type}{width} and \qic{type}{height} are obtained from\\
the image file
\end{itemize}
\demo{qml-composing-uis/ex-elements/images.qml}

\end{slide}

%----------------------------------------------------------------------

\begin{slide}{1689}\frametitle{Image Scaling}

\flushedImage{qml-composing-uis/images/image-scaling.png}
% declarative-uis/composing-uis/image-scaling.qml
\begin{qml}
\qtt{\qc{lightgray}{import~QtQuick~2.0}}\\
\vspace*{0.5em}
\qtt{\qc{lightgray}{Rectangle~\{}}\\
\qtt{~~~~\qc{lightgray}{width:~400;~height:~400}}\\
\qtt{~~~~\qc{lightgray}{color:~"black"}}\\
\vspace*{0.5em}
\qtt{~~~~\qc{lightgray}{Image~\{}}\\
\qtt{~~~~~~~~\qc{lightgray}{x:~150;~y:~150}}\\
\qtt{~~~~~~~~\qc{lightgray}{source:~"../images/rocket.png"}}\\
\qtt{~~~~~~~~\qc{type}{scale}:~\qc{number}{2.0}}\\
\qtt{~~~~\qc{lightgray}{\}}}\\
\qtt{\qc{lightgray}{\}}}
\end{qml}

\begin{itemize}
\item Set the \qic{type}{scale} property
\item By default, the center of the item remains in the same place
\end{itemize}                                    
\demo{qml-composing-uis/ex-elements/image-scaling.qml}

\end{slide}

%----------------------------------------------------------------------

\begin{slide}{1688}\frametitle{Image Rotation}

\flushedImage{qml-composing-uis/images/image-rotation.png}
% declarative-uis/composing-uis/image-rotation.qml
\inputqml{qml-composing-uis/colorized/image-rotation}

\begin{itemize}
\item Set the \qic{type}{rotate} property
\item By default, the center of the item remains in the same place
\end{itemize}                                     
\demo{qml-composing-uis/ex-elements/image-rotation.qml}
\end{slide}                                       

%----------------------------------------------------------------------

\begin{slide}{1687}\frametitle{Image Rotation}

\flushedImage{qml-composing-uis/images/image-rotation-top.png}
% declarative-uis/composing-uis/image-rotation-top.qml
\inputqml{qml-composing-uis/colorized/image-rotation-top}

\begin{itemize}
\item Set the \qic{type}{transformOrigin} property
\item Now the image rotates about the top of the item
\end{itemize}

\end{slide}

%----------------------------------------------------------------------

\begin{slide}{1686}\frametitle{Gradients}

Define a gradient using the \qic{type}{gradient} property:

\begin{itemize}
\item With a \qic{class}{Gradient} element as the value
\item Containing \qic{class}{GradientStop} elements, each with
  \begin{itemize}
  \item a \qic{type}{position}: a number between 0 (start point) and 1 (end point)
  \item a \qic{type}{color}
  \end{itemize}
\item The start and end points
  \begin{itemize}
    \item are on the top and bottom edges of the item
    \item cannot be repositioned
  \end{itemize}
\item Issues with gradients:
  \begin{itemize}
  \item rendering is CPU intensive
  \item gradients may not be animated as you expect
  \end{itemize}
\item Gradients override \qic{type}{color} definitions
\end{itemize}

\doc{qml-gradient.html}{QML Gradient Element Reference}
\end{slide}

%----------------------------------------------------------------------

\begin{slide}{1684}\frametitle{Gradients}

\flushedImage{qml-composing-uis/images/gradients.png}
% declarative-uis/composing-uis/gradients.qml
\inputqml{qml-composing-uis/colorized/gradients}

\begin{itemize}
\item Note the definition of an element as a property value
\end{itemize}                                

\demo{qml-composing-uis/ex-elements/gradients.qml}

\end{slide}

%----------------------------------------------------------------------

\begin{slide}{1683}\frametitle{Gradient Images}

\flushedImage{qml-composing-uis/images/image-gradients.png}
% declarative-uis/composing-uis/image-gradients.qml
\inputqml{qml-composing-uis/colorized/image-gradients}

\begin{itemize}
\item It is often faster to use images instead of real gradients
\item Artists can create the desired gradients
\end{itemize}
\demo{qml-composing-uis/ex-elements/image-gradients.qml}

\end{slide}

%----------------------------------------------------------------------
\begin{slide}{4050}\frametitle{Border Images}

\flushedImageDoubleWidth{qml-composing-uis/images/image-border.png}

\begin{itemize}
\item Create border using part of an image:
  \begin{itemize}
  \item corners (region 1, 3, 7, 9) are not scaled
  \item horizontal borders (2 and 8) are scaled according to \qic{type}{horizontalTileMode}
  \item vertical borders (4 and 6) are scaled according to \qic{type}{verticalTileMode}
  \item middle region (5) is scaled according to both mode
  \end{itemize}
\item There are 3 different scale modes
  \begin{itemize}
  \item \qic{type}{Stretch}: scale the image to fit to the available area.
  \item \qic{type}{Repeat}: tile the image until there is no more space.
  \item \qic{type}{Round}: like \qic{type}{Repeat}, but scales the images down\\
    to ensure that the last image is not cropped
  \end{itemize}
\end{itemize}

\end{slide}

%----------------------------------------------------------------------

\begin{slide}{4051}\frametitle{Border Images}

\centeredImageFullWidth{qml-composing-uis/images/image-border-example.png}

\begin{qml}
\qtt{\qc{class}{BorderImage}~\{}\\
\qtt{~~~~\qc{type}{source}:~\qc{string}{"content/colors.png"}}\\
\qtt{~~~~\qc{type}{border}~\{~\qc{type}{left}:~\qc{number}{30};~\qc{type}{top}:~\qc{number}{30};~\qc{type}{right}:~\qc{number}{30};~\qc{type}{bottom}:~\qc{number}{30};~\}}\\
\qtt{~~~~\qc{type}{horizontalMode}:~\qc{class}{BorderImage}.\qc{type}{Stretch}}\\
\qtt{~~~~\qc{type}{verticalMode}:~\qc{class}{BorderImage}.\qc{type}{Repeat}}\\
\qtt{~~~~...}\\
\qtt{\}}
\end{qml}

\qtdemo{examples/qtdeclarative/qtquick/imageelements/borderimage.qml}

\end{slide}
