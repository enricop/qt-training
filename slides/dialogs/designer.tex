%%%%%%%%%%%%%%%%%%%%%%%%%%%%%%%%%%%%%%%%%%%%%%%%%%%%%%%%%%%%%%%%%%%%%%%%%%
%
% Copyright (c) 2008-2011, Nokia Corporation and/or its subsidiary(-ies).
% All rights reserved.
%
% This work, unless otherwise expressly stated, is licensed under a
% Creative Commons Attribution-ShareAlike 2.5.
%
% The full license document is available from
% http://creativecommons.org/licenses/by-sa/2.5/legalcode .
%
%%%%%%%%%%%%%%%%%%%%%%%%%%%%%%%%%%%%%%%%%%%%%%%%%%%%%%%%%%%%%%%%%%%%%%%%%%

\subsection{Qt Designer}
\newcommand{\uicProgram}{\iConcept{uic}\xspace}

\label{qt_designer}

%----------------------------------------------------------------------
\begin{slide}[fragile]{0431}\frametitle{Qt Designer}
\flushedImage{dialogs/images/designer}
  \begin{itemize}
\item Design UI forms visually
  \end{itemize}
  \vfill
 \begin{itemize}
 \item Visual Editor for
   \begin{itemize}
  \item Signal/slot connections
  \item Actions
  \item Tab handling
  \item Buddy widgets
  \item Widget properties
  \item Integration of custom widgets
  \item Resource files
  \end{itemize}
\end{itemize}
\end{slide}

%----------------------------------------------------------------------
\begin{slide}{1707}\frametitle{Designer Views}
  \centeredImageDoubleWidth{dialogs/images/designer-views}
\end{slide}

%----------------------------------------------------------------------
\begin{slide}{1706}\frametitle{Designer's Editing Modes}
  \begin{itemize}
  \item \inlinedImage{dialogs/images/designer-widget-tool} Widget Editing 
    \begin{itemize}
    \item Change appearance of form
    \item Add layouts
    \item Edit properties of widgets
    \end{itemize}
  \item \inlinedImage{dialogs/images/designer-connection-tool} Signal and Slots Editing 
    \begin{itemize}
    \item Connect widgets together with signals \& slots
    \end{itemize}
 \item \inlinedImage{dialogs/images/designer-buddy-tool} Buddy Editing
   \begin{itemize}
   \item Assign buddy widgets to label
   \item \textit{Buddy widgets help keyboard focus handling correctly}
   \end{itemize}
  \item \inlinedImage{dialogs/images/designer-tab-order-tool} Tab Order Editing
    \begin{itemize}
    \item Set order for widgets to receive the keyboard focus
    \end{itemize}

 \end{itemize}  
\end{slide}

%----------------------------------------------------------------------
\begin{slide}[fragile]{1705}
  \frametitle{Designer UI Form Files}
\flushedImage{dialogs/images/ui-xml}
\begin{itemize}
\item Form stored in .ui file
  \begin{itemize}
  \item format is XML 
  \end{itemize}
\end{itemize}
\begin{itemize}
\item \texttt{uic} tool generates code
  \begin{itemize}
  \item From \texttt{myform.ui}
    \begin{itemize}
    \item to \texttt{ui\_myform.h}
    \end{itemize}
  \item Class created in in the \iCls{Ui} namespace
  \end{itemize}
\item List form \texttt{ui} in project file (\texttt{.pro})
  \begin{itemize}
  \item \texttt{FORMS += mainwindow.ui}
  \end{itemize}
\end{itemize}
\end{slide}

% ----------------------------------------------------------------------
\begin{slide}[fragile]{1704}
\frametitle{From .ui to C++}
\centeredImageFullWidth{dialogs/images/designer-steps} 
\end{slide}

% ----------------------------------------------------------------------
\begin{slide}[fragile]{1703}
\frametitle{Qt Creator - Form Wizards}
\begin{itemize}
\item \textbf{Add New...} "Designer Form"
  \begin{itemize}
  \item or "Designer Form Class" (for C++ integration)
  \end{itemize}
\end{itemize}
\medskip
\centeredImage{dialogs/images/designer-new-form.png}
\end{slide}

% ----------------------------------------------------------------------
\begin{slide}{1702}
\frametitle{Naming Widgets}
\begin{enumerate}
\item Place widgets on form
\item Edit \texttt{objectName} property
\end{enumerate}
\medskip
\centeredImage{dialogs/images/designer-object-inspector.png}
\begin{itemize}
\item \textit{\texttt{objectName} defines member name in generated code}
\end{itemize}
\end{slide}

% ----------------------------------------------------------------------
\begin{slide}[fragile]{1701}
\frametitle{Form layout in Designer}
\begin{itemize}
\item \iCls{QFormLayout}: Suitable for most input forms
\end{itemize}
\medskip
\centeredImage{dialogs/images/designer-form-layout}
\end{slide}

% ----------------------------------------------------------------------
\begin{slide}[fragile]{1700}
\frametitle{Top-Level Layout}
\begin{enumerate}
\item First layout child widgets
\item Finally select empty space and set top-level layout
\end{enumerate}
\medskip
\centeredImage{dialogs/images/designer-toplevel-layout}
\end{slide}

% ----------------------------------------------------------------------
\begin{slide}[fragile]{1699}
\frametitle{Preview Widget in Preview Mode}
\begin{itemize}
\item Check that widget is nicely resizable
\end{itemize}
\medskip
\centeredImage{dialogs/images/designer-preview-form}    
\end{slide}

%----------------------------------------------------------------------
\begin{slide}[fragile]{1698}\frametitle{Code Integration - Header File}
   \begin{cpp}
// orderform.h
class Ui::OrderForm;

class OrderForm : public QDialog {
private:
   Ui::OrderForm *ui;    // pointer to UI object
};
\end{cpp}
  \begin{itemize}
  \item Host widget derives from appropriate base class
  \item \textbf{\texttt{*ui}} member encapsulate UI class
    \begin{itemize}
    \item Makes header independent of designer generated code 
    \end{itemize}
\end{itemize}
\end{slide}


% ----------------------------------------------------------------------
\begin{slide}[fragile]{1697}\frametitle{Code Integration - Implementation File}
  \begin{itemize}
   \begin{cpp}
// orderform.cpp
#include "ui_orderform.h"

OrderForm::OrderForm(QWidget *parent)
: QDialog(parent), ui(new Ui::OrderForm) {
  ui->setupUi(this);
}

OrderForm::~OrderForm() {
  delete ui; ui=0;
}      
\end{cpp}
\item \textit{Default behavior in Qt Creator}
  \end{itemize}
\end{slide}

%----------------------------------------------------------------------
\begin{slide}{0441}\frametitle{Signals and Slots in Designer}
  \begin{itemize}
  \item Widgets are available as public members
    \begin{itemize}
    \item \texttt{ui->fileName->setText("image.png")}
      \begin{itemize}
      \item \textit{Name based on widget's objectName property}
      \end{itemize}
    \end{itemize}
  \item You can set up signals \& slots traditionally...
    \begin{itemize}
    \item \texttt{connect(ui->okButton, SIGNAL(clicked()), ...}
    \end{itemize}
  \end{itemize}
  \begin{itemize}
  \item Auto-connection facility for custom slots
    \begin{itemize}
    \item Automatically connect signals to slots in your code
      \begin{itemize}
      \item Based on object name and signal 
      \end{itemize}
    \item \texttt{void on\_\emph{objectName}\_\emph{signal}(\emph{parameters});}
      \begin{itemize}
      \item Example: \textit{\texttt{on\_okButton\_clicked()}} slot
      \end{itemize}
    \item \doc{designer-using-a-ui-file.html\#automatic-connections}{Automatic Connections}
    \end{itemize}
  \end{itemize}
  \begin{itemize}
  \item Qt Creator: right-click on widget and "Go To Slot"
    \begin{itemize}
    \item Generates a slot using auto-connected name
    \end{itemize}
  \end{itemize}

\end{slide}



%----------------------------------------------------------------------
\begin{slide}[fragile]{0442}\frametitle{Using Custom Widgets in Designer}\label{designer_plugins}
\xConcept{Qt Designer!Handling your own classes}
\xConcept{Qt Designer!Plug-ins}
Choices for Custom Widgets
\begin{enumerate}
\flushedImage{dialogs/images/customWidget}
\item \emph{Promote to Custom Widget}
  \begin{itemize}
  \item Choose the widget closest
  \item From context menu choose \\ \emph{Promote to Custom Widget}
  \item Code generated will now refer to given class name
  \end{itemize}
\item \emph{Implement a Designer plug-in}
  \begin{itemize}
  \item \qtdemo{examples/designer/customwidgetplugin}
  \item \doc{designer-creating-custom-widgets.html}{Creating Custom Widgets for Qt Designer}
  \end{itemize}
\end{enumerate}
\end{slide}

%----------------------------------------------------------------------
\begin{slide}[fragile]{0445}\frametitle{Dynamically loading \texttt{.ui} files}
\begin{itemize}
\item Forms can be processed at runtime
  \begin{itemize}
  \item Produces dynamically generated user interfaces
  \end{itemize}
\item Disadvantages
  \begin{itemize}
  \item Slower, harder to maintain
  \item Risk: \texttt{.ui} file not available at runtime
  \end{itemize}
\item[] \doc{designer-using-a-ui-file.html\#run-time-form-processing}{Run Time Form Processing}
\item Loading \texttt{.ui} file
 \begin{cpp}
QUiLoader loader;
QFile file("forms/textfinder.ui");
file.open(QFile::ReadOnly);
QWidget *formWidget = loader.load(&file, this);
\end{cpp}
\item Locate objects in form
\begin{cpp}
ui_okButton = qFindChild<QPushButton*>(this, "okButton");
\end{cpp}
\item \emph{Handle with care!}

\end{itemize}
\end{slide}


% ----------------------------------------------------------------------
\begin{slide}[fragile]{1236}
\frametitle{Lab: Designer Order Form}
\begin{itemize}
\item Create an order form dialog
  \begin{itemize}
  \item With fields for price, quantity and total.
  \item Total field updates itself to reflect quantity and price entered
  \end{itemize}
\end{itemize} 

\centeredImageDoubleWidth{dialogs/images/designer-orderform-screenshot}

\lab{dialogs/lab-orderform}
\end{slide}

% ----------------------------------------------------------------------
\begin{slide}\frametitle{Lab: Upgrade editor to use a Designer UI file}
\begin{itemize}
\item Starting with your previous editor, rework it to use Designer UI files
  \begin{itemize}
  \item Menus and toolbars should be created in Designer
  \item Resources should be managed by Designer
  \end{itemize}
\item Optional: Implement a Find Dialog for the editor
\end{itemize} 
\end{slide}

%%% See ticket #320. Please add once you have a valid handout.
%%% %----------------------------------------------------------------------
%%% \begin{slide}{0448}\frametitle{Lab (Optional): Options Dialog}
%%%   \label{qt_designer_exercise}
%%%  \begin{itemize}
%%%   \item Create options dialog for a simple text editor
%%%   \item Use designer for dialog and editor UI
%%%  \item Try out different approaches
%%%     \begin{itemize}
%%%     \item Direct, Single and Multiple Inheritance
%%%     \end{itemize}
%%%  \flushedImage{dialogs/images/lab-optionsdialog}
%%%  \item Use Auto-Connection Signal\&Slots
%%%   \item Optional
%%%     \begin{itemize}
%%%     \item Implement real functionality
%%%     \item Save dialog settings
%%%     \end{itemize}
%%%   \item[] \lab{dialogs/lab-optionsdialog}
%%%  \end{itemize}
%%% \end{slide}
