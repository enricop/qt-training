%%%%%%%%%%%%%%%%%%%%%%%%%%%%%%%%%%%%%%%%%%%%%%%%%%%%%%%%%%%%%%%%%%%%%%%%%%
%
% Copyright (c) 2008-2011, Nokia Corporation and/or its subsidiary(-ies).
% All rights reserved.
%
% This work, unless otherwise expressly stated, is licensed under a
% Creative Commons Attribution-ShareAlike 2.5.
%
% The full license document is available from
% http://creativecommons.org/licenses/by-sa/2.5/legalcode .
%
%%%%%%%%%%%%%%%%%%%%%%%%%%%%%%%%%%%%%%%%%%%%%%%%%%%%%%%%%%%%%%%%%%%%%%%%%%

% ----------------------------------------------------------------------
\subsection{Widgets in a Scene}
% ----------------------------------------------------------------------
\begin{slide}{1975}
\frametitle{Widgets in a Scene}
\centeredImage{graphicsview2/images/padnavigator} \\
\qtdemo{examples/graphicsview/padnavigator}
\end{slide}
% ----------------------------------------------------------------------
\begin{slide}{1974}
\frametitle{Items are not widgets}
\begin{itemize}
\item \iCls{QGraphicsItem}:

    \begin{itemize}
    \item Lightweight compared to \iCls{QWidget}

    \item No signals/slots/properties
    \item Scenes can easily contain thousands of Items
    \item Uses different \iCls{QEvent} sub-hierarchy (derived from \iCls{QGraphicsSceneEvent})

    \item Supports transformations directly
    \end{itemize}
\item \iCls{QWidget}:

    \begin{itemize}
    \item Derived from \iCls{QObject} (less light-weight)

    \item supports signals, slots, properties, etc
    \item can be embedded in a \iCls{QGraphicsScene} with a \iCls{QGraphicsProxyWidget}

    \end{itemize}
\end{itemize}
\end{slide}

% ----------------------------------------------------------------------
\begin{slide}{1973}
\frametitle{QGraphicsWidget}
\begin{itemize}
\item Advanced functionality graphics item
\item Provides signals/slots, layouts, geometry, palette, etc.
\item \textit{Not} a QWidget!
\item Base class for \texttt{QGraphicsProxyWidget}
\end{itemize}
\centeredImageFullWidth{graphicsview2/images/qgraphicswidget-uml}
\end{slide}


% ----------------------------------------------------------------------
\begin{slide}{1972}
\frametitle{QGraphicsProxyWidget}
\begin{itemize}
\item \texttt{QGraphicsItem} that can embed a \texttt{QWidget} in a \texttt{QGraphicsScene}
\item \textit{Still} not a \texttt{QWidget}, just a proxy for a \texttt{QWidget}
\item Handles complex widgets like \texttt{QFileDialog}
\item Takes \textit{ownership} of related widget
    \begin{itemize}
    \item Synchronizes states/properties:
        \begin{itemize}
        \item \texttt{visible}, \texttt{enabled}, \texttt{geometry}, \texttt{style}, \texttt{palette}, \texttt{font}, \texttt{cursor}, \texttt{sizeHint}, etc
        \item Proxies events between Widget and GraphicsView
        \item ... including the paint events
        \end{itemize}
    \item If either (widget or proxy) is deleted, the other is also!
    \end{itemize}
\item Widget must not already have a parent
\item []
\item \textit{Warning: Overuse can lead to performance problems}
\end{itemize}
\end{slide}


% ----------------------------------------------------------------------
\begin{slide}[fragile]{1966}


\frametitle{Embedded Widget Example}
\begin{cpp}
#include <QtGui>
int main(int argc, char **argv) {
   QApplication app(argc, argv);

   QCalendarWidget *calendar = new QCalendarWidget;

   QGraphicsScene scene;
   QGraphicsProxyWidget *proxy = scene.addWidget(calendar);

   QGraphicsView view(&scene);
   view.show();

   return app.exec();
}
\end{cpp}
\end{slide}

% ----------------------------------------------------------------------
\begin{slide}{1971}
\frametitle{QGraphicsLayout}

\begin{itemize}

\item For layout of \texttt{QGraphicsLayoutItem} (+derived) classes in \texttt{QGraphicsView}
\item Concrete classes:
    \begin{itemize}
    \item \texttt{QGraphicsLinearLayout}: equivalent to \texttt{QBoxLayout}, arranges items horizontally or vertically
    \item \texttt{QGraphicsGridLayout}: equivalent to \texttt{QGridLayout}, arranges items in a grid
    \end{itemize}
\item \texttt{QGraphicsWidget::setLayout()} - set layout for child items of this \texttt{QGraphicsWidget}
\end{itemize}
\demo{graphicsview/ex-layoutitems}
\end{slide}

% ----------------------------------------------------------------------
\begin{slide}{1970}
\frametitle{Lab: Widgets in a Scene}
\begin{itemize}
\item Starting with the \texttt{graphicsview/lab-mapviewer} handout, add zooming controls.
\item Suggested widgets:
    \begin{itemize}
    \item \texttt{QPushButtons} for +/-
    \item \texttt{QSlider} for selecting zoom level directly. 
    \end{itemize}
\item Use \texttt{QGraphicsLayout} to lay out the widgets
\item Make the mouse work like a "hand-grab" tool on drag, so we can see different zoomed areas. 
\end{itemize}
\centeredImage{graphicsview2/images/mapviewer}
\end{slide}


