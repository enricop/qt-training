%%%%%%%%%%%%%%%%%%%%%%%%%%%%%%%%%%%%%%%%%%%%%%%%%%%%%%%%%%%%%%%%%%%%%%%%%%
%
% Copyright (c) 2008-2011, Nokia Corporation and/or its subsidiary(-ies).
% All rights reserved.
%
% This work, unless otherwise expressly stated, is licensed under a
% Creative Commons Attribution-ShareAlike 2.5.
%
% The full license document is available from
% http://creativecommons.org/licenses/by-sa/2.5/legalcode .
%
%%%%%%%%%%%%%%%%%%%%%%%%%%%%%%%%%%%%%%%%%%%%%%%%%%%%%%%%%%%%%%%%%%%%%%%%%%

% ----------------------------------------------------------------------
\subsection{Drag and Drop}
% ----------------------------------------------------------------------
\begin{slide}{1988}


\frametitle{Drag and Drop}
\begin{itemize}
\item Items can be:
    \begin{itemize}
    \item Dragged
    \item Dropped onto other items
    \item Dropped onto scenes
        \begin{itemize}
        \item for handling empty drop areas
        \end{itemize}
    \end{itemize}
\end{itemize}

\end{slide}

% ----------------------------------------------------------------------
\begin{slide}{1987}

\frametitle{Start Drag}
Starting an item drag is similar to dragging from a QWidget.
 \begin{itemize}
 \item Override event handlers:
     \begin{itemize}
     \item \texttt{mousePressEvent()}
     \item \texttt{mouseMoveEvent()}
     \end{itemize}
 \item In \texttt{mouseMoveEvent()}, decide if drag started? if so:
    \begin{itemize}
    \item Create a \iCls{QDrag} instance

    \item Attach a \iCls{QMimeData} to it

        \begin{itemize}
        \item See section on Drag and Drop for \texttt{QMimeData info}
        \end{itemize}
    \item Call \iClsFn{QDrag}{exec}

        \begin{itemize}
        \item Function returns when user drops
        \item Does not block event loop
        \end{itemize}
    \end{itemize}
\end{itemize}
\end{slide}

% ----------------------------------------------------------------------

\begin{slide}{1986}

\frametitle{Drop on a scene}
\begin{itemize}
\item Override \iClsFn{QGraphicsScene}{dropEvent}

    \begin{itemize}
    \item To accept drop:
        \begin{itemize}
        \item \texttt{acceptProposedAction()}
        \item \texttt{setDropAction(Qt::DropAction); accept();}
        \end{itemize}
    \end{itemize}
\item Override \iClsFn{QGraphicsScene}{dragMoveEvent}

\item Optional overrides:
    \begin{itemize}
    \item \texttt{dragEnterEvent()}, \texttt{dragLeaveEvent()}
    \end{itemize}
\end{itemize}
\end{slide}

\begin{slide}[fragile]{1985}


\frametitle{startDrag example}
\begin{cpp}
void startDrag( Qt::DropActions supportedActions ) {
    if ( selectedItems().size()>0 ) {
        QListWidgetItem *item = selectedItems()[0];
        QDrag* drag = new QDrag( this );
        QMimeData *mimeData = new QMimeData; [...]
        QGraphicsItem* gitem = 
               DiagramItem::createItem( item->toolType() );
        mimeData->setData( "application/x-qgraphicsitem-ptr",
                           QByteArray::number( ( qulonglong )gitem ) );
        drag->setMimeData( mimeData );
        QPixmap pix = item->icon().pixmap( 111,111 );
        drag->setPixmap( pix );
        drag->setHotSpot( pix.rect().center() );
        if ( drag->exec(supportedActions) == Qt::IgnoreAction ) {
            delete gitem; // drag cancelled, must delete item
        }
\end{cpp}
\demo{graphicsview/ex-dragdrop}
\end{slide}

% ----------------------------------------------------------------------
\begin{slide}[fragile]{1984}


\frametitle{dropEvent() on a scene}
\begin{cpp}
void DiagramScene::dropEvent( QGraphicsSceneDragDropEvent* event ) {
    if (event->mimeData()->hasFormat("application/x-qgraphicsitem-ptr")) {
         QGraphicsItem* item = reinterpret_cast<QGraphicsItem*>(
              event->mimeData()->data (
              "application/x-qgraphicsitem-ptr").toULongLong() );
         if ( item ) {
             addItem( item );
             item->setFlag( QGraphicsItem::ItemIsMovable );
             item->setFlag( QGraphicsItem::ItemIsSelectable );
             item->setFlag( QGraphicsItem::ItemIsFocusable );
             item->setPos( event->scenePos() );
             event->acceptProposedAction();
         }
    } else
        /* Call baseclass to allow per-item dropEvent */

        QGraphicsScene::dropEvent( event );
}
\end{cpp}
\demo{graphicsview/ex-dragdrop}
\end{slide}


% ----------------------------------------------------------------------
\begin{slide}[fragile]{1983}


\frametitle{Drop on an item}
\begin{itemize}
\item To drop into an item:
    \begin{itemize}
    \item Override \texttt{dragEnterEvent()}
    \item Optional override: \texttt{dragMoveEvent()} 
       (if the item can only accept drops in some parts of its area)
    \end{itemize}

\begin{cpp}
void DiagramItem::dragEnterEvent(QGraphicsSceneDragDropEvent* e){
    if ( e->mimeData()->hasColor() )
        e->acceptProposedAction();
}

void DiagramScene::dragEnterEvent(QGraphicsSceneDragDropEvent* e){
     if (e->mimeData()->hasFormat(
     "application/x-qgraphicsitem-ptr"))
         e->acceptProposedAction();
     else
         QGraphicsScene::dragEnterEvent(e);
}
\end{cpp}
\demo{graphicsview/ex-dragdrop}
\end{itemize}
\end{slide}


