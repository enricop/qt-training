%%%%%%%%%%%%%%%%%%%%%%%%%%%%%%%%%%%%%%%%%%%%%%%%%%%%%%%%%%%%%%%%%%%%%%%%%%
%
% Copyright (c) 2008-2011, Nokia Corporation and/or its subsidiary(-ies).
% All rights reserved.
%
% This work, unless otherwise expressly stated, is licensed under a
% Creative Commons Attribution-ShareAlike 2.5.
%
% The full license document is available from
% http://creativecommons.org/licenses/by-sa/2.5/legalcode .
%
%%%%%%%%%%%%%%%%%%%%%%%%%%%%%%%%%%%%%%%%%%%%%%%%%%%%%%%%%%%%%%%%%%%%%%%%%%

\subsection{Performance Tuning}



% ----------------------------------------------------------------------
\begin{slide}[fragile]{1982}

\frametitle{Level of Detail}
\begin{itemize}
\item Don't draw what you can't see!
\item \iCls{QStyleOptionGraphicsItem} passed to \texttt{paint()}

    \begin{itemize}
    \item Contains \texttt{palette}, \texttt{state}, \texttt{matrix
    } members
    \item \texttt{qreal levelOfDetailFromTransform(QTransform T)} method   
    \end{itemize}
\item "levelOfDetail" is max width/height of the unity rectangle
needed to draw this shape onto a QPainter with a QTransform of T.
\item use \texttt{QPainter::worldTransform()}for current transform.
    \begin{itemize}
\begin{cpp}
double levelOfDetail = 
levelOfDetailFromTransform(painter->worldTransform());
\end{cpp}
    \item Zoomed out: \texttt{levelOfDetail} < 1.0
    \item Zoomed in: \texttt{levelOfDetail} > 1.0
    \end{itemize}
\end{itemize}
\end{slide}

% ----------------------------------------------------------------------
\begin{slide}[fragile]{1981}


\frametitle{Level of detail: Chip demo}
\centeredImage{graphicsview2/images/chips}

\qtdemo{demos/chip}
\end{slide}


% ----------------------------------------------------------------------
\begin{slide}[fragile]{1980}


\frametitle{Level of detail: Chip demo 2}
\begin{cpp}
void Chip::paint(QPainter *painter, 
    const QStyleOptionGraphicsItem *option, QWidget *)
{
    const qreal lod = option->levelOfDetailFromTransform(
                    painter->worldTransform());
    [ ... ]
    if (lod >= 2) {
        QFont font("Times", 10);
        font.setStyleStrategy(QFont::ForceOutline);
        painter->save();        
        painter->setFont(font);
        painter->scale(0.1, 0.1);
        painter->drawText(170, 180, QString("Model: VSC-2000 ..."
        painter->drawText(170, 220, QString("Manufacturer: ..."
        painter->restore();
    }
\end{cpp}
\qtdemo{demos/chip}
\end{slide}


% ----------------------------------------------------------------------
\begin{slide}[fragile]{1979}


\frametitle{Caching tips}
\begin{itemize}
\item Cache item painting into a pixmap
    \begin{itemize}
    \item So \texttt{paint()} runs faster
    \end{itemize}
\item Cache \texttt{boundingRect()} and \texttt{shape()}
    \begin{itemize}
    \item Avoid recomputing expensive operations that stay the same
    \item Be sure to invalidate manually cached items after zooming and other transforms
    \end{itemize}
\begin{cpp}
QRectF MyItem::boundingRect() const {
    if (m_rect.isNull()) calculateBoundingRect();
    return m_rect;
}

QPainterPath MyItem::shape() const {
    if (m_shape.isEmpty()) calculateShape();
    return m_shape;
}
\end{cpp}
\end{itemize}
\end{slide}


% ----------------------------------------------------------------------
\begin{slide}[fragile]{1978}


\frametitle{setCacheMode()}
\begin{itemize}
\item Property of \iCls{QGraphicsView} and \iCls{QGraphicsItem}

\item Allows caching of pre-rendered content in a \iCls{QPixmap}

    \begin{itemize}
    \item Drawn on the viewport
    \item Especially useful for gradient shape backgrounds
    \item Invalidated whenever view is transformed.
    \end{itemize}
\begin{cpp}
 QGraphicsView view;
 view.setBackgroundBrush(QImage(":/images/backgroundtile.png"));
 view.setCacheMode(QGraphicsView::CacheBackground);
\end{cpp}
\end{itemize}
\end{slide}



% ----------------------------------------------------------------------
\begin{slide}{1977}

\frametitle{Tweaking}
The following methods allow you to tweak performance of view/scene/items:
\begin{itemize}
\item \iClsFn{QGraphicsView}{setViewportUpdateMode}

\item \iClsFn{QGraphicsView}{setOptimizationFlags}

\item \iClsFn{QGraphicsScene}{setItemIndexMethod}

\item \iClsFn{QGraphicsScene}{setBspTreeDepth}

\item \iClsFn{QGraphicsItem}{setFlags}

    \begin{itemize}
    \item \texttt{ItemDoesntPropagateOpacityToChildren} and \texttt{ItemIgnoresParentOpacity} especially recommended if your items are opaque! 
    \end{itemize}
\end{itemize}
See API documentation for details. 
\end{slide}



% ----------------------------------------------------------------------
\begin{slide}{1976}

\frametitle{Tips for better performance}
\begin{itemize}
   \item \texttt{boundingRect()} and \texttt{shape()} are called frequently so they should run fast!
        \begin{itemize}
            \item \texttt{boundingRect()} should be as small as possible
            \item \texttt{shape()} should return simplest reasonable path
        \end{itemize}
   \item Try to avoid drawing gradients on the painter. Consider using pre-rendered backgrounds from images instead.
   \item It is costly to dynamically insert/remove items from the scene.  Consider hiding and reusing items instead. 
   \item Embedded widgets in a scene is costly.
   \item Try using a different paint engine (OpenGL, Direct3D, etc)
        \begin{itemize}
        \item \texttt{setViewport (new QGLWidget);}
        \end{itemize}
   \item Avoid curved and dashed lines
   \item Alpha blending and antialiasing are expensive

   
\end{itemize} 
\end{slide}

   
  
