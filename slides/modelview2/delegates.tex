% %%%%%%%%%%%%%%%%%%%%%%%%%%%%%%%%%%%%%%%%%%%%%%%%%%%%%%%%%%%%%%%%%%%%%%%%%%
%
% Copyright (c) 2008-2011, Nokia Corporation and/or its subsidiary(-ies).
% All rights reserved.
%
% This work, unless otherwise expressly stated, is licensed under a
% Creative Commons Attribution-ShareAlike 2.5.
%
% The full license document is available from
% http://creativecommons.org/licenses/by-sa/2.5/legalcode .
%
%%%%%%%%%%%%%%%%%%%%%%%%%%%%%%%%%%%%%%%%%%%%%%%%%%%%%%%%%%%%%%%%%%%%%%%%%%

\subsection {Delegates}

\begin{slide}{1320}
  \frametitle{ Item Delegates}
\flushedImage{modelview/images/modelview-overview}
  \begin{itemize}
 \item \iCls{QAbstractItemDelegate} subclasses
    \begin{itemize}
    \item Control appearance of items in views
    \item Provide edit and display mechanisms
   \end{itemize}
  \item \iCls{QItemDelegate}, \iCls{QStyledItemDelegate} (preferred)
    \begin{itemize}
    \item Default editors for dates, times and numbers 
    \item Easy to extend with custom editors for other types
    \end{itemize}
    \medskip
  \item When to go for Custom Delegates?
    \begin{itemize}
    \item More control over appearance of items
    \end{itemize}
  \end{itemize}
\end{slide}

\begin{slide}{1319}
  \frametitle{Item Appearance via Data Roles}
  \flushedImage{modelview2/images/itemappearance}
  \textit{Data table shown has no custom delegate}
  \vskip 10em
  \begin{itemize}
  \item No need for custom delegate!
  \item Use \texttt{Qt::ItemDataRole} to customize appearance
  \end{itemize}
\end{slide}

\begin{slide}[fragile]{1318}
  \frametitle{Required overrides of \iCls{QAbstractItemDelegate}}
  \begin{cpp}
class BarGraphDelegate : public QAbstractItemDelegate {
public:
  void paint(QPainter *painter, 
             const QStyleOptionViewItem &option,
             const QModelIndex &index) const;
  QSize sizeHint(const QStyleOptionViewItem &option, 
                 const QModelIndex &index) const;
};
  \end{cpp}
  
  \flushedImage{modelview2/images/bargraphdelegate}

\doc{qabstractitemdelegate.html}{QAbstractItemDelegate}

\end{slide}

\begin{slide}[fragile]{1317}
  \frametitle{Custom painting from Delegates}
  \begin{cpp}
void BarGraphDelegate::paint(painter, option, index) const {
  if(index.data(Qt::EditRole).userType() == QVariant::Int) {
    int value = index.data(Qt::EditRole).toInt();
    // prepare rect with a width proportional to value
    QRect rect(option.rect.adjusted(4,4,-4,-4));
    rect.setWidth(rect.width()*value/MAX_VALUE);
    // draw the value bar
    painter->fillRect(rect, QColor("steelblue").lighter(value));
    painter->drawText(option.rect, index.data().toString());
  }
}

QSize BarGraphDelegate::sizeHint(option, index) const {
    Q_UNUSED(index)
    return QSize(MIN_BAR_WIDTH, option.fontMetrics.height());
}
  \end{cpp}
\demo{modelview2/ex-bargraphdelegate}
\end{slide}

\subsection{Editing item data}

\begin{slide}[fragile]{13171}
\frametitle{Editing Dates, Times, and Numbers}

\begin{itemize}
\item The \iCls{QStyledItemDelegate} has an \iCls{QItemEditorFactory}
\item Creates default editors based on type of the item \texttt{data()}.
\begin{itemize}
\item \iCls{QDateEdit} for \iCls{QDate}
\item \iCls{QTimeEdit} for \iCls{QTime}
\item \iCls{QDateTimeEdit} for \iCls{QDateTime}
\item \iCls{QSpinBox} for \texttt{int}
\item \iCls{QDoubleSpinBox} for \texttt{double}
\end{itemize}
\end{itemize}
\demo{modelview2/ex-bargraphdelegate}


\end{slide}

\begin{slide}[fragile]{1316}
  \frametitle{Custom Editing from Delegate}
  \flushedImage{modelview2/images/editordelegate}
 \begin{itemize}
  \item Provides \iCls{QComboBox}
    \begin{itemize}
    \item for editing a series of values
    \end{itemize}

 \end{itemize}
 \vskip 2em

 \begin{cpp}
class CountryDelegate : public QStyledItemDelegate
{
public:
  // returns editor for editing data
  QWidget *createEditor( parent, option, index ) const;
  // sets data from model to editor
  void setEditorData( editor, index ) const;
  // sets data from editor to model
  void setModelData( editor, model, index ) const;
  // updates geometry of editor for index
  void updateEditorGeometry( editor, option, index ) const;
};   
 \end{cpp}

\end{slide}

\begin{slide}[fragile]{1315}
  \frametitle{Providing a custom editor}
  \begin{itemize}
  \item Create editor by index
 \begin{cpp}
QWidget *CountryDelegate::createEditor( ... ) const {
  QComboBox *editor = new QComboBox(parent);
  editor->addItems( m_countries );
  return editor;
}    
  \end{cpp}
\item Set data to editor
  \begin{cpp}
void CountryDelegate::setEditorData( ... ) const {
  QComboBox* combo = static_cast<QComboBox*>( editor );
  QString country = index.data().toString();
  int idx = m_countries.indexOf( country );
  combo->setCurrentIndex( idx );
}
  \end{cpp}
  \end{itemize}
\end{slide}

\begin{slide}[fragile]{1314}
  \frametitle{Submitting data to the model}
  \begin{itemize}
  \item When user finished editing
    \begin{itemize}
    \item view asks delegate to store data into model
    \end{itemize}

  \begin{cpp}
void CountryDelegate::setModelData(editor, model, index) const {
  QComboBox* combo = static_cast<QComboBox*>( editor );
  model->setData( index, combo->currentText() );
}    
  \end{cpp}

\item If editor has finished editing
  \begin{cpp}
// copy edtitors data to model
emit commitData( editor );
// close/destroy editor
emit closeEditor( editor, hint );
// hint: indicates action performed next to editing
 \end{cpp}
\end{itemize}

\end{slide}

\begin{slide}[fragile]{1313}
  \frametitle{Updating the editor's geometry}
  \begin{itemize}
  \item Delegate manages editor's geometry
  \item View provides geometry information
    \begin{itemize}
    \item \iCls{QStyleOptionViewItem}
    \end{itemize}
  \end{itemize}

  \begin{cpp}
void CountryDelegate::updateEditorGeometry( ... ) const {
  // don't allow to get smaller than editors sizeHint()
  QSize size = option.rect.size().expandedTo(editor->sizeHint());
  QRect rect(QPoint(0,0), size); 
  rect.moveCenter(option.rect.center());
  editor->setGeometry( rect );
}    
  \end{cpp}
  \demo{modelview2/ex-editordelegate}
  \begin{itemize}
  \item Case of multi-index editor
    \begin{itemize}
    \item Position editor in relation to indexes
    \end{itemize}

  \end{itemize}
\end{slide}


\begin{slide}{1312}
  \frametitle{Setting Delegates on Views}
  \begin{itemize}
  \item \texttt{view->setItemDelegate( ... )}
  \item \texttt{view->setItemDelegateForColumn( ... )}
  \item \texttt{view->setItemDelegateForRow(... )}
 \end{itemize}
\end{slide}

\begin{slide}[fragile]{1311}
  \frametitle{Type Based Delegates}
\flushedImage{modelview2/images/coloreditor}
\begin{itemize}
\item A color editor widget
\begin{cpp}
class ColorListEditor : public QComboBox {
 ...
}; 
\end{cpp}
\medskip
\item Registering editor for type QVariant::Color
\begin{cpp}
QItemEditorFactory *factory = new QItemEditorFactory;
QItemEditorCreatorBase *creator = 
        new QStandardItemEditorCreator<ColorListEditor>();
// registers item editor creator given type of data
factory->registerEditor(QVariant::Color, creator);
// new and existing delegates will use new factory
QItemEditorFactory::setDefaultFactory(factory);
\end{cpp}
\end{itemize}
\qtdemo{examples/itemviews/coloreditorfactory}
\end{slide}

