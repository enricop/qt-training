%%%%%%%%%%%%%%%%%%%%%%%%%%%%%%%%%%%%%%%%%%%%%%%%%%%%%%%%%%%%%%%%%%%%%%%%%%
%
% Copyright (c) 2008-2011, Nokia Corporation and/or its subsidiary(-ies).
% All rights reserved.
%
% This work, unless otherwise expressly stated, is licensed under a
% Creative Commons Attribution-ShareAlike 2.5.
%
% The full license document is available from
% http://creativecommons.org/licenses/by-sa/2.5/legalcode .
%
%%%%%%%%%%%%%%%%%%%%%%%%%%%%%%%%%%%%%%%%%%%%%%%%%%%%%%%%%%%%%%%%%%%%%%%%%%

\subsection{Custom Tree Models}

%----------------------------------------------------------------------
\begin{slide}{1305}\frametitle{Creating a Custom Tree Model}
  \begin{itemize}
  \item Tree models must also implement:
    \begin{itemize}
    \item \iClsFnPar{QAbstractItemModel}{index}{row, column, parent}
       \begin{itemize}
       \item Return an index for the given row, column, and parent
       \end{itemize}
    \item \iClsFnPar{QAbstractItemModel}{parent}{index}
       \begin{itemize}
       \item Return an index to the parent of the given index
       \end{itemize}
    \end{itemize}
  \image{modelview2/images/itemviews-editabletreemodel-items}
  \end{itemize}
  \demo{modelview2/ex-widgetmodel}
\end{slide}

%----------------------------------------------------------------------
\begin{slide}[fragile]{1305}\frametitle{Tree Model Data}
  \begin{itemize}
  \item Inefficient to repeatedly traverse a tree structure
  \item Solution is to store an internal handle to the item
    \begin{itemize}
    \item Direct access to the data
    \item \iClsFn{QModelIndex}{internalPointer}
    \item \iClsFn{QModelIndex}{internalId}
    \end{itemize}
  \item \iCls{QModelIndex} does not have a constructor
    \begin{itemize}
    \item Must be created with \iClsFn{QAbstractItemModel}{createIndex}
      \begin{itemize}
      \item Takes a row, column, and internal pointer or ID
      \end{itemize}
      \begin{cpp}
if (item->isValid()) {
    return createIndex(row, column, item);
}
      \end{cpp}
    \end{itemize}
  \end{itemize}
\end{slide}

%----------------------------------------------------------------------
\begin{slide}[fragile]{1304}
  \frametitle{\textbf{\texttt{QAbstractItemModel::index()}}}
  \begin{cpp}
QModelIndex TreeModel::index(int row, int column,
                             const QModelIndex &parent) const
{
    if (!hasIndex(row, column, parent)) return QModelIndex();

    TreeItem *parentItem;
    if (!parent.isValid())
        parentItem = rootItem;
    else
        parentItem = static_cast<TreeItem*>(parent.internalPointer());

    TreeItem *childItem = parentItem->child(row);
    if (childItem)
        return createIndex(row, column, childItem);
    else
        return QModelIndex();
}
  \end{cpp}
  \qtdemo{widgets/itemviews/simpletreemodel}
\end{slide}

%----------------------------------------------------------------------
\begin{slide}[fragile]{1303}
  \frametitle{\textbf{\texttt{QAbstractItemModel::parent()}}}
  \begin{cpp}
QModelIndex TreeModel::parent(const QModelIndex &index) const
{
    if (!index.isValid())
        return QModelIndex();

    TreeItem *childItem = static_cast<TreeItem*>(index.internalPointer());
    TreeItem *parentItem = childItem->parent();

    if (parentItem == rootItem) return QModelIndex();

    return createIndex(parentItem->row(), 0, parentItem);
}
  \end{cpp}
  \qtdemo{widgets/itemviews/simpletreemodel}
\end{slide}
%----------------------------------------------------------------------
\begin{slide}[fragile]{1302}\frametitle{Using the Internal Pointer}
  \begin{cpp}
QVariant TreeModel::data(const QModelIndex &index, int role) const
{
    if (!index.isValid()) return QVariant();
    if (role != Qt::DisplayRole) return QVariant();

    TreeItem *item = static_cast<TreeItem*>(index.internalPointer());

    return item->data(index.column());
}
  \end{cpp}
  \medskip
  \flushedImage{modelview2/images/simpletreemodel-example}
\end{slide}

%----------------------------------------------------------------------
% \begin{slide}{1299}
%   \frametitle{Lab: Object Inspector}
%   \begin{itemize}
%   \item Read-Only Object Inspector
%     \medskip
%   \item Optional:
%     \begin{itemize}
%     \item Editable Object Inspector
%     \end{itemize}
%   \end{itemize}


% \end{slide}
