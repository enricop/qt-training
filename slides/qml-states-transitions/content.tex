%%%%%%%%%%%%%%%%%%%%%%%%%%%%%%%%%%%%%%%%%%%%%%%%%%%%%%%%%%%%%%%%%%%%%%%%%%
%
% Copyright (c) 2008-2011, Nokia Corporation and/or its subsidiary(-ies).
% All rights reserved.
%
% This work, unless otherwise expressly stated, is licensed under a
% Creative Commons Attribution-ShareAlike 2.5.
%
% The full license document is available from
% http://creativecommons.org/licenses/by-sa/2.5/legalcode .
%
%%%%%%%%%%%%%%%%%%%%%%%%%%%%%%%%%%%%%%%%%%%%%%%%%%%%%%%%%%%%%%%%%%%%%%%%%%

\section{States and Transitions}

%----------------------------------------------------------------------
\begin{slide}{1511}\frametitle{Objectives}

Can define user interface behavior using states and transitions:

\begin{itemize}
\item Provides a way to formally specify a user interface
\item Useful way to organize application logic
\item Helps to determine if all functionality is covered
\item Can extend transitions with animations and visual effects
\end{itemize}

States and transitions are covered in the Qt documentation

\end{slide}

%%%%%%%%%%%%%%%%%%%%%%%%%%%%%%%%%%%%%%%%%%%%%%%%%%%%%%%%%%%%%%%%%%%%%%%%%%
%
% Copyright (c) 2008-2011, Nokia Corporation and/or its subsidiary(-ies).
% All rights reserved.
%
% This work, unless otherwise expressly stated, is licensed under a
% Creative Commons Attribution-ShareAlike 2.5.
%
% The full license document is available from
% http://creativecommons.org/licenses/by-sa/2.5/legalcode .
%
%%%%%%%%%%%%%%%%%%%%%%%%%%%%%%%%%%%%%%%%%%%%%%%%%%%%%%%%%%%%%%%%%%%%%%%%%%

\subsection{States}

%----------------------------------------------------------------------

\begin{slide}{1510}\frametitle{States}

States manage named items

\begin{itemize}
\item Represented by the \qic{class}{State} element
\item Each item can define a set of states
  \begin{itemize}
  \item with the \qic{type}{states} property
  \item current state is set with the \qic{type}{state} property
  \end{itemize}
\item Properties are set when a state is entered
\item Can also
  \begin{itemize}
  \item modify anchors
  \item change the parents of items
  \item run scripts
  \end{itemize}
\end{itemize}

\doc{qdeclarativestates.html}{QML States}
\end{slide}

%----------------------------------------------------------------------

\begin{slide}{1509}\frametitle{States Example}

\flushedImage{qml-states-transitions/images/states-stop-light.png}
% declarative-uis/states-transitions/states.qml
\begin{qml}
\qtt{\qc{keyword}{import}~\qc{class}{QtQuick}~\qc{number}{2.0}}\\
\vspace*{0.5em}
\qtt{\qc{class}{Rectangle}~\{}\\
\qtt{~~\qc{type}{width}:~\qc{number}{150};~\qc{type}{height}:~\qc{number}{250}}\\
\vspace*{0.5em}
\qtt{~~\qc{class}{Rectangle}~\{}\\
\qtt{~~~~~~\qc{type}{id}:~stop\_light}\\
\qtt{~~~~~~\qc{type}{x}:~\qc{number}{25};~\qc{type}{y}:~\qc{number}{15};~\qc{type}{width}:~\qc{number}{100};~\qc{type}{height}:~\qc{number}{100}}\\
\qtt{~~\}}\\
\qtt{~~\qc{class}{Rectangle}~\{}\\
\qtt{~~~~~~\qc{type}{id}:~go\_light}\\
\qtt{~~~~~~\qc{type}{x}:~\qc{number}{25};~\qc{type}{y}:~\qc{number}{135};~\qc{type}{width}:~\qc{number}{100};~\qc{type}{height}:~\qc{number}{100}}\\
\qtt{~~\}}\\
\qtt{~~...}\\
\end{qml}

\begin{itemize}
\item Prepare each item with an \qic{type}{id}
\item Set up properties not modified by states
\end{itemize}      


\end{slide}

%----------------------------------------------------------------------

\begin{slide}{1508}\frametitle{Defining States}

% declarative-uis/states-transitions/states.qml
\begin{qml}
\qtt{~~\qc{type}{states}:~[}\\
\qtt{~~~~\qc{class}{State}~\{}\\
\qtt{~~~~~~\qc{type}{name}:~\qc{string}{"stop"}}\\
\qtt{~~~~~~\qc{class}{PropertyChanges}~\{~\qc{type}{target}:~stop\_light;~\qc{type}{color}:~\qc{string}{"red"}~\}}\\
\qtt{~~~~~~\qc{class}{PropertyChanges}~\{~\qc{type}{target}:~go\_light;~\qc{type}{color}:~\qc{string}{"black"}~\}}\\
\qtt{~~~~\},}\\
\qtt{~~~~\qc{class}{State}~\{}\\
\qtt{~~~~~~\qc{type}{name}:~\qc{string}{"go"}}\\
\qtt{~~~~~~\qc{class}{PropertyChanges}~\{~\qc{type}{target}:~stop\_light;~\qc{type}{color}:~\qc{string}{"black"}~\}}\\
\qtt{~~~~~~\qc{class}{PropertyChanges}~\{~\qc{type}{target}:~go\_light;~\qc{type}{color}:~\qc{string}{"green"}~\}}\\
\qtt{~~~~\}}\\
\qtt{~~]}
\end{qml}

\begin{itemize}
\item Define states with names: \qic{string}{"stop"} and \qic{string}{"go"}
\item Set up properties for each state with \qic{class}{PropertyChanges}
  \begin{itemize}
  \item defining differences from the default values
  \end{itemize}
\end{itemize}
                                
\demo{qml-states-transitions/ex-states/states.qml}

\end{slide}

%----------------------------------------------------------------------

\begin{slide}{1507}\frametitle{Setting the State}

\flushedImage{qml-states-transitions/images/states-stop-go-lights.pdf}

Define an initial state:

\begin{qml}
\qtt{~~\qc{type}{state}:~\qc{string}{"stop"}}
\end{qml}

\vspace*{1em}
Use a \qic{class}{MouseArea} to switch between\\
states:

\vspace*{0.5em}
\begin{qml}
\qtt{~~\qc{class}{MouseArea}~\{}\\
\qtt{~~~~~~\qc{type}{anchors.fill}:~parent}\\
\qtt{~~~~~~\qc{type}{onClicked}:~parent.state~\qc{operator}{==}~\qc{string}{"stop"}~?}\\
\qtt{~~~~~~~~~~~~~~~~~parent.state~\qc{operator}{=}~\qc{string}{"go"}~:~parent.state~\qc{operator}{=}~\qc{string}{"stop"}}\\
\qtt{~~\}}
\end{qml}

\vspace*{0.5em}
\begin{itemize}
\item Reacts to a click on the user interface
  \begin{itemize}
  \item toggles the parent's \qic{type}{state} property
  \item between \qic{string}{"stop"} and \qic{string}{"go"} states
  \end{itemize}
\end{itemize}

\end{slide}

%----------------------------------------------------------------------

\begin{slide}{1506}\frametitle{Changing Properties}

States change properties with the \qic{class}{PropertyChanges} element:

\begin{qml}
\qtt{~~~~\qc{class}{State}~\{}\\
\qtt{~~~~~~\qc{type}{name}:~\qc{string}{"go"}}\\
\qtt{~~~~~~\qc{class}{PropertyChanges}~\{~\qc{type}{target}:~stop\_light;~\qc{type}{color}:~\qc{string}{"black"}~\}}\\
\qtt{~~~~~~\qc{class}{PropertyChanges}~\{~\qc{type}{target}:~go\_light;~\qc{type}{color}:~\qc{string}{"green"}~\}}\\
\qtt{~~~~\}}\\
\end{qml}

\begin{itemize}
\item Acts on a target element named using the \qic{type}{target} property
  \begin{itemize}
  \item the \qic{type}{target} refers to an \qic{type}{id}
  \end{itemize}
\item Applies the other property definitions to the target element
  \begin{itemize}
  \item one \qic{class}{PropertyChanges} element can redefine multiple properties
  \end{itemize}
\item Property definitions are evaluated when the state is entered
\item \qic{class}{PropertyChanges} describes new property values for an item
  \begin{itemize}
  \item new values are assigned to items when the state is entered
  \item \textit{properties left unspecified are assigned their default values}
  \end{itemize}
\end{itemize}

% Use of signals (MouseArea) to change the state property of an Item.
% Or states can be set using JavaScript.
% Use of states property, defining State objects which reference elements
% by their id properties.

% When are the property changes evaluated?
% Can you update the target property of a target element?

\end{slide}

%%%%%%%%%%%%%%%%%%%%%%%%%%%%%%%%%%%%%%%%%%%%%%%%%%%%%%%%%%%%%%%%%%%%%%%%%%
%
% Copyright (c) 2008-2011, Nokia Corporation and/or its subsidiary(-ies).
% All rights reserved.
%
% This work, unless otherwise expressly stated, is licensed under a
% Creative Commons Attribution-ShareAlike 2.5.
%
% The full license document is available from
% http://creativecommons.org/licenses/by-sa/2.5/legalcode .
%
%%%%%%%%%%%%%%%%%%%%%%%%%%%%%%%%%%%%%%%%%%%%%%%%%%%%%%%%%%%%%%%%%%%%%%%%%%

\subsection{State Conditions}

%----------------------------------------------------------------------

\begin{slide}{1496}\frametitle{State Conditions}

Another way to use states:

\begin{itemize}
\item Let the \qic{class}{State} decide when to be active
  \begin{itemize}
  \item using conditions to determine if a state is active
  \end{itemize}
\item Define the \qic{type}{when} property
  \begin{itemize}
  \item using an expression that evaluates to \qic{number}{true} or
  \qic{number}{false}
  \end{itemize}
\item Only one state in a \qic{type}{states} list should be active
  \begin{itemize}
  \item Ensure \qic{type}{when} is \qic{number}{true} for only one state
  \end{itemize}
\end{itemize}

\centeredImage{qml-states-transitions/images/state-conditions.pdf}

\demo{qml-states-transitions/ex-states/states-when.qml}

% Use of the State's when property to define when a state is active.
% Don't define the when condition in terms of the state.
% Use of the Item's state property to define a condition that chooses a state for an
% element. (JavaScript)

% Easier to use signals to switch between states than to try and use when
% conditions.

\end{slide}

%----------------------------------------------------------------------

\begin{slide}{1495}\frametitle{State Conditions Example}

\flushedImage{qml-states-transitions/images/state-conditions-with-text.pdf}
% declarative-uis/states-transitions/states-when.qml
\begin{qml}
\qtt{\qc{keyword}{import}~\qc{class}{QtQuick}~\qc{number}{2.0}}\\
\vspace*{0.5em}
\qtt{\qc{class}{Rectangle}~\{}\\
\qtt{~~~~\qc{type}{width}:~\qc{number}{250};~\qc{type}{height}:~\qc{number}{50};~\qc{type}{color}:~\qc{string}{"\#ccffcc"}}\\
\vspace*{0.5em}
\qtt{~~~~\qc{class}{TextInput}~\{~\qc{type}{id}:~text\_field}\\
\qtt{~~~~~~~~~~~~~~~~\qc{type}{text}:~\qc{string}{"Enter~text..."}~...~\}}\\
\vspace*{0.5em}
\qtt{~~~~\qc{class}{Image}~\{}\\
\qtt{~~~~~~~~\qc{type}{id}:~clear\_button}\\
\qtt{~~~~~~~~\qc{type}{source}:~\qc{string}{"../images/clear.svg"}}\\
\qtt{~~~~~~~~...}\\
\vspace*{0.5em}
\qtt{~~~~~~~~\qc{class}{MouseArea}~\{~\qc{type}{anchors.fill}:~parent}\\
\qtt{~~~~~~~~~~~~~~~~~~~~\qc{type}{onClicked}:~text\_field.text~\qc{operator}{=}~""~\}}\\
\qtt{~~~~\}}\\
\qtt{~~~~...}
\end{qml}

\begin{itemize}
\item Define default property values and actions
\end{itemize}

\end{slide}

%----------------------------------------------------------------------

\begin{slide}{1494}\frametitle{State Conditions Example}

\flushedImage{qml-states-transitions/images/state-conditions-with-text.pdf}
% declarative-uis/states-transitions/states-when.qml
\begin{qml}
\qtt{~~~~\qc{type}{states}:~[}\\
\qtt{~~~~~~\qc{class}{State}~\{}\\
\qtt{~~~~~~~~\qc{type}{name}:~\qc{string}{"with~text"}}\\
\qtt{~~~~~~~~\qc{type}{when}:~text\_field.text~\qc{operator}{!=}~""}\\
\qtt{~~~~~~~~\qc{class}{PropertyChanges}~\{}\\
\qtt{~~~~~~~~~~~~\qc{type}{target}:~clear\_button;~\qc{type}{opacity}:~\qc{number}{1.0}~\}}\\
\qtt{~~~~~~\},}
\end{qml}
\flushedImage{qml-states-transitions/images/state-conditions-without-text.pdf}
\begin{qml}
\qtt{~~~~~~\qc{class}{State}~\{}\\
\qtt{~~~~~~~~\qc{type}{name}:~\qc{string}{"without~text"}}\\
\qtt{~~~~~~~~\qc{type}{when}:~text\_field.text~\qc{operator}{==}~""}\\
\qtt{~~~~~~~~\qc{class}{PropertyChanges}~\{}\\
\qtt{~~~~~~~~~~~~\qc{type}{target}:~clear\_button;~\qc{type}{opacity}:~\qc{number}{0.25}~\}}\\
\qtt{~~~~~~~~\qc{class}{PropertyChanges}~\{}\\
\qtt{~~~~~~~~~~~~\qc{type}{target}:~text\_field;~\qc{type}{focus}:~\qc{number}{true}~\}}\\
\qtt{~~~~~~\}}\\
\qtt{~~~~]}\\
\end{qml}

\vspace*{0.25em}
\begin{itemize}
\item A clear button that fades out when there is no text
\item Do not need to define \qic{type}{state}
\end{itemize}

\end{slide}


%%%%%%%%%%%%%%%%%%%%%%%%%%%%%%%%%%%%%%%%%%%%%%%%%%%%%%%%%%%%%%%%%%%%%%%%%%
%
% Copyright (c) 2008-2011, Nokia Corporation and/or its subsidiary(-ies).
% All rights reserved.
%
% This work, unless otherwise expressly stated, is licensed under a
% Creative Commons Attribution-ShareAlike 2.5.
%
% The full license document is available from
% http://creativecommons.org/licenses/by-sa/2.5/legalcode .
%
%%%%%%%%%%%%%%%%%%%%%%%%%%%%%%%%%%%%%%%%%%%%%%%%%%%%%%%%%%%%%%%%%%%%%%%%%%

%----------------------------------------------------------------------

\subsection{Transitions}
\begin{slide}{1505}\frametitle{Transitions}

\begin{itemize}
\item Define how items change when switching states
\item Applied to two or more states
\item Usually describe how items are animated
\end{itemize}

\vspace*{0.5em}
\centeredImage{qml-states-transitions/images/states-stop-go-transitions.pdf}

\vspace*{0.5em}
\begin{itemize}
\item Let's add transitions to a previous example...
\end{itemize}

\demo{qml-states-transitions/ex-transitions/transitions.qml}

\end{slide}

%----------------------------------------------------------------------

\begin{slide}{1504}\frametitle{Transitions Example}

\begin{qml}
\qtt{~~\qc{type}{transitions}:~[}\\
\qtt{~~~~\qc{class}{Transition}~\{}\\
\qtt{~~~~~~~~\qc{type}{from}:~\qc{string}{"stop"};~\qc{type}{to}:~\qc{string}{"go"}}\\
\qtt{~~~~~~~~\qc{lightgray}{PropertyAnimation~\{}}\\
\qtt{~~~~~~~~~~~~\qc{lightgray}{target:~stop\_light}}\\
\qtt{~~~~~~~~~~~~\qc{lightgray}{properties:~"color";~duration:~1000}}\\
\qtt{~~~~~~~~\qc{lightgray}{\}}}\\
\qtt{~~~~\},}\\
\qtt{~~~~\qc{class}{Transition}~\{}\\
\qtt{~~~~~~~~\qc{type}{from}:~\qc{string}{"go"};~\qc{type}{to}:~\qc{string}{"stop"}}\\
\qtt{~~~~~~~~\qc{lightgray}{PropertyAnimation~\{}}\\
\qtt{~~~~~~~~~~~~\qc{lightgray}{target:~go\_light}}\\
\qtt{~~~~~~~~~~~~\qc{lightgray}{properties:~"color";~duration:~1000}}\\
\qtt{~~~~~~~~\qc{lightgray}{\}}}\\
\qtt{~~~~\}~]}
\end{qml}

\begin{itemize}
\item The \qic{type}{transitions} property defines a list of transitions
\item Transitions between \qic{string}{"stop"} and \qic{string}{"go"} states
\end{itemize}     

\end{slide}

%----------------------------------------------------------------------

\begin{slide}{1503}\frametitle{Wildcard Transitions}

\flushedImage{qml-states-transitions/images/states-stop-go-wildcard-transitions.pdf}
\begin{qml}
\qtt{~~\qc{type}{transitions}:~[}\\
\qtt{~~~~\qc{class}{Transition}~\{}\\
\qtt{~~~~~~~~\qc{type}{from}:~\qc{string}{"*"};~\qc{type}{to}:~\qc{string}{"*"}}\\
\qtt{~~~~~~~~\qc{lightgray}{PropertyAnimation}~\{}\\
\qtt{~~~~~~~~~~~~\qc{lightgray}{target:~stop\_light}}\\
\qtt{~~~~~~~~~~~~\qc{lightgray}{properties:~"color";~duration:~1000}}\\
\qtt{~~~~~~~~\qc{lightgray}{\}}}\\
\qtt{~~~~~~~~\qc{lightgray}{PropertyAnimation~\{}}\\
\qtt{~~~~~~~~~~~~\qc{lightgray}{target:~go\_light}}\\
\qtt{~~~~~~~~~~~~\qc{lightgray}{properties:~"color";~duration:~1000}}\\
\qtt{~~~~~~~~\qc{lightgray}{\}}}\\
\qtt{~~~~\}~]}\\
\end{qml}

\begin{itemize}
\item Use \qic{string}{"*"} to represent any state
\item Now the same transition is used whenever the state changes
\item Both lights fade at the same time
\end{itemize}
                      
\demo{qml-states-transitions/ex-transitions/transitions-multi.qml}

% Possible questions about using "*" for only one state.

\end{slide}

%----------------------------------------------------------------------

\begin{slide}{1502}\frametitle{Reversible Transitions}

\begin{itemize}
\item Useful when two transitions operate on the same properties
\end{itemize}

\flushedImage{qml-states-transitions/images/state-conditions-with-text.png}
\begin{qml}
\qtt{~~~~\qc{type}{transitions}:~[}\\
\qtt{~~~~~~~~\qc{class}{Transition}~\{}\\
\qtt{~~~~~~~~~~~~\qc{type}{from}:~\qc{string}{"with~text"};~\qc{type}{to}:~\qc{string}{"without~text"}}\\
\qtt{~~~~~~~~~~~~\qc{type}{reversible}:~\qc{number}{true}}\\
\qtt{~~~~~~~~~~~~\qc{class}{PropertyAnimation}~\{}\\
\qtt{~~~~~~~~~~~~~~~~\qc{type}{target}:~clear\_button}\\
\qtt{~~~~~~~~~~~~~~~~\qc{type}{properties}:~\qc{string}{"opacity"};~\qc{type}{duration}:~\qc{number}{1000}}\\
\qtt{~~~~~~~~~~~~\}}\\
\qtt{~~~~~~~~\}}\\
\qtt{~~~~]}
\end{qml}

\begin{itemize}
\item Transition applies from \qic{string}{"with text"} to \qic{string}{"without text"}
  \begin{itemize}
  \item and back again from \qic{string}{"without text"} to \qic{string}{"with text"}
  \end{itemize}
\item No need to define two separate transitions
\end{itemize}

\demo{qml-states-transitions/ex-transitions/transitions-reversible.qml}

\end{slide}

%----------------------------------------------------------------------

\begin{slide}{1501}\frametitle{Using States and Transitions}

\begin{itemize}
\item Avoid defining complex statecharts
  \begin{itemize}
  \item not just one statechart to manage the entire UI
  \item usually defined individually for each component
  \item link together components with internal states
  \end{itemize}
\item Setting state with script code
  \begin{itemize}
  \item easy to do, but might be difficult to manage
  \end{itemize}
\item Setting state with state conditions
  \begin{itemize}
  \item more declarative style
  \item can be difficult to specify conditions
  \end{itemize}
\item Using animations in transitions
  \begin{itemize}
  \item do not specify \qic{type}{from} and \qic{type}{to} properties
  \item use \qic{class}{PropertyChanges} elements in state definitions
  \end{itemize}
\end{itemize}

% Scope - where are states defined? Depends on encapsulation
% (inside the Item they are defined for vs. inside an Item that manages its children).

\end{slide}

%----------------------------------------------------------------------

\begin{slide}{1500}\frametitle{Summary~\textendash~States}

\qic{class}{State} items manage properties of other items:

\begin{itemize}
\item Items define states using the \qic{type}{states} property
  \begin{itemize}
  \item must define a unique \qic{type}{name} for each state
  \end{itemize}
\item Useful to assign \qic{type}{id} properties to items
  \begin{itemize}
  \item use \qic{class}{PropertyChanges} to modify items
  \end{itemize}
\item The \qic{type}{state} property contains the current state
  \begin{itemize}
  \item set this using JavaScript code, or
  \item define a \qic{type}{when} condition for each state
  \end{itemize}
\end{itemize}

\end{slide}

%----------------------------------------------------------------------

\begin{slide}{1499}\frametitle{Summary~\textendash~Transitions}

\qic{class}{Transition} items describe how items change between states:

\begin{itemize}
\item Items define transitions using the \qic{type}{transitions} property
\item Transitions refer to the states they are between
  \begin{itemize}
  \item using the \qic{type}{from} and \qic{type}{to} properties
  \item using a wildcard value, \qic{string}{"*"}, to mean any state
  \end{itemize}
\item Transitions can be reversible
  \begin{itemize}
  \item used when the \qic{type}{from} and \qic{type}{to} properties are reversed
  \end{itemize}
\end{itemize}

\end{slide}             



%----------------------------------------------------------------------
\begin{slide}{1498}\frametitle{Exercise~\textendash~States~and~Transitions}

\begin{itemize}
\item How do you define a set of states for an item?
% Define the states property, using a list of State elements.
\item What defines the current state?
% The state property.
\item Do you need to define a name for all states?
% Yes, states need names to distinguish them.
\item Do state names need to be globally unique?
% No, just unique for the item they are defined for.
\end{itemize}

\end{slide}

%----------------------------------------------------------------------
\begin{slide}{1497}\frametitle{Lab~\textendash~Light~Switch}

\vspace*{-1em}
\begin{center}
\imageDoubleWidth{qml-states-transitions/images/switch-off.png}
\hspace*{0.5em}
\imageDoubleWidth{qml-states-transitions/images/switch-on.png}
\end{center}

\vspace*{-0.5em}
\begin{itemize}
\item Using the partial solutions as hints, create a user interface similar
to the one shown above.
\end{itemize}

\begin{center}
\imageDoubleWidth{qml-states-transitions/images/switch-hint1.png}
\hspace*{0.5em}
\imageDoubleWidth{qml-states-transitions/images/switch-hint2.png}
\hspace*{0.5em}
\imageDoubleWidth{qml-states-transitions/images/switch-hint3.png}
\end{center}

\vspace*{-0.5em}
\begin{itemize}
\item Adapt the reversible transition code from earlier and add it to the
example. 

\lab{qml-states-transitions/lab-switch}
\end{itemize}

\end{slide}

                                                                   

