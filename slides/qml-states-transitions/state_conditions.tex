%%%%%%%%%%%%%%%%%%%%%%%%%%%%%%%%%%%%%%%%%%%%%%%%%%%%%%%%%%%%%%%%%%%%%%%%%%
%
% Copyright (c) 2008-2011, Nokia Corporation and/or its subsidiary(-ies).
% All rights reserved.
%
% This work, unless otherwise expressly stated, is licensed under a
% Creative Commons Attribution-ShareAlike 2.5.
%
% The full license document is available from
% http://creativecommons.org/licenses/by-sa/2.5/legalcode .
%
%%%%%%%%%%%%%%%%%%%%%%%%%%%%%%%%%%%%%%%%%%%%%%%%%%%%%%%%%%%%%%%%%%%%%%%%%%

\subsection{State Conditions}

%----------------------------------------------------------------------

\begin{slide}{1496}\frametitle{State Conditions}

Another way to use states:

\begin{itemize}
\item Let the \qic{class}{State} decide when to be active
  \begin{itemize}
  \item using conditions to determine if a state is active
  \end{itemize}
\item Define the \qic{type}{when} property
  \begin{itemize}
  \item using an expression that evaluates to \qic{number}{true} or
  \qic{number}{false}
  \end{itemize}
\item Only one state in a \qic{type}{states} list should be active
  \begin{itemize}
  \item Ensure \qic{type}{when} is \qic{number}{true} for only one state
  \end{itemize}
\end{itemize}

\centeredImage{qml-states-transitions/images/state-conditions.pdf}

\demo{qml-states-transitions/ex-states/states-when.qml}

% Use of the State's when property to define when a state is active.
% Don't define the when condition in terms of the state.
% Use of the Item's state property to define a condition that chooses a state for an
% element. (JavaScript)

% Easier to use signals to switch between states than to try and use when
% conditions.

\end{slide}

%----------------------------------------------------------------------

\begin{slide}{1495}\frametitle{State Conditions Example}

\flushedImage{qml-states-transitions/images/state-conditions-with-text.png}
% declarative-uis/states-transitions/states-when.qml
\begin{qml}
\qtt{\qc{keyword}{import}~\qc{class}{QtQuick}~\qc{number}{2.0}}\\
\vspace*{0.5em}
\qtt{\qc{class}{Rectangle}~\{}\\
\qtt{~~~~\qc{type}{width}:~\qc{number}{250};~\qc{type}{height}:~\qc{number}{50};~\qc{type}{color}:~\qc{string}{"\#ccffcc"}}\\
\vspace*{0.5em}
\qtt{~~~~\qc{class}{TextInput}~\{~\qc{type}{id}:~text\_field}\\
\qtt{~~~~~~~~~~~~~~~~\qc{type}{text}:~\qc{string}{"Enter~text..."}~...~\}}\\
\vspace*{0.5em}
\qtt{~~~~\qc{class}{Image}~\{}\\
\qtt{~~~~~~~~\qc{type}{id}:~clear\_button}\\
\qtt{~~~~~~~~\qc{type}{source}:~\qc{string}{"../images/clear.svg"}}\\
\qtt{~~~~~~~~...}\\
\vspace*{0.5em}
\qtt{~~~~~~~~\qc{class}{MouseArea}~\{~\qc{type}{anchors.fill}:~parent}\\
\qtt{~~~~~~~~~~~~~~~~~~~~\qc{type}{onClicked}:~text\_field.text~\qc{operator}{=}~""~\}}\\
\qtt{~~~~\}}\\
\qtt{~~~~...}
\end{qml}

\begin{itemize}
\item Define default property values and actions
\end{itemize}

\end{slide}

%----------------------------------------------------------------------

\begin{slide}{1494}\frametitle{State Conditions Example}

\flushedImage{qml-states-transitions/images/state-conditions-with-text.png}
% declarative-uis/states-transitions/states-when.qml
\begin{qml}
\qtt{~~~~\qc{type}{states}:~[}\\
\qtt{~~~~~~\qc{class}{State}~\{}\\
\qtt{~~~~~~~~\qc{type}{name}:~\qc{string}{"with~text"}}\\
\qtt{~~~~~~~~\qc{type}{when}:~text\_field.text~\qc{operator}{!=}~""}\\
\qtt{~~~~~~~~\qc{class}{PropertyChanges}~\{~\qc{type}{target}:~clear\_button;~\qc{type}{opacity}:~\qc{number}{1.0}~\}}\\
\qtt{~~~~~~\},}
\end{qml}
\flushedImage{qml-states-transitions/images/state-conditions-without-text.png}
\begin{qml}
\qtt{~~~~~~\qc{class}{State}~\{}\\
\qtt{~~~~~~~~\qc{type}{name}:~\qc{string}{"without~text"}}\\
\qtt{~~~~~~~~\qc{type}{when}:~text\_field.text~\qc{operator}{==}~""}\\
\qtt{~~~~~~~~\qc{class}{PropertyChanges}~\{~\qc{type}{target}:~clear\_button;~\qc{type}{opacity}:~\qc{number}{0.25}~\}}\\
\qtt{~~~~~~~~\qc{class}{PropertyChanges}~\{~\qc{type}{target}:~text\_field;~\qc{type}{focus}:~\qc{number}{true}~\}}\\
\qtt{~~~~~~\}}\\
\qtt{~~~~]}\\
\qtt{\}}
\end{qml}

\vspace*{0.25em}
\begin{itemize}
\item A clear button that fades out when there is no text
\item Do not need to define \qic{type}{state}
\end{itemize}

\end{slide}

%----------------------------------------------------------------------

\begin{slide}{1493}\frametitle{State Conditions Example}

\flushedImage{qml-states-transitions/images/state-conditions-with-text.png}
% declarative-uis/states-transitions/states-when.qml
\begin{qml}
\qtt{~~~~\qc{lightgray}{states:~[}}\\
\qtt{~~~~~~\qc{class}{State}~\{}\\
\qtt{~~~~~~~~\qc{type}{name}:~\qc{string}{"with~text"}}\\
\qtt{~~~~~~~~\qc{type}{when}:~text\_field.text~\qc{operator}{!=}~""}\\
\qtt{~~~~~~~~\qc{lightgray}{PropertyChanges~\{~target:~clear\_button;~opacity:~1.0~\}}}\\
\qtt{~~~~~~\},}
\end{qml}
\flushedImage{qml-states-transitions/images/state-conditions-without-text.png}
\begin{qml}
\qtt{~~~~~~\qc{class}{State}~\{}\\
\qtt{~~~~~~~~\qc{type}{name}:~\qc{string}{"without~text"}}\\
\qtt{~~~~~~~~\qc{type}{when}:~text\_field.text~\qc{operator}{==}~""}\\
\qtt{~~~~~~~~\qc{lightgray}{PropertyChanges~\{~target:~clear\_button;~opacity:~0.25~\}}}\\
\qtt{~~~~~~~~\qc{lightgray}{PropertyChanges~\{~target:~text\_field;~focus:~true~\}}}\\
\qtt{~~~~~~\}}\\
\qtt{~~~~\qc{lightgray}{]}}
\end{qml}

\begin{itemize}
\item Two states with mutually-exclusive \qic{type}{when} conditions
\item Initial state is the state with a true \qic{type}{when} condition
  \begin{itemize}
  \item \qic{type}{state} property is updated with the current state \qic{type}{name}
  \end{itemize}
\end{itemize}

% Still need to define state names because the underlying mechanism appears
% to simply apply the name of the state which passes its condition to the
% state property. If no names are defined, there is no way to distinguish
% between states.

\end{slide}
