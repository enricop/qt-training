%%%%%%%%%%%%%%%%%%%%%%%%%%%%%%%%%%%%%%%%%%%%%%%%%%%%%%%%%%%%%%%%%%%%%%%%%%
%
% Copyright (c) 2008-2011, Nokia Corporation and/or its subsidiary(-ies).
% All rights reserved.
%
% This work, unless otherwise expressly stated, is licensed under a
% Creative Commons Attribution-ShareAlike 2.5.
%
% The full license document is available from
% http://creativecommons.org/licenses/by-sa/2.5/legalcode .
%
%%%%%%%%%%%%%%%%%%%%%%%%%%%%%%%%%%%%%%%%%%%%%%%%%%%%%%%%%%%%%%%%%%%%%%%%%%

%----------------------------------------------------------------------

\subsection{Transitions}
\begin{slide}{1505}\frametitle{Transitions}

\begin{itemize}
\item Define how items change when switching states
\item Applied to two or more states
\item Usually describe how items are animated
\end{itemize}

\vspace*{0.5em}
\centeredImage{qml-states-transitions/images/states-stop-go-transitions.pdf}

\vspace*{0.5em}
\begin{itemize}
\item Let's add transitions to a previous example...
\end{itemize}

\demo{qml-states-transitions/ex-transitions/transitions.qml}

\end{slide}

%----------------------------------------------------------------------

\begin{slide}{1504}\frametitle{Transitions Example}

\begin{qml}
\qtt{~~\qc{type}{transitions}:~[}\\
\qtt{~~~~\qc{class}{Transition}~\{}\\
\qtt{~~~~~~~~\qc{type}{from}:~\qc{string}{"stop"};~\qc{type}{to}:~\qc{string}{"go"}}\\
\qtt{~~~~~~~~\qc{lightgray}{PropertyAnimation~\{}}\\
\qtt{~~~~~~~~~~~~\qc{lightgray}{target:~stop\_light}}\\
\qtt{~~~~~~~~~~~~\qc{lightgray}{properties:~"color";~duration:~1000}}\\
\qtt{~~~~~~~~\qc{lightgray}{\}}}\\
\qtt{~~~~\},}\\
\qtt{~~~~\qc{class}{Transition}~\{}\\
\qtt{~~~~~~~~\qc{type}{from}:~\qc{string}{"go"};~\qc{type}{to}:~\qc{string}{"stop"}}\\
\qtt{~~~~~~~~\qc{lightgray}{PropertyAnimation~\{}}\\
\qtt{~~~~~~~~~~~~\qc{lightgray}{target:~go\_light}}\\
\qtt{~~~~~~~~~~~~\qc{lightgray}{properties:~"color";~duration:~1000}}\\
\qtt{~~~~~~~~\qc{lightgray}{\}}}\\
\qtt{~~~~\}~]}
\end{qml}

\begin{itemize}
\item The \qic{type}{transitions} property defines a list of transitions
\item Transitions between \qic{string}{"stop"} and \qic{string}{"go"} states
\end{itemize}     

\end{slide}

%----------------------------------------------------------------------

\begin{slide}{1503}\frametitle{Wildcard Transitions}

\flushedImage{qml-states-transitions/images/states-stop-go-wildcard-transitions.pdf}
\begin{qml}
\qtt{~~\qc{type}{transitions}:~[}\\
\qtt{~~~~\qc{class}{Transition}~\{}\\
\qtt{~~~~~~~~\qc{type}{from}:~\qc{string}{"*"};~\qc{type}{to}:~\qc{string}{"*"}}\\
\qtt{~~~~~~~~\qc{lightgray}{PropertyAnimation}~\{}\\
\qtt{~~~~~~~~~~~~\qc{lightgray}{target:~stop\_light}}\\
\qtt{~~~~~~~~~~~~\qc{lightgray}{properties:~"color";~duration:~1000}}\\
\qtt{~~~~~~~~\qc{lightgray}{\}}}\\
\qtt{~~~~~~~~\qc{lightgray}{PropertyAnimation~\{}}\\
\qtt{~~~~~~~~~~~~\qc{lightgray}{target:~go\_light}}\\
\qtt{~~~~~~~~~~~~\qc{lightgray}{properties:~"color";~duration:~1000}}\\
\qtt{~~~~~~~~\qc{lightgray}{\}}}\\
\qtt{~~~~\}~]}\\
\end{qml}

\begin{itemize}
\item Use \qic{string}{"*"} to represent any state
\item Now the same transition is used whenever the state changes
\item Both lights fade at the same time
\end{itemize}
                      
\demo{qml-states-transitions/ex-transitions/transitions-multi.qml}

% Possible questions about using "*" for only one state.

\end{slide}

%----------------------------------------------------------------------

\begin{slide}{1502}\frametitle{Reversible Transitions}

\begin{itemize}
\item Useful when two transitions operate on the same properties
\end{itemize}

\flushedImage{qml-states-transitions/images/state-conditions-with-text.pdf}
\begin{qml}
\qtt{~~~~\qc{type}{transitions}:~[}\\
\qtt{~~~~~~~~\qc{class}{Transition}~\{}\\
\qtt{~~~~~~~~~~~~\qc{type}{from}:~\qc{string}{"with~text"};~\qc{type}{to}:~\qc{string}{"without~text"}}\\
\qtt{~~~~~~~~~~~~\qc{type}{reversible}:~\qc{number}{true}}\\
\qtt{~~~~~~~~~~~~\qc{class}{PropertyAnimation}~\{}\\
\qtt{~~~~~~~~~~~~~~~~\qc{type}{target}:~clear\_button}\\
\qtt{~~~~~~~~~~~~~~~~\qc{type}{properties}:~\qc{string}{"opacity"};~\qc{type}{duration}:~\qc{number}{1000}}\\
\qtt{~~~~~~~~~~~~\}}\\
\qtt{~~~~~~~~\}}\\
\qtt{~~~~]}
\end{qml}

\begin{itemize}
\item Transition applies from \qic{string}{"with text"} to \qic{string}{"without text"}
  \begin{itemize}
  \item and back again from \qic{string}{"without text"} to \qic{string}{"with text"}
  \end{itemize}
\item No need to define two separate transitions
\end{itemize}

\demo{qml-states-transitions/ex-transitions/transitions-reversible.qml}

\end{slide}

%----------------------------------------------------------------------

\begin{slide}{2142}\frametitle{Parent Changes}

\begin{itemize}
\item Used to animate an element when its parent changes
\end{itemize}

\fontsize{8}{8}\fontseries{c}\selectfont
\begin{qml}
\fontsize{8}{8}\fontseries{c}\selectfont
 \qtt{\qc{type}{states}:~\qc{class}{State}~\{}\\
\qtt{~~~~\qc{type}{name}:~\qc{string}{"reparented"}}\\
\qtt{~~~~ParentChange~\{}\\
\qtt{~~~~~~~~\qc{type}{target}:~myRect}\\
\qtt{~~~~~~~~parent~:~yellowRect}\\
\qtt{~~~~~~~~x~:~\qc{number}{60}}\\
\qtt{~~~~~~~~y~:~\qc{number}{20}}\\
\qtt{~~~~\}}\\
\qtt{\}}\\
\vspace*{0.5em}
\qtt{\qc{type}{transitions}:~\qc{class}{Transition}~\{}\\
\qtt{~~~~ParentAnimation~\{}\\
\qtt{~~~~~~~~\qc{class}{NumberAnimation}~\{}\\
\qtt{~~~~~~~~~~~~properties~:~\qc{string}{"x,y"}}\\
\qtt{~~~~~~~~~~~~\qc{type}{duration}:~\qc{number}{1000}}\\
\qtt{~~~~~~~~\}}\\
\qtt{~~~~\}}\\
\qtt{\}}\\
\end{qml}

\begin{itemize}
\item ParentAnimation applies only when changing the parent with ParentChange in a state change
\end{itemize}

\demo{qml-animations/ex-animations/parent-animation.qml}

\end{slide}

%----------------------------------------------------------------------

\begin{slide}{4203}\frametitle{Anchor Changes}
\begin{itemize}
\item Used to animate an element when its anchors change
\end{itemize}

\fontsize{8}{8}\fontseries{c}\selectfont
\begin{qml}
\fontsize{8}{8}\fontseries{c}\selectfont
\qtt{\qc{type}{states}:~\qc{class}{State}~\{}\\
\qtt{~~~~\qc{type}{name}:~\qc{string}{"reanchored"}}\\
\qtt{~~~~AnchorChanges~\{}\\
\qtt{~~~~~~~~target~:~myRect}\\
\qtt{~~~~~~~~anchors.left~:~parent.left}\\
\qtt{~~~~~~~~anchors.right~:~parent.right}\\
\qtt{~~~~\}}\\
\qtt{\}}\\
\vspace*{0.5em}
\qtt{\qc{type}{transitions}:~\qc{class}{Transition}~\{}\\
\qtt{~~~~AnchorAnimation~\{}\\
\qtt{~~~~~~~~duration~:~\qc{number}{1000}}\\
\qtt{~~~~\}}\\
\qtt{\}}\\
\end{qml}

\begin{itemize}
\item AnchorAnimation applies only when changing the anchors with AnchorChanges in a state change
\end{itemize}

\demo{qml-animations/ex-animations/anchors-animation.qml}\\\smallskip
\demo{qml-animations/ex-animations/parent-anchors-animation.qml}

\end{slide}

%----------------------------------------------------------------------

\begin{slide}{1501}\frametitle{Using States and Transitions}

\begin{itemize}
\item Avoid defining complex statecharts
  \begin{itemize}
  \item not just one statechart to manage the entire UI
  \item usually defined individually for each component
  \item link together components with internal states
  \end{itemize}
\item Setting state with script code
  \begin{itemize}
  \item easy to do, but might be difficult to manage
  \end{itemize}
\item Setting state with state conditions
  \begin{itemize}
  \item more declarative style
  \item can be difficult to specify conditions
  \end{itemize}
\item Using animations in transitions
  \begin{itemize}
  \item do not specify \qic{type}{from} and \qic{type}{to} properties
  \item use \qic{class}{PropertyChanges} elements in state definitions
  \end{itemize}
\end{itemize}

% Scope - where are states defined? Depends on encapsulation
% (inside the Item they are defined for vs. inside an Item that manages its children).

\end{slide}

%----------------------------------------------------------------------

\begin{slide}{1500}\frametitle{Summary~\textendash~States}

\qic{class}{State} items manage properties of other items:

\begin{itemize}
\item Items define states using the \qic{type}{states} property
  \begin{itemize}
  \item must define a unique \qic{type}{name} for each state
  \end{itemize}
\item Useful to assign \qic{type}{id} properties to items
  \begin{itemize}
  \item use \qic{class}{PropertyChanges} to modify items
  \end{itemize}
\item The \qic{type}{state} property contains the current state
  \begin{itemize}
  \item set this using JavaScript code, or
  \item define a \qic{type}{when} condition for each state
  \end{itemize}
\end{itemize}

\end{slide}

%----------------------------------------------------------------------

\begin{slide}{1499}\frametitle{Summary~\textendash~Transitions}

\qic{class}{Transition} items describe how items change between states:

\begin{itemize}
\item Items define transitions using the \qic{type}{transitions} property
\item Transitions refer to the states they are between
  \begin{itemize}
  \item using the \qic{type}{from} and \qic{type}{to} properties
  \item using a wildcard value, \qic{string}{"*"}, to mean any state
  \end{itemize}
\item Transitions can be reversible
  \begin{itemize}
  \item used when the \qic{type}{from} and \qic{type}{to} properties are reversed
  \end{itemize}
\end{itemize}

\end{slide}             



%----------------------------------------------------------------------
\begin{slide}{1498}\frametitle{Exercise~\textendash~States~and~Transitions}

\begin{itemize}
\item How do you define a set of states for an item?
% Define the states property, using a list of State elements.
\item What defines the current state?
% The state property.
\item Do you need to define a name for all states?
% Yes, states need names to distinguish them.
\item Do state names need to be globally unique?
% No, just unique for the item they are defined for.
\item Remember the thumbnail explorer \pageHyperlink{animations-showcase}. Which states and transitions would you use for it?
% Good time to look at the code of this showcase to put it all together (animations, states & transitions, etc.)
\end{itemize}

\end{slide}

%----------------------------------------------------------------------
\begin{slide}{1497}\frametitle{Lab~\textendash~Light~Switch}

\vspace*{-1em}
\begin{center}
\imageDoubleWidth{qml-states-transitions/images/switch-off.png}
\hspace*{0.5em}
\imageDoubleWidth{qml-states-transitions/images/switch-on.png}
\end{center}

\vspace*{-0.5em}
\begin{itemize}
\item Using the partial solutions as hints, create a user interface similar
to the one shown above.
\item Adapt the reversible transition code from earlier and add it to the
example. 

\lab{qml-states-transitions/lab-switch}
\end{itemize}

\end{slide}

                                                                   
