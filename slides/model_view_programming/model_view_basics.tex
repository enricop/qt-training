%%%%%%%%%%%%%%%%%%%%%%%%%%%%%%%%%%%%%%%%%%%%%%%%%%%%%%%%%%%%%%%%%%%%%%%%%%
%
% Copyright (c) 2008-2011, Nokia Corporation and/or its subsidiary(-ies).
% All rights reserved.
%
% This work, unless otherwise expressly stated, is licensed under a
% Creative Commons Attribution-ShareAlike 2.5.
%
% The full license document is available from
% http://creativecommons.org/licenses/by-sa/2.5/legalcode .
%
%%%%%%%%%%%%%%%%%%%%%%%%%%%%%%%%%%%%%%%%%%%%%%%%%%%%%%%%%%%%%%%%%%%%%%%%%%

\subsection{Item Widgets}

%----------------------------------------------------------------------
\begin{slide}[fragile]{0865}\frametitle{Item Views}
Qt provides a set of widgets for lists, trees and tables:

\vspace*{1em}
\centeredImage{Model_View_Programming/images/standard-views}

\vspace*{1em}
These are available in two forms:

\begin{itemize}
\item Item-based widgets.
\item Views that display the contents of models.
\end{itemize}

We will look at the widgets first.
\end{slide}

%----------------------------------------------------------------------
\begin{slide}[fragile]{0867}\frametitle{Introducing the Widgets}
  \label{introduction_to_model_view_widgets}
\xConcept{Model/View!Convenience Widgets}

The item view widgets:

\begin{itemize}
\item Provide a simple to use API.
\item Are practical for small-to-medium data sets.
\item Act as containers for items.
\end{itemize}

\begin{itemize}
\item \textbf{List:} \iCls{QListWidget}
\item \textbf{Table:} \iCls{QTableWidget}
\item \textbf{Tree:} \iCls{QTreeWidget}
\end{itemize}

These are all derived from corresponding view classes.
\end{slide}

%----------------------------------------------------------------------
\begin{slide}{0868}\frametitle{QListWidget}
\xConcept{Model/View!QListWidget}
\xConcept{Model/View!QListView modes}

\begin{itemize}
 \item Displays a list of items.
   \begin{itemize}
   \item \iCls{QListWidgetItem} is item in \iCls{QListWidget}
   \item \iClsFn{QListWidget}{addItem} adds item.
   \end{itemize}
 \item  \texttt{setViewMode()}: \textit{Select the display mode using}
   \begin{itemize}
   \item \texttt{ListMode}: a single column of text plus optional icons
   \item \texttt{IconMode}: a grid of movable icons
   \end{itemize}
   \end{itemize}
\vfill
\hfill\image{Model_View_Programming/images/listwidget-listmode}\hfill\image{Model_View_Programming/images/listwidget-iconmode}\hfill\strut

\end{slide}

%----------------------------------------------------------------------
\begin{slide}[fragile]{46868}\frametitle{QListWidgetItem}

  \begin{itemize}
  \item \textbf{Example:} Adding items to a list widget:
  \end{itemize}
  \begin{lstlisting}
    QListWidget *widget = new QListWidget;
    // adding using item class
    QListWidgetItem *item = new QListWidgetItem("Item-0");
    widget->addItem(item);
    
    for (int i = 1; i < 10; ++i) {
      // convienence method
      widget->addItem(QString("Item %1").arg(i));
    }
    // retrieve item from widget
    widget->item(0)->text(); // returns "Item-0"
  \end{lstlisting}
  \begin{itemize}
  \item \textit{The widget takes ownership of the items}
 \end{itemize}
\demo{modelview/ex-QListWidget}
\end{slide}

%----------------------------------------------------------------------
\begin{slide}{0869}\frametitle{QTableWidget}
\xConcept{Model/View!QTableWidget}

Displays a table of items.

\begin{itemize}
\item \texttt{new QTableWidget(rows, columns, parent)}
  \begin{itemize}
  \item Usually constructed with predefined dimensions.
  \end{itemize}
\item class \iCls{QTableWidgetItem} represents table item
\item \texttt{setItem(row, column, item)} to set cell data.
\item Memory-efficient
  \begin{itemize}
  \item \textit{ Only cells with contents need memory.}
  \end{itemize}
\end{itemize}

\vspace*{0.5em}
\centeredImage{Model_View_Programming/images/tablewidget}

\end{slide}

%----------------------------------------------------------------------
\begin{slide}[fragile]{46870}\frametitle{QTableWidgetItem}

\flushedImage{Model_View_Programming/images/naughts-and-crosses}
\textbf{Example:} Setting items in a table widget:

\vspace*{0.5em}
\begin{lstlisting}
 QTableWidget *w = new QTableWidget(3, 3);
\end{lstlisting}

\vspace*{0.5em}
Creating items and setting them for a given row and column:

\begin{cpp}
QTableWidgetItem *item1 = new QTableWidgetItem("X");
QTableWidgetItem *item2 = new QTableWidgetItem("O");
QTableWidgetItem *item3 = new QTableWidgetItem("X");
w->setItem(1, 1, item1);
w->setItem(1, 2, item2);
w->setItem(2, 0, item3);
\end{cpp}

\vspace*{0.5em}
\textit{The widget takes ownership of the items.}

\vfill
\demo{modelview/ex-QTableWidget}
\end{slide}

%----------------------------------------------------------------------
\begin{slide}{0870}\frametitle{QTreeWidget}
\xConcept{Model/View!QTreeWidget}

Displays a tree of items with branches.

\begin{itemize}
\item Tree items are instances of \iCls{QTreeWidgetItem}.
\item One item per row, supports multiple columns.
\begin{itemize}
\item \hClsFn{QTreeWidget}{setColumnCount}
\end{itemize}
\item Branches can be expanded and collapsed.
\item Allows sorting by the data in one column.
\begin{itemize}
\item \hClsFn{QTreeWidget}{setSortingEnabled}
\end{itemize}
\item Columns can be rearranged.
\end{itemize}

\vspace*{0.5em}
\centeredImage{Model_View_Programming/images/qtreewidget}

\end{slide}

%----------------------------------------------------------------------
\begin{slide}[fragile]{46871}\frametitle{QTreeWidgetItem}
\xConcept{Model/View!QTreeWidget}

\textbf{Example:} Adding items to a tree widget.

\vspace*{0.5em}
Creating items as children of the tree widget:

\vspace*{0.5em}
\begin{lstlisting}
 QTreeWidget *w = new QTreeWidget;
 QTreeWidgetItem *top = new QTreeWidgetItem(w);
 top->setText(0, "Top A");
 top->setText(1, "Top B");
\end{lstlisting}

\begin{itemize}
\item The item is constructed with \iCls{QTreeWidget} as the parent.
\item Each item has two columns of text.
\end{itemize}

\vspace*{0.5em}
The widget takes ownership of the items.
\vfill
\demo{modelview/ex-QTreeWidget}
\end{slide}

%----------------------------------------------------------------------
\begin{slide}[fragile]{46872}\frametitle{QTreeWidgetItem}

\vspace*{0.5em}
Adding top-level items later with \hClsFn{QTreeWidget}{addTopLevelItem}:

\vspace*{0.5em}
\begin{cpp}
QTreeWidgetItem *top2 = new QTreeWidgetItem();
top2->setText(0, "Second Top A");
top2->setText(1, "Second Top B");
w->addTopLevelItem(top2);
\end{cpp}

\vspace*{1em}
\begin{itemize}
\item The item is constructed without a parent.
\item It is added explicitly to the tree widget.
\end{itemize}

\vfill
\demo{modelview/ex-QTreeWidget}
\end{slide}

%----------------------------------------------------------------------
\begin{slide}[fragile]{46873}\frametitle{QTreeWidgetItem}
\xConcept{Model/View!QTreeWidget}

\textbf{Example:} Creating items as children of a tree widget item:

\vspace*{0.5em}
\begin{cpp}
QTreeWidgetItem *player1 = new QTreeWidgetItem(teamItem);
player1->setText(0, "James");
player1->setText(1, "Goalkeeper");
player1->setText(2, "0");
\end{cpp}

\vspace*{0.5em}
Adding child items later with \hClsFn{QTreeWidgetItem}{addChild}:

\vspace*{0.5em}
\begin{cpp}
QTreeWidgetItem *player7 = new QTreeWidgetItem();
layer7->setText(0, "Bernie");
player7->setText(1, "Striker");
player7->setText(2, "4");
teamItem->addChild(player7);
\end{cpp}

\vspace*{1em}

\demo{modelview/ex-tree-child-items}

\end{slide}

%----------------------------------------------------------------------
\begin{slide}{1165}\frametitle{Items}
\xConcept{Model/View!Item classes}

\begin{itemize}
\item Widget Items have similar APIs.
\item Common setter functions:
  \begin{itemize}
  \item Text: \hClsFn{QTreeWidgetItem}{setText}
  \item Icon: \hClsFn{QTreeWidgetItem}{setIcon}
  \item Tool tip: \hClsFn{QTreeWidgetItem}{setToolTip}
  \end{itemize}
\item Hiding, configuring of Items:
  \begin{itemize}
  \item \hClsFn{QTreeWidgetItem}{setHidden}
  \item \hClsFn{QTreeWidgetItem}{setFlags}
  \end{itemize}
  
\item \iCls{QListWidgetItem}, \iCls{QTreeWidgetItem}
  \begin{itemize}
  \item Functions take column number as first argument
  \end{itemize}
\end{itemize}

\end{slide}

%----------------------------------------------------------------------
\begin{slide}{46994}\frametitle{Item Flags}
\xConcept{Model/View!Item classes}

\begin{itemize}
\item Flags describe how user can interact with items.
\item Items can be a combination of these flags:
  \begin{itemize}
  \item Enabled: \iNsEnum{Qt}{ItemIsEnabled} 
  \item Editable: \iNsEnum{Qt}{ItemIsEditable} 
  \item Selectable: \iNsEnum{Qt}{ItemIsSelectable} 
  \item Checkable: \iNsEnum{Qt}{ItemIsUserCheckable} 
    \begin{itemize}
    \item \iNsEnum{Qt}{ItemIsTristate} 
    \end{itemize}
  \item Dragged and dropped:
    \begin{itemize}
    \item \iNsEnum{Qt}{ItemIsDragEnabled} 
    \item \iNsEnum{Qt}{ItemIsDropEnabled} 
    \end{itemize}
  \end{itemize}
\item Items can also be completely disabled:
  \begin{itemize}
  \item \iNsEnum{Qt}{NoItemFlags} 
  \end{itemize}
\end{itemize}

\end{slide}

%----------------------------------------------------------------------
\begin{slide}{0873}\frametitle{Headers}

\iCls{QTreeWidget} and \iCls{QTableWidget} have headers.

\vspace*{1em}
\iCls{QTreeWidget} has a horizontal header:
\begin{itemize}
\item Accessed using \hClsFn{QTreeWidget}{headerItem}.
\item Only a single column is visible to begin with.
\item Change the number of columns with \hClsFn{QTreeWidget}{setColumnCount} or
\iClsFn{QTreeWidget}{setHeaderLabels}.
\end{itemize}

\vspace*{0.5em}
\iCls{QTableWidget} has horizontal and vertical headers:
\begin{itemize}
\item Accessed using \hClsFn{QTreeWidget}{horizontalHeaderItem}
and \hClsFn{QTreeWidget}{verticalHeaderItem}.
\item The number of headers match the dimensions of the table.
\end{itemize}
\end{slide}

% Notes:
% \item Set a custom header with \iClsFn{QTreeWidget}{setHeaderItem} for the whole
% header.
% \item Set custom headers with \iClsFn{QTableWidget}{setHorizontalHeaderItem}
% and \iClsFn{QTableWidget}{setHorizontalHeaderItem} for each header section.

%----------------------------------------------------------------------
\begin{slide}{0874}\frametitle{Selection and Keyboard Focus}
\xConcept{Model/View!Focus}
\xConcept{Model/View!Selection}
\begin{itemize}
\item Keyboard focus is shown using a line around the item.
\item Selection is shown by inverting the item colors.
\item Only one item can have keyboard focus while several items can be
  selected (depending on the selection mode)
\item Five different selection modes exist: \hClsEnum{QAbstractItemView}{NoSelection},
  \hClsEnum{QAbstractItemView}{Single}, \hClsEnum{QAbstractItemView}{Multi},
  \hClsEnum{QAbstractItemView}{Extended}, and \hClsEnum{QAbstractItemView}{ContiguousSelection}.
\end{itemize}
\end{slide}

%----------------------------------------------------------------------
\begin{slide}{0875}\frametitle{Selection and Keyboard Focus}
\begin{itemize}
\item \hClsEnum{QAbstractItemView}{Multi},
  \hClsEnum{QAbstractItemView}{Extended}, and \hClsEnum{QAbstractItemView}{ContiguousSelection} allow the user to select 
  several items. In \hClsEnum[\textbf]{QAbstractItemView}{Extended} mode, selecting one item deselects
  other selected items unless Ctrl or Shift is held. This is not
  the case for
  \hClsEnum[\textbf]{QAbstractItemView}{Multi}-mode. \hClsEnum[\textbf]{QAbstractItemView}{ContiguousSelection} is similar to extended, you can however, only select a contiguous range.
\item Keyboard focus is retrieved and set using
  \iClsFn{QTreeWidget}{currentItem} and \iClsFn{QTreeWidget}{setCurrentItem}
  respectively.
\item Selection is retrieved and set using
  \iClsFn{QTreeWidget}{selectedItems} and
  \iClsFnPar{QTreeWidgetItem}{setSelected}{bool} respectively.
\end{itemize}
\end{slide}


%----------------------------------------------------------------------
% \begin{slide}{969}\frametitle{Models and Views}

% \begin{itemize}
% \item \iCls{QListView}, \iCls{QTreeView}, \iCls{QTableView}
% \end{itemize}

% \end{slide}

%----------------------------------------------------------------------
\begin{slide}{0876}\frametitle{Sorting}
\xConcept{Model/View!Sorting}
\begin{itemize}
\item The user can sort the items by clicking in the header.
\item This can be enabled by calling
  \iClsFnPar{QTreeWidget}{setSortingEnabled}{true}.
\item The tree view shows an arrow in the header to indicate sorting
  direction. This can be disabled by calling
  \iClsFnPar{QHeaderView}{setShowSortIndicator}{false}.
\item You can sort the items programmatically using
  \iClsFnPar{QListView}{sortItems}{int column, Qt::SortOrder}.
\item You can change the sorting algorithm by overriding
  \iClsFn{QTreeWidgetItem}{operator<}.
\end{itemize}
\end{slide}

%----------------------------------------------------------------------
\begin{slide}{0877}\frametitle{Events}
\xConcept{Model/View!Events}
\begin{itemize}
\item \iCls{QTreeWidgetItem} does \emph{not} inherit from
  \iCls{QWidget} or \iCls{QObject}, and therefore does not get any
  events nor emit any signals.
\item Events can be handled by \iCls{QTreeWidget}, and can be mapped to the
  item in question using \iClsFnPar{QTreeWidget}{itemAt}{QPoint}
\item \iCls{QTreeWidget} does part of this job for you by emitting a large
  number of signals notifying you about changes to the items:
  \hClsFn{QTreeWidget}{itemClicked}, \hClsFn{QTreeWidget}{itemEntered}, \hClsFn{QTreeWidget}{itemExpanded},
  \ldots
\item When constructing a \iCls{QTreeWidgetItem}, you can specify a type,
  which is basically a number. This type can later be retrieved using the
  method \hClsFn{QTreeWidgetItem}{type}. This can be used for RTTI.
\end{itemize}
\end{slide}

\subsection{Model/View  Concept}
%----------------------------------------------------------------------
\begin{slide}{1017}\frametitle{Concept}\label{model_view_concepts}
\xConcept{Model/View!Concept}
\oooFlushedImage{model-view-overview}
\begin{itemize}
\item \textbf{Model} Provides an interface\\
  to the real data.
\item \textbf{View} Displays the data.
\item \textbf{Delegate} Renders the items of data,\\
  and supports editing of data.
\end{itemize}
\hspace{30mm}
\end{slide}

%----------------------------------------------------------------------
\begin{slide}[fragile]{0879}\frametitle{Concept}
\begin{itemize}
\item Views can almost always be used as-is
\item Exception to add a new view
\item Often you will subclass a model class
\item Occasionally create own delegate
  \begin{itemize}
  \item  Provides control over data display and edit
  \end{itemize}
\end{itemize}
\demo{modelview/ex-simple} \demo{modelview/ex-QColumnView}
\end{slide}

%----------------------------------------------------------------------
\begin{slide}[fragile]{0880}\frametitle{Model Classes}
\xConcept{Model/View!Model classes}
\vfill
\oooCenteredImage{model-view-qabstractitemmodel}
\vfill
\end{slide}

%----------------------------------------------------------------------
\begin{slide}{0881}\frametitle{Model Indexes}\label{modelIndexes}
\xConcept{Model/View!Model indexes}
\begin{itemize}
\item Each piece of information that can be obtained via a model is
  represented by a \emph{model index}.
\item Two kinds of model index exists:
  \begin{itemize}
  \item Temporary indexes that should not be kept around
    (\iCls{QModelIndex})
  \item Persistent model indexes, which are updated by the model, when it
    internally reorganizes (\iCls{QPersistentModelIndex})\\
    \underline{Warning:} they might be expensive for Qt to maintain.
  \end{itemize}
  \item An index consists of three parameters:\\
    \emph{row}, \emph{column}, and \emph{parent index}.
\end{itemize}
\end{slide}

%----------------------------------------------------------------------
\begin{slide}{0882}\frametitle{Model Indexes}
\begin{block}{Imagine}
A world \textbf{without} model indexes
\end{block}

\begin{itemize}
\item \textbf{View:} Dear model, how many items are under the first item?
  \begin{itemize}
  \item \textbf{Model:} 12
  \end{itemize}
\item \textbf{View:} Great, thanks pal. Say, how many children does
  the first row of your first row have?
  \begin{itemize}
  \item \textbf{Model:} 2
  \end{itemize}
\item \textbf{View:} Awesome. Erhh how many children does the first row of the
  first row of the first row have?
  \begin{itemize}
  \item \textit{(much later)}
  \end{itemize}
  \item \textbf{View} \textit{(Very tired now)}\textbf{:} One last thing: What is
the text in the third column of the 15th row inside the second row inside
the 8th row inside the \ldots?
\end{itemize}
\end{slide}

%----------------------------------------------------------------------
\begin{slide}[fragile]{0883}
\frametitle{Table Models}
\xConcept{Model/View!Table models}
\flushedImage{Model_View_Programming/images/modelview-tablemodel}
\begin{itemize}
\item In the simplest form a model can\\
  be represented as a table.
\item In that case, the parent index\\
  is not used.
\item Example:
\begin{alltt}
\iCls{QModelIndex} index = 
model->index(2,1, QModelIndex())
\end{alltt}
\end{itemize}
\end{slide}

%----------------------------------------------------------------------
\begin{slide}[fragile]{0884}
\frametitle{Tree Models}
\xConcept{Model/View!Tree models}
\flushedImage{Model_View_Programming/images/modelview-treemodel}
\begin{alltt}
\iCls{QModelIndex} indexA = 
  model->index(0, 0, QModelIndex());

QModelIndex indexB = 
  model->index(1, 0, indexA);

QModelIndex indexC = 
  model->index(2, 1, QModelIndex());
\end{alltt}
\end{slide}

%----------------------------------------------------------------------
\begin{slide}{0885}\frametitle{Item Roles}\label{itemRoles}
\xConcept{Model/View!Item roles}
\flushedImage{Model_View_Programming/images/modelview-roles}
\begin{itemize}
\item Each item in the model\\
  can contain different data\\
  (aka. performing\\
  different kind of\\
  \emph{roles}).
\item You get the data for\\
  a given role using\\
  \iClsFn{QAbstractItemModel}{data},\\
  providing an index,\\
  and a \texttt{DisplayRole} enum.
\end{itemize}
\end{slide}

%----------------------------------------------------------------------
\begin{slide}{0886}\frametitle{Item Roles}
\begin{itemize}
\item The existing views and delegates understand a lot of roles:
  \begin{itemize}
  \item \textbf{General purpose:} DisplayRole, DecorationRole,
    EditRole, ToolTipRole, StatusTipRole, WhatsThisRole
  \item \textbf{Appearance and meta data:} FontRole, TextAlignmentRole,
    BackgroundColorMode, TextColorRole, CheckStateRole, SizeHintRole
  \item \textbf{Accessibility:} AccessibleTextRole, AccessibleDescriptionRole
  \end{itemize}
\item You can also provide your own roles, though Qt's Views and Delegates
  might not make use of them.
\end{itemize}
\end{slide}

\subsection{Custom Models}
%----------------------------------------------------------------------
\begin{slide}{0887}\frametitle{Implementing a Model}\label{model_view_creating_your_own_model}
\xConcept{Model/View!Creating your own models}
\begin{itemize}
\item When implementing a model, we have a variety of classes to
  choose from, each with their advantages:
  \begin{itemize}
  \item \iCls[\textbf]{QAbstractListModel} One dimensional list.
  \item \iCls[\textbf]{QAbstractTableModel} Two-dimensional tables.
  \item \iCls[\textbf]{QAbstractItemModel} Generic model class.
  \item \iCls[\textbf]{QStringListModel} Simple one-dimensional model that works on a string
  \item \iCls[\textbf]{QStandardItemModel} A simple model that stores the data.
    string list.
  \end{itemize}
\item Notice that you need to subclass each of the above classes except
  \iCls{QStandardItemModel} and \iCls{QStringListModel}.
\end{itemize}
\end{slide}

%----------------------------------------------------------------------
\begin{slide}{0888}\frametitle{Implemeting a ListModel}
\begin{itemize}
\item Using \iCls{QAbstractListModel}:
  \begin{itemize}
  \item Inherit from the class
  \item Implement \texttt{int} \hClsFn{QAbstractItemModel}{rowCount} and
     \iCls{QVariant} \hClsFn{QAbstractItemModel}{data}
  \item Optionally reimplement \iCls{QVariant} \hClsFn{QAbstractItemModel}{headerData} to make the
    model work better with \iCls{QTableView} and \iCls{QTreeView}.
  \item If the model is editable, reimplement  \iClsEnum{Qt}{ItemFlags} \hClsFn{QAbstractItemModel}{flags},
    and \hClsFn{QAbstractItemModel}{setData}.
  \item Optionally, to allow row insertion and removal, implement
    \hClsFn{QAbstractItemModel}{insertRows} and \hClsFn{QAbstractItemModel}{removeRows}
  \end{itemize}
\end{itemize}
\demo{modelview/ex-stringlistmodel-4}
\end{slide}

%----------------------------------------------------------------------
\begin{slide}{0889}\frametitle{Implemeting a TableModel}
\begin{itemize}
\item Using \iCls{QAbstractTableModel}:
  \begin{itemize}
  \item Everything from \iCls{QAbstractListModel} applies here, too.
  \item In addition, you must implement \texttt{int} \hClsFn{QAbstractItemModel}{columnCount}
  \item Optionally also implement \hClsFn{QAbstractItemModel}{insertColumns} and
    \hClsFn{QAbstractItemModel}{removeColumns}.
  \end{itemize}
\end{itemize}
\end{slide}

%----------------------------------------------------------------------
\begin{slide}{0890}\frametitle{Handle Changes}
\begin{itemize}
\item Inform the view of changes in the model:
  \begin{itemize}
  \item \iCls{QAbstractItemModel} provides signals to inform the views when its contents change :
    \hClsFn{QAbstractItemModel}{dataChanged}, \hClsFn{QAbstractItemModel}{headerDataChanged} 
  \item There are also signals to inform the views about structural changes :
    \hClsFn{QAbstractItemModel}{rowsInserted}, \hClsFn{QAbstractItemModel}{rowsRemoved},
    \hClsFn{QAbstractItemModel}{columnsInserted}, \hClsFn{QAbstractItemModel}{columnsRemoved}
    \hClsFn{QAbstractItemModel}{layoutChanged}, \hClsFn{QAbstractItemModel}{modelReset}
    (and their "AboutToBe" counterparts)
  \item Don't forget to emit the right signal when you change data in the model,
    to make sure the views will be properly updated. For example, you should emit
    \hClsFn{QAbstractItemModel}{dataChanged} from \hClsFn{QAbstractItemModel}{setData}.
  \end{itemize}
\end{itemize}
\end{slide}

%----------------------------------------------------------------------
\begin{slide}{0891}\frametitle{Lab: \iProject{Object Browser - step 1}}
\flushedImage{Model_View_Programming/images/object-browser-step1}
\begin{itemize}
\item In the following exercises we\\
  will implement an object browser\\
  that allows us to see the \\
  parent/child hierarchy of a dialog.
\item In the first step you must\\
  implement a table model. 
\item Your model will take as argument a widget which should be displayed.
\item Slow teams can get a kickstart by using the implementation in the
  file \emph{extra-help}.
\end{itemize}
\lab{modelview/lab-object-browser-step1}
\end{slide}

%----------------------------------------------------------------------
\begin{slide}{0892}\frametitle{Implementing Your Own Model}\label{model_indexes_start}
\begin{itemize}
\item Using \iCls{QAbstractItemModel}:
  \begin{itemize}
  \item Everything from \iCls{QAbstractTableModel} applies here, too.
  \item Implement \iCls{QModelIndex} \hClsFnPar{QAbstractItemModel}{index}{row, column, parent}, where
    \emph{parent} is itself a model index.
  \item Implement \iCls{QModelIndex} \hClsFnPar{QAbstractItemModel}{parent}{QModelIndex}.
  \end{itemize}
\item Indexes can be created using the protected method 
  \iClsFn{QAbstractItemModel}{createIndex}, which is overloaded to
  take either an integer index, or a void pointer (usually internal data
  pointer to your data structure).
\end{itemize}
\end{slide}

%----------------------------------------------------------------------
\begin{slide}{0893}\frametitle{Implementing Your Own Model
   }\label{model_indexes_end}
\xConcept{Model/View!Model indexes}
\begin{itemize}
\item With the void pointer, you can associate a model index with your internal
  data structures.
\item Example: making a model around a linked list: you need the pointer to each
  item inside the index representing it.
\item You can ask an index for its id or pointer using
  \iClsFn{QModelIndex}{internalId} and \iClsFn{QModelIndex}{internalPointer}
\end{itemize}
\qtdemo{examples/itemviews/simpletreemodel}
\end{slide}

%----------------------------------------------------------------------
\begin{slide}[fragile]{0894}
  \frametitle{Summary of Methods Needed}
  \begin{itemize}
  \item Methods to implement \textit{(simplified)}
  \end{itemize}
  \begin{cpp}
class CustomModel : public QAbstractItemModel
{
public:
  int columnCount( QModelIndex parent );
  int rowCount( QModelIndex parent );

  QVariant data( QModelIndex index, int role );

  QModelIndex index( int row, int column, QModelIndex parent );

  QModelIndex parent( QModelIndex index );
};
\end{cpp}
\end{slide}
  
%----------------------------------------------------------------------
\begin{slide}{0895}\frametitle{Lab: \iProject{Object Browser - step 2}}
\flushedImage{Model_View_Programming/images/object-browser-step2}
\begin{itemize}
\item Implement an object browser \\ that uses
  a tree view.
\item First step is to verify \\ this code
  against your solution.
\item To help you a lot of test cases\\
 are available in main.cpp.
\end{itemize}

\textit{Hints on next slide...}

\lab{modelview/lab-object-browser-step2}

\end{slide}

%----------------------------------------------------------------------
\begin{slide}{0896}\frametitle{Lab: Object Browser - step 2}
\label{model_view_exercise_step2}
\begin{itemize}
\item Follow these steps carefully.
  \begin{enumerate}
  \item On a piece of paper formulate what goes into the internal
    pointer. You will likely not succeed this exercise if you skip this
    step.
  \item Implement the method \texttt{qobjectFromModelIndex()}.
    \begin{itemize}
    \item This maps from a model index to the associated widget.
    \end{itemize}
  \item Update \texttt{rowCount()} and
    \texttt{data()}
    \begin{itemize}
    \item using the \texttt{qobjectFromModelIndex()} method.
    \end{itemize}
  \item Implement the \texttt{index()} method.
  \item Compile and hope for the best
    \begin{itemize}
    \item if it doesn't work, try to check the
      test cases in main.cpp
    \end{itemize}
  \end{enumerate}
\item Optional:
  \begin{itemize}
  \item Throw out the implementation for \texttt{parent()}
    and implement it yourself.
  \end{itemize}
\end{itemize}
\end{slide}

%----------------------------------------------------------------------
\begin{slide}[fragile]{0897}\frametitle{QStandardItemModel and QStringListModel}
\begin{itemize}
\item There are two model classes that do not need subclassing:
  \iCls{QStandardItemModel} and \iCls{QStringListModel}.
\item Generally you do not gain much from model/view programming if you
  give the model all your data upfront, so please consider these two
  classes as stubs which help you migrate to model/view.
\end{itemize}
\end{slide}

%----------------------------------------------------------------------
\begin{slide}{0898}\frametitle{Proxy Models - Motivation}\label{proxymodels}
\xConcept{Model/View!Proxy models}
  \begin{itemize}
  \item Imagine you had two different views on the same data. One view
    should show the \emph{highlights} (i.e.\ was somehow filtered). The other
    view should show all the data.
  \item How would you implement that using model/view?
    \begin{itemize}
    \item Clutter the view with information about filtering?
    \item Modify the model? If so, how would the model know when to return what?
    \end{itemize}
  \item Solution: Use a stack of models. Intermediate models are called Proxy
    Models. 
  \end{itemize}
\end{slide}

%----------------------------------------------------------------------
\begin{slide}[fragile]{0899}\frametitle{Proxy Models - Introduction}
  \begin{itemize}
  \item Proxy Models take a source model and transform it, usually by manipulating
    the model indexes.
  \item Common use cases:
    \begin{itemize}
    \item Customizable sorting by column.
    \item Filtering of rows matching a regular expression or wildcard.
    \item Reordering of cells by manipulating the indexes.
    \item Extracting part of a model (Imagine that the model shows different
      data in different branches of a tree, and you only want to operate on
      one of the branches).
    \end{itemize}
  \end{itemize}
\demo{modelview/ex-DecorationProxy} \demo{modelview/ex-ModelFlip}
\end{slide}

%----------------------------------------------------------------------
\begin{slide}[fragile]{0900}\frametitle{Proxy Model Classes}
  \begin{itemize}
  \item \iCls[\textbf]{QAbstractProxyModel}: Abstract Proxy Model class that can be used to
    implement any index, row or column based model transformation. Has to be
    reimplemented, as it contains pure virtual methods.
  \item \iCls[\textbf]{QSortFilterProxyModel}: Generic implementation of a Proxy Model that
    allows sorting and filtering of the model data. 
  \item Proxy models may never give out indexes of their source models. In
    other words \hClsFn{QAbstractItemModel}{index} and \hClsFn{QAbstractItemModel}{parent} must always call \hClsFn{QAbstractItemModel}{createIndex}.
  \end{itemize}
\end{slide}

%----------------------------------------------------------------------
\begin{slide}{0901}\frametitle{Proxy Models}
\begin{itemize}
\item Proxy models sometimes need to know everything about the underlying
  model in order to be able to implement its API. Just think about a filtering model
  that needs to implement \hClsFn{QAbstractItemModel}{rowCount}.
\item It is therefore very wise to cache information---the views do ask a
  lot of questions!
\item Be careful to update your cache when the underlying model
  changes. Signals will tell you when that is, including
  \hClsSig{QAbstractItemModel}{dataChanged},
  \hClsSig{QAbstractItemModel}{endInsertRows},
  \hClsSig{QAbstractItemModel}{endRemoveColumns}, etc
\end{itemize}
\end{slide}


%----------------------------------------------------------------------
\begin{slide}[fragile]{0902}\frametitle{QSortFilterProxyModel}
  \begin{itemize}
  \item Can be used to display filtered, sorted model data.
  \item Model rows can be filtered using regular expressions or wildcards. 
  \item Model rows can be sorted by different colummns. 
  \item Sorting order can be customized by reimplementing the \texttt{bool lessThan()}
    function. 
  \item needs to be reimplemented to overload \hClsFn{QSortFilterProxyModel}{lessThan},
    \hClsFn{QSortFilterProxyModel}{filterAcceptsColumn}, \hClsFn{QSortFilterProxyModel}{filterAcceptsRow}
  \end{itemize}
\demo{modelview/ex-QSortFilterProxyModel}
\end{slide}


\subsection{Item Views}
%----------------------------------------------------------------------
\begin{slide}{0903}\frametitle{Views}\label{model_view_views}
\xConcept{Model/View!Views}
\vfill
\oooCenteredImage{model-view-qabstractitemview}
\vfill
\end{slide}

%-------------------------------------------------------------------
\begin{slide}{0904}\frametitle{Views}
\begin{itemize}
\item Using \iClsFn{QAbstractItemView}{setEditTriggers}, you specify when
  editing starts, the possible options are:
  \begin{itemize}
    \item \textbf{NoEditTriggers} No editing possible.
    \item \textbf{CurrentChanged} Start whenever current item changes.
    \item \textbf{DoubleClicked} Start when an item is double clicked.
    \item \textbf{SelectedClicked} Start when clicking on an already selected item.
    \item \textbf{EditKeyPressed} Start when an edit key has been pressed over an item.
    \item \textbf{AnyKeyPressed} Start when any key is pressed over an item.
    \item \textbf{AllEditTriggers} Start for all above actions.
  \end{itemize}
\end{itemize}
\end{slide}

%----------------------------------------------------------------------
\begin{slide}{0905}\frametitle{QDataWidgetMapper}
\begin{itemize}
\item Sometimes the three predefined views are not exactly the kind of view
  you need. In those cases you have two possibilities:
  \begin{itemize}
  \item Subclass \iCls{QAbstractItemView}
  \item Implement a view with a number of regular \iCls{QWidgets} and \iCls{QDataWidgetMapper}
  \end{itemize}
\end{itemize}
\demo{modelview/ex-QDataWidgetMapper}\\[1cm]
\hfill\image{Model_View_Programming/images/qdatawidgetmapper-2}\hfill\image{images/qdatawidgetmapper-1}\hfill\strut
\end{slide}


%----------------------------------------------------------------------
\begin{slide}[fragile]{0906}\frametitle{Delegates}\label{model_view_delegates}
\xConcept{Model/View!Delegates}
\begin{itemize}
\item Rendering and editing of items in the views are handled by subclasses
  of \iCls{QAbstractItemDelegate}.
\item As we have seen so far, a suitable delegate is already in place,
  which renders the items, and allows editing using an appropriate widget
  for the data type in question (\iCls{QItemEditorFactory} decides what
  is \emph{appropriate}).
\item Many editing properties are specified directly on the items in the
  model using \iClsFn{QAbstractItemModel}{setData} with appropriate
  roles%
, as specified on \pageHyperlink{itemRoles}.%
\end{itemize}
\end{slide}

%----------------------------------------------------------------------
\begin{slide}{0907}\frametitle{Delegates}
\begin{itemize}
\item New input methods can be implemented by subclassing
  \iCls{QAbstractItemDelegate} or \iCls{QItemDelegate}
\item Subclassing from \iCls{QAbstractItemDelegate} requires that you
  also implement rendering methods.
\item To create a custom editor, the methods to reimplement are \hClsFn{QAbstractItemDelegate}{createEditor},
  \hClsFn{QAbstractItemDelegate}{setEditorData}, \hClsFn{QAbstractItemDelegate}{setModelData}, and optionally
  \hClsFn{QAbstractItemDelegate}{updateEditorGeometry}.
\end{itemize}
\end{slide}

%----------------------------------------------------------------------
\begin{slide}{0908}\frametitle{Delegates}
\begin{itemize}
\item To improve usability it is possible to let the delegate emit the
  signal \hClsFn{QAbstractItemDelegate}{closeEditor} with a hint on where focus should go.
\item This signal should be emitted when the editor is done editing, an
  event filter might be useful to figure out when that is.
\end{itemize}
\demo{modelview/ex-delegate}\xExample{Model/View!Delegate}
\end{slide}

%----------------------------------------------------------------------
\begin{slide}{0909}\frametitle{Selection}\label{model_view_selection}
\xConcept{Model/View!Selection}
\begin{itemize}
\item Selection is handled by an instance of the class
  \iCls{QItemSelectionModel}.
\item The selection model can be retrieved using
  \iClsFn{QAbstractItemView}{selectionModel}, and set using
  \iClsFn{QAbstractItemView}{setSelectionModel}.
\item A selection model can be shared between several views.
\end{itemize}
\end{slide}

%----------------------------------------------------------------------
\begin{slide}{0910}\frametitle{Selection}
\begin{itemize}
\item The current selection can be retrieved using\\
  \iClsFn[QModelIndexList]{QItemSelectionModel}{selectedIndexes}.
\item You build selections using the class \iCls{QItemSelection}, and
  apply them using \iClsFn{QItemSelectionModel}{select}.
\item The select method takes as a second parameter
  \texttt{SelectionFlags}, which describes how the selection is to be
  applied.
\end{itemize}
\end{slide}

%----------------------------------------------------------------------
\begin{slide}{0911}\frametitle{Selection}
\begin{itemize}
\item \iClsEnum{QItemSelectionModel}{SelectionFlag}:
\begin{itemize}
\item \textbf{Select} All specified indexes will be selected.
\item \textbf{Deselect} All specified indexes will be deselected.
\item \textbf{Toggle} All specified indexes will be selected or deselected, depending on their current state.\\

\item \textbf{Rows} All indexes will be expanded to span rows.
\item \textbf{Columns} All indexes will be expanded to span columns.\\
\end{itemize}
\end{itemize}
\end{slide}

%----------------------------------------------------------------------
\begin{slide}[fragile]{0912}
\frametitle{Selection}
\begin{itemize}
\item \iClsEnum{QItemSelectionModel}{SelectionFlag}:
\begin{itemize}
\item \textbf{Clear} The complete selection will be cleared.
\item \textbf{Current} The current selection will be updated.\\

\item \textbf{SelectCurrent} Select \verb!|! Current 
\item \textbf{ToggleCurrent} Toggle \verb!|! Current
\item \textbf{ClearAndSelect} Clear \verb!|! Select
  \end{itemize}
\end{itemize}
\end{slide}

%----------------------------------------------------------------------
\begin{slide}{0913}\frametitle{Selection}
\begin{itemize}
\item The selection model emits the following signals:
  \begin{itemize}
  \item currentChanged ( const QModelIndex \& current,\\
    \strut\hspace{28.5mm}const QModelIndex \& previous )
  \item selectionChanged ( const QItemSelection \& selected,\\
    \strut\hspace{30.5mm}const QItemSelection \& deselected )
  \item currentColumnChanged ( const QModelIndex \& current,\\
    \strut\hspace{40.5mm}const QModelIndex \& previous )
  \item currentRowChanged ( const QModelIndex \& current,\\
    \strut\hspace{35mm}const QModelIndex \& previous )
  \end{itemize}
\end{itemize}
\demo{modelview/ex-selection}\xExample{Model/View!Selection}
\end{slide}

%----------------------------------------------------------------------
\begin{slide}{0914}\frametitle{Drag and Drop}\label{model_view_dragndrop}
\xConcept{Model/View!Drag and drop}
\begin{itemize}
\item The model/view classes support drag and drop to a certain extent, so
  you do not have to implement it from scratch.
\item To enable dragging:
  \begin{itemize}
  \item Call \iClsFnPar{QAbstractItemView}{setDragEnabled}{true} on each of
    the views.
  \item Reimplement \iClsFn{QAbstactItemModel}{mimeTypes} to return a
    list of mime types you support
  \item Reimplement \iClsFn{QAbstractItemModel}{mimeData} to encode data
    in your supported formats.
  \end{itemize}
\item You only need to implement \texttt{mimeTypes()} and
  \texttt{mimeData()} if you want to be able to drop outside of
  \iCls{QAbstractItemViews}.
\end{itemize}
\end{slide}

%----------------------------------------------------------------------
\begin{slide}{0915}\frametitle{Drag and Drop}
\begin{itemize}
\item To enable dropping of data:
  \begin{itemize}
  \item Implement \iClsFn{QAbstractItemModel}{dropMimeData} to handle
    drops of the types specified with
    \iClsFn{QAbstactItemModel}{mimeTypes}.
  \item Optionally, implement \hClsFn{QAbstractItemModel}{supportedDropActions} to specify
    which drop operations you support. (Default is copy)
  \end{itemize}
\item For drag and drop only inside the views, you do not need to implement
  \texttt{dropMimeData()}.
\end{itemize}
\end{slide}
