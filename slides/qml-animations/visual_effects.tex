%%%%%%%%%%%%%%%%%%%%%%%%%%%%%%%%%%%%%%%%%%%%%%%%%%%%%%%%%%%%%%%%%%%%%%%%%%
%
% Copyright (c) 2008-2011, Nokia Corporation and/or its subsidiary(-ies).
% All rights reserved.
%
% This work, unless otherwise expressly stated, is licensed under a
% Creative Commons Attribution-ShareAlike 2.5.
%
% The full license document is available from
% http://creativecommons.org/licenses/by-sa/2.5/legalcode .
%
%%%%%%%%%%%%%%%%%%%%%%%%%%%%%%%%%%%%%%%%%%%%%%%%%%%%%%%%%%%%%%%%%%%%%%%%%%

\subsection{Visual Effects}
\begin{slide}{1273}\frametitle{Visual Effects}

Effects are applied to items:

\begin{itemize}
\item Applied using the \qic{type}{effect} property
\item Four kinds of visual effect are provided:
  \begin{itemize}
  \item \qic{class}{Blur} applies a Gaussian blur
  \item \qic{class}{Colorize} adjusts the item's colors towards a single color
  \item \qic{class}{DropShadow} adds a shadow behind the item
  \item \qic{class}{Opacity} makes items solid or see-through
  \end{itemize}
\item Effects can be combined
\end{itemize}

\end{slide}

%----------------------------------------------------------------------

\begin{slide}{1272}\frametitle{Visual Effects~\textendash~Blur}

\flushedImage{qml-animations/images/blur-effect.png}
% declarative-uis/effects/blur-effect.qml
\inputqml{qml-animations/colorized/blur-effect}

\demo{qml-animations/ex-effects/blur-effect.qml}
\end{slide}

%----------------------------------------------------------------------

\begin{slide}{1271}\frametitle{Visual Effects~\textendash~Blur}

The standard \qic{class}{Blur} effect

\begin{itemize}
\item Samples pixels within a circular area
\item The \qic{type}{blurRadius} controls the radius of the sample area
  \begin{itemize}
  \item the amount of blurring
  \end{itemize}
\end{itemize}

\vspace*{0.5em}
\centeredImage{qml-animations/images/blur-values.png}

\vspace*{1em}
Choose speed or accuracy with the \qic{type}{blurHint} property:

\begin{itemize}
\item Default is \qtt{PerformanceHint} for quick rendering
\item Alternative is \qtt{QualityHint} when accuracy is more important
\end{itemize}

\end{slide}

%----------------------------------------------------------------------

\begin{slide}{1270}\frametitle{Visual Effects~\textendash~Colorize}

\flushedImage{qml-animations/images/colorize-effect.png}
% declarative-uis/effects/colorize-effect.qml
\inputqml{qml-animations/colorized/colorize-effect}    

\demo{qml-animations/ex-effects/colorize-effect.qml}


\end{slide}

%----------------------------------------------------------------------

\begin{slide}{1269}\frametitle{Visual Effects~\textendash~Colorize}

The standard \qic{class}{Colorize} effect

\begin{itemize}
\item Adjusts the colors used in an element
\item Towards the color specified in the \qic{type}{color} property
\end{itemize}

\vspace*{0.5em}
\centeredImage{qml-animations/images/colorize-values.png}

\vspace*{1em}
The \qic{type}{strength} property

\begin{itemize}
\item Describes the extent of the recoloring
\item \qic{number}{0.0} is no recoloring
\item \qic{number}{1.0} (default) is complete recoloring
\item Choose values between these for different effects
\end{itemize}

\end{slide}

%----------------------------------------------------------------------

\begin{slide}{1268}\frametitle{Visual Effects~\textendash~Drop Shadow}

\flushedImage{qml-animations/images/drop-shadow-effect.png}
% declarative-uis/effects/drop-shadow-effect.qml
\inputqml{qml-animations/colorized/drop-shadow-effect}  

\demo{qml-animations/ex-effects/drop-shadow-effect.qml}


\end{slide}

%----------------------------------------------------------------------

\begin{slide}{1267}\frametitle{Visual Effects~\textendash~Drop Shadow}

The standard \qic{class}{DropShadow} effect

\begin{itemize}
\item Displays a shadow behind an item
\item With horizontal and vertical offsets: \qic{type}{xOffset} and \qic{type}{yOffset}
\item With a color: \qic{type}{color}
\item \qic{type}{blurRadius} describes the blurring applied to the shadow
\end{itemize}

\vspace*{0.5em}
\centeredImage{qml-animations/images/drop-shadow-values.png}

\vspace*{1em}
Default values:

\begin{itemize}
\item Default \qic{type}{blurRadius} of 1 (a sharp shadow)
\item The default \qic{type}{color} is black
\item Default value of 8 pixels for \qic{type}{xOffset} and \qic{type}{yOffset}
\end{itemize}

\end{slide}

%----------------------------------------------------------------------

\begin{slide}{1266}\frametitle{Visual Effects~\textendash~Opacity}

\flushedImage{qml-animations/images/opacity-effect.png}
% declarative-uis/effects/opacity-effect.qml
\inputqml{qml-animations/colorized/opacity-effect}  

\demo{qml-animations/ex-effects/opacity-effect.qml}


\end{slide}

%----------------------------------------------------------------------

\begin{slide}{1265}\frametitle{Visual Effects~\textendash~Opacity}

The standard \qic{class}{Opacity} effect

\begin{itemize}
\item Uniformly changes the opacity of an item
\item Specified with an \qic{type}{opacity} property
\end{itemize}

\vspace*{0.5em}
\centeredImage{qml-animations/images/opacity-values.png}

\vspace*{1em}
Default value:

\begin{itemize}
\item Default \qic{type}{opacity} of 0.7
\end{itemize}

\end{slide}

%----------------------------------------------------------------------

\begin{slide}{1264}\frametitle{Combining Visual Effects}

Effects can be combined by using nested elements

\begin{itemize}
\item Apply an effect to an item
\item Place the item inside another
  \begin{itemize}
  \item apply an effect to the outer item
  \end{itemize}
\end{itemize}

\end{slide}

%----------------------------------------------------------------------

\begin{slide}{1263}\frametitle{Combining Visual Effects}

\flushedImage{qml-animations/images/combining-effects.png}
% declarative-uis/animations/combining-effects.qml
\inputqml{qml-animations/colorized/combining-effects}   

\demo{qml-animations/ex-effects/combining-effect.qml}


\end{slide}

%----------------------------------------------------------------------

\begin{slide}{1262}\frametitle{Animating Visual Effects}

Animations and visual effects can be combined

\begin{itemize}
\item Fading and transformations are often used together
\item Effects parameters can be changed over time
\end{itemize}

\end{slide}
