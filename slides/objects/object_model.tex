%%%%%%%%%%%%%%%%%%%%%%%%%%%%%%%%%%%%%%%%%%%%%%%%%%%%%%%%%%%%%%%%%%%%%%%%%%
%
% Copyright (c) 2008-2011, Nokia Corporation and/or its subsidiary(-ies).
% All rights reserved.
%
% This work, unless otherwise expressly stated, is licensed under a
% Creative Commons Attribution-ShareAlike 2.5.
%
% The full license document is available from
% http://creativecommons.org/licenses/by-sa/2.5/legalcode .
%
%%%%%%%%%%%%%%%%%%%%%%%%%%%%%%%%%%%%%%%%%%%%%%%%%%%%%%%%%%%%%%%%%%%%%%%%%%

\subsection{Common Features of Qt's Object Model}

% ----------------------------------------------------------------------
\begin{slide}{1244}
  \frametitle{Qt's C++ Object Model - \iCls{QObject}}
  \begin{itemize}
  \item \iCls{QObject} is the heart of Qt's object model
  \item Adds features to C++, like ...
    \begin{itemize}
    \item Signals and slots
    \item Properties
    \item Event handling
    \item Memory management
    \item ...
   \end{itemize}
 \item Some features are standard C++
    \begin{itemize}
    \item Some use Qt's meta-object system 
    \end{itemize}
  \item \iCls{QObject} has no visual representation

 \end{itemize}
\end{slide}


% ----------------------------------------------------------------------
\begin{slide}[fragile]{0550} \label{parent_child} \frametitle{Object Tree}
  \xConcept{Parent/child relationship}
\begin{itemize}
  \item \iCls{QObjects} organize themselves in object trees
    \begin{itemize}
    \item Based on parent-child relationship
    \end{itemize}
  \flushedImage{ooo/reference}
  \item \texttt{QObject(QObject *parent = 0)}
    \begin{itemize}
    \item Parent adds object to list of children
    \item Parent owns children
    \end{itemize}
  \item Construction/Destruction
    \begin{itemize}
   \item Tree can be constructured in any order
    \item Tree can be destroyed in any order
      \begin{itemize}
      \item if object has parent: object first removed from parent
      \item if object has children: deletes each child first
      \item No object is deleted twice
      \end{itemize}
    \end{itemize}
 \end{itemize}
 \medskip
  \textit{Note: Parent-child relationship is NOT inheritance}
\end{slide}


% ----------------------------------------------------------------------
\begin{slide}[fragile]{0552}\frametitle{Creating Objects}
  \xConcept{QObject!allocating}
 \begin{itemize}

 \item \textbf{On Heap} - \iCls{QObject} with parent:
   \begin{itemize}
   \begin{cpp}
QLabel *label = new QLabel("Some Text", parent);
    \end{cpp}
\item \texttt{QLayout::addWidget()} and \texttt{QWidget::setLayout()} reparent children automatically
    \end{itemize}

    \item \textbf{On Stack} - \texttt{QObject} without parent:
    \begin{itemize}
    \item \texttt{QFile}, usually local to a function
    \item \texttt{QApplication} (local to \texttt{main()})
    \item Top level \texttt{QWidget}s: \texttt{QMainWindow}
    \end{itemize}
    \medskip
\item \textbf{On Stack} - "value" types  \doc{qvariant.html\#Type-enum}{\texttt{QVariant::Type}}
    \begin{itemize}    
     \begin{cpp}
QString name;
QStringList list;
QColor color;
    \end{cpp}
    \item Do not inherit QObject
    \item Passed by value everywhere
    \item Exception: QString is implicitly shared (COW strategy)
          % which is different from std::string (nowadays deep-copied for simplicity)
   \end{itemize}
\item \textbf{Stack or Heap} - \texttt{QDialog} - depending on lifetime 
  \end{itemize}
\end{slide}

% ----------------------------------------------------------------------

\begin{slide}{1243}
  \frametitle{Qt's Widget Model - \texttt{QWidget}}
 \flushedImage{ooo/qobject-inherited}
  \begin{itemize}
 \item \textbf{Derived from \texttt{QObject}}
    \begin{itemize}
    \item Adds visual representation
    \end{itemize}
  \item \textbf{Base of user interface objects}
  \item \textbf{Receives events}
    \begin{itemize}
    \item e.g. mouse, keyboard events
    \end{itemize}
  \item \textbf{Paints itself on screen}
    \begin{itemize}
    \item Using styles
    \end{itemize}

\end{itemize}
\end{slide}

% ----------------------------------------------------------------------
\begin{slide}[fragile]{0551}\frametitle{Object Tree and QWidget}
  \begin{itemize}
  \item \textbf{\texttt{new QWidget(0)}}
    \begin{itemize}
    \item Widget with no parent $=$ "window"
    \end{itemize}
  \item \textbf{QWidget's children}
    \begin{itemize}
    \item Positioned in parent's coordinate system
    \item Clipped by parent's boundaries
    \end{itemize}
  \item \textbf{QWidget parent }
    \begin{itemize}
    \item Propagates state changes
      \begin{itemize}
      \item hides/shows children when it is hidden/shown itself
      \item enables/disables children when it is enabled/disabled itself
      \end{itemize}
    \end{itemize}
  \item \textbf{\textit{Tristate} mechanism}
    \begin{itemize}
      \item For hide/show and enable/disable, ensures that e.g. an
        explicitly hidden child is not shown when the parent is shown.
      \item \demo{objects/ex-showhide}
      \end{itemize}
 \end{itemize}
\end{slide}


% ----------------------------------------------------------------------
\begin{slide}{0553}\frametitle{Widgets that contain other widgets}
  \xConcept{Geometry Management!Basic}
  \flushedImage{objects/images/widgetGrouping}
  
  \begin{itemize}
  \item Container Widget
    \begin{itemize}
    \item Aggregates other child-widgets
    \end{itemize}
  \item Use layouts for aggregation
    \begin{itemize}
    \item In this example: \iCls{QHBoxLayout} and \\ \iCls{QVBoxLayout}
    \item Note: Layouts are \emph{not} widgets
    \end{itemize}
  \item Layout Process
    \begin{itemize}
    \item Add widgets to layout
    \item Layouts may be nested
    \item Set layout on container widget
    \end{itemize}
  \end{itemize}
\end{slide}

% ----------------------------------------------------------------------
\begin{slide}[fragile]{0555}
  \frametitle{Example Container Widget}
  \begin{cpp}
// container (window) widget creation
QWidget* container = new QWidget; // top-level widget has no parent
QLabel* label = new QLabel("Note:", container);
QTextEdit* edit = new QTextEdit(container);
QPushButton* clear = new QPushButton("Clear", container);
QPushButton* save = new QPushButton("Save", container);
    \end{cpp} 
  \flushedImage{objects/images/widgetGrouping}
   \begin{cpp}
// widget layout
QVBoxLayout* outer = new QVBoxLayout();
outer->addWidget(label);
outer->addWidget(edit);
QHBoxLayout* inner = new QHBoxLayout();
inner->addWidget(clear);
inner->addWidget(save);
    \end{cpp} 
  \begin{cpp}
container->setLayout(outer);
outer->addLayout(inner); // nesting layouts
    \end{cpp}
  \demo{objects/ex-simplelayout}
\end{slide}



% ----------------------------------------------------------------------
\begin{slide}[fragile]{976}
  \frametitle{Questions And Answers}\label{parentChildQuestions}
  \begin{questionize}
  \item What is an object tree?
  \item Which pointers to QObjects do you need to keep around?
  \item What does it mean when a QWidget has no parent?
%   \item What is \texttt{qmake}, and when is it a good idea to use it?
  \item Allocate on heap or stack? \\ \texttt{QWidget; QStringList;
      QApplication; QString; QFile}
%   \item Name places where you can find answers about Qt problems
  \item Name some layout managers and when to use them
  \item What does it mean to nest layouts?
 \end{questionize}
\end{slide}

% ----------------------------------------------------------------------
\begin{slide}{0564}\frametitle{Lab: Your first Qt Application}
  \label{first_app}
  \xProject{Your first Qt Application}
  \flushedImage{objects/images/firstApplication}
  \begin{itemize}
  \item \textbf{Implement the application shown here}
    \begin{itemize}
   \item Search the widgets
      \begin{itemize}    
      \item \doc{gallery-windowsvista.html}{Qt Widget Gallery}
      \item ... and choose your os style
      \end{itemize}
    \item Layouts: \texttt{QHBoxLayout, QVBoxLayout}
      \begin{itemize}
      \item See previous slides how to use them
      \end{itemize}
    \end{itemize}
  \item \textbf{Optionally}
    \begin{itemize} 
    \item Provide a window title
    \item Add Edit, Remove buttons
      \begin{itemize}
      \item  On the right of list
      \end{itemize}
    \item Use group box to provide list caption
    \end{itemize}
 \end{itemize}
  \medskip
  \lab{objects/lab-firstapp} \newline
\end{slide}
