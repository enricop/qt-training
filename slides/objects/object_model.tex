%%%%%%%%%%%%%%%%%%%%%%%%%%%%%%%%%%%%%%%%%%%%%%%%%%%%%%%%%%%%%%%%%%%%%%%%%%
%
% Copyright (c) 2008-2011, Nokia Corporation and/or its subsidiary(-ies).
% All rights reserved.
%
% This work, unless otherwise expressly stated, is licensed under a
% Creative Commons Attribution-ShareAlike 2.5.
%
% The full license document is available from
% http://creativecommons.org/licenses/by-sa/2.5/legalcode .
%
%%%%%%%%%%%%%%%%%%%%%%%%%%%%%%%%%%%%%%%%%%%%%%%%%%%%%%%%%%%%%%%%%%%%%%%%%%

\subsection{Qt's Object Model}

% ----------------------------------------------------------------------
\begin{slide}{1244}
  \frametitle{Qt's C++ Object Model - \iCls{QObject}}
  \begin{itemize}
  \item \iCls{QObject} is the heart of Qt's object model
  \item Include these features:
    \begin{itemize}
    \item Memory management
    \item Object properties
    \item Signals and slots
    \item Event handling
   \end{itemize}
  \item \iCls{QObject} has no visual representation

 \end{itemize}
\end{slide}


% ----------------------------------------------------------------------
\begin{slide}{0550} \label{parent_child} \frametitle{Object Tree}
  \xConcept{Parent/child relationship}
\begin{itemize}
  \item \iCls{QObjects} organize themselves in object trees
    \begin{itemize}
    \item Based on parent-child relationship
    \end{itemize}\medskip
  \flushedImage{ooo/reference}

  \item \texttt{QObject(QObject *parent = 0)}
    \begin{itemize}
    \item Parent adds object to list of children
    \item Parent owns children
    \end{itemize}\medskip

  \item Construction/Destruction
    \begin{itemize}
    \item Trees can be constructed in any order
    \item Trees can be destroyed in any order
      \begin{itemize}
      \item If object has a parent: object is first removed from the parent
      \item If object has children: deletes each child first
      \item No object is deleted twice
      \end{itemize}
    \end{itemize}
 \end{itemize}
 \medskip
  \textit{Note: Parent-child relationship is NOT inheritance}
\end{slide}


% ----------------------------------------------------------------------
\begin{slide}[fragile]{0552}\frametitle{Creating Objects - General Guideline}
  \xConcept{QObject!allocating}
 \begin{itemize}

 \item \textbf{On Heap} - \iCls{QObject} with parent
   \begin{itemize}
   \item[]
   \begin{cpp}
QTimer* timer = new QTimer(this);
    \end{cpp}
   \end{itemize}\medskip

    \item \textbf{On Stack} - \texttt{QObject} without parent:
    \begin{itemize}
    \item \texttt{QFile}, usually local to a function
    \item \texttt{QApplication} (local to \iCls{main())}
    \end{itemize}
    \medskip

\item \textbf{On Stack} - value types
    \begin{itemize}
    \item QString, QStringList, QColor
   \end{itemize}\medskip

\item \textbf{Stack or Heap} - \texttt{QDialog} - depending on lifetime 
  \end{itemize}
\end{slide}

% ----------------------------------------------------------------------

\begin{slide}{1243}
  \frametitle{Qt's Widget Model - \texttt{QWidget}}
 \flushedImage{ooo/qobject-inherited}
  \begin{itemize}
 \item Derived from \iCls{QObject}
    \begin{itemize}
    \item Adds visual representation
    \end{itemize}\medskip
  \item Receives events
    \begin{itemize}
    \item e.g. mouse, keyboard events
    \end{itemize}\medskip
  \item Paints itself on screen
    \begin{itemize}
    \item Using styles
    \item Or custom painting routines
    \end{itemize}
  \item Traditional user interface object
\end{itemize}
\end{slide}

% ----------------------------------------------------------------------

\begin{slide}[fragile]{0551}\frametitle{Object Tree and QWidget}
  \begin{itemize}
  \item \texttt{new QWidget(0)}
    \begin{itemize}
    \item Widget with no parent $=$ "window"
    \end{itemize}\medskip

  \item QWidget's children
    \begin{itemize}
    \item Positioned in parent's coordinate system
    \item Clipped by parent's boundaries
    \end{itemize}\medskip

  \item QWidget parent
    \begin{itemize}
    \item Propagates state changes
      \begin{itemize}
      \item hides/shows children when it is hidden/shown itself
      \item enables/disables children when it is enabled/disabled itself
      \end{itemize}
    \end{itemize}
 \end{itemize}
\end{slide}

% ----------------------------------------------------------------------
\begin{slide}{0553}\frametitle{Widgets that contain other widgets}
  \xConcept{Geometry Management!Basic}
  \flushedImage{objects/images/widgetGrouping}
  
  \begin{itemize}
  \item Container Widget
    \begin{itemize}
    \item Aggregates other child-widgets
    \end{itemize}\medskip

  \item Use layouts for aggregation
    \begin{itemize}
    \item In this example: \iCls{QHBoxLayout} and \\ \iCls{QVBoxLayout}
    \item Note: Layouts are \emph{not} widgets
    \end{itemize}\medskip

  \item Layout Process
    \begin{itemize}
    \item Add widgets to layout
    \item Layouts may be nested
    \item Set layout on container widget
    \end{itemize}
  \end{itemize}
\end{slide}

% ----------------------------------------------------------------------
\begin{slide}[fragile]{0555}\frametitle{Example Container Widget}
  \flushedImage{objects/images/widgetGrouping}
  \begin{cpp}
// container (window) widget creation
QWidget* container = new QWidget;
QLabel* label = new QLabel("Note:", container);
QTextEdit* edit = new QTextEdit(container);
QPushButton* clear = new QPushButton("Clear", container);
QPushButton* save = new QPushButton("Save", container);

// widget layout
QVBoxLayout* outer = new QVBoxLayout();
outer->addWidget(label);
outer->addWidget(edit);
QHBoxLayout* inner = new QHBoxLayout();
inner->addWidget(clear);
inner->addWidget(save);

container->setLayout(outer);
outer->addLayout(inner); // nesting layouts
    \end{cpp}
  \demo{coretypes/ex-simplelayout}
\end{slide}
