%%%%%%%%%%%%%%%%%%%%%%%%%%%%%%%%%%%%%%%%%%%%%%%%%%%%%%%%%%%%%%%%%%%%%%%%%%
%
% Copyright (c) 2008-2011, Nokia Corporation and/or its subsidiary(-ies).
% All rights reserved.
%
% This work, unless otherwise expressly stated, is licensed under a
% Creative Commons Attribution-ShareAlike 2.5.
%
% The full license document is available from
% http://creativecommons.org/licenses/by-sa/2.5/legalcode .
%
%%%%%%%%%%%%%%%%%%%%%%%%%%%%%%%%%%%%%%%%%%%%%%%%%%%%%%%%%%%%%%%%%%%%%%%%%%

\subsection{Handling Events in Qt}

%----------------------------------------------------------------------
\begin{slide}{970}
\frametitle{Event Processing}
\label{events_introduction}

\begin{block}{Qt is an event-driven UI toolkit}
\iClsFn{QApplication}{exec} runs the \textit{event loop}
\end{block}
\begin{enumerate}
\item \textbf{Generate Events}
  \begin{itemize}
  \item by input devices: keyboard, mouse, etc.
  \item by Qt itself (e.g. timers)
  \end{itemize}
\item \textbf{Queue Events}
  \begin{itemize}
  \item by event loop
  \end{itemize}
\item \textbf{Dispatch Events}
  \begin{itemize}
  \item by \iCls{QApplication} to receiver: \iCls{QObject}
    \begin{itemize}
    \item \textit{Key events sent to widget with focus}
    \item \textit{Mouse events sent to widget under cursor}
    \end{itemize}
 \end{itemize}
\item \textbf{Handle Events}
  \begin{itemize}
  \item by \iCls{QObject} event handler methods
  \end{itemize}
\end{enumerate}
\end{slide}

%----------------------------------------------------------------------
\begin{slide}{1080}
\frametitle{Event Handling}

\begin{itemize}
\item QObject::event(QEvent *event)
  \begin{itemize}
  \item Handles all events for this object
 \end{itemize}
\item Specialized event handlers
  \begin{itemize}
  \item \iClsFn{QWidget}{mousePressEvent} for mouse clicks
  \item \iClsFn{QWidget}{keyPressEvent} for key presses     
 \end{itemize}
\item Accepting an Event
  \begin{itemize}
  \item \lstinline{event->accept()} / \lstinline{event->ignore()}
    \begin{itemize}
    \item Accepts or ignores the event
    \item Accepted is the default.
    \end{itemize}
  \end{itemize}
\item Event propagation
  \begin{itemize}
  \item Happens if event is ignored
  \item Might be propagated to parent widget
  \end{itemize}
\end{itemize}
\demo{objects/ex-allevents}
\end{slide}

%----------------------------------------------------------------------
\begin{slide}[fragile]{1083}
\frametitle{Example of Event Handling}

\begin{itemize}
\item \iCls{QCloseEvent} delivered to top level widgets (windows)
\item Accepting event allows window to close
\item Ignoring event keeps window open
\item[]
\begin{cpp}
void MyWidget::closeEvent(QCloseEvent *event) {
  if (maybeSave()) {
    writeSettings();
    event->accept(); // close window
  } else {
    event->ignore(); // keep window
  }
}
\end{cpp}  
\end{itemize}
\demo{objects/ex-closeevent}

\end{slide}

%----------------------------------------------------------------------
\begin{slide}{1088}
\frametitle{Events and Signals}
Signals and slots are used instead of events:

\begin{itemize}
\item To communicate between components.
\item In cases where there is a well-defined sender and receiver.
  \begin{itemize}
  \item For example: a button and a slot to handle clicks.
  \end{itemize}
\item For some events, there is no sender in Qt.
  \begin{itemize}
  \item For example: redraw, keyboard and mouse events.
  \end{itemize}
\item To describe high level logic and control flow.
\end{itemize}

Developers can create custom events if they need to.
\end{slide}
