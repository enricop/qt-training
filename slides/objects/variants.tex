%%%%%%%%%%%%%%%%%%%%%%%%%%%%%%%%%%%%%%%%%%%%%%%%%%%%%%%%%%%%%%%%%%%%%%%%%%
%
% Copyright (c) 2008-2011, Nokia Corporation and/or its subsidiary(-ies).
% All rights reserved.
%
% This work, unless otherwise expressly stated, is licensed under a
% Creative Commons Attribution-ShareAlike 2.5.
%
% The full license document is available from
% http://creativecommons.org/licenses/by-sa/2.5/legalcode .
%
%%%%%%%%%%%%%%%%%%%%%%%%%%%%%%%%%%%%%%%%%%%%%%%%%%%%%%%%%%%%%%%%%%%%%%%%%%

\subsection{Variants}

%----------------------------------------------------------------------
\begin{slide}[fragile]{0238}\frametitle{QVariant}
\begin{itemize}
  \item \iCls{QVariant} 
  \begin{itemize}
    \item Union for common Qt "value types" (copyable, assignable)
    \item Supports implicit sharing (fast copying)
    \item Supports user types
  \end{itemize}\medskip
\item A generic data object
\item Use cases:
\begin{cpp}
QVariant property(const char* name) const;
void setProperty(const char* name, const QVariant &value);
\end{cpp}\medskip

\begin{cpp}
class QAbstractItemModel {
  virtual QVariant data( const QModelIndex& index, int role );
  ...
}
\end{cpp}
\end{itemize}
\end{slide}

%----------------------------------------------------------------------
\begin{slide}[fragile]{1020}\frametitle{QVariant}
\begin{itemize}
  \item For \texttt{QtCore} types  
  \item[] \begin{cpp}
QVariant variant(42);
int value = variant.toInt(); // read back as integer
QString text = variant.toString(); // read back as string
qDebug() << variant.typeName(); // int
\end{cpp}\vspace*{3mm}
\item For non-core and custom types:
  \item[] \begin{cpp}
QVariant variant = QVariant::fromValue(QColor(Qt::red));
QColor color = variant.value<QColor>(); // read back
qDebug() << variant.typeName(); // "QColor"
\end{cpp}\vspace*{3mm}
\item[] \doc{qvariant.html\#details}{QVariant}
\end{itemize}
\end{slide}

%----------------------------------------------------------------------
\begin{slide}[fragile]{0240}
\frametitle{Custom data types in variants}
\begin{itemize}
\item[]
\begin{cpp}
#include <QMetaType>

class Contact
{
  public:
    void setName(const QString & name);
    QString name() const;
  ...
};

Q_DECLARE_METATYPE(Contact);
\end{cpp}\medskip
\item Type must support default construction, copy and assignment.\medskip
\item Q\_DECLARE\_METATYPE should be after class definition in header file.

\medskip
\doc{qmetatype.html\#Q_DECLARE_METATYPE}{Q\_DECLARE\_METATYPE}
\end{itemize}
\end{slide}

% ----------------------------------------------------------------------
\begin{slide}[fragile]{1254}
\frametitle{Custom Types and QVariant}

\begin{itemize}
\item[]
\begin{cpp}
#include "Contact.h"
#include <QDebug>
#include <QVariant>

int main(int argc, char* argv[])
{
    Contact contact;
    contact.setName("Peter");

    const QVariant variant = QVariant::fromValue(contact);

    const Contact otherContact = variant.value<Contact>();
    qDebug() << otherContact.name(); // "Peter"
    qDebug() << variant.typeName();  // prints "Contact"

    return 0;
}
\end{cpp}
\demo{coretypes/ex\_custom\_types}
\end{itemize}
\end{slide}

% ----------------------------------------------------------------------
\begin{slide}{1253}
\frametitle{qRegisterMetaType}
\begin{itemize}
  \item Q\_DECLARE\_METATYPE makes the type known to \iCls{QVariant} and template based functions
  \item If instances of the type need to be marshalled or serialized they must 
        {\it also} be registered
  \item \iCls{qRegisterMetaType} will register the type
  \begin{itemize}
    \item Instances can be created dynamically at runtime
    \item Type can be used with property system and as signal parameters in
            \textit{queued} connections 
  \end{itemize}
\end{itemize}
\end{slide}

% ----------------------------------------------------------------------
\begin{slide}[fragile]{1253}
\frametitle{qRegisterMetaType Example}
\begin{itemize}
\item[]
\begin{cpp}
int main(int argc, char* argv[])
{
    // Register string typename:
    const int typeId = qRegisterMetaType<Contact>();

    Contact contact;
    contact.setName("Peter");

    // Create copy of object in a generic way
    void *object = QMetaType::construct(typeId, &contact);

    Contact *otherContact = reinterpret_cast<Contact*>(object);
    qDebug() << otherContact->name();

    return 0;
}
\end{cpp}
\doc{qmetatype.html\#qRegisterMetaType-2}{qRegisterMetaType}
\doc{qmetatype.html\#construct}{construct}
\end{itemize}
\end{slide}
