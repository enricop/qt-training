%%%%%%%%%%%%%%%%%%%%%%%%%%%%%%%%%%%%%%%%%%%%%%%%%%%%%%%%%%%%%%%%%%%%%%%%%%
%
% Copyright (c) 2008-2011, Nokia Corporation and/or its subsidiary(-ies).
% All rights reserved.
%
% This work, unless otherwise expressly stated, is licensed under a
% Creative Commons Attribution-ShareAlike 2.5.
%
% The full license document is available from
% http://creativecommons.org/licenses/by-sa/2.5/legalcode .
%
%%%%%%%%%%%%%%%%%%%%%%%%%%%%%%%%%%%%%%%%%%%%%%%%%%%%%%%%%%%%%%%%%%%%%%%%%%

\subsection{Object Communication using Signals \& Slots}

%----------------------------------------------------------------------
\begin{slide}[fragile]{0795}\frametitle{Callbacks}
  \begin{block}{General Problem}
    How do you get from "the user clicks a button" to your business logic?
  \end{block}
  \begin{itemize}
 \item Possible solutions
    \begin{itemize}
    \item Callbacks
      \begin{itemize}
      \item Based on function pointers
      \item Not type-safe
     \end{itemize}
    \item Observer Pattern (Listener)
      \begin{itemize}
      \item Based on interface classes
      \item Needs listener registraction
      \item Many interface classes
      \end{itemize}  
    \end{itemize}
  \item Qt uses
    \begin{itemize}
    \item Signals and slots for high-level (semantic) callbacks
    \item Virtual methods for low-level (syntactic) events.
    \end{itemize}
  \end{itemize}
\end{slide}

%----------------------------------------------------------------------
\begin{slide}{0796}\frametitle{Signals \& Slots}
  \xConcept{Signals and Slots} \xConcept{MOC}
  \label{signals_and_slots}
  \begin{itemize}
  \item Object Communication
    \begin{itemize}
    \item \textbf{Signal} - \textit{emitted to notify other objects}
    \item \textbf{Slot} - \textit{method called in response to signal}
   \end{itemize}
  \item Provides type-safe callbacks
  \item After getting used to it, they are
    \begin{itemize}
    \item easier to use than message maps,
    \item more secure than callbacks,
    \item more flexible than virtual methods
    \end{itemize}
  \item Fosters component-based programming
 \end{itemize}
\end{slide}

%----------------------------------------------------------------------

% I have to shrink these pages in 1024x768, there are no way we can fit it
% in otherwise.
% \ifthenelse{\equal{\screenwidth}{1024}}{
%   \renewcommand{\paperToScreenFactor}{0.28346484375}
%   \renewcommand{\doublePaperToScreenFactor}{0.566929687}
% }{}

\begin{slide}[fragile]{0797}\frametitle{Connecting Signals to Slots}
\begin{overprint}
  \onslide<1| handout:0> \centeredImageFullHeight{objects/images/signal-slot-cpp-step1}
  \onslide<2| handout:0> \centeredImageFullHeight{objects/images/signal-slot-cpp-step2}
  \onslide<3| handout:0> \centeredImageFullHeight{objects/images/signal-slot-cpp-step3}
  \onslide<4| handout:0> \centeredImageFullHeight{objects/images/signal-slot-cpp-step4}
  \onslide<5| handout:0> \centeredImageFullHeight{objects/images/signal-slot-cpp-step5}
  \onslide<6| handout:0> \centeredImageFullHeight{objects/images/signal-slot-cpp-step6}
  \onslide<7| handout:0> \centeredImageFullHeight{objects/images/signal-slot-cpp-step7}
  \onslide<8| handout:1> \centeredImageFullHeight{objects/images/signal-slot-cpp-all} \\ \demo{objects/ex-connect}

\end{overprint}
\end{slide}

% \ifthenelse{\equal{\screenwidth}{1024}}{
%   \renewcommand{\paperToScreenFactor}{0.354331054688}
%   \renewcommand{\doublePaperToScreenFactor}{0.708662109}
% }{}

%----------------------------------------------------------------------
\begin{slide}[fragile]{0798}\frametitle{Custom Slots}
 \label{codeForSlots1}
\xConcept{connect}
\xConcept{Q\_OBJECT}
\begin{itemize}
\item File: \textbf{myclass.h} 
  \begin{cpp}
class MyClass : public QObject 
{
  Q_OBJECT // marker for moc
  // ...
public slots:
  void setValue(int value); // a custom slot
};
  \end{cpp}
\item File: \textbf{myclass.cpp}
  \begin{cpp}
void MyClass::setValue(int value) {
  // slot implementation
}

  \end{cpp}
\end{itemize}
\end{slide}

%----------------------------------------------------------------------
\begin{slide}{0226}\frametitle{Example of Custom Slots}\label{qtimer}

\begin{itemize}
  \item \iCls{QTimer} is a class for executing functions at a later time.
  \item Connect the QTimer's signal \hClsFn{QTimer}{timeout} to a custom slot.
  \item Call \hClsFn{QTimer}{setSingleShot} for a single-shot timer.
  \item Finally, call \hClsFnPar{QTimer}{start}{int msec} on the timer to start it.
  \item For a one-time non-cancellable single-shot timer: \iClsFnPar{QTimer}{singleShot}{1000, this, SLOT(doit())}
  \end{itemize}

\demo{objects/ex-stop-watch}

\end{slide}

%----------------------------------------------------------------------

\begin{slide}[fragile]{07981}\frametitle{Custom Signals}
\label{codeForSignals1}
\xConcept{Q\_OBJECT}
\begin{itemize}
\item File: \textbf{myclass.h} 
  \begin{cpp}
class MyClass : public QObject 
{
  Q_OBJECT // marker for moc                       
  // ...
signals:
  void valueChanged(int value); // a custom signal
};
  \end{cpp}
\item File: \textbf{myclass.cpp}
  \begin{cpp}
// No implementation for a signal

  \end{cpp}
\item Sending a signal
\begin{cpp}
emit valueChanged(value);
\end{cpp}
\end{itemize}

\demo{objects/ex-quotebutton}

\end{slide}


%----------------------------------------------------------------------


\begin{slide}[fragile]{0799}\frametitle{\texttt{Q\_OBJECT} - flag for MOC}
\begin{itemize}
\item \textbf{\texttt{Q\_OBJECT}}
  \begin{itemize}
  \item Enhances QObject with meta-object information
  \item Required for Signals \& Slots
  \item \textbf{moc} creates meta-object information
    \begin{itemize}
    \item[] \begin{shell}
moc -o moc_myclass.cpp myclass.h
c++ -c myclass.cpp; c++ -c moc_myclass.cpp
c++ -o myapp moc_myclass.o myclass.o
    \end{shell}
   \end{itemize}
 \item \texttt{qmake} takes care of moc files for you
 \end{itemize}
\item \textbf{Analyze definition of} 
  \begin{itemize}
  \item \texttt{Q\_OBJECT}
  \item \texttt{signals} and \texttt{slots}
  \item \texttt{emit}
  \item At \textit{\texttt{\$QTDIR/src/corelib/kernel/qobjectdefs.h}}
  \end{itemize}        
\item \textbf{Look at moc generated files}
  \begin{itemize}
  \item \demo{objects/ex-signalslots}
  \end{itemize} 
\end{itemize}

\end{slide}

%----------------------------------------------------------------------
\begin{slide}[fragile]{0802}\frametitle{Back to the Original Problem}
  \xConcept{connect}
  \begin{block}{We asked some slides ago...}
    How to react to a button being clicked?
  \end{block}
\begin{itemize}
\item Solution:
  \begin{itemize}
  \item \emph{Implement a slot in your widget}
  \item \emph{Connect the button's clicked signal to the slot}
  \end{itemize}
\item Connect statement
\begin{lstlisting}
connect(sender, signal, receiver, slot);
\end{lstlisting}

\item Example
\begin{cpp}
connect(button, SIGNAL(clicked()), this, SLOT(onClicked()));
\end{cpp}
\end{itemize}
\end{slide}


%----------------------------------------------------------------------
\begin{slide}[fragile]{0807}
\frametitle{Lab: Connect to Click}
\label{project_signal_slot}
\xProject{Color Tester}
\flushedImageDoubleWidth{objects/images/colorTester}
\begin{itemize}
\item \textbf{Create an application as shown here}
  \begin{itemize}
  \item Clicking on ``Select Color'' \\
    updates label with color's name.
  \end{itemize}
\item \textbf{Hints}
  \begin{itemize}
  \item \iClsFn{QColorDialog}{getColor} to fetch a color
  \item \iClsFn{QColor}{name} to get the color name
 \end{itemize}
\item \textbf{Optional}
  \begin{itemize}
  \item In \iCls{QColorDialog}, honor the user clicking
    ``cancel'', and provide it with the current color to start from.
  \item Set the selected color as the label's background (Hint: see \iCls{QPalette})
  \end{itemize}
\end{itemize}
\vfill
\lab{objects/lab-selectcolor}
\end{slide}

%----------------------------------------------------------------------

\subsection{Signal/Slot Variations}

%----------------------------------------------------------------------
\begin{slide}[fragile]{0803}\frametitle{Variations of Signal/Slot
    Connections}
  \begin{center}
 \begin{tabular}{|c|c|c|}
\hline
\textbf{Signal(s)}                & \textbf{Connect to}   & \textbf{Slot(s)} \\\hline
\hline
one & \correct & many \\\hline
many & \correct & one \\\hline
one & \correct & another signal \\\hline
  \end{tabular}
\end{center}
\begin{itemize}
\item Signal to Signal connection
 \item[] \begin{cpp}
connect(bt, SIGNAL(clicked()), this, SIGNAL(okSignal()));
  \end{cpp}

\item \textbf{Not} allowed to name parameters
\begin{tabbing}
connect( \=m\_slider, \=SIGNAL( valueChanged( int \incorrecttext{value} ) )\\
         \>this,      \>SLOT( setValue( int \incorrecttext{newValue} ) ) )
  \end{tabbing}
\end{itemize}
\end{slide}

%----------------------------------------------------------------------
\begin{slide}{1957}
  \frametitle{Making the Connection}
  \begin{block}{Rule for Signal/Slot Connection}
    Can ignore arguments, but not create values from nothing
  \end{block}
  \begin{center}
    
 \begin{tabular}{r|c|l}
\textbf{Signal}                &    & \textbf{Slot} \\
\hline
\multirow{3}{*}{rangeChanged(int,int)} & \uncover<2->{\correct} & setRange(int,int) \\
 & \uncover<3->{\correct} & setValue(int) \\
 & \uncover<4->{\correct} & updateUi() \\
\hline

\multirow{3}{*}{valueChanged(int)}     & \uncover<5->{\correct} & setValue(int) \\
 & \uncover<6->{\correct} & updateUi() \\
 & \uncover<7->{\incorrect} & setRange(int,int) \\ 
 & \uncover<8->{\incorrect} & setValue(float) \\
\hline                                   

textChanged(QString) & \uncover<9->{\incorrect} & setValue(int) \\
\hline

\multirow{2}{*}{clicked()}  & \uncover<10->{\correct} & updateUi() \\
& \uncover<11->{\incorrect} & setValue(int) \\

  \end{tabular}
  \end{center}
\end{slide}

%----------------------------------------------------------------------
\begin{slide}{0806}\frametitle{Questions And Answers}\label{signalSlotsQuestions}
\begin{questionize}
\item How do you connect a signal to a slot?
\item How would you implement a slot?
\item Name some possible signal/slot connection combinations?
\item How would you emit a signal?
\item Do you need a class to implement a slot?
\item Can you return a value from a slot?
\item When do you need to run \texttt{qmake}?
\item Where do you place the \texttt{Q\_OBJECT} macro and when do you
need it?
\end{questionize}
\end{slide}


%----------------------------------------------------------------------
\begin{slide}{0808}
\frametitle{Lab: Source Compatibility}
\label{project_slider_with_indicator}
\flushedImageDoubleWidth{objects/images/SliderWithValue}
 \begin{itemize}
  \item \textbf{Implement custom slider}
    \begin{itemize}
    \item API compatible with \iCls{QSlider}
    \item Shows current value of slider
    \end{itemize}
 \item \textbf{To create custom slider}
    \begin{itemize}
    \item use \iCls{QSlider} and \iCls{QLabel}
    \end{itemize}
  \item \textbf{To test slider}
    \begin{itemize}
    \item \texttt{main.cpp} provides test code
    \item \iCls{QLCDNumber} is part of test code
    \end{itemize}
  \item \textbf{Optional:}
    \begin{itemize}
    \item  Discuss pros and cons of inheriting from
      QSlider instead of using an instance in a layout.
    \end{itemize}
  \end{itemize}
  \vfill
  \lab{objects/lab-slider}
  
\end{slide}

