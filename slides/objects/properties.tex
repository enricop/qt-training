%%%%%%%%%%%%%%%%%%%%%%%%%%%%%%%%%%%%%%%%%%%%%%%%%%%%%%%%%%%%%%%%%%%%%%%%%%
%
% Copyright (c) 2008-2011, Nokia Corporation and/or its subsidiary(-ies).
% All rights reserved.
%
% This work, unless otherwise expressly stated, is licensed under a
% Creative Commons Attribution-ShareAlike 2.5.
%
% The full license document is available from
% http://creativecommons.org/licenses/by-sa/2.5/legalcode .
%
%%%%%%%%%%%%%%%%%%%%%%%%%%%%%%%%%%%%%%%%%%%%%%%%%%%%%%%%%%%%%%%%%%%%%%%%%%

\subsection{Properties}

\begin{slide}[fragile]{1021}
\frametitle{Properties}
\begin{itemize}
\item Qt Quick example\\[2mm]

\inputqml{qml-intro/colorized/rectangle}

\item Generic property access:\\
\begin{cpp}
QObject* root = view->rootObject();
if (root != NULL) {
    QString color = root->property("color").toString();
}
\end{cpp}
\end{itemize}
\end{slide}

% ----------------------------------------------------------------------
\begin{slide}[fragile]{2084}
\frametitle{Properties}
\begin{itemize}
  \item Q\_PROPERTY is a macro:
\begin{cpp}
  Q_PROPERTY( type name READ getFunction [WRITE setFunction]
  [RESET resetFunction] [NOTIFY notifySignal] [DESIGNABLE bool]
  [SCRIPTABLE bool] [STORED bool] )
\end{cpp}
  \medskip
  \item Property access methods:
\begin{cpp}
  QVariant property(const char* name) const;
  void setProperty(const char* name, const QVariant &value);
\end{cpp}
  \medskip
  \item If name is not declared as a \texttt{Q\_PROPERTY}
  \begin{itemize}
    \item Stored as a \textbf{dynamic property}
    \item Not accessible from Qt Quick
  \end{itemize}
  \medskip
  \item Note:
  \begin{itemize}
    \item Q\_OBJECT macro is required for properties to work
    \item \texttt{QMetaObject} knows nothing about dynamic properties
  \end{itemize}
\end{itemize}
\end{slide}

% ----------------------------------------------------------------------
\begin{slide}[fragile]{1422}
\frametitle{Providing Properties from QObject}
\begin{cpp}
class Customer : public QObject
{
    Q_OBJECT

    Q_PROPERTY(QString id READ getId WRITE setId NOTIFY idChanged);

  public:
     QString getId() const;
     void setId(const QString& id);
  
  signals:
     void idChanged();

  ...
};
\end{cpp}
\end{slide}

% ----------------------------------------------------------------------
\begin{slide}[fragile]{1022}
\frametitle{Enum properties}
\begin{cpp}
class Customer : public QObject
{
    Q_OBJECT

    Q_PROPERTY(CustomerType type READ getType WRITE setType 
               NOTIFY typeChanged);

  public:
    enum CustomerType {
      Corporate, Individual, Educational, Government
    };

    Q_ENUMS(CustomerType);
    
  ...
};
\end{cpp}
\end{slide}

% ----------------------------------------------------------------------
\begin{slide}[fragile]{1423}
\frametitle{Using Properties}
\begin{itemize}
\item \iCls{QMetaObject} support property introspection\smallskip
  \begin{cpp}
  const QMetaObject *metaObject = object->metaObject();
  const QString className = metaObject->className();
  const int propertyCount = metaObject->propertyCount();
  for ( int i=0; i<propertyCount; ++i ) {
    const QMetaProperty metaProperty = metaObject->property(i);
    const QString typeName = metaProperty.typeName()
    const QString propertyName = metaProperty.name();
    const QVariant value = object->property(metaProperty.name());
  }
  \end{cpp}
\end{itemize}
\demo{coretypes/ex-properties}
\end{slide}
