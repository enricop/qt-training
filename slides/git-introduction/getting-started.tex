%%%%%%%%%%%%%%%%%%%%%%%%%%%%%%%%%%%%%%%%%%%%%%%%%%%%%%%%%%%%%%%%%%%%%%%%%%
%
% Copyright (c) 2008-2011, Nokia Corporation and/or its subsidiary(-ies).
% All rights reserved.
%
% This work, unless otherwise expressly stated, is licensed under a
% Creative Commons Attribution-ShareAlike 2.5.
%
% The full license document is available from
% http://creativecommons.org/licenses/by-sa/2.5/legalcode .
%
%%%%%%%%%%%%%%%%%%%%%%%%%%%%%%%%%%%%%%%%%%%%%%%%%%%%%%%%%%%%%%%%%%%%%%%%%%

\subsection{Getting Started}

\begin{slide}[fragile]
  \frametitle{Minimal Setup}
  \begin{itemize}
    \item Let's create a small application...
    \item[]
    \consolecode{%
mkdir MyBlog\\
cd MyBlog\\
git init
    }
    \item Pooring down the first lines of code
    \item As always: Commit frequently, so you can go back...
    \item[]
    \consolecode{%
touch main.cpp MyBlog.hpp MyBlog.cpp \\
qmake \\
git status \\
git add *.*pp \\
git commit -m "Done smth." \\
}
  \end{itemize}
\end{slide}

\begin{slide}[fragile]
  \frametitle{Under the Hood}
  \begin{itemize}
    \item Git is a patch/commit database...
    \item[]
    \consolecode{%
ls .git\\
cat .git/refs/heads/master\\
git log|head
}
    \item The entire repository is contained in the .git subdirectory
    \item Cloning means copying the entire database
    \item Commit = Patch + Contributor Info + Message + Hash
    \item Commit history is a tree
    \begin{itemize}
      \item Heads: Pointers to the leafs/ends of the commit tree
      \item Each commit identified by SH1 Id (40 chars, the hash)
      \item Tags: Names for hashes
      \item Default head: "master"
    \end{itemize}
  \end{itemize}
\end{slide}

\begin{slide}[fragile]
  \frametitle{Git Terms}
  \centeredImageFullHeight{git-introduction/ooo/simple_terms}
  \begin{itemize}
     \item Active branch (which is checked out): {\footnotesize{\verb+.git/HEAD+}}
     \item The default branch is called "master": {\footnotesize{\verb+.git/refs/heads/master+}}
     \item Remote tracking branches: {\footnotesize{\verb+.git/refs/remotes+}}
  \end{itemize}
\end{slide}

\begin{slide}
  \frametitle{Example Tree}
  \centeredImage{git-introduction/images/gitk_qt_47}
\end{slide}

\begin{slide}[fragile]
  \frametitle{Configuring Git}
  \begin{itemize}
    \item Git keeps track of trees and locations here:
    \item[]
    \consolecode{%
cat \$MYPROJECT/.git/config
}
    \item User specific setup (identity and preferences):
    \item[]
    \consolecode{%
cat 1> \textasciitilde/.gitconfig\\
{[}user{]}\\
    name = Frank Mertens\\
    email = ext-frank.mertens{@}nokia.com\\
\emph{Ctrl-D}}
    \item Optional: gpg setup (if needed)
  \end{itemize}
\end{slide}

\begin{slide}[fragile]
  \frametitle{Setting up Console Colors}
  \begin{itemize}
  \item Generate color escapes if writing to TTY or pipe...
  \item[]
    \consolecode{%
cat 1>{>} \textasciitilde/.gitconfig\\
{[}color{]}\\
    diff = auto\\
    status = auto\\
    branch = auto\\
\emph{Ctrl-D}}
  \item And tell GNU less:
  \item[]
    \consolecode{%
cat 1>{>} \textasciitilde/.gitconfig\\
{[}core{]}\\
    pager = less -FRSX\\
    editor = vi\\
\emph{Ctrl-D}}
  \end{itemize}
\end{slide}

\begin{slide}[fragile]
  \frametitle{Reverting Changes}
  \begin{itemize}
    \item Uncommitted:
    \begin{itemize}
      \item[]
      \item Delete affecting files and do:
      \item[]
      \consolecode{%
git checkout
}
      \item Or reset the index (and implicitly restore working tree):
      \item[]
      \consolecode{%
git reset -{-}hard HEAD
}
    \end{itemize}
    \item Committed: Revert the commit
    \item[]
    \consolecode{%
git revert <commit-id>
}
  \end{itemize}
\end{slide}
