%%%%%%%%%%%%%%%%%%%%%%%%%%%%%%%%%%%%%%%%%%%%%%%%%%%%%%%%%%%%%%%%%%%%%%%%%%
%
% Copyright (c) 2008-2011, Nokia Corporation and/or its subsidiary(-ies).
% All rights reserved.
%
% This work, unless otherwise expressly stated, is licensed under a
% Creative Commons Attribution-ShareAlike 2.5.
%
% The full license document is available from
% http://creativecommons.org/licenses/by-sa/2.5/legalcode .
%
%%%%%%%%%%%%%%%%%%%%%%%%%%%%%%%%%%%%%%%%%%%%%%%%%%%%%%%%%%%%%%%%%%%%%%%%%%

\section{Core Classes}

%----------------------------------------------------------------------
\begin{slide}{1256}{Module Objectives}
\label{basic-types}

Qt provides a set of basic data types:

\begin{itemize}
    \item \textbf{String handling classes:}
      \begin{itemize}
      \item Unicode-aware string and character classes
      \item Used throughout the Qt API
      \item Regular expression engine for pattern matching
      \end{itemize}
\item \textbf{Container classes:}
  \begin{itemize}
  \item Common containers: lists, sets, maps, arrays, ...
  \item Including STL and Java-style iterators
  \end{itemize}
\item \textbf{File handling classes:}
  \begin{itemize}
  \item Reading and Writing Files
  \item Files and Text Streams  
\end{itemize}
\item \textbf{Variants:}
  \begin{itemize}
  \item Variant basics
  \item Application notes
\end{itemize}

\end{itemize}

\end{slide}

%----------------------------------------------------------------------

%%%%%%%%%%%%%%%%%%%%%%%%%%%%%%%%%%%%%%%%%%%%%%%%%%%%%%%%%%%%%%%%%%%%%%%%%%
%
% Copyright (c) 2008-2011, Nokia Corporation and/or its subsidiary(-ies).
% All rights reserved.
%
% This work, unless otherwise expressly stated, is licensed under a
% Creative Commons Attribution-ShareAlike 2.5.
%
% The full license document is available from
% http://creativecommons.org/licenses/by-sa/2.5/legalcode .
%
%%%%%%%%%%%%%%%%%%%%%%%%%%%%%%%%%%%%%%%%%%%%%%%%%%%%%%%%%%%%%%%%%%%%%%%%%%

\subsection{String Handling}

%----------------------------------------------------------------------

\begin{slide}[fragile]{0205}\frametitle{Text Processing with \iCls{QString}}
  \xConcept{Strings}
  \label{text_processing}
  Strings can be created in a number of ways:
  \begin{itemize}
  \item Conversion constructor and assignment operators:
 \begin{cpp}
QString str("abc");
str = "def";
  \end{cpp}
 \item From a number using a static function:
 \begin{cpp}
QString n = QString::number(1234);
  \end{cpp}
 \item From a char pointer using the static functions:
 \begin{cpp}
QString text = QString::fromLatin1("Hello Qt");
QString text = QString::fromUtf8(inputText);
QString text = QString::fromLocal8Bit(cmdLineInput);
  \end{cpp}
\item From char pointer with translations:
  \begin{cpp}
QString text = tr("Hello Qt");
  \end{cpp}
\end{itemize}
\end{slide}

%----------------------------------------------------------------------
\begin{slide}[fragile]{0206}\frametitle{Text Processing with \iCls{QString}}
  \begin{itemize}
    \item \texttt{operator+} and \texttt{operator+=}    
 \begin{cpp}    
QString str = str1 + str2;
fileName += ".txt";
  \end{cpp}\medskip
 \item \hClsFn{QString}{simplified} // removes duplicate whitespace
 \item\hClsFn{QString}{left}, \hClsFn{QString}{mid},
   \hClsFn{QString}{right} // part of a string\medskip
  \item \hClsFn{QString}{leftJustified},
    \hClsFn{QString}{rightJustified} // padded version
 \begin{cpp}
QString s = "apple";
QString t = s.leftJustified(8, '.'); // t == "apple..."
  \end{cpp}
  \end{itemize}
\end{slide}

%----------------------------------------------------------------------
\begin{slide}[fragile]{0207}\frametitle{Text Processing with \iCls{QString}}
  Data can be extracted from strings.
  \vspace*{1em}
  
  \begin{itemize}
  \item Numbers:
 \begin{cpp}
QString text = ...; 
int value = text.toInt();
float value = text.toFloat();
  \end{cpp}
 \item Strings:
 \begin{cpp}
QString text = ...;
QByteArray bytes = text.toLatin1();
QByteArray bytes = text.toUtf8();
QByteArray bytes = text.toLocal8Bit();
  \end{cpp}
  \end{itemize}
\end{slide}

% Slide note: There are many more functions available for conversions to types.

% ----------------------------------------------------------------------
\begin{slide}[fragile]{2079}
\frametitle{Formatted output with QString::arg()}
\begin{itemize}
\item[]
\begin{cpp}
int i = ...;
int total = ...;
QString fileName = ...;
QString status = tr("Processing file %1 of %2: %3")
                .arg(i).arg(total).arg(fileName);
double d = 12.34;
QString str = QString::fromLatin1("delta: %1").arg(d,0,'E',3);
// str == "delta: 1.234E+01"               
\end{cpp}\medskip

\item Safer: \hClsFnPar{QString}{arg}{QString,...,QString} (``multi-arg()'').
  \begin{itemize}
  \item But: only works with \iCls{QString} arguments.
  \end{itemize}
\end{itemize}
\end{slide}

%----------------------------------------------------------------------
\begin{slide}[fragile]{1255}\frametitle{Text Processing with \iCls{QString}}
  \begin{itemize}
  \item Obtaining raw character data from a \iCls{QByteArray}:
  
  \begin{cpp}
char *str = bytes.data();
const char *str = bytes.constData();
  \end{cpp}
  \end{itemize}\bigskip

WARNING:
  \begin{itemize}  
  \item Character data is only valid for the lifetime of the byte array.
  \item Calling a non-\texttt{const} member of \texttt{bytes} also invalidates \texttt{ptr}.
  \item Either copy the character data or keep a copy of the byte array.
  \end{itemize}
\end{slide}

% ----------------------------------------------------------------------
\begin{slide}{0208}\frametitle{Text Processing with \iCls{QString}}
    \begin{itemize}
      \item \hClsFn{QString}{length} 
      \item \hClsFn{QString}{endsWith} and \hClsFn{QString}{startsWith} 
      \item \hClsFn{QString}{contains}, \hClsFn{QString}{count}
      \item \hClsFn{QString}{indexOf} and \hClsFn{QString}{lastIndexOf} 
    \end{itemize}
    Expression can be characters, strings, or regular expressions
\end{slide}

%----------------------------------------------------------------------
\begin{slide}{0209}\frametitle{Text Processing with \iCls{QStringList}}
\xConcept{List of strings}
\begin{itemize}
\item \iClsFn{QString}{split}, \iClsFn{QStringList}{join}
\item \iClsFn{QStringList}{replaceInStrings} 
\item \iClsFn{QStringList}{filter} 
\end{itemize}
\end{slide}

%----------------------------------------------------------------------
\begin{slide}[fragile]{0211}\frametitle{Working with \iConcept{Regular Expressions}}\label{qregexp}
\begin{itemize}
  \item \iCls{QRegExp} supports
  \begin{itemize}
    \item Regular expression matching
    \item Wildcard matching
  \end{itemize}
  \item \iCls{QString} \hClsFnPar{QRegExp}{cap}{int}\\
    \iCls{QStringList} \hClsFn{QRegExp}{capturedTexts}\medskip
 \begin{cpp}
QRegExp rx("^\\d\\d?$");    // match integers 0 to 99
rx.indexIn("123");          // returns -1 (no match)
rx.indexIn("-6");           // returns -1 (no match)
rx.indexIn("6");            // returns 0 (matched as position 0)
    \end{cpp}                      
  \end{itemize}                 
  \doc{qregexp.html}{QRegExp}
\end{slide}


%----------------------------------------------------------------------
%----------------------------------------------------------------------
% Local Variables: 
% mode: latex 
% TeX-master: "course"
% End: 

%%%%%%%%%%%%%%%%%%%%%%%%%%%%%%%%%%%%%%%%%%%%%%%%%%%%%%%%%%%%%%%%%%%%%%%%%%
%
% Copyright (c) 2008-2011, Nokia Corporation and/or its subsidiary(-ies).
% All rights reserved.
%
% This work, unless otherwise expressly stated, is licensed under a
% Creative Commons Attribution-ShareAlike 2.5.
%
% The full license document is available from
% http://creativecommons.org/licenses/by-sa/2.5/legalcode .
%
%%%%%%%%%%%%%%%%%%%%%%%%%%%%%%%%%%%%%%%%%%%%%%%%%%%%%%%%%%%%%%%%%%%%%%%%%%

\subsection{Container Classes}


%----------------------------------------------------------------------
\begin{slide}{0282}\frametitle{Container Classes}
  General purpose template-based container classes
  \begin{itemize}
  \item \lstinline!QList<QString>! - \textit{Sequence Container}
    \begin{itemize}
    \item Other: QLinkedList, QStack , QQueue ...
    \end{itemize}
  \item \lstinline!QMap<int, QString>! - \textit{Associative Container}
    \begin{itemize}
    \item Other: QHash, QSet, QMultiMap, QMultiHash
    \end{itemize}
  \end{itemize}
  \medskip  Qt's Container Classes compared to STL
  \begin{itemize}
  \item Lighter, safer, and easier to use than STL containers
  \item If you prefer STL, feel free to continue using it.
  \item Methods exist that convert between Qt and STL
    \begin{itemize}
    \item e.g. you need to pass \texttt{std::list} to a Qt method
    \end{itemize}
  \end{itemize}
\end{slide}

%----------------------------------------------------------------------
\begin{slide}[fragile]{1997}\frametitle{Using Containers}
  \begin{itemize}
  \item Using \iCls{QList}
    \begin{cpp}
QList<QString> list;
list << "one" << "two" << "three";
QString item1 = list[1]; // "two"
for(int i=0; i<list.count(); i++) { 
  const QString &item2 = list.at(i);
}
int index = list.indexOf("two"); // returns 1
    \end{cpp}
  \item Using \iCls{QMap}
    \begin{cpp}
QMap<QString, int> map;
map["Norway"] = 5; map["Italy"] = 48;
int value = map["France"]; // inserts key if not exists
if(map.contains("Norway")) {
  int value2 = map.value("Norway"); // recommended lookup
}
  \end{cpp}

  \end{itemize}
\end{slide}

%----------------------------------------------------------------------
\begin{slide}[fragile]{0286}\frametitle{Algorithm Complexity}
  \begin{block}{Concern}
    How fast is a function when number of items grow  
  \end{block}
  \begin{itemize}
  \item Sequential Container
  \item[] \begin{tabular}{l|c|c|c|c|}
        & \textbf{Lookup} & \textbf{Insert} & \textbf{Append} & \textbf{Prepend}\\\hline
        \textbf{QList} & O(1) & O(n) & O(1) & O(1)\\\hline
        \textbf{QVector} & O(1) & O(n) & O(1) & O(n)\\\hline
        \textbf{QLinkedList} & O(n) & O(1) & O(1) & O(1)\\\hline
      \end{tabular}
      \medskip
  \item Associative Container
      \medskip
  \item[] \begin{tabular}{l|l|l|}
        & \textbf{Lookup} & \textbf{Insert} \\\hline
        \textbf{QMap} & O(log(n)) &  O(log(n)) \\
        \textbf{QHash} & O(1) & O(1) \\\hline
      \end{tabular}
  \end{itemize}
  \vspace{-5mm}\hfill\textit{all complexities are amortized}
\end{slide}

%----------------------------------------------------------------------
\begin{slide}[fragile]{0287}\frametitle{Storing Classes in Qt Container}
\begin{itemize}
  \item Class must be an \emph{assignable data type}
  \item Class is \emph{assignable}, if:
  \item[] \begin{cpp}
class Contact {
public:
  Contact() {} // default constructor
  Contact(const Contact &other); // copy constructor
  // assignment operator
  Contact &operator=(const Contact &other);
};
\end{cpp}
\item \emph{If copy constructor or assignment operator is not provided} 
\begin{itemize}
	\item C++ will provide one (uses member copying)
\end{itemize}
\item \emph{If no constructors provided}
\begin{itemize}
  \item Empty default constructor provided by C++
\end{itemize}
\end{itemize}
\end{slide}


%----------------------------------------------------------------------
\begin{slide}[fragile]{0288}\frametitle{Requirements on Container Keys}
\begin{itemize}
  \item Type \texttt{K} as key for \iCls{QMap}:
    \begin{itemize}
    \item \texttt{bool K::operator$<$( const K\& )} or\\
      \texttt{bool operator$<$( const K\&, const K\& )}
        \item[] \begin{cpp}
bool Contact::operator<(const Contact& c);
bool operator<(const Contact& c1, const Contact& c2);
      \end{cpp}
      \doc{qmap.html}{QMap}
    \end{itemize} \medskip

  \item Type \texttt{K} as key for \iCls{QHash} or \iCls{QSet}:
  \begin{itemize}
  \item \texttt{bool K::operator==( const K\& )} or\\
    \texttt{bool operator==( const K\&, const K\& )}
  \item \texttt{uint qHash( const K\& )}\\
  \doc{qhash.html}{QHash}
  \end{itemize}
\end{itemize}
\end{slide}

%----------------------------------------------------------------------
\begin{slide}{0289}\frametitle{\iConcept{Iterators}}
\begin{itemize}
  \item Allow reading a container's content sequentially
 \item \textbf{Java-style iterators}: simple and easy to use
  \begin{itemize}
    \item \texttt{QListIterator$<$...$>$} for read
    \item \texttt{QMutableListIterator$<$...$>$} for read-write
 \end{itemize}
  \item \textbf{STL-style iterators} slightly more efficient
  \begin{itemize}
    \item \texttt{QList::const\_iterator} for read
    \item \iClsFn{QList}{iterator} for read-write
 \end{itemize}
    \item Same works for QSet, QMap, QHash, ...
\end{itemize}
\end{slide}

%----------------------------------------------------------------------
\begin{slide}[fragile]{0291}\frametitle{\iConcept{Iterators} {Java style}}\label{java-style-iterators}
\xExample{Iterators!Java style}
\flushedImage{coretypes/images/javaiterators1}
\begin{itemize}
  \item Example \xConcept{Iterators}{QList} iterator
  \begin{cpp}
QList<QString> list;
list << "A" << "B" << "C" << "D";
QListIterator<QString> it(list);
  \end{cpp}
  \item Forward iteration
  \begin{cpp}
while(it.hasNext()) {
  qDebug() << it.next();    // A B C D
}  
  \end{cpp}
  \item Backward iteration
  \begin{cpp}
it.toBack(); // position after the last item
while(it.hasPrevious()) {
  qDebug() << it.previous(); // D C B A
}  
  \end{cpp}  
  \item[] \doc{qlistiterator.html\#details}{QListIterator}
\end{itemize}  
\end{slide}

%----------------------------------------------------------------------
\begin{slide}[fragile]{0293}\frametitle{Modifying During Iteration}
\xConcept{Iterators!Modifying}
\begin{itemize}
  \item Use \emph{mutable} versions of the iterators
  \begin{itemize}
	  \item e.g. \iCls{QMutableListIterator}.
    \item[] \begin{cpp}
QList<int> list; 
list << 1 << 2 << 3 << 4;
QMutableListIterator<int> i(list);
while (i.hasNext()) {
  if (i.next() % 2 != 0)
    i.remove();
}
// list now 2, 4  
    \end{cpp}
  \end{itemize}
  \item \hClsFn{QMutableListIterator}{remove} and \hClsFn{QMutableListIterator}{setValue}
  \begin{itemize}
    \item  Operate on items just jumped over using next()/previous()
  \end{itemize}
  \item \hClsFn{QMutableListIterator}{insert} 
  \begin{itemize}
    \item Inserts item at current position in sequence \\ 
    \item \hClsFn{QMutableListIterator}{previous} reveals just inserted item
  \end{itemize}
\end{itemize}    
\end{slide}

%----------------------------------------------------------------------
\begin{slide}[fragile]{0294}\frametitle{Iterating Over QMap and QHash}
\xConcept{Iterators!QMap}
\begin{itemize}
\item \hClsFn{QMapIterator}{next} and \hClsFn{QMapIterator}{previous} 
  \begin{itemize}  
  	\item Return Item class  with \texttt{key()} and \texttt{value()}
  \end{itemize}  
\item Alternatively use \hClsFn{QMapIterator}{key} and \hClsFn{QMapIterator}{value} from iterator
\end{itemize} 
\xExample{Iterators!QMap}                                                               
\begin{cpp}
QMap<QString, QString> map;
map["Paris"] = "France";
map["Guatemala City"] = "Guatemala";
map["Mexico City"] = "Mexico";
map["Moscow"] = "Russia";

QMutableMapIterator<QString, QString> i(map);
while (i.hasNext()) {
  if (i.next().key().endsWith("City"))
    i.remove();
}          
// map now "Paris", "Moscow"
\end{cpp}
\demo{core-types/ex-container}
\end{slide}

%----------------------------------------------------------------------
\begin{slide}[fragile]{0297}\frametitle{\iConcept{STL-style} Iterators}
\xExample{Iterators!STL style}
\flushedImage{coretypes/images/stliterators1}
\vspace*{-4mm}
\begin{itemize}
  \item Iterator points at item
  \item Example \xConcept{Iterators}{QList} iterator
\begin{cpp}
QList<QString> list;
list << "A" << "B" << "C" << "D";
QList<QString>::iterator i;
\end{cpp}
\item Forward mutable iteration
\begin{cpp}
for (i = list.begin(); i != list.end(); ++i) {
    *i = (*i).toLower();
}
\end{cpp}
\item Backward mutable iteration
\begin{cpp}
i = list.end();
while (i != list.begin()) {
    --i;
    *i = (*i).toLower();
}
\end{cpp}  
\item \lstinline{QList<QString>::const_iterator} for read-only
\end{itemize}
\end{slide}


%----------------------------------------------------------------------
\begin{slide}[fragile]{0298}\frametitle{The \texttt{foreach} Keyword}
\xConcept{foreach keyword}                                                                     
\xExample{Iterators!foreach}
\begin{itemize}
  \item It is a macro, feels like a keyword
  \item[] \fbox{\texttt{foreach} ( \textit{variable}, \textit{container} ) \textit{statement} }  
  \item[] \begin{cpp}
foreach (const QString& str, list) {
  if (str.isEmpty())
    break;
  qDebug() << str;
}    
  \end{cpp}
  \item \iCls{break} and \iCls{continue} as normal
  \item Modifying the container while iterating
  \begin{itemize}
  	\item results in container being copied
  	\item iteration continues in unmodified version
  \end{itemize}
  \item Not possible to modify item
  \begin{itemize}
  	\item iterator variable is a const reference.
  \end{itemize}
\end{itemize}
\end{slide}

%----------------------------------------------------------------------
\begin{slide}{0300}\frametitle{\iConcept{Algorithms}}
\begin{itemize}
  \item STL-style iterators are compatible with the STL algorithms
  \begin{itemize}
    \item Defined in the STL $<$algorithm$>$ header
  \end{itemize}
  \item Qt has own algorithms
  \begin{itemize}
    \item Defined in $<$QtAlgorithms$>$ header
  \end{itemize}
  \item \emph{If STL is available on all your supported platforms you can
      choose to use the STL algorithms}
  \begin{itemize}
    \item The collection is much larger than the one in Qt.
  \end{itemize}
\end{itemize}
\end{slide}

%----------------------------------------------------------------------
\begin{slide}[fragile]{0301}\frametitle{Algorithms}
\begin{itemize}
\item \iCls[\textbf]{qSort}(begin, end) sort items in range
\item \iCls[\textbf]{qFind}(begin, end, value) find value
\item \iCls[\textbf]{qEqual}(begin1, end1, begin2) checks two ranges
\item \iCls[\textbf]{qCopy}(begin1, end1, begin2) from one range to another
\item \iCls[\textbf]{qCount}(begin, end, value, n) occurrences of value in range
\item and more ...
\end{itemize}
\doc{qtalgorithms.html\#details}{QtAlgorithms}
\end{slide}

%----------------------------------------------------------------------
\begin{slide}[fragile]{0301}\frametitle{Algorithms}
Counting 1's in list          
\begin{cpp}
QList<int> list;
list << 1 << 2 << 3 << 1;
int count = 0;
qCount(list, 1, count); // count the 1's
qDebug() << count;      // 2  (means 2 times 1)
\end{cpp}
\begin{itemize}
  \item For parallel (ie. multi-threaded) algorithms
  \begin{itemize}  
  	\item \doc{threads.html\#qtconcurrent-intro}{\iCls{QtConcurrent}}
  \end{itemize}
\end{itemize}    
\end{slide}

%----------------------------------------------------------------------
\begin{slide}[fragile]{0303}\frametitle{Algorithms Examples}
\xExample{Algorithms!qCopy}
\begin{itemize}
  \item Copy list to vector example
  \item[] \begin{cpp}
QList<QString> list;
list << "one" << "two" << "three";
QVector<QString> vector(3);
qCopy(list.begin(), list.end(), vector.begin());
// vector: [ "one", "two", "three" ]
  \end{cpp}      
  \item Case insensitive sort example
  \item[] \begin{cpp}
bool lessThan(const QString& s1, const QString& s2) {
    return s1.toLower() < s2.toLower();
}
// ...
QList<QString> list;
list << "AlPha" << "beTA" << "gamma" << "DELTA";
qSort(list.begin(), list.end(), lessThan);
// list: [ "AlPha", "beTA", "DELTA", "gamma" ]
\end{cpp}
  
\end{itemize}
\end{slide}

%----------------------------------------------------------------------
\begin{slide}[fragile]{0305}\frametitle{\iConcept{Implicitly Sharing and Containers}}
\xConcept{Reference counting} 
\begin{block}{Implicit Sharing}
If an object is copied, then its data is copied \textit{only when} the data of one of
the objects is changed
\end{block}
\begin{itemize}
  \item Shared class has a pointer to shared data block
  \begin{itemize}
    \item Shared data block = reference counter and actual data
 \end{itemize}  
  \item Assignment is a shallow copy
  \item Changing results into deep copy (detach)
\begin{cpp}
QList<int> l1, l2; l1 << 1 << 2;
l2 = l1; // shallow-copy: l2 shares date with l1
l2 << 3; // deep-copy: change triggers detach from l1
\end{cpp}  
\end{itemize}
\textit{Important to remember when inserting items into a container, or when returning a container.}
\end{slide}

%%%%%%%%%%%%%%%%%%%%%%%%%%%%%%%%%%%%%%%%%%%%%%%%%%%%%%%%%%%%%%%%%%%%%%%%%%
%
% Copyright (c) 2008-2011, Nokia Corporation and/or its subsidiary(-ies).
% All rights reserved.
%
% This work, unless otherwise expressly stated, is licensed under a
% Creative Commons Attribution-ShareAlike 2.5.
%
% The full license document is available from
% http://creativecommons.org/licenses/by-sa/2.5/legalcode .
%
%%%%%%%%%%%%%%%%%%%%%%%%%%%%%%%%%%%%%%%%%%%%%%%%%%%%%%%%%%%%%%%%%%%%%%%%%%

\subsection{File Handling}

%----------------------------------------------------------------------
\begin{slide}[fragile]{0242}\frametitle{Working With Files} \label{working_with_files}
\xConcept{Reading files}\xConcept{Writing files}\xConcept{I/O}
\begin{block}{Rule}
For portable file access do not use the native
functions like \texttt{open()} or \texttt{CreateFile()}, but Qt classes
instead.  
\end{block}
File Handling 
\begin{itemize}
\item \iCls{QFile}
  \begin{itemize}    
  \item Interface for reading from and writing to files
  \item Inherits QIODevice \emph{(base interface class of all I/O devices)}
  \end{itemize}    
\item \iCls{QTextStream} 
  \begin{itemize}    
  \item Interface for reading and writing text
  \end{itemize}
\item \iCls{QDataStream} 
  \begin{itemize}    
  \item Serialization of binary data 
  \end{itemize}
\end{itemize}
Additional
\begin{itemize}
\item \iCls{QFileInfo} - System-independent file information
\item \iCls{QDir} - Access to directory structures and their contents
\end{itemize}
\end{slide}
%----------------------------------------------------------------------

\begin{slide}[fragile]{0243}
\frametitle{Reading/Writing a File}\label{streaming_to_files}
\begin{itemize}
  \item Writing with text stream
    \begin{cpp}
QFile file("myfile.txt");
if (file.open(QIODevice::WriteOnly)) {
  QTextStream stream(&file);
  stream << "HelloWorld " << 4711;
  file.close();
}
    \end{cpp}     
  \item Reading with text stream
    \begin{cpp}
{
  QString text; int value;
  QFile file("myfile.txt");
  if (file.open(QIODevice::ReadOnly)) {
    QTextStream stream(&file);
    stream >> text >> value;
  }
} // file closed automatically
    \end{cpp}    
  \item \demo{coretypes/ex-qtextstream}
  \end{itemize}
\end{slide}

%----------------------------------------------------------------------
\begin{slide}{0245}\frametitle{Text and Encodings}
\xConcept{Streaming}
\begin{itemize}
  \item Using \iCls{QTextStream}
  \begin{itemize}
    \item Be aware of encoding
    \item Need to call \iClsFnPar{QTextStream}{setCodec}{codec}
  \end{itemize}
  \item Alternative ways of reading a file
    \begin{itemize}
    \item \iClsFn{QFile}{readAll} \emph{returns \texttt{QByteArray} }
    \item \iClsFnPar{QFile}{readLine}{maxlen} \emph{returns \texttt{QByteArray} }
    \item \iCls{QTextStream::readAll()} \emph{returns \texttt{QString} }
    \item \iClsFnPar{QTextStream}{readLine}{maxlen} \emph{returns \texttt{QString} }
    \end{itemize}
  \end{itemize}
\end{slide}

%----------------------------------------------------------------------
\begin{slide}[fragile]{0246}\frametitle{Reading/Writing Data - \texttt{QDataStream}}
\begin{itemize}  
  \item \iCls{Class QDataStream}  
  \begin{itemize}
    \item Alternative to \iCls{QTextStream}      
    \item Adds extra information about the data
    \item Portable between hardware architectures and operating systems
    \item Not human-readable.
  \end{itemize}
  \begin{cpp}
QFile file("file.dat");
if (file.open(QIODevice::WriteOnly)) {
  QDataStream out(&file);
  out << "Blue"; // string
  out << QColor(Qt::blue); // as QColor
}
  \end{cpp}  
  \item QDataStream can serialize many Qt classes
    \begin{itemize}    
    \item Not the case with \iCls{QTextStream}
    \end{itemize}    
  \item Common for \iCls{QTextStream} \& \iCls{QDataStream}
    \begin{itemize}
    	\item Both can write onto memory or sockets (i.e. any QIODevice)
    \end{itemize}
  \end{itemize}         
\end{slide}

%----------------------------------------------------------------------

\begin{slide}{0247}\frametitle{File Convenient Methods}
  \begin{itemize}
  \item Media methods: \texttt{load(fileName)}, \texttt{save(fileName)}
  \begin{itemize}
  \item for \iCls{QPixmap}, \iCls{QImage}, \iCls{QPicture}, \iCls{QIcon}
  \end{itemize}
  \item \iCls{QFileDialog}
  \begin{itemize}
    \item \iClsFn{QFileDialog}{getExistingDirectory}
    \item \iClsFn{QFileDialog}{getOpenFileName}
    \item \iClsFn{QFileDialog}{getSaveFileName}
  \end{itemize}
  \item \iClsFnPar{QDesktopServices}{storageLocation}{type}
  \begin{itemize}
    \item returns default system directory where files of type belong
  \end{itemize}
  \item File operations
  \begin{itemize}
    \item \iClsFnPar{QFile}{exists}{fileName}
    \item \iClsFnPar{QFile}{rename}{oldName, newName}
    \item \iClsFnPar{QFile}{copy}{oldName, newName}
    \item \iClsFnPar{QFile}{remove}{fileName}
  \end{itemize}          
  \item Directory Information
  \begin{itemize}
    \item \iClsFnPar{QDir}{tempPath}{}
    \item \iClsFnPar{QDir}{home}{}
    \item \iClsFnPar{QDir}{drives}{}
  \end{itemize}
  \end{itemize}
\end{slide}

%%%%%%%%%%%%%%%%%%%%%%%%%%%%%%%%%%%%%%%%%%%%%%%%%%%%%%%%%%%%%%%%%%%%%%%%%%
%
% Copyright (c) 2008-2011, Nokia Corporation and/or its subsidiary(-ies).
% All rights reserved.
%
% This work, unless otherwise expressly stated, is licensed under a
% Creative Commons Attribution-ShareAlike 2.5.
%
% The full license document is available from
% http://creativecommons.org/licenses/by-sa/2.5/legalcode .
%
%%%%%%%%%%%%%%%%%%%%%%%%%%%%%%%%%%%%%%%%%%%%%%%%%%%%%%%%%%%%%%%%%%%%%%%%%%

\subsubsection{Variants}

%----------------------------------------------------------------------
\begin{slide}[fragile]{0238}\frametitle{QVariant}
\begin{itemize}
  \item \iCls{QVariant} 
  \begin{itemize}
    \item Union for common Qt "value types" (copyable, assignable)
    \item Supports implicit sharing (fast copying)
    \item Supports user types
  \end{itemize}\medskip
\item Use cases:
\end{itemize}
\begin{cpp}
QVariant property(const char* name) const;
void setProperty(const char* name, const QVariant &value);
\end{cpp}\medskip

\begin{cpp}
class QAbstractItemModel {
  virtual QVariant data( const QModelIndex& index, int role );
  ...
}
\end{cpp}
\end{slide}

%----------------------------------------------------------------------
\begin{slide}[fragile]{1020}\frametitle{QVariant}
\begin{itemize}
  \item For \texttt{QtCore} types  
  \item[] \begin{cpp}
QVariant variant(42);
int value = variant.toInt(); // read back
qDebug() << variant.typeName(); // int
\end{cpp}\vspace*{3mm}
\item For non-core and custom types:
  \item[] \begin{cpp}
QVariant variant = QVariant::fromValue(QColor(Qt::red));
QColor color = variant.value<QColor>(); // read back
qDebug() << variant.typeName(); // "QColor"
\end{cpp}\vspace*{3mm}
\item[] \doc{qvariant.html\#details}{QVariant}
\end{itemize}
\end{slide}

%----------------------------------------------------------------------
\begin{slide}[fragile]{0240}
\frametitle{Custom data types in variants}
\begin{itemize}
\item[]
\begin{cpp}
#include <QMetaType>

class Contact
{
  public:
    void setName(const QString & name);
    QString name() const;
  ...
};

Q_DECLARE_METATYPE(Contact);
\end{cpp}\medskip
\item Type must support default construction, copy and assignment.\medskip
\item Q\_DECLARE\_METATYPE shoud after class definition in header file.

\medskip
\doc{qmetatype.html\#Q_DECLARE_METATYPE}{Q\_DECLARE\_METATYPE}
\end{itemize}
\end{slide}

% ----------------------------------------------------------------------
\begin{slide}[fragile]{1254}
\frametitle{Custom Types and QVariant}

\begin{itemize}
\item[]
\begin{cpp}
#include "Contact.h"
#include <QDebug>
#include <QVariant>

int main(int argc, char* argv[])
{
    Contact contact;
    contact.setName("Peter");

    const QVariant variant = QVariant::fromValue(contact);

    const Contact otherContact = variant.value<Contact>();
    qDebug() << otherContact.name(); // "Peter"
    qDebug() << variant.typeName();  // prints "Contact"

    return 0;
}
\end{cpp}
\demo{coretypes/ex\_custom\_types}
\end{itemize}
\end{slide}

% ----------------------------------------------------------------------
\begin{slide}[fragile]{1253}
\frametitle{qRegisterMetaType}
\begin{itemize}
\item[]
\begin{cpp}
int main(int argc, char* argv[])
{
    // Register string typename:
    const int typeId = qRegisterMetaType<Contact>();

    Contact contact;
    contact.setName("Peter");

    // Create copy of object in a generic way
    void *object = QMetaType::construct(typeId, &contact);

    Contact *otherContact = reinterpret_cast<Contact*>(object);
    qDebug() << otherContact->name();

    return 0;
}
\end{cpp}
\doc{qmetatype.html\#qRegisterMetaType-2}{qRegisterMetaType}
\doc{qmetatype.html\#construct}{construct}
\end{itemize}
\end{slide}


% ----------------------------------------------------------------------
\subsubsection{Properties}
\begin{slide}[fragile]{1021}
\frametitle{Properties}
\begin{itemize}
\item Qt Quick example\\[2mm]
\begin{EXCLUDE}
Rectangle {
    objectName: "myRect"
    height: 100
    ...
}
\end{EXCLUDE}
\begin{qml}
\qtt{\qc{class}{Rectangle}~\{}\\
\qtt{~~~~\qc{type}{objectName}:~\qc{string}{"myRect"}}\\
\qtt{~~~~\qc{type}{height}:~\qc{number}{100}}\\
\qtt{~~~~...}\\
\qtt{\}}\\
\end{qml}\medskip

\item Direct access (Broken, due to private headers)
  \begin{cpp}
QQuickRectangle* rectangle
  = root->findChild<QQuickRectangle*>("myRect");
int height = rectangle->height();
  \end{cpp}\medskip

\item Generic property access:\\
\begin{cpp}
QObject* rectangle = root->findChild<QObject*>("myRect");
int height = rectangle->property("height").value<int>();
\end{cpp}
\end{itemize}
\hfill\tiny\textcolor{red}{Using findChild is almost always a bad idea!}
\end{slide}

% ----------------------------------------------------------------------
\begin{slide}[fragile]{1422}
\frametitle{Providing properties from QObject}
\begin{cpp}
class Customer : public QObject
{
    Q_OBJECT

    Q_PROPERTY(QString id READ getId WRITE setId NOTIFY idChanged);

  public:
     QString getId() const;
     void setId(const QString& id);
  
  signals:
     void idChanged();

  ...
};
\end{cpp}
\end{slide}

% ----------------------------------------------------------------------
\begin{slide}[fragile]{1022}
\frametitle{Enum properties}
\begin{cpp}
class Customer : public QObject
{
    Q_OBJECT

    Q_PROPERTY(CustomerType type READ getType WRITE setType 
               NOTIFY typeChanged);

  public:
    enum CustomerType {
      Corporate, Individual, Educational, Government
    };

    Q_ENUMS(CustomerType);
    
  ...
};
\end{cpp}
\end{slide}

% ----------------------------------------------------------------------
\begin{slide}[fragile]{2084}
\frametitle{Properties}
\begin{itemize}
  \item Q\_Property is a macro:
\begin{cpp}
  Q_PROPERTY( type name READ getFunction [WRITE setFunction]
  [RESET resetFunction] [NOTIFY notifySignal] [DESIGNABLE bool]
  [SCRIPTABLE bool] [STORED bool] )
\end{cpp}
  \medskip
  \item Property access methods:
\begin{cpp}
  QVariant property(const char* name) const;
  void setProperty(const char* name, const QVariant &value);
\end{cpp}
  \medskip
  \item If name is not declared as a \texttt{Q\_PROPERTY}
  \begin{itemize}
    \item -> \textbf{dynamic property}
    \item Not accessible from Qt Quick.
  \end{itemize}
  \medskip
  \item Note:
  \begin{itemize}
    \item Q\_OBJECT macro required for properties to work
    \item \texttt{QMetaObject} knows nothing about dynamic properties
  \end{itemize}
\end{itemize}
\end{slide}

% ----------------------------------------------------------------------
\begin{slide}[fragile]{1423}
\frametitle{Using Properties}
\begin{itemize}
\item \iCls{QMetaObject} support property introspection\smallskip
  \begin{cpp}
  const QMetaObject *metaObject = object->metaObject();
  const QString className = metaObject->className();
  const int propertyCount = metaObject->propertyCount();
  for ( int i=0; i<propertyCount; ++i ) {
    const QMetaProperty metaProperty = metaObject->property(i);
    const QString typeName = metaProperty.typeName()
    const QString propertyName = metaProperty.name();
    const QVariant value = object->property(metaProperty.name());
  }
  \end{cpp}
\end{itemize}
\demo{coretypes/ex-properties}
\end{slide}

