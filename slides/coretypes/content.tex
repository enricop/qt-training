%%%%%%%%%%%%%%%%%%%%%%%%%%%%%%%%%%%%%%%%%%%%%%%%%%%%%%%%%%%%%%%%%%%%%%%%%%
%
% Copyright (c) 2008-2011, Nokia Corporation and/or its subsidiary(-ies).
% All rights reserved.
%
% This work, unless otherwise expressly stated, is licensed under a
% Creative Commons Attribution-ShareAlike 2.5.
%
% The full license document is available from
% http://creativecommons.org/licenses/by-sa/2.5/legalcode .
%
%%%%%%%%%%%%%%%%%%%%%%%%%%%%%%%%%%%%%%%%%%%%%%%%%%%%%%%%%%%%%%%%%%%%%%%%%%

\section{Core Classes}

%----------------------------------------------------------------------
\begin{slide}{1256}{Module Objectives}
\label{basic-types}

Qt provides a set of basic data types:

\begin{itemize}
    \item \textbf{String handling classes:}
      \begin{itemize}
      \item Unicode-aware string and character classes
      \item Used throughout the Qt API
      \item Regular expression engine for pattern matching
      \end{itemize}
\item \textbf{Container classes:}
  \begin{itemize}
  \item Common containers: lists, sets, maps, arrays, ...
  \item Including STL and Java-style iterators
  \end{itemize}
\item \textbf{File handling classes:}
  \begin{itemize}
  \item Reading and Writing Files
  \item Files and Text Streams  
\end{itemize}
\item \textbf{Variants:}
  \begin{itemize}
  \item Variant basics
  \item Application notes
\end{itemize}

\end{itemize}

\end{slide}

%----------------------------------------------------------------------

%%%%%%%%%%%%%%%%%%%%%%%%%%%%%%%%%%%%%%%%%%%%%%%%%%%%%%%%%%%%%%%%%%%%%%%%%%
%
% Copyright (c) 2008-2011, Nokia Corporation and/or its subsidiary(-ies).
% All rights reserved.
%
% This work, unless otherwise expressly stated, is licensed under a
% Creative Commons Attribution-ShareAlike 2.5.
%
% The full license document is available from
% http://creativecommons.org/licenses/by-sa/2.5/legalcode .
%
%%%%%%%%%%%%%%%%%%%%%%%%%%%%%%%%%%%%%%%%%%%%%%%%%%%%%%%%%%%%%%%%%%%%%%%%%%

\subsection{Qt's Object Model}

\subsubsection{QObject}
% ----------------------------------------------------------------------
\begin{slide}{1244}
  \frametitle{Qt's C++ Object Model - \iCls{QObject}}
  \begin{itemize}
  \item \iCls{QObject} is the heart of Qt's object model
  \item Include these features:
    \begin{itemize}
    \item Memory management
    \item Object properties
    \item Signals and slots
    \item Event handling
   \end{itemize}
  \item \iCls{QObject} has no visual representation

 \end{itemize}
\end{slide}


% ----------------------------------------------------------------------
\begin{slide}{0550} \label{parent_child} \frametitle{Object Tree}
  \xConcept{Parent/child relationship}
\begin{itemize}
  \item \iCls{QObjects} organize themselves in object trees
    \begin{itemize}
    \item Based on parent-child relationship
    \end{itemize}
  \flushedImage{ooo/reference}
  \item \texttt{QObject(QObject *parent = 0)}
    \begin{itemize}
    \item Parent adds object to list of children
    \item Parent owns children
    \end{itemize}
  \item Widget Centric
    \begin{itemize}
    \item Used intensively with QtWidget
    \item Less so when using Qt/C++ from QML
    \end{itemize}
 \end{itemize}
 \medskip
  \textit{Note: Parent-child relationship is NOT inheritance}
\end{slide}


% ----------------------------------------------------------------------
\begin{slide}[fragile]{0552}\frametitle{Creating Objects}
  \xConcept{QObject!allocating}
 \begin{itemize}

 \item \textbf{On Heap} - \iCls{QObject} with parent:
   \begin{itemize}
   \begin{cpp}
QLabel *label = new QLabel("Some Text", parent);
    \end{cpp}
\item \texttt{QLayout::addWidget()} and \texttt{QWidget::setLayout()} reparent children automatically
    \end{itemize}

    \item \textbf{On Stack} - \texttt{QObject} without parent:
    \begin{itemize}
    \item \texttt{QFile}, usually local to a function
    \item \texttt{QApplication} (local to \texttt{main()})
    \item Top level \texttt{QWidget}s: \texttt{QMainWindow}
    \end{itemize}
    \medskip
\item \textbf{On Stack} - "value" types  \doc{qvariant.html\#Type-enum}{\texttt{QVariant::Type}}
    \begin{itemize}    
     \begin{cpp}
QString name;
QStringList list;
QColor color;
    \end{cpp}
    \item Do not inherit QObject
    \item Passed by value everywhere
    \item Exception: QString is implicitly shared (COW strategy)
          % which is different from std::string (nowadays deep-copied for simplicity)
   \end{itemize}
\item \textbf{Stack or Heap} - \texttt{QDialog} - depending on lifetime 
  \end{itemize}
\end{slide}

% ----------------------------------------------------------------------
\subsubsection{QWidget}
\begin{slide}{1243}
  \frametitle{Qt's Widget Model - \texttt{QWidget}}
 \flushedImage{ooo/qobject-inherited}
  \begin{itemize}
 \item \textbf{Derived from \iCls{QObject}}
    \begin{itemize}
    \item Adds visual representation
    \end{itemize}
  \item \textbf{Base of user interface objects}
  \item \textbf{Receives events}
    \begin{itemize}
    \item e.g. mouse, keyboard events
    \end{itemize}
  \item \textbf{Paints itself on screen}
    \begin{itemize}
    \item Using styles
    \end{itemize}
\end{itemize}
\end{slide}

% ----------------------------------------------------------------------

\begin{slide}[fragile]{0551}\frametitle{Object Tree and QWidget}
  \begin{itemize}
  \item \textbf{\texttt{new QWidget(0)}}
    \begin{itemize}
    \item Widget with no parent $=$ "window"
    \end{itemize}
  \item \textbf{QWidget's children}
    \begin{itemize}
    \item Positioned in parent's coordinate system
    \item Clipped by parent's boundaries
    \end{itemize}
  \item \textbf{QWidget parent }
    \begin{itemize}
    \item Propagates state changes
      \begin{itemize}
      \item hides/shows children when it is hidden/shown itself
      \item enables/disables children when it is enabled/disabled itself
      \end{itemize}
    \end{itemize}
 \end{itemize}
\end{slide}

% ----------------------------------------------------------------------
\begin{slide}{0553}\frametitle{Widgets that contain other widgets}
  \xConcept{Geometry Management!Basic}
  \flushedImage{coretypes/images/widgetGrouping}
  
  \begin{itemize}
  \item Container Widget
    \begin{itemize}
    \item Aggregates other child-widgets
    \end{itemize}
  \item Use layouts for aggregation
    \begin{itemize}
    \item In this example: \iCls{QHBoxLayout} and \\ \iCls{QVBoxLayout}
    \item Note: Layouts are \emph{not} widgets
    \end{itemize}
  \item Layout Process
    \begin{itemize}
    \item Add widgets to layout
    \item Layouts may be nested
    \item Set layout on container widget
    \end{itemize}
  \end{itemize}
\end{slide}

% ----------------------------------------------------------------------
\begin{slide}[fragile]{0555}
  \frametitle{Example Container Widget}
  \begin{cpp}
// container (window) widget creation
QWidget* container = new QWidget; // top-level widget has no parent
QLabel* label = new QLabel("Note:", container);
QTextEdit* edit = new QTextEdit(container);
QPushButton* clear = new QPushButton("Clear", container);
QPushButton* save = new QPushButton("Save", container);
    \end{cpp} 
  \flushedImage{coretypes/images/widgetGrouping}
   \begin{cpp}
// widget layout
QVBoxLayout* outer = new QVBoxLayout();
outer->addWidget(label);
outer->addWidget(edit);
QHBoxLayout* inner = new QHBoxLayout();
inner->addWidget(clear);
inner->addWidget(save);
    \end{cpp} 
  \begin{cpp}
container->setLayout(outer);
outer->addLayout(inner); // nesting layouts
    \end{cpp}
  \demo{objects/ex-simplelayout}
\end{slide}



% ----------------------------------------------------------------------
\begin{slide}[fragile]{976}
  \frametitle{Questions And Answers}\label{parentChildQuestions}
  \begin{questionize}
  \item What is an object tree?
  \item Which pointers to QObjects do you need to keep around?
  \item What does it mean when a QWidget has no parent?
%   \item What is \texttt{qmake}, and when is it a good idea to use it?
  \item Allocate on heap or stack? \\ \texttt{QWidget; QStringList;
      QApplication; QString; QFile}
%   \item Name places where you can find answers about Qt problems
  \item Name some layout managers and when to use them
  \item What does it mean to nest layouts?
 \end{questionize}
\end{slide}

% ----------------------------------------------------------------------
\begin{slide}{0564}\frametitle{Lab: Your first Qt Application}
  \label{first_app}
  \xProject{Your first Qt Application}
  \flushedImage{coretypes/images/firstApplication}
  \begin{itemize}
  \item \textbf{Implement the application shown here}
    \begin{itemize}
   \item Search the widgets
      \begin{itemize}    
      \item \doc{gallery-windowsvista.html}{Qt Widget Gallery}
      \item ... and choose your os style
      \end{itemize}
    \item Layouts: \texttt{QHBoxLayout, QVBoxLayout}
      \begin{itemize}
      \item See previous slides how to use them
      \end{itemize}
    \end{itemize}
  \item \textbf{Optionally}
    \begin{itemize} 
    \item Provide a window title
    \item Add Edit, Remove buttons
      \begin{itemize}
      \item  On the right of list
      \end{itemize}
    \item Use group box to provide list caption
    \end{itemize}
 \end{itemize}
  \medskip
  \lab{objects/lab-firstapp} \newline
\end{slide}

%%%%%%%%%%%%%%%%%%%%%%%%%%%%%%%%%%%%%%%%%%%%%%%%%%%%%%%%%%%%%%%%%%%%%%%%%%
%
% Copyright (c) 2008-2011, Nokia Corporation and/or its subsidiary(-ies).
% All rights reserved.
%
% This work, unless otherwise expressly stated, is licensed under a
% Creative Commons Attribution-ShareAlike 2.5.
%
% The full license document is available from
% http://creativecommons.org/licenses/by-sa/2.5/legalcode .
%
%%%%%%%%%%%%%%%%%%%%%%%%%%%%%%%%%%%%%%%%%%%%%%%%%%%%%%%%%%%%%%%%%%%%%%%%%%

\subsubsection{Variants}

%----------------------------------------------------------------------
\begin{slide}[fragile]{0238}\frametitle{QVariant}
\begin{itemize}
  \item \iCls{QVariant} 
  \begin{itemize}
    \item Union for common Qt "value types" (copyable, assignable)
    \item Supports implicit sharing (fast copying)
    \item Supports user types
  \end{itemize}\vspace*{3mm}
  \item For \texttt{QtCore} types  
  \item[] \begin{cpp}
QVariant variant(42);
int value = variant.toInt(); // read back
qDebug() << variant.typeName(); // int
\end{cpp}\vspace*{3mm}
\item For non-core and custom types:
  \item[] \begin{cpp}
QVariant variant = QVariant::fromValue(QColor(Qt::red));
QColor color = variant.value<QColor>(); // read back
qDebug() << variant.typeName(); // "QColor"
\end{cpp}\vspace*{3mm}
\item[] \doc{qvariant.html\#details}{QVariant}
\end{itemize}
\end{slide}

%----------------------------------------------------------------------
\begin{slide}[fragile]{0240}
\frametitle{Q\_DECLARE\_METATYPE}
\begin{itemize}
\item[]
\begin{cpp}
#include <QMetaType>

class Contact
{
  public:
    void setName(const QString & name);
    QString name() const;
  ...
};

Q_DECLARE_METATYPE(Contact);
\end{cpp}\vspace*{5mm}
\item Type must support default construction, copy and assignment.
\item Should appear after class definition in header file.
\doc{qmetatype.html\#Q_DECLARE_METATYPE}{Q\_DECLARE\_METATYPE}
\end{itemize}
\end{slide}

% ----------------------------------------------------------------------
\begin{slide}[fragile]{1254}
\frametitle{Custom Types and QVariant}

\begin{itemize}
\item[]
\begin{cpp}
#include "Contact.h"
#include <QDebug>
#include <QVariant>

int main(int argc, char* argv[])
{
    Contact contact;
    contact.setName("Peter");

    const QVariant variant = QVariant::fromValue(contact);

    const Contact otherContact = variant.value<Contact>();
    qDebug() << otherContact.name(); // "Peter"
    qDebug() << variant.typeName();  // prints "Contact"

    return 0;
}
\end{cpp}
\end{itemize}
\end{slide}

% ----------------------------------------------------------------------
\begin{slide}[fragile]{1253}
\frametitle{qRegisterMetaType}
\begin{itemize}
\item[]
\begin{cpp}
int main(int argc, char* argv[])
{
    // Register string typename:
    const int typeId = qRegisterMetaType<Contact>();

    Contact contact;
    contact.setName("Peter");

    // Create copy of object in a generic way
    void *object = QMetaType::construct(typeId, &contact);

    Contact *otherContact = reinterpret_cast<Contact*>(object);
    qDebug() << otherContact->name();

    return 0;
}
\end{cpp}
\end{itemize}
\doc{qmetatype.html\#qRegisterMetaType-2}{qRegisterMetaType}
\doc{qmetatype.html\#construct}{construct}
\end{slide}

% ----------------------------------------------------------------------
\subsubsection{Properties}
\begin{frame}[fragile]
\frametitle{Properties}
\begin{cpp}
    Q_PROPERTY( type name READ getFunction [WRITE setFunction]
    [RESET resetFunction] [NOTIFY notifySignal] [DESIGNABLE bool]
    [SCRIPTABLE bool] [STORED bool] )
\end{cpp}
\begin{itemize}
\item Macro for mapping property names to getter/setter functions
    \begin{itemize}
    \item optional - emit NOTIFY signals for change notification 
    \end{itemize}
\item An indirect way to call getters/setters of \texttt{QObject}s
    \begin{itemize}
    \item \texttt{property(char*)} calls a getter, returns value in \iCls{QVariant}
    \item \texttt{setProperty(char*, QVariant)} calls a setter
    \end{itemize}
\item Q\_OBJECT macro required for properties to work

\item If name is not declared as a \texttt{Q\_PROPERTY}
    \begin{itemize}
    \item store as \textbf{dynamic property} in \texttt{QObject}
    \item using \iCls{QVariantMap}  
    \end{itemize}
\item Class-reflection provided by \iCls{QMetaObject}, \iCls
{QMetaProperty}
    \begin{itemize}
    \item \texttt{QMetaObject} knows nothing about dynamic properties
    \end{itemize}
\end{itemize}
\end{frame}

% ----------------------------------------------------------------------
\begin{slide}[fragile]{1422}
\frametitle{QObject with Properties Example}
\begin{cpp}
class Customer : public QObject {
    Q_OBJECT  /* macro required for moc to preprocess class */

    Q_PROPERTY( QString id READ getId WRITE setId);
    Q_PROPERTY( QString name READ getName WRITE setName);
    Q_PROPERTY( QString address READ getAddress WRITE setAddress);
    // Read-only
    Q_PROPERTY( QDate dateEstablished READ getDateEstablished );
    Q_PROPERTY( CustomerType type READ getType WRITE setType);   
  public:
    enum CustomerType
    { Corporate, Individual, Educational, Government }; /* enum type
        must be in the same class as Q_ENUMS macro. */
    Q_ENUMS( CustomerType ) ;  /* generates string-to-enum
        conversion functions - also must be in same class */
\end{cpp}
\end{slide}

% ----------------------------------------------------------------------
\begin{slide}[fragile]{1423}
\frametitle{Using Properties}
\begin{itemize}
\item Direct vs indirect setting/getting:
\end{itemize}
\begin{cpp}
Customer cust;
cust.setProperty("name", QString("qt.nokia.com"));
cust.setAddress("54B East Middlesex Turnpike");
QVariant qv = cust.property("address");
Q_ASSERT(qv.toString() == cust.address());
Q_ASSERT(cust.name() == QString("qt.nokia.com"));
\end{cpp}
\end{slide}

% ----------------------------------------------------------------------
\begin{slide}[fragile]{1424}
\frametitle{Iterating through properties}                    
\begin{cpp}
QString objToString(const QObject* obj) {
    QStringList result;
    const QMetaObject *meta = obj->metaObject(); 
    result += QString("class %1 : public %2 {")
              .arg(meta->className())
              .arg(meta->superClass()->className());
    for (int i=0; i < meta->propertyCount(); ++i) {
        const QMetaProperty qmp = meta->property(i);
        QVariant value = obj->property(qmp.name());     
        result += QString("  %1 %2 = %3;")
            .arg(qmp.typeName()).arg(qmp.name())
            .arg(value.toString()); }
    result += "};";
    return result.join("\n");
}
\end{cpp}
\demo{coretypes/ex-properties}
\end{slide}

% ----------------------------------------------------------------------
\begin{slide}[fragile]{1425}
\frametitle{Example QObject with properties}
\begin{cpp}
int main (int argc, char* argv[]) {
    QApplication app(argc, argv);
    app.setOrganizationName("nokia");
    app.setApplicationName("testproperties");
    app.setOrganizationDomain("com");
    qDebug() << objToString(&app);
}
\end{cpp}
\demo{coretypes/ex-properties}
\end{slide}

% ----------------------------------------------------------------------
\begin{slide}[fragile]{1426}
\frametitle{Iterating through properties example}
\begin{cpp}
class QApplication : public QCoreApplication {
  QString objectName = ex-properties;
  QString applicationName = testproperties;
  QString applicationVersion = ;
  QString organizationName = nokia;
  QString organizationDomain = com;
  Qt::LayoutDirection layoutDirection = 0;
  int cursorFlashTime = 1000;
  int doubleClickInterval = 400;
  int keyboardInputInterval = 400;
  int wheelScrollLines = 3;
  int startDragTime = 500;
  int startDragDistance = 4;
  bool quitOnLastWindowClosed = true;
  QString styleSheet = ;
  bool autoSipEnabled = true;
};
\end{cpp}
\demo{coretypes/ex-properties}
\end{slide}

%%%%%%%%%%%%%%%%%%%%%%%%%%%%%%%%%%%%%%%%%%%%%%%%%%%%%%%%%%%%%%%%%%%%%%%%%%
%
% Copyright (c) 2008-2011, Nokia Corporation and/or its subsidiary(-ies).
% All rights reserved.
%
% This work, unless otherwise expressly stated, is licensed under a
% Creative Commons Attribution-ShareAlike 2.5.
%
% The full license document is available from
% http://creativecommons.org/licenses/by-sa/2.5/legalcode .
%
%%%%%%%%%%%%%%%%%%%%%%%%%%%%%%%%%%%%%%%%%%%%%%%%%%%%%%%%%%%%%%%%%%%%%%%%%%

\subsection{String Handling}

%----------------------------------------------------------------------

\begin{slide}[fragile]{0205}\frametitle{Text Processing with \iCls{QString}}
  \xConcept{Strings}
  \label{text_processing}
  Strings can be created in a number of ways:
  \begin{itemize}
  \item Conversion constructor and assignment operators:
 \begin{cpp}
QString str("abc");
str = "def";
  \end{cpp}
 \item Create a numerical string using a static function:
 \begin{cpp}
QString n = QString::number(1234);
  \end{cpp}
 \item From a char pointer using the static functions:
 \begin{cpp}
QString text = QString::fromLatin1("Hello Qt");
QString text = QString::fromUtf8(inputText);
QString text = QString::fromLocal8Bit(cmdLineInput);
QString text = QStringLiteral("Literal string"); (Assumed to be UTF-8)
  \end{cpp}
\item From char pointer with translations:
  \begin{cpp}
QString text = tr("Hello Qt");
  \end{cpp}
\end{itemize}
\end{slide}

%----------------------------------------------------------------------
\begin{slide}[fragile]{0206}\frametitle{Text Processing with \iCls{QString}}
  \begin{itemize}
    \item Concatenation: \texttt{operator+} and \texttt{operator+=}    
 \begin{cpp}    
QString str = str1 + str2;
fileName += ".txt";
  \end{cpp}\medskip
 \item \hClsFn{QString}{simplified} // removes duplicate whitespace
 \item\hClsFn{QString}{left}, \hClsFn{QString}{mid},
   \hClsFn{QString}{right} // part of a string\medskip
  \item \hClsFn{QString}{leftJustified},
    \hClsFn{QString}{rightJustified} // padded version
 \begin{cpp}
QString s = "apple";
QString t = s.leftJustified(8, '.'); // t == "apple..."
  \end{cpp}
  \end{itemize}
\end{slide}

%----------------------------------------------------------------------
\begin{slide}[fragile]{0207}\frametitle{Text Processing with \iCls{QString}}
  Data can be extracted from strings.
  \vspace*{1em}
  
  \begin{itemize}
  \item Numbers:
 \begin{cpp}
QString text = ...; 
int value = text.toInt(bool *ok);
float value = text.toDouble(bool *ok);
  \end{cpp}
 \item Strings:
 \begin{cpp}
QString text = ...;
QByteArray bytes = text.toLatin1();
QByteArray bytes = text.toUtf8();
QByteArray bytes = text.toLocal8Bit();
  \end{cpp}
  \end{itemize}
\end{slide}

% Slide note: There are many more functions available for conversions to types.

% ----------------------------------------------------------------------
\begin{slide}[fragile]{2079}
\frametitle{Formatted output with QString::arg()}
\begin{itemize}
\item[]
\begin{cpp}
int i = ...;
int total = ...;
QString fileName = ...;
QString status = tr("Processing file %1 of %2: %3")
                .arg(i).arg(total).arg(fileName);
double d = 12.34;
QString str = QString::fromLatin1("delta: %1").arg(d,0,'E',3);
// str == "delta: 1.234E+01"               
\end{cpp}\medskip

\item Convenience: \hClsFnPar{QString}{arg}{QString,...,QString} (``multi-arg'').
  \begin{itemize}
  \item Only works with \iCls{QString} arguments.
  \end{itemize}
\end{itemize}
\end{slide}

%----------------------------------------------------------------------
\begin{slide}[fragile]{1255}\frametitle{Text Processing with \iCls{QString}}
  \begin{itemize}
  \item Obtaining raw character data from a \iCls{QByteArray}:
  
  \begin{cpp}
char *str = bytes.data();
const char *str = bytes.constData();
  \end{cpp}
  \end{itemize}\bigskip

WARNING:
  \begin{itemize}  
  \item Character data is only valid for the lifetime of the byte array.
  \item Calling a non-\texttt{const} member of \texttt{bytes} also invalidates \texttt{ptr}.
  \item Either copy the character data or keep a copy of the byte array.
  \end{itemize}
\end{slide}

% ----------------------------------------------------------------------
\begin{slide}{0208}\frametitle{Text Processing with \iCls{QString}}
    \begin{itemize}
      \item \hClsFn{QString}{length} 
      \item \hClsFn{QString}{endsWith} and \hClsFn{QString}{startsWith} 
      \item \hClsFn{QString}{contains}, \hClsFn{QString}{count}
      \item \hClsFn{QString}{indexOf} and \hClsFn{QString}{lastIndexOf} 
    \end{itemize}
    Expression can be characters, strings, or regular expressions
\end{slide}

%----------------------------------------------------------------------
\begin{slide}{0209}\frametitle{Text Processing with \iCls{QStringList}}
\xConcept{List of strings}
\begin{itemize}
\item \iClsFn{QString}{split}, \iClsFn{QStringList}{join}
\item \iClsFn{QStringList}{replaceInStrings} 
\item \iClsFn{QStringList}{filter} 
\end{itemize}
\end{slide}

%----------------------------------------------------------------------
\begin{slide}[fragile]\frametitle{Null Strings}
\begin{itemize}
\item \iCls{QString()} constructs a null string
\item Null strings are empty and invalid
  \begin{itemize}
  \item \iCls{QString()} is empty and invalid
  \item \iCls{QString("")} is empty but is still a valid string
  \end{itemize}
\item Common idiom to use the default constructor in comparisons
\begin{cpp}
if (filename != QString()) {
   // ... load file
}
\end{cpp}
\item Several other Qt classes have similar default constructors
  \begin{itemize}
  \item \iCls{QByteArray}, \iCls{QVariant}, \iCls{QModelIndex}
  \end{itemize}
\end{itemize}
\end{slide}

%----------------------------------------------------------------------
\begin{slide}[fragile]{0211}\frametitle{Working with \iConcept{Regular Expressions}}\label{qregexp}
\begin{itemize}
  \item \iCls{QRegExp} supports
  \begin{itemize}
    \item Regular expression matching
    \item Wildcard matching
  \end{itemize}
  \item \iCls{QString} \hClsFnPar{QRegExp}{cap}{int}\\
    \iCls{QStringList} \hClsFn{QRegExp}{capturedTexts}\medskip
 \begin{cpp}
QRegExp rx("^\\d\\d?$");    // match integers 0 to 99
rx.indexIn("123");          // returns -1 (no match)
rx.indexIn("-6");           // returns -1 (no match)
rx.indexIn("6");            // returns 0 (matched as position 0)
    \end{cpp}
  \end{itemize}                 
  \doc{qregexp.html}{QRegExp}\medskip\medskip
  \begin{itemize}
  \item \textit{There is also a new alternative class, \iCls{QRegularExpression}}
  \end{itemize}                 
\end{slide}


%----------------------------------------------------------------------
%----------------------------------------------------------------------
% Local Variables: 
% mode: latex 
% TeX-master: "course"
% End: 

%%%%%%%%%%%%%%%%%%%%%%%%%%%%%%%%%%%%%%%%%%%%%%%%%%%%%%%%%%%%%%%%%%%%%%%%%%
%
% Copyright (c) 2008-2011, Nokia Corporation and/or its subsidiary(-ies).
% All rights reserved.
%
% This work, unless otherwise expressly stated, is licensed under a
% Creative Commons Attribution-ShareAlike 2.5.
%
% The full license document is available from
% http://creativecommons.org/licenses/by-sa/2.5/legalcode .
%
%%%%%%%%%%%%%%%%%%%%%%%%%%%%%%%%%%%%%%%%%%%%%%%%%%%%%%%%%%%%%%%%%%%%%%%%%%

\subsection{Container Classes}


%----------------------------------------------------------------------
\begin{slide}{0282}\frametitle{Container Classes}
  General purpose template-based container classes
  \begin{itemize}
  \item \lstinline!QList<QString>! - \textit{Sequence Container}
    \begin{itemize}
    \item Other: QLinkedList, QStack , QQueue ...
    \end{itemize}
  \item \lstinline!QMap<int, QString>! - \textit{Associative Container}
    \begin{itemize}
    \item Other: QHash, QSet, QMultiMap, QMultiHash
    \end{itemize}
  \end{itemize}
  \medskip  Qt's Container Classes compared to STL
  \begin{itemize}
  \item Lighter, safer, and easier to use than STL containers
  \item If you prefer STL, feel free to continue using it.
  \item Methods exist that convert between Qt and STL
    \begin{itemize}
    \item e.g. you need to pass \texttt{std::list} to a Qt method
    \end{itemize}
  \end{itemize}
\end{slide}

%----------------------------------------------------------------------
\begin{slide}[fragile]{1997}\frametitle{Using Containers}
  \begin{itemize}
  \item Using \iCls{QList}
    \begin{cpp}
QList<QString> list;
list << "one" << "two" << "three";
QString item1 = list[1]; // "two"
for(int i=0; i<list.count(); i++) { 
  const QString &item2 = list.at(i);
}
int index = list.indexOf("two"); // returns 1
    \end{cpp}
  \item Using \iCls{QMap}
    \begin{cpp}
QMap<QString, int> map;
map["Norway"] = 5; map["Italy"] = 48;
int value = map["France"]; // inserts key if not exists
if(map.contains("Norway")) {
  int value2 = map.value("Norway"); // recommended lookup
}
  \end{cpp}

  \end{itemize}
\end{slide}

%----------------------------------------------------------------------
\begin{slide}[fragile]{0286}\frametitle{Algorithm Complexity}
  \begin{block}{Concern}
    How fast is a function when number of items grow  
  \end{block}
  \begin{itemize}
  \item Sequential Container
  \item[] \begin{tabular}{l|c|c|c|c|}
        & \textbf{Lookup} & \textbf{Insert} & \textbf{Append} & \textbf{Prepend}\\\hline
        \textbf{QList} & O(1) & O(n) & O(1) & O(1)\\\hline
        \textbf{QVector} & O(1) & O(n) & O(1) & O(n)\\\hline
        \textbf{QLinkedList} & O(n) & O(1) & O(1) & O(1)\\\hline
      \end{tabular}
      \medskip
  \item Associative Container
      \medskip
  \item[] \begin{tabular}{l|l|l|}
        & \textbf{Lookup} & \textbf{Insert} \\\hline
        \textbf{QMap} & O(log(n)) &  O(log(n)) \\
        \textbf{QHash} & O(1) & O(1) \\\hline
      \end{tabular}
  \end{itemize}
  \vspace{-5mm}\hfill\textit{all complexities are amortized}
\end{slide}

%----------------------------------------------------------------------
\begin{slide}[fragile]{0287}\frametitle{Storing Classes in Qt Container}
\begin{itemize}
  \item Class must be an \emph{assignable data type}
  \item Class is \emph{assignable}, if:
  \item[] \begin{cpp}
class Contact {
public:
  Contact() {} // default constructor
  Contact(const Contact &other); // copy constructor
  // assignment operator
  Contact &operator=(const Contact &other);
};
\end{cpp}
\item \emph{If copy constructor or assignment operator is not provided} 
\begin{itemize}
	\item C++ will provide one (uses member copying)
\end{itemize}
\item \emph{If no constructors provided}
\begin{itemize}
  \item Empty default constructor provided by C++
\end{itemize}
\end{itemize}
\end{slide}


%----------------------------------------------------------------------
\begin{slide}[fragile]{0288}\frametitle{Requirements on Container Keys}
\begin{itemize}
  \item Type \texttt{K} as key for \iCls{QMap}:
    \begin{itemize}
    \item \texttt{bool K::operator$<$( const K\& )} or\\
      \texttt{bool operator$<$( const K\&, const K\& )}
        \item[] \begin{cpp}
bool Contact::operator<(const Contact& c);
bool operator<(const Contact& c1, const Contact& c2);
      \end{cpp}    

      \item \doc{qmap.html}{QMap}
    \end{itemize} 
  \item Type \texttt{K} as key for \iCls{QHash} or \iCls{QSet}:
  \begin{itemize}
  \item \texttt{bool K::operator==( const K\& )} or\\
    \texttt{bool operator==( const K\&, const K\& )}
  \item \texttt{uint qHash( const K\& )}
  \item \doc{qmap.html}{QHash}
  \end{itemize}
\end{itemize}
\end{slide}

%----------------------------------------------------------------------
\begin{slide}{0289}\frametitle{\iConcept{Iterators}}
\begin{itemize}
  \item Allow reading a container's content sequentially
 \item \textbf{Java-style iterators}: simple and easy to use
  \begin{itemize}
    \item \texttt{QListIterator$<$...$>$} for read
    \item \texttt{QMutableListIterator$<$...$>$} for read-write
 \end{itemize}
  \item \textbf{STL-style iterators} slightly more efficient
  \begin{itemize}
    \item \texttt{QList::const\_iterator} for read
    \item \iClsFn{QList}{iterator} for read-write
 \end{itemize}
    \item Same works for QSet, QMap, QHash, ...
\end{itemize}
\end{slide}

%----------------------------------------------------------------------
\begin{slide}[fragile]{0291}\frametitle{\iConcept{Iterators} {Java style}}\label{java-style-iterators}
\xExample{Iterators!Java style}
\flushedImage{coretypes/images/javaiterators1}
\begin{itemize}
  \item Iterator points between items
  \item Example \xConcept{Iterators}{QList} iterator
  \begin{cpp}
QList<QString> list;
list << "A" << "B" << "C" << "D";
QListIterator<QString> it(list);
  \end{cpp}
  \item Forward iteration
  \begin{cpp}
while(it.hasNext()) {
  qDebug() << it.next();    // A B C D
}  
  \end{cpp}
  \item Backward iteration
  \begin{cpp}
it.toBack(); // position after the last item
while(it.hasPrevious()) {
  qDebug() << it.previous(); // D C B A
}  
  \end{cpp}  
  \item[] \doc{qlistiterator.html\#details}{QListIterator}
\end{itemize}  
\end{slide}

%----------------------------------------------------------------------
\begin{slide}[fragile]{0293}\frametitle{Modifying During Iteration}
\xConcept{Iterators!Modifying}
\begin{itemize}
  \item Use \emph{mutable} versions of the iterators
  \begin{itemize}
	  \item e.g. \iCls{QMutableListIterator}.
    \item[] \begin{cpp}
QList<int> list; 
list << 1 << 2 << 3 << 4;
QMutableListIterator<int> i(list);
while (i.hasNext()) {
  if (i.next() % 2 != 0)
    i.remove();
}
// list now 2, 4  
    \end{cpp}
  \end{itemize}
  \item \hClsFn{QMutableListIterator}{remove} and \hClsFn{QMutableListIterator}{setValue}
  \begin{itemize}
    \item  Operate on items just jumped over using next()/previous()
  \end{itemize}
  \item \hClsFn{QMutableListIterator}{insert} 
  \begin{itemize}
    \item Inserts item at current position in sequence \\ 
    \begin{itemize}    
    	\item \emph{Remember iterator points between item}
    \end{itemize}    
    \item \hClsFn{QMutableListIterator}{previous} reveals just inserted item
  \end{itemize}
\end{itemize}    
\end{slide}

%----------------------------------------------------------------------
\begin{slide}[fragile]{0294}\frametitle{Iterating Over QMap and QHash}
\xConcept{Iterators!QMap}
\begin{itemize}
\item \hClsFn{QMapIterator}{next} and \hClsFn{QMapIterator}{previous} 
  \begin{itemize}  
  	\item Return Item class  with \texttt{key()} and \texttt{value()}
  \end{itemize}  
\item Alternatively use \hClsFn{QMapIterator}{key} and \hClsFn{QMapIterator}{value} from iterator
\end{itemize} 
\xExample{Iterators!QMap}                                                               
\begin{cpp}
QMap<QString, QString> map;
map["Paris"] = "France";
map["Guatemala City"] = "Guatemala";
map["Mexico City"] = "Mexico";
map["Moscow"] = "Russia";

QMutableMapIterator<QString, QString> i(map);
while (i.hasNext()) {
  if (i.next().key().endsWith("City"))
    i.remove();
}          
// map now "Paris", "Moscow"
\end{cpp}
\demo{core-types/ex-container}
\end{slide}

%----------------------------------------------------------------------
\begin{slide}[fragile]{0297}\frametitle{\iConcept{STL-style} Iterators}
\xExample{Iterators!STL style}
\flushedImage{coretypes/images/stliterators1}
\begin{itemize}
  \item Iterator points at item
  \item Example \xConcept{Iterators}{QList} iterator
\begin{cpp}
QList<QString> list;
list << "A" << "B" << "C" << "D";
QList<QString>::iterator i;
\end{cpp}
\item Forward mutable iteration
\begin{cpp}
for (i = list.begin(); i != list.end(); ++i) {
    *i = (*i).toLower();
}
\end{cpp}
\item Backward mutable iteration
\begin{cpp}
i = list.end();
while (i != list.begin()) {
    --i;
    *i = (*i).toLower();
}
\end{cpp}  
\item \lstinline{QList<QString>::const_iterator} for read-only
\end{itemize}
\end{slide}


%----------------------------------------------------------------------
\begin{slide}[fragile]{0298}\frametitle{The \texttt{foreach} Keyword}
\xConcept{foreach keyword}                                                                     
\xExample{Iterators!foreach}
\begin{itemize}
  \item It is a macro, feels like a keyword
  \item[] \fbox{\texttt{foreach} ( \textit{variable}, \textit{container} ) \textit{statement} }  
  \item[] \begin{cpp}
foreach (const QString& str, list) {
  if (str.isEmpty())
    break;
  qDebug() << str;
}    
  \end{cpp}
  \item \iCls{break} and \iCls{continue} as normal
  \item Modifying the container while iterating
  \begin{itemize}
  	\item results in container being copied
  	\item iteration continues in unmodified version
  \end{itemize}
  \item Not possible to modify item
  \begin{itemize}
  	\item iterator variable is a const reference.
  \end{itemize}
\end{itemize}
\end{slide}

%----------------------------------------------------------------------
\begin{slide}{0300}\frametitle{\iConcept{Algorithms}}
\begin{itemize}
  \item STL-style iterators are compatible with the STL algorithms
  \begin{itemize}
    \item Defined in the STL $<$algorithm$>$ header
  \end{itemize}
  \item Qt has own algorithms
  \begin{itemize}
    \item Defined in $<$QtAlgorithms$>$ header
  \end{itemize}
  \item \emph{If STL is available on all your supported platforms you can
      choose to use the STL algorithms}
  \begin{itemize}
    \item The collection is much larger than the one in Qt.
  \end{itemize}
\end{itemize}
\end{slide}

%----------------------------------------------------------------------
\begin{slide}[fragile]{0301}\frametitle{Algorithms}
\begin{itemize}
\item \iCls[\textbf]{qSort}(begin, end) sort items in range
\item \iCls[\textbf]{qFind}(begin, end, value) find value
\item \iCls[\textbf]{qEqual}(begin1, end1, begin2) checks two ranges
\item \iCls[\textbf]{qCopy}(begin1, end1, begin2) from one range to another
\item \iCls[\textbf]{qCount}(begin, end, value, n) occurrences of value in range
\item and more ...
\end{itemize}
Counting 1's in list          
\begin{cpp}
QList<int> list;
list << 1 << 2 << 3 << 1;
int count = 0;
qCount(list, 1, count); // count the 1's
qDebug() << count;      // 2  (means 2 times 1)
\end{cpp}
\begin{itemize}
  \item For parallel (ie. multi-threaded) algorithms
  \begin{itemize}  
  	\item \doc{threads.html\#qtconcurrent-intro}{\iCls{QtConcurrent}}
  \end{itemize}
\end{itemize}    
  \doc{qtalgorithms.html\#details}{QtAlgorithms}
\end{slide}

%----------------------------------------------------------------------
\begin{slide}[fragile]{0303}\frametitle{Algorithms Examples}
\xExample{Algorithms!qCopy}
\begin{itemize}
  \item Copy list to vector example
  \item[] \begin{cpp}
QList<QString> list;
list << "one" << "two" << "three";
QVector<QString> vector(3);
qCopy(list.begin(), list.end(), vector.begin());
// vector: [ "one", "two", "three" ]
  \end{cpp}      
  \item Case insensitive sort example
  \item[] \begin{cpp}
bool lessThan(const QString& s1, const QString& s2) {
    return s1.toLower() < s2.toLower();
}
// ...
QList<QString> list;
list << "AlPha" << "beTA" << "gamma" << "DELTA";
qSort(list.begin(), list.end(), lessThan);
// list: [ "AlPha", "beTA", "DELTA", "gamma" ]
\end{cpp}
  
\end{itemize}
\end{slide}

%----------------------------------------------------------------------
\begin{slide}[fragile]{0305}\frametitle{\iConcept{Implicitly Sharing and Containers}}
\xConcept{Reference counting} 
\begin{block}{Implicit Sharing}
If an object is copied, then its data is copied \textit{only when} the data of one of
the objects is changed
\end{block}
\begin{itemize}
  \item Shared class has a pointer to shared data block
  \begin{itemize}
    \item Shared data block = reference counter and actual data
 \end{itemize}  
  \item Assignment is a shallow copy
  \item Changing results into deep copy (detach)
\begin{cpp}
QList<int> l1, l2; l1 << 1 << 2;
l2 = l1; // shallow-copy: l2 shares date with l1
l2 << 3; // deep-copy: change triggers detach from l1
\end{cpp}  
\end{itemize}
\textit{Important to remember when inserting items into a container, or when returning a container.}
\end{slide}

\subsection{File Handling}

%----------------------------------------------------------------------

\begin{slide}\frametitle{QIODevice}
  Base class for all I/O devices in Qt
  \begin{itemize}
    \item Files, buffers, sockets, network replies, etc.
    \item Devices may be random access or sequential
    \item \iCls{open}(), \iCls{close}(), \iCls{read}() \iCls{readLine}(), \iCls{write}(), ...
    \item emits \iCls{readyRead}() when new data is available for reading
  \end{itemize}

\end{slide}

%----------------------------------------------------------------------

\begin{slide}\frametitle{Working With Files}
  Use Qt I/O classes for portable cross-platform file access
  \begin{itemize}
    \item \iCls{QFile}
    \begin{itemize}
      \item Interface for reading from and writing to files
      \item Inherits \iCls{QIODevice}
    \end{itemize}
    \item \iCls{QTextStream}
    \begin{itemize}
      \item Interface for reading and writing text
    \end{itemize}
    \item \iCls{QDataStream}
    \begin{itemize}
      \item Serialization of binary data
    \end{itemize}  \end{itemize}
  Additional
  \begin{itemize}
    \item \iCls{QFileInfo} - System-independent file information
    \item \iCls{QDir} - Access to directory structures and their contents
  \end{itemize}
\end{slide}

%----------------------------------------------------------------------

\begin{slide}[fragile]\frametitle{Reading/Writing a File}
  \begin{itemize}
    \item Writing with text stream
    \begin{cpp}
QFile file("myfile.txt");
if (file.open(QIODevice::WriteOnly)) {
  QTextStream stream(&file);
  stream << "HelloWorld " << 4711;
  file.close();
}
    \end{cpp}
    \item Reading with text stream
    \begin{cpp}
{
  QString text; int value;
  QFile file("myfile.txt");
  if (file.open(QIODevice::ReadOnly)) {
    QTextStream stream(&file);
    stream >> text >> value;
  }
} // file closed automatically
    \end{cpp}  
  \end{itemize}
\end{slide}

%----------------------------------------------------------------------

\begin{slide}\frametitle{Text and Encodings}
  \begin{itemize}
  \item Using QTextStream
    \begin{itemize}
    \item Be aware of encoding
    \item Need to call \iClsFnPar{QTextStream}{setCodec}{codec}
    \end{itemize}
  \item Alternative ways of reading a file
    \begin{itemize}
    \item \iClsFn{QFile}{readAll} \textit{returns \iCls{QByteArray}}
    \item \iClsFnPar{QFile}{readLine}{maxlen} \textit{returns \iCls{QByteArray}}
     \item \iClsFn{QTextStream}{readAll} \textit{returns \iCls{QString}}
    \item \iClsFnPar{QTextStream}{readLine}{maxlen} \textit{returns \iCls{QString}}   
    \end{itemize}
  \end{itemize}
\end{slide}

%----------------------------------------------------------------------

\begin{slide}[fragile]\frametitle{Reading/Writing data - QDataStream}
  \begin{itemize}
  \item Class \iCls{QDataStream}
    \begin{itemize}
    \item Alternative to \iCls{QTextStream}
    \item Adds extra information about the data
    \item Portable between hardware architectures and operating systems
    \item Not human readable
    \end{itemize}
  \item \iCls{QDataStream} can serialize many Qt classes
    \begin{itemize}
    \item Not the case with \iCls{QTextStream}
    \end{itemize}
  \begin{cpp}
QFile file("file.dat");
if (file.open(QIODevice::WriteOnly)) {
  QDataStream out(&file);
  out << "Blue"; // string
  out << QColor(Qt::blue); // as color
}
  \end{cpp}
  \end{itemize}
\end{slide}

%----------------------------------------------------------------------

\begin{slide}\frametitle{File Convenient Methods}
  \begin{itemize}
  \item Media methods: \hClsFnPar{}{load}{fileName}, \hClsFnPar{}{save}{fileName}
    \begin{itemize}
    \item for \iCls{QPixmap}, \iCls{QImage}, \iCls{QIcon}
    \end{itemize}
  \item \iCls{QFileDialog}
    \begin{itemize}
    \item \iClsFn{QFileDialog}{getExistingDirectory}
    \item \iClsFn{QFileDialog}{getOpenFileName}
    \item \iClsFn{QFileDialog}{getSaveFileName}
    \end{itemize}
  \item \iClsFnPar{QDesktopServices}{storageLocation}{type}
    \begin{itemize}
    \item Returns the default system directory where files of \textit{type} belong
    \end{itemize}
  \item File Operations
    \begin{itemize}
    \item \iClsFnPar{QFile}{exists}{fileName}
    \item \iClsFnPar{QFile}{copy}{oldName, newName}
    \item \iClsFnPar{QFile}{remove}{fileName}
    \end{itemize}
  \item Directory Information
    \begin{itemize}
    \item \iClsFn{QDir}{home}
    \item \iClsFn{QDir}{drives}
    \end{itemize}
  \end{itemize}
\end{slide}

%----------------------------------------------------------------------
%----------------------------------------------------------------------
% Local Variables: 
% mode: latex 
% TeX-master: "course"
% End: 

