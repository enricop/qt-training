%%%%%%%%%%%%%%%%%%%%%%%%%%%%%%%%%%%%%%%%%%%%%%%%%%%%%%%%%%%%%%%%%%%%%%%%%%
%
% Copyright (c) 2008-2011, Nokia Corporation and/or its subsidiary(-ies).
% All rights reserved.
%
% This work, unless otherwise expressly stated, is licensed under a
% Creative Commons Attribution-ShareAlike 2.5.
%
% The full license document is available from
% http://creativecommons.org/licenses/by-sa/2.5/legalcode .
%
%%%%%%%%%%%%%%%%%%%%%%%%%%%%%%%%%%%%%%%%%%%%%%%%%%%%%%%%%%%%%%%%%%%%%%%%%%

\subsection{String Handling}

%----------------------------------------------------------------------

\begin{slide}[fragile]{0205}\frametitle{Text Processing with \iCls{QString}}
  \xConcept{Strings}
  \label{text_processing}
  Strings can be created in a number of ways:
  \begin{itemize}
  \item Conversion constructor and assignment operators:
 \begin{cpp}
QString str("abc");
str = "def";
  \end{cpp}
 \item From a number using a static function:
 \begin{cpp}
QString n = QString::number(1234);
  \end{cpp}
 \item From a char pointer using the static functions:
 \begin{cpp}
QString text = QString::fromLatin1("Hello Qt");
QString text = QString::fromUtf8(inputText);
QString text = QString::fromLocal8Bit(cmdLineInput);
  \end{cpp}
\item From char pointer with translations:
  \begin{cpp}
QString text = tr("Hello Qt");
  \end{cpp}
\end{itemize}
\end{slide}

%----------------------------------------------------------------------
\begin{slide}[fragile]{0206}\frametitle{Text Processing with \iCls{QString}}
  \begin{itemize}
    \item \texttt{operator+} and \texttt{operator+=}    
 \begin{cpp}    
QString str = str1 + str2;
fileName += ".txt";
  \end{cpp}\medskip
 \item \hClsFn{QString}{simplified} // removes duplicate whitespace
 \item\hClsFn{QString}{left}, \hClsFn{QString}{mid},
   \hClsFn{QString}{right} // part of a string\medskip
  \item \hClsFn{QString}{leftJustified},
    \hClsFn{QString}{rightJustified} // padded version
 \begin{cpp}
QString s = "apple";
QString t = s.leftJustified(8, '.'); // t == "apple..."
  \end{cpp}
  \end{itemize}
\end{slide}

%----------------------------------------------------------------------
\begin{slide}[fragile]{0207}\frametitle{Text Processing with \iCls{QString}}
  Data can be extracted from strings.
  \vspace*{1em}
  
  \begin{itemize}
  \item Numbers:
 \begin{cpp}
QString text = ...; 
int value = text.toInt();
float value = text.toFloat();
  \end{cpp}
 \item Strings:
 \begin{cpp}
QString text = ...;
QByteArray bytes = text.toLatin1();
QByteArray bytes = text.toUtf8();
QByteArray bytes = text.toLocal8Bit();
  \end{cpp}
  \end{itemize}
\end{slide}

% Slide note: There are many more functions available for conversions to types.

% ----------------------------------------------------------------------
\begin{slide}[fragile]{2079}
\frametitle{Formatted output with QString::arg()}
\begin{itemize}
\item[]
\begin{cpp}
int i = ...;
int total = ...;
QString fileName = ...;
QString status = tr("Processing file %1 of %2: %3")
                .arg(i).arg(total).arg(fileName);
double d = 12.34;
QString str = QString::fromLatin1("delta: %1").arg(d,0,'E',3);
// str == "delta: 1.234E+01"               
\end{cpp}\medskip

\item Safer: \hClsFnPar{QString}{arg}{QString,...,QString} (``multi-arg()'').
  \begin{itemize}
  \item But: only works with \iCls{QString} arguments.
  \end{itemize}
\end{itemize}
\end{slide}

%----------------------------------------------------------------------
\begin{slide}[fragile]{1255}\frametitle{Text Processing with \iCls{QString}}
  \begin{itemize}
  \item Obtaining raw character data from a \iCls{QByteArray}:
  
  \begin{cpp}
char *str = bytes.data();
const char *str = bytes.constData();
  \end{cpp}
  \end{itemize}\bigskip

WARNING:
  \begin{itemize}  
  \item Character data is only valid for the lifetime of the byte array.
  \item Calling a non-\texttt{const} member of \texttt{bytes} also invalidates \texttt{ptr}.
  \item Either copy the character data or keep a copy of the byte array.
  \end{itemize}
\end{slide}

% ----------------------------------------------------------------------
\begin{slide}{0208}\frametitle{Text Processing with \iCls{QString}}
    \begin{itemize}
      \item \hClsFn{QString}{length} 
      \item \hClsFn{QString}{endsWith} and \hClsFn{QString}{startsWith} 
      \item \hClsFn{QString}{contains}, \hClsFn{QString}{count}
      \item \hClsFn{QString}{indexOf} and \hClsFn{QString}{lastIndexOf} 
    \end{itemize}
    Expression can be characters, strings, or regular expressions
\end{slide}

%----------------------------------------------------------------------
\begin{slide}{0209}\frametitle{Text Processing with \iCls{QStringList}}
\xConcept{List of strings}
\begin{itemize}
\item \iClsFn{QString}{split}, \iClsFn{QStringList}{join}
\item \iClsFn{QStringList}{replaceInStrings} 
\item \iClsFn{QStringList}{filter} 
\end{itemize}
\end{slide}

%----------------------------------------------------------------------
\begin{slide}[fragile]{0211}\frametitle{Working with \iConcept{Regular Expressions}}\label{qregexp}
\begin{itemize}
  \item \iCls{QRegExp} supports
  \begin{itemize}
    \item Regular expression matching
    \item Wildcard matching
  \end{itemize}
  \item \iCls{QString} \hClsFnPar{QRegExp}{cap}{int}\\
    \iCls{QStringList} \hClsFn{QRegExp}{capturedTexts}\medskip
 \begin{cpp}
QRegExp rx("^\\d\\d?$");    // match integers 0 to 99
rx.indexIn("123");          // returns -1 (no match)
rx.indexIn("-6");           // returns -1 (no match)
rx.indexIn("6");            // returns 0 (matched as position 0)
    \end{cpp}                      
  \end{itemize}                 
  \doc{qregexp.html}{QRegExp}
\end{slide}


%----------------------------------------------------------------------
%----------------------------------------------------------------------
% Local Variables: 
% mode: latex 
% TeX-master: "course"
% End: 
