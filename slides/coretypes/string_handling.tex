%%%%%%%%%%%%%%%%%%%%%%%%%%%%%%%%%%%%%%%%%%%%%%%%%%%%%%%%%%%%%%%%%%%%%%%%%%
%
% Copyright (c) 2008-2011, Nokia Corporation and/or its subsidiary(-ies).
% All rights reserved.
%
% This work, unless otherwise expressly stated, is licensed under a
% Creative Commons Attribution-ShareAlike 2.5.
%
% The full license document is available from
% http://creativecommons.org/licenses/by-sa/2.5/legalcode .
%
%%%%%%%%%%%%%%%%%%%%%%%%%%%%%%%%%%%%%%%%%%%%%%%%%%%%%%%%%%%%%%%%%%%%%%%%%%

\subsection{String Handling}

%----------------------------------------------------------------------

\begin{slide}[fragile]{0205}\frametitle{Text Processing with \iCls{QString}}
  \xConcept{Strings}
  \label{text_processing}
  Strings can be created in a number of ways:
  \begin{itemize}
  \item Conversion constructor and assignment operators:
 \begin{cpp}
QString str("abc");
str = "def";
  \end{cpp}
 \item From a number using a static function:
 \begin{cpp}
QString n = QString::number(1234);
  \end{cpp}
 \item From a char pointer using the static functions:
 \begin{cpp}
QString text = QString::fromLatin1("Hello Qt");
QString text = QString::fromUtf8(inputText);
QString text = QString::fromLocal8Bit(cmdLineInput);
  \end{cpp}
\end{itemize}
\end{slide}

%----------------------------------------------------------------------
\begin{slide}[fragile]{0206}\frametitle{Text Processing with QString}
  \begin{itemize}
  \item Created from other strings
    \begin{itemize}
    \item Using \texttt{operator+} and \texttt{operator+=}    
    \end{itemize}
 \begin{cpp}    
QString str = str1 + str2;
fileName += ".txt";
  \end{cpp}
 \item \hClsFn{QString}{simplified} // removes duplicate whitespace
 \item\hClsFn{QString}{left}, \hClsFn{QString}{mid},
   \hClsFn{QString}{right} // part of a string
  \item \hClsFn{QString}{leftJustified},
    \hClsFn{QString}{rightJustified} // padded version
 \begin{cpp}
QString s = "apple";
QString t = s.leftJustified(8, '.'); // t == "apple..."
  \end{cpp}
  \end{itemize}
\end{slide}

%----------------------------------------------------------------------
\begin{slide}[fragile]{0207}\frametitle{Text Processing with QString}
  Data can be extracted from strings.
  \vspace*{1em}
  
  \begin{itemize}
  \item Numbers:
 \begin{cpp}
int value = QString::toInt();
float value = QString::toFloat();
  \end{cpp}
 \item Strings:
 \begin{cpp}
QString text = ...;
QByteArray bytes = text.toLatin1();
QByteArray bytes = text.toUtf8();
QByteArray bytes = text.toLocal8Bit();
  \end{cpp}
  \end{itemize}
\end{slide}

% Slide note: There are many more functions available for conversions to types.

% ----------------------------------------------------------------------
\begin{slide}[fragile]
\frametitle{Formatted output with QString::arg()}
\begin{itemize}
\item Instead of \texttt{printf()}, we have \texttt{arg()}.
\item Overloaded to handle many types
\item Optional width, precision, and floating point format 
\item Format codes are simply \%1 .. \%9
\end{itemize}
\begin{cpp}
     QString i;         
     QString total;     
     QString fileName;  
     QString status = QString("Processing file %1 of %2: %3")
                     .arg(i).arg(total).arg(fileName);
     double d = 12.34;
     QString str = QString("delta: %1").arg(d, 0, 'E', 3);
     // str == "delta: 1.234E+01"               
\end{cpp}
\end{slide}

%----------------------------------------------------------------------
\begin{slide}[fragile]{1255}\frametitle{Text Processing with QString}
  Obtaining raw character data from a \iCls{QByteArray}:
  
  \begin{lstlisting}
    char *str = bytes.data();
    const char *str = bytes.constData();
  \end{lstlisting}
  
  Care must be taken:
  \begin{itemize}
  \item Character data is only valid for the lifetime of the byte array.
  \item Either copy the character data or keep a copy of the byte array.
  \end{itemize}
\end{slide}

% ----------------------------------------------------------------------
\begin{slide}{0208}\frametitle{Text Processing with \iCls{QString}}
    Strings can be tested with:
    \begin{itemize}
      \item \hClsFn{QString}{length} 
      \begin{itemize}    
        \item returns the length of the string.
      \end{itemize}
      \item \hClsFn{QString}{endsWith} and \hClsFn{QString}{startsWith} 
      \begin{itemize}    
        \item test whether string starts or ends with an other string
      \end{itemize}    
      \item \hClsFn{QString}{contains} 
      \begin{itemize}
        \item returns whether the string matches a given expression
        \item \hClsFn{QString}{count} tells you how many times.
      \end{itemize}
      \item \hClsFn{QString}{indexOf} and \hClsFn{QString}{lastIndexOf} 
      \begin{itemize}    
        \item search for next matching expressions, and return its index
      \end{itemize}    
    \end{itemize}
    Expression can be characters, strings, or regular expressions
\end{slide}

%----------------------------------------------------------------------
\begin{slide}{0209}\frametitle{Text Processing with \iCls{QStringList}}
\xConcept{List of strings}
\begin{itemize}
\item \iClsFn{QString}{split} and \iClsFn{QStringList}{join}
\begin{itemize}
  \item \texttt{split} one string or \texttt{join} many strings using a substring
\end{itemize} 
\item \iClsFn{QStringList}{replaceInStrings} 
  \begin{itemize}
    \item search/replace on all strings in a list
  \end{itemize} 
  \item \iClsFn{QStringList}{filter} 
    \begin{itemize}
      \item return list of items matching given pattern or substring
    \end{itemize} 
\end{itemize}
\end{slide}

%----------------------------------------------------------------------
\begin{slide}[fragile]{0210}\frametitle{\iConcept{URLs}}
\begin{itemize}
\item \iCls{QUrl} handles URLs
\begin{cpp}
QUrl url("http://qt.nokia.com");
url.scheme()  // http
url.host();   // qt.nokia.com
\end{cpp}

\item Convenience methods 
\begin{itemize}
  \item Constructing from components (user, password, protocol, etc.)
  \item Splitting URL into components
\end{itemize}
\item \iCls{QUrlInfo} provides info about the content the URL points to
  \begin{itemize}
  	\item similar to what \iCls{QFileInfo} does for local files.
  \end{itemize}  
  \item \iCls{QDesktopServices}
  \begin{itemize}  
  	\item can \emph{open} URL, and configure how URLs are opened. 
  	\begin{cpp}
QDesktopServices::openUrl(QUrl("http://qt.nokia.com"));
  	\end{cpp}
  \end{itemize}  
\end{itemize}
\end{slide}

%----------------------------------------------------------------------
\begin{slide}[fragile]{0211}\frametitle{Working with \iConcept{Regular Expressions}}\label{qregexp}
\begin{itemize}
  \item \iCls{QRegExp} supports
  \begin{itemize}
    \item Regular expression matching
    \item Wildcard matching
  \end{itemize}
  \item Most regular expression features are supported.
  \item \iCls{QRegExp} objects can also be used for searching
    strings.
  \item \iCls{QRegExp} offers text capturing regular expressions, where
    a subexpression can be extracted using \iCls{QString} \hClsFnPar{QRegExp}{cap}{int} and
    \iCls{QStringList} \hClsFn{QRegExp}{capturedTexts}.
 \begin{cpp}
QRegExp rx("^\\d\\d?$");    // match integers 0 to 99
rx.indexIn("123");          // returns -1 (no match)
rx.indexIn("-6");           // returns -1 (no match)
rx.indexIn("6");            // returns 0 (matched as position 0)
    \end{cpp}                      
  \end{itemize}                 
  \doc{qregexp.html}{QRegExp}
\end{slide}


%----------------------------------------------------------------------
%----------------------------------------------------------------------
% Local Variables: 
% mode: latex 
% TeX-master: "course"
% End: 
