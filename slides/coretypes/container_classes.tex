%%%%%%%%%%%%%%%%%%%%%%%%%%%%%%%%%%%%%%%%%%%%%%%%%%%%%%%%%%%%%%%%%%%%%%%%%%
%
% Copyright (c) 2008-2011, Nokia Corporation and/or its subsidiary(-ies).
% All rights reserved.
%
% This work, unless otherwise expressly stated, is licensed under a
% Creative Commons Attribution-ShareAlike 2.5.
%
% The full license document is available from
% http://creativecommons.org/licenses/by-sa/2.5/legalcode .
%
%%%%%%%%%%%%%%%%%%%%%%%%%%%%%%%%%%%%%%%%%%%%%%%%%%%%%%%%%%%%%%%%%%%%%%%%%%

\subsection{Container Classes}


%----------------------------------------------------------------------
\begin{slide}{0282}\frametitle{Container Classes}
  General purpose template-based container classes
  \begin{itemize}
  \item \lstinline!QList<QString>! - \textit{Sequence Container}
    \begin{itemize}
    \item Other: QLinkedList, QStack , QQueue ...
    \end{itemize}
  \item \lstinline!QMap<int, QString>! - \textit{Associative Container}
    \begin{itemize}
    \item Other: QHash, QSet, QMultiMap, QMultiHash
    \end{itemize}
  \end{itemize}
  \medskip  Qt's Container Classes compared to STL
  \begin{itemize}
  \item Lighter, safer, and easier to use than STL containers
  \item If you prefer STL, feel free to continue using it.
  \item Methods exist that convert between Qt and STL
    \begin{itemize}
    \item e.g. you need to pass \texttt{std::list} to a Qt method
    \end{itemize}
  \end{itemize}
\end{slide}

%----------------------------------------------------------------------
\begin{slide}[fragile]{1997}\frametitle{Using Containers}
  \begin{itemize}
  \item Using \iCls{QList}
    \begin{cpp}
QList<QString> list;
list << "one" << "two" << "three";
QString item1 = list[1]; // "two"
for(int i=0; i<list.count(); i++) { 
  const QString &item2 = list.at(i);
}
int index = list.indexOf("two"); // returns 1
    \end{cpp}
  \item Using \iCls{QMap}
    \begin{cpp}
QMap<QString, int> map;
map["Norway"] = 5; map["Italy"] = 48;
int value = map["France"]; // inserts key if not exists
if(map.contains("Norway")) {
  int value2 = map.value("Norway"); // recommended lookup
}
  \end{cpp}

  \end{itemize}
\end{slide}

%----------------------------------------------------------------------
\begin{slide}[fragile]{0286}\frametitle{Algorithm Complexity}
  \begin{block}{Concern}
    How fast is a function when number of items grow  
  \end{block}
  \begin{itemize}
  \item Sequential Container
  \item[] \begin{tabular}{l|c|c|c|c|}
        & \textbf{Lookup} & \textbf{Insert} & \textbf{Append} & \textbf{Prepend}\\\hline
        \textbf{QList} & O(1) & O(n) & O(1) & O(1)\\\hline
        \textbf{QVector} & O(1) & O(n) & O(1) & O(n)\\\hline
        \textbf{QLinkedList} & O(n) & O(1) & O(1) & O(1)\\\hline
      \end{tabular}
      \medskip
  \item Associative Container
      \medskip
  \item[] \begin{tabular}{l|l|l|}
        & \textbf{Lookup} & \textbf{Insert} \\\hline
        \textbf{QMap} & O(log(n)) &  O(log(n)) \\
        \textbf{QHash} & O(1) & O(1) \\\hline
      \end{tabular}
  \end{itemize}
  \medskip
  \textit{all complexities are amortized}
\end{slide}

%----------------------------------------------------------------------
\begin{slide}[fragile]{0287}\frametitle{Storing Classes in Qt Container}
\begin{itemize}
  \item Class must be an \emph{assignable data type}
  \item Class is \emph{assignable}, if:
  \item[] \begin{cpp}
class Contact {
public:
  Contact() {} // default constructor
  Contact(const Contact &other); // copy constructor
  // assignment operator
  Contact &operator=(const Contact &other);
};
\end{cpp}
\item \emph{If copy constructor or assignment operator is not provided} 
\begin{itemize}
	\item C++ will provide one (uses member copying)
\end{itemize}
\item \emph{If no constructors provided}
\begin{itemize}
  \item Empty default constructor provided by C++
\end{itemize}
\end{itemize}
\end{slide}


%----------------------------------------------------------------------
\begin{slide}[fragile]{0288}\frametitle{Requirements on Container Keys}
\begin{itemize}
  \item Type \texttt{K} as key for \iCls{QMap}:
    \begin{itemize}
    \item \texttt{bool K::operator$<$( const K\& )} or\\
      \texttt{bool operator$<$( const K\&, const K\& )}
        \item[] \begin{cpp}
bool Contact::operator<(const Contact& c);
bool operator<(const Contact& c1, const Contact& c2);
      \end{cpp}    

      \item \doc{qmap.html}{QMap}
    \end{itemize} 
  \item Type \texttt{K} as key for \iCls{QHash} or \iCls{QSet}:
  \begin{itemize}
  \item \texttt{bool K::operator==( const K\& )} or\\
    \texttt{bool operator==( const K\&, const K\& )}
  \item \texttt{uint qHash( const K\& )}
  \item \doc{qmap.html}{QHash}
  \end{itemize}
\end{itemize}
\end{slide}

%----------------------------------------------------------------------
\begin{slide}{0289}\frametitle{\iConcept{Iterators}}
\begin{itemize}
  \item Allow reading a container's content sequentially
 \item \textbf{Java-style iterators}: simple and easy to use
  \begin{itemize}
    \item \texttt{QListIterator$<$...$>$} for read
    \item \texttt{QMutableListIterator$<$...$>$} for read-write
 \end{itemize}
  \item \textbf{STL-style iterators} slightly more efficient
  \begin{itemize}
    \item \texttt{QList::const\_iterator} for read
    \item \iClsFn{QList}{iterator} for read-write
 \end{itemize}
    \item Same works for QSet, QMap, QHash, ...
\end{itemize}
\end{slide}

%----------------------------------------------------------------------
\begin{slide}[fragile]{0291}\frametitle{\iConcept{Iterators} {Java style}}\label{java-style-iterators}
\xExample{Iterators!Java style}
\flushedImage{coretypes/images/javaiterators1}
\begin{itemize}
  \item Iterator points between items
  \item Example \xConcept{Iterators}{QList} iterator
  \begin{cpp}
QList<QString> list;
list << "A" << "B" << "C" << "D";
QListIterator<QString> it(list);
  \end{cpp}
  \item Forward iteration
  \begin{cpp}
while(it.hasNext()) {
  qDebug() << it.next();    // A B C D
}  
  \end{cpp}
  \item Backward iteration
  \begin{cpp}
it.toBack(); // position after the last item
while(it.hasPrevious()) {
  qDebug() << it.previous(); // D C B A
}  
  \end{cpp}  
  \item[] \doc{qlistiterator.html\#details}{QListIterator}
\end{itemize}  
\end{slide}

%----------------------------------------------------------------------
\begin{slide}[fragile]{0293}\frametitle{Modifying During Iteration}
\xConcept{Iterators!Modifying}
\begin{itemize}
  \item Use \emph{mutable} versions of the iterators
  \begin{itemize}
	  \item e.g. \iCls{QMutableListIterator}.
    \item[] \begin{cpp}
QList<int> list; 
list << 1 << 2 << 3 << 4;
QMutableListIterator<int> i(list);
while (i.hasNext()) {
  if (i.next() % 2 != 0)
    i.remove();
}
// list now 2, 4  
    \end{cpp}
  \end{itemize}
  \item \hClsFn{QMutableListIterator}{remove} and \hClsFn{QMutableListIterator}{setValue}
  \begin{itemize}
    \item  Operate on items just jumped over using next()/previous()
  \end{itemize}
  \item \hClsFn{QMutableListIterator}{insert} 
  \begin{itemize}
    \item Inserts item at current position in sequence \\ 
    \begin{itemize}    
    	\item \emph{Remember iterator points between item}
    \end{itemize}    
    \item \hClsFn{QMutableListIterator}{previous} reveals just inserted item
  \end{itemize}
\end{itemize}    
\end{slide}

%----------------------------------------------------------------------
\begin{slide}[fragile]{0294}\frametitle{Iterating Over QMap and QHash}
\xConcept{Iterators!QMap}
\begin{itemize}
\item \hClsFn{QMapIterator}{next} and \hClsFn{QMapIterator}{previous} 
  \begin{itemize}  
  	\item Return Item class  with \texttt{key()} and \texttt{value()}
  \end{itemize}  
\item Alternatively use \hClsFn{QMapIterator}{key} and \hClsFn{QMapIterator}{value} from iterator
\end{itemize} 
\xExample{Iterators!QMap}                                                               
\begin{cpp}
QMap<QString, QString> map;
map["Paris"] = "France";
map["Guatemala City"] = "Guatemala";
map["Mexico City"] = "Mexico";
map["Moscow"] = "Russia";

QMutableMapIterator<QString, QString> i(map);
while (i.hasNext()) {
  if (i.next().key().endsWith("City"))
    i.remove();
}          
// map now "Paris", "Moscow"
\end{cpp}
\demo{core-types/ex-container}
\end{slide}

%----------------------------------------------------------------------
\begin{slide}[fragile]{0297}\frametitle{\iConcept{STL-style} Iterators}
\xExample{Iterators!STL style}
\flushedImage{coretypes/images/stliterators1}
\begin{itemize}
  \item Iterator points at item
  \item Example \xConcept{Iterators}{QList} iterator
\begin{cpp}
QList<QString> list;
list << "A" << "B" << "C" << "D";
QList<QString>::iterator i;
\end{cpp}
\item Forward mutable iteration
\begin{cpp}
for (i = list.begin(); i != list.end(); ++i) {
    *i = (*i).toLower();
}
\end{cpp}
\item Backward mutable iteration
\begin{cpp}
i = list.end();
while (i != list.begin()) {
    --i;
    *i = (*i).toLower();
}
\end{cpp}  
\item \lstinline{QList<QString>::const_iterator} for read-only
\end{itemize}
\end{slide}


%----------------------------------------------------------------------
\begin{slide}[fragile]{0298}\frametitle{The \texttt{foreach} Keyword}
\xConcept{foreach keyword}                                                                     
\xExample{Iterators!foreach}
\begin{itemize}
  \item It is a macro, feels like a keyword
  \item[] \fbox{\texttt{foreach} ( \textit{variable}, \textit{container} ) \textit{statement} }  
  \item[] \begin{cpp}
foreach (const QString& str, list) {
  if (str.isEmpty())
    break;
  qDebug() << str;
}    
  \end{cpp}
  \item \iCls{break} and \iCls{continue} as normal
  \item Modifying the container while iterating
  \begin{itemize}
  	\item results in container being copied
  	\item iteration continues in unmodified version
  \end{itemize}
  \item Not possible to modify item
  \begin{itemize}
  	\item iterator variable is a const reference.
  \end{itemize}
\end{itemize}
\end{slide}

%----------------------------------------------------------------------
\begin{slide}{0300}\frametitle{\iConcept{Algorithms}}
\begin{itemize}
  \item STL-style iterators are compatible with the STL algorithms
  \begin{itemize}
    \item Defined in the STL $<$algorithm$>$ header
  \end{itemize}
  \item Qt has own algorithms
  \begin{itemize}
    \item Defined in $<$QtAlgorithms$>$ header
  \end{itemize}
  \item \emph{If STL is available on all your supported platforms you can
      choose to use the STL algorithms}
  \begin{itemize}
    \item The collection is much larger than the one in Qt.
  \end{itemize}
\end{itemize}
\end{slide}

%----------------------------------------------------------------------
\begin{slide}[fragile]{0301}\frametitle{Algorithms}
\begin{itemize}
\item \iCls[\textbf]{qSort}(begin, end) sort items in range
\item \iCls[\textbf]{qFind}(begin, end, value) find value
\item \iCls[\textbf]{qEqual}(begin1, end1, begin2) checks two ranges
\item \iCls[\textbf]{qCopy}(begin1, end1, begin2) from one range to another
\item \iCls[\textbf]{qCount}(begin, end, value, n) occurrences of value in range
\item and more ...
\end{itemize}
Counting 1's in list          
\begin{cpp}
QList<int> list;
list << 1 << 2 << 3 << 1;
int count = 0;
qCount(list, 1, count); // count the 1's
qDebug() << count;      // 2  (means 2 times 1)
\end{cpp}
\begin{itemize}
  \item For parallel (ie. multi-threaded) algorithms
  \begin{itemize}  
  	\item \doc{threads.html\#qtconcurrent-intro}{\iCls{QtConcurrent}}
  \end{itemize}
\end{itemize}    
  \doc{qtalgorithms.html\#details}{QtAlgorithms}
\end{slide}

%----------------------------------------------------------------------
\begin{slide}[fragile]{0303}\frametitle{Algorithms Examples}
\xExample{Algorithms!qCopy}
\begin{itemize}
  \item Copy list to vector example
  \item[] \begin{cpp}
QList<QString> list;
list << "one" << "two" << "three";
QVector<QString> vector(3);
qCopy(list.begin(), list.end(), vector.begin());
// vector: [ "one", "two", "three" ]
  \end{cpp}      
  \item Case insensitive sort example
  \item[] \begin{cpp}
bool lessThan(const QString& s1, const QString& s2) {
    return s1.toLower() < s2.toLower();
}
// ...
QList<QString> list;
list << "AlPha" << "beTA" << "gamma" << "DELTA";
qSort(list.begin(), list.end(), lessThan);
// list: [ "AlPha", "beTA", "DELTA", "gamma" ]
\end{cpp}
  
\end{itemize}
\end{slide}

%----------------------------------------------------------------------
\begin{slide}[fragile]{0305}\frametitle{\iConcept{Implicitly Sharing and Containers}}
\xConcept{Reference counting} 
\begin{block}{Implicit Sharing}
If an object is copied, then its data is copied \textit{only when} the data of one of
the objects is changed
\end{block}
\begin{itemize}
  \item Shared class has a pointer to shared data block
  \begin{itemize}
    \item Shared data block = reference counter and actual data
 \end{itemize}  
  \item Assignment is a shallow copy
  \item Changing results into deep copy (detach)
\begin{cpp}
QList<int> l1, l2; l1 << 1 << 2;
l2 = l1; // shallow-copy: l2 shares date with l1
l2 << 3; // deep-copy: change triggers detach from l1
\end{cpp}  
\end{itemize}
\textit{Important to remember when inserting items into a container, or when returning a container.}
\end{slide}
