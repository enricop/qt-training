%%%%%%%%%%%%%%%%%%%%%%%%%%%%%%%%%%%%%%%%%%%%%%%%%%%%%%%%%%%%%%%%%%%%%%%%%%
%
% Copyright (c) 2008-2011, Nokia Corporation and/or its subsidiary(-ies).
% All rights reserved.
%
% This work, unless otherwise expressly stated, is licensed under a
% Creative Commons Attribution-ShareAlike 2.5.
%
% The full license document is available from
% http://creativecommons.org/licenses/by-sa/2.5/legalcode .
%
%%%%%%%%%%%%%%%%%%%%%%%%%%%%%%%%%%%%%%%%%%%%%%%%%%%%%%%%%%%%%%%%%%%%%%%%%%

\subsection{XML Models}

\begin{slide}{1450}\frametitle{XML List Models}

\begin{itemize}
\item Many data sources provide data in XML formats
\item \qic{class}{XmlListModel} is used to supply XML data to views
  \begin{itemize}
  \item using a mechanism that maps data to properties
  \item using XPath queries
  \end{itemize}
\item Views and delegates do not need to know about XML
  \begin{itemize}
  \item use a \qic{class}{ListView} or \qic{class}{Repeater} to access data
  \end{itemize}
\end{itemize}

\end{slide}

%----------------------------------------------------------------------

\begin{slide}{1449}\frametitle{Defining an XML List Model}

% declarative-uis/models-views/xml-list-model.tex
\begin{qml}
\qtt{\qc{class}{XmlListModel}~\{}\\
\qtt{~~~~\qc{type}{id}:~xmlModel}\\
\qtt{~~~~\qc{type}{source}:~\qc{string}{"files/items.xml"}}\\
\qtt{~~~~\qc{type}{query}:~\qc{string}{"//item"}}\\
\vspace*{0.5em}
\qtt{~~~~XmlRole~\{~\qc{type}{name}:~\qc{string}{"title"};~\qc{type}{query}:~\qc{string}{"string()"}~\}}\\
\qtt{~~~~XmlRole~\{~\qc{type}{name}:~\qc{string}{"link"};~\qc{type}{query}:~\qc{string}{"@link/string()"}~\}}\\
\qtt{\}}
\end{qml}

\begin{itemize}
\item Set the \qic{type}{id} property so the model can be referenced
\item Specify the \qic{type}{source} of the XML
\item The \qic{type}{query} identifies pieces of data in the model
\item Each piece of data is queried by \qic{class}{XmlRole} elements
\end{itemize}

\demo{qml-presenting-data/ex-models-views/xml-list-model.qml}

\end{slide}

%----------------------------------------------------------------------

\begin{slide}{1448}\frametitle{XML Roles}

\centeredImageDoubleWidth{qml-presenting-data/images/xml-list-model-xml-role.pdf}

\vspace*{1.0em}
\begin{itemize}
\item \qic{class}{XmlRole} associates names with data obtained
      using XPath queries
\item Made available to delegates as properties
  \begin{itemize}
  \item \qic{type}{title} and \qic{type}{link} in the above example
  \end{itemize}
\end{itemize}

\end{slide}

%----------------------------------------------------------------------

\begin{slide}{1447}\frametitle{Using an XML List Model}

% declarative-uis/models-views/xml-list-model.tex
\begin{qml}
\qtt{TitleDelegate~\{}\\
\qtt{~~~~\qc{type}{id}:~xmlDelegate}\\
\qtt{\}}\\
\vspace*{0.5em}
\qtt{\qc{class}{ListView}~\{}\\
\qtt{~~~~\qc{type}{anchors.fill}:~parent}\\
\qtt{~~~~\qc{type}{anchors.margins}:~\qc{number}{4}}\\
\qtt{~~~~\qc{type}{model}:~xmlModel}\\
\qtt{~~~~\qc{type}{delegate}:~xmlDelegate}\\
\qtt{\}}\\
\end{qml}

\begin{itemize}
\item Specify the \qic{type}{model} and \qic{type}{delegate} as usual
\item Ensure that the view is positioned and given a size
\item \qtt{TitleDelegate} element is defined in \qtt{\small TitleDelegate.qml}
  \begin{itemize}
  \item Must be defined using a \qic{class}{Component} element
  \end{itemize}
\end{itemize}

\vspace*{0.5em}
\demo{qml-presenting-data/ex-models-views/TitleDelegate.qml}

\end{slide}

%----------------------------------------------------------------------

\begin{slide}{1446}\frametitle{Defining a Delegate}

% declarative-uis/models-views/TitleDelegate.tex
\begin{qml}
\qtt{\qc{class}{Component}~\{}\\
\qtt{~~~\qc{class}{Item}~\{}\\
\qtt{~~~~~~\qc{type}{width}:~parent.width;~\qc{type}{height}:~\qc{number}{64}}\\
\vspace*{0.5em}
\qtt{~~~~~~\qc{class}{Rectangle}~\{}\\
\qtt{~~~~~~~~~\qc{type}{width}:~Math.max(childrenRect.width~\qc{operator}{+}~\qc{number}{16},~parent.width)}\\
\qtt{~~~~~~~~~\qc{type}{height}:~\qc{number}{60};~\qc{type}{clip}:~\qc{number}{true}}\\
\qtt{~~~~~~~~~\qc{type}{color}:~\qc{string}{"\#505060"};~\qc{type}{border.color}:~\qc{string}{"\#8080b0"};~\qc{type}{radius}:~\qc{number}{8}}\\
\vspace*{0.5em}
\qtt{~~~~~~~~~\qc{class}{Column}~\{~\qc{class}{Text}~\{~\qc{type}{x}:~\qc{number}{6};~\qc{type}{color}:~\qc{string}{"white"}}\\
\qtt{~~~~~~~~~~~~~~~~~~~~~~~~~\qc{type}{font.pixelSize}:~\qc{number}{32};~\qc{type}{text}:~title~\}}\\
\qtt{~~~~~~~~~~~~~~~~~~\qc{class}{Text}~\{~\qc{type}{x}:~\qc{number}{6};~\qc{type}{color}:~\qc{string}{"white"}}\\
\qtt{~~~~~~~~~~~~~~~~~~~~~~~~~\qc{type}{font.pixelSize}:~\qc{number}{16};~\qc{type}{text}:~link~\}~\}}\\
\qtt{~~~~~~\}}\\
\qtt{~~~\}}\\
\qtt{\}}
\end{qml}

%%% parents of delegates (scoping)

\begin{itemize}
\item \qic{type}{parent} refers to the view where it is used
\item \qtt{title} and \qtt{link} are properties exported by the model
\end{itemize}

\end{slide}
