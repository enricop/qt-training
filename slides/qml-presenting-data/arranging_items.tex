%%%%%%%%%%%%%%%%%%%%%%%%%%%%%%%%%%%%%%%%%%%%%%%%%%%%%%%%%%%%%%%%%%%%%%%%%%
%
% Copyright (c) 2008-2011, Nokia Corporation and/or its subsidiary(-ies).
% All rights reserved.
%
% This work, unless otherwise expressly stated, is licensed under a
% Creative Commons Attribution-ShareAlike 2.5.
%
% The full license document is available from
% http://creativecommons.org/licenses/by-sa/2.5/legalcode .
%
%%%%%%%%%%%%%%%%%%%%%%%%%%%%%%%%%%%%%%%%%%%%%%%%%%%%%%%%%%%%%%%%%%%%%%%%%%

\subsection{Arranging Items}

%----------------------------------------------------------------------

\begin{slide}{1489}\frametitle{Arranging Items}

Positioners and repeaters make it easier to work with many items

\begin{itemize}
\item Positioners arrange items in standard layouts
  \begin{itemize}
  \item in a column: \qic{class}{Column}
  \item in a row: \qic{class}{Row}
  \item in a grid: \qic{class}{Grid}
  \item like words on a page: \qic{class}{Flow}
  \end{itemize}
\item Repeaters create items from a template
  \begin{itemize}
  \item for use with positioners
  \item using data from a model
  \end{itemize}
\item Combining these make it easy to lay out lots of items
\end{itemize}

\end{slide}

%----------------------------------------------------------------------

\begin{slide}{1488}\frametitle{Positioning Items}

\flushedImage{qml-presenting-data/images/grid-rectangles.png}
% declarative-uis/arranging-items/grid-rectangles.qml
\inputqml{qml-presenting-data/colorized/grid-rectangles}


\begin{itemize}
\item Items inside a positioner are automatically arranged
  \begin{itemize}
  \item in a 2 by 2 \qic{class}{Grid}
  \item with horizontal/vertical spacing of 20 pixels
  \end{itemize}
\item \qic{type}{x}, \qic{type}{y} is the position of the first item
\item Like layouts in Qt
\end{itemize}    

\demo{qml-presenting-data/ex-arranging-items/grid-rectangles.qml}

\end{slide}

%----------------------------------------------------------------------

\begin{slide}{1487}\frametitle{Repeating Items}

\begin{qml}
\qtt{\qc{keyword}{import}~\qc{class}{QtQuick}~\qc{number}{2.0}}\\
\vspace*{0.5em}
\qtt{\qc{class}{Rectangle}~\{}\\
\qtt{~~~~\qc{type}{width}:~\qc{number}{400};~\qc{type}{height}:~\qc{number}{400};~\qc{type}{color}:~\qc{string}{"black"}}\\
\vspace*{0.5em}
\qtt{~~~~\qc{class}{Grid}~\{}\\
\qtt{~~~~~~~~...}\\
\qtt{~~~~~~~~\qc{class}{Repeater}~\{~...~\}}\\
\qtt{~~~~\}}\\
\qtt{\}}
\end{qml}

\begin{itemize}
\item The \qic{class}{Repeater} creates items
\item The \qic{class}{Grid} arranges them within its parent item
\item The outer \qic{class}{Rectangle} item provides
  \begin{itemize}
  \item the space for generated items
  \item a local coordinate system
  \end{itemize}
\end{itemize}

\end{slide}

%----------------------------------------------------------------------

\begin{slide}{1486}\frametitle{Repeating Items}

\flushedImage{qml-presenting-data/images/repeater-grid.png}
% declarative-uis/arranging-items/repeater-grid.qml
\inputqml{qml-presenting-data/colorized/repeater-grid}

\begin{itemize}
\item \qic{class}{Repeater} takes data from a model
  \begin{itemize}
  \item just a number in this case
  \end{itemize}
\item Creates items based on the template item
  \begin{itemize}
  \item a light green rectangle
  \end{itemize}
\end{itemize} 

\demo{qml-presenting-data/ex-arranging-items/repeater-grid.qml}

\end{slide}

%----------------------------------------------------------------------

\begin{slide}{1485}\frametitle{Indexing Items}

\flushedImage{qml-presenting-data/images/repeater-grid-index}
% declarative-uis/arranging-items/repeater-grid-index.qml
\begin{qml}
\qtt{\qc{lightgray}{import~QtQuick~2.0}}\\
\vspace*{0.5em}
\qtt{\qc{lightgray}{Rectangle~\{}}\\
\qtt{~~~~\qc{lightgray}{width:~400;~height:~400;~color:~"black"}}\\
\vspace*{0.5em}
\qtt{~~~~\qc{lightgray}{Grid~\{}}\\
\qtt{~~~~~~~~\qc{lightgray}{x:~5;~y:~5}}\\
\qtt{~~~~~~~~\qc{lightgray}{rows:~5;~columns:~5;~spacing:~10}}\\
\vspace*{0.5em}
\qtt{~~~~~~~~\qc{class}{Repeater}~\qc{lightgray}{\{~model:~24}}\\
\qtt{~~~~~~~~~~~~~~~~~~~\qc{lightgray}{Rectangle~\{~width:~70;~height:~70}}\\
\qtt{~~~~~~~~~~~~~~~~~~~~~~~~~~~~~~~\qc{lightgray}{color:~"lightgreen"}}\\
\vspace*{0.5em}
\qtt{~~~~~~~~~~~~~~~~~~~~~~~~~~~~~~~\qc{class}{Text}~\{~\qc{type}{text}:~index}\\
\qtt{~~~~~~~~~~~~~~~~~~~~~~~~~~~~~~~~~~~~~~\qc{type}{font.pointSize}:~\qc{number}{30}}\\
\qtt{~~~~~~~~~~~~~~~~~~~~~~~~~~~~~~~~~~~~~~\qc{type}{anchors.centerIn}:~parent~\}~\}}\\
\qtt{~~~~~~~~\}~...}
\end{qml}

\begin{itemize}
\item \qic{class}{Repeater} provides an \qtt{index} for each item it creates
\end{itemize}             

\demo{qml-presenting-data/ex-arranging-items/repeater-grid-index.qml} 

\end{slide}

%----------------------------------------------------------------------

\begin{slide}{1484}\frametitle{Positioner Hints and Tips}

\begin{itemize}
\item Anchors in the \qic{class}{Row}, \qic{class}{Column} or \qic{class}{Grid}
  \begin{itemize}
  \item apply to all the items they contain
  \end{itemize}
\end{itemize}

\end{slide}
                            

%----------------------------------------------------------------------

\begin{slide}{1483}\frametitle{Lab~\textendash~Chess Board}

\flushedImage{qml-presenting-data/images/chess-board.png}

\begin{itemize}
\item Start by creating a chess board using\\
      a \qic{class}{Grid} and a \qic{class}{Repeater}
  \begin{itemize}
  \item use the \qic{type}{index} to create a checker pattern
  \end{itemize}
\item Use the \qtt{\small knight.png} image to create a\\
      piece that can be placed on any square
  \begin{itemize}
  \item bind its \qic{type}{x} and \qic{type}{y} properties to custom\\
        \qic{type}{cx} and \qic{type}{cy} properties
  \end{itemize}
\item Make each square clickable
  \begin{itemize}
  \item move the piece when a suitable square is clicked
  \end{itemize}
\item Make the model an \qtt{Array} that records which squares have been
      visited
\item Make the board and piece separate components
\end{itemize}

\end{slide}

%----------------------------------------------------------------------

\begin{slide}{1482}\frametitle{Lab~\textendash~Calendar}

\flushedImage{qml-presenting-data/images/calendar}

\begin{itemize}
\item Start by creating a grid of squares using\\
      a \qic{class}{Grid} and a \qic{class}{Repeater}
  \begin{itemize}
  \item put the grid inside an \qic{class}{Item}
  \item use the \qic{type}{index} to give each square a number
  \end{itemize}
\item Place a title above the grid
\item Ensure that the current date is highlighted
\item Use the \qtt{\small left.png} and \qtt{\small right.png} images to
      create buttons on each side of the title
\item Make the buttons navigate to the next and previous months
\item Add a header showing the days of the week
\end{itemize}

\end{slide}
