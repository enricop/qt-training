%%%%%%%%%%%%%%%%%%%%%%%%%%%%%%%%%%%%%%%%%%%%%%%%%%%%%%%%%%%%%%%%%%%%%%%%%%
%
% Copyright (c) 2008-2011, Nokia Corporation and/or its subsidiary(-ies).
% All rights reserved.
%
% This work, unless otherwise expressly stated, is licensed under a
% Creative Commons Attribution-ShareAlike 2.5.
%
% The full license document is available from
% http://creativecommons.org/licenses/by-sa/2.5/legalcode .
%
%%%%%%%%%%%%%%%%%%%%%%%%%%%%%%%%%%%%%%%%%%%%%%%%%%%%%%%%%%%%%%%%%%%%%%%%%%

\subsection{Using Views}
%----------------------------------------------------------------------

\begin{slide}{1481}\frametitle{Views}

\begin{itemize}
\item \qic{class}{ListView} shows a classic list of items
  \begin{itemize}
  \item with horizontal or vertical placing of items
  \end{itemize}
\item \qic{class}{GridView} displays items in a grid
  \begin{itemize}
  \item like an file manager's icon view
  \end{itemize}
%\item \qic{class}{PathView} displays items on a path
%  \begin{itemize}
%  \item the items move instead of the selection
%  \end{itemize}
\end{itemize}

\end{slide}

%----------------------------------------------------------------------

\begin{slide}{1480}\frametitle{List Views}

Take the model and delegate from before:

\vspace*{0.5em}
\begin{qml}
\qtt{\qc{class}{ListModel}~\{}\\
\qtt{~~~~\qc{type}{id}:~nameModel}\\
\qtt{~~~~ListElement~\{~\qc{type}{name}:~\qc{string}{"Alice"}~\}}\\
\qtt{~~~~ListElement~\{~\qc{type}{name}:~\qc{string}{"Bob"}~\}}\\
\qtt{~~~~ListElement~\{~\qc{type}{name}:~\qc{string}{"Jane"}~\}}\\
\qtt{~~~~ListElement~\{~\qc{type}{name}:~\qc{string}{"Victor"}~\}}\\
\qtt{~~~~ListElement~\{~\qc{type}{name}:~\qc{string}{"Wendy"}~\}}\\
\qtt{\}}\\
\vspace*{0.5em}
\qtt{\qc{class}{Component}~\{}\\
\qtt{~~~~\qc{type}{id}:~nameDelegate}\\
\qtt{~~~~\qc{class}{Text}~\{}\\
\qtt{~~~~~~~~\qc{type}{text}:~name;}\\
\qtt{~~~~~~~~\qc{type}{font.pixelSize}:~\qc{number}{32}}\\
\qtt{~~~~\}}\\
\qtt{\}}
\end{qml}

\end{slide}

%----------------------------------------------------------------------

\begin{slide}{1479}\frametitle{List Views}

\flushedImageDoubleWidth{qml-presenting-data/images/list-model-list-view.png}
% declarative-uis/models-views/list-model-list-view.qml
\begin{qml}
\qtt{\qc{class}{ListView}~\{}\\
\qtt{~~~~\qc{type}{anchors.fill}:~parent}\\
\qtt{~~~~\qc{type}{model}:~nameModel}\\
\qtt{~~~~\qc{type}{delegate}:~nameDelegate}\\
\qtt{~~~~\qc{type}{clip}:~\qc{number}{true}}\\
\qtt{\}}
\end{qml}

\begin{itemize}
  \item No default delegate
  \item Unclipped views paint outside their areas
  \begin{itemize}
  \item set the \qic{type}{clip} property to enable clipping
  \end{itemize}
\item Views are positioned like other items
  \begin{itemize}
  \item the above view fills its parent
  \end{itemize}
\end{itemize}

\demo{qml-presenting-data/ex-models-views/list-model-list-view.qml}


\end{slide}

%----------------------------------------------------------------------

\begin{slide}{1478}\frametitle{Decoration and Navigation}

\flushedImageDoubleWidth{qml-presenting-data/images/list-view-decoration.pdf}

By default, \qic{class}{ListView} is
\begin{itemize}
\item undecorated
\item a flickable surface
  \begin{itemize}
  \item (can be dragged and flicked)
  \end{itemize}
\end{itemize}\medskip

\begin{itemize}
\item To add decoration:
  \begin{itemize}
  \item with a \qic{type}{header} and \qic{type}{footer}
  \item with a \qic{type}{highlight} item to show the current item
  \end{itemize}

\medskip
\item To configure for navigation:
  \begin{itemize}
  \item set \qic{type}{focus} to allow keyboard navigation
  \item \qic{type}{highlight} also helps the user with navigation
  \item unset \qic{type}{interactive} to disable dragging and flicking
  \end{itemize}
\end{itemize}

\vspace*{0.5em}
\demo{qml-presenting-data/ex-models-views/list-view-decoration.qml}

\end{slide}

%----------------------------------------------------------------------

\begin{slide}{1477}\frametitle{Decoration and Navigation}

\flushedImageDoubleWidth{qml-presenting-data/images/list-view-decoration.pdf}
% declarative-uis/models-views/list-view-decoration.qml
\begin{qml}
\qtt{\qc{class}{ListView}~\{}\\
\qtt{~~~\qc{type}{anchors.fill}:~parent}\\
\qtt{~~~\qc{type}{model}:~nameModel}\\
\qtt{~~~\qc{type}{delegate}:~nameDelegate}\\
\qtt{~~~\qc{type}{focus}:~\qc{number}{true}}\\
\qtt{~~~\qc{type}{clip}:~\qc{number}{true}}\\
\qtt{~~~\qc{type}{header}:~\qc{class}{Rectangle}~\{}\\
\qtt{~~~~~~\qc{type}{width}:~parent.width;~\qc{type}{height}:~\qc{number}{10}}\\
\qtt{~~~~~~\qc{type}{color}:~\qc{string}{"pink"}}\\
\qtt{~~~\}}\\
\qtt{~~~\qc{type}{footer}:~\qc{class}{Rectangle}~\{}\\
\qtt{~~~~~~\qc{type}{width}:~parent.width;~\qc{type}{height}:~\qc{number}{10}}\\
\qtt{~~~~~~\qc{type}{color}:~\qc{string}{"lightblue"}}\\
\qtt{~~~\}}\\
\qtt{~~~\qc{type}{highlight}:~\qc{class}{Rectangle}~\{}\\
\qtt{~~~~~~\qc{type}{width}:~parent.width;~\qc{type}{color}:~\qc{string}{"lightgray"}}\\
\qtt{~~~\}}\\
\qtt{\}}
\end{qml}

\end{slide}

%----------------------------------------------------------------------

\begin{slide}{1476}\frametitle{Decoration and Navigation}

\flushedImageDoubleWidth{qml-presenting-data/images/list-view-current-item.png}
Each \qic{class}{ListView} exposes its current item:

\vspace*{0.75em}
% declarative-uis/models-views/list-model-current-item.qml
\begin{qml}
\qtt{\qc{class}{ListView}~\{}\\
\qtt{~~~~\qc{type}{id}:~listView}\\
\qtt{~~~~...}\\
\qtt{\}}\\
\vspace*{0.5em}
\qtt{\qc{class}{Text}~\{}\\
\qtt{~~~~\qc{type}{id}:~label}\\
\qtt{~~~~\qc{type}{anchors.bottom}:~parent.bottom}\\
\qtt{~~~~\qc{type}{anchors.horizontalCenter}:~parent.horizontalCenter}\\
\qtt{~~~~\qc{type}{text}:~\qc{string}{"<b>"}~\qc{operator}{+}~listView.currentItem.text~\qc{operator}{+}}\\
\qtt{~~~~~~~~~~\qc{string}{"</b>~is~current"}}\\
\qtt{~~~~\qc{type}{font.pixelSize}:~\qc{number}{16}}\\
\qtt{\}}
\end{qml}

\begin{itemize}
\item Recall that, in this case, each item has a \qic{type}{text} property
  \begin{itemize}
  \item re-use the \qtt{listView}'s \qic{type}{currentItem}'s \qic{type}{text}
  \end{itemize}
\end{itemize}

\demo{qml-presenting-data/ex-models-views/list-view-current-item.qml}

\end{slide}

%----------------------------------------------------------------------

\begin{slide}{1475}\frametitle{Adding Sections}

\flushedImageDoubleWidth{qml-presenting-data/images/list-view-sections.png}
\begin{itemize}
\item Data in a \qic{class}{ListView} can be ordered by section
\item Categorize the list items by
  \begin{itemize}
  \item choosing a property name; e.g. \qic{type}{team}
  \item adding this property to each \qic{class}{ListElement}
  \item storing the section in this property
  \end{itemize}
\end{itemize}

\vspace*{1.5em}
\begin{qml}
\qtt{\qc{class}{ListModel}~\{}\\
\qtt{~~~~\qc{type}{id}:~nameModel}\\
\qtt{~~~~\qc{class}{ListElement}~\{~\qc{type}{name}:~\qc{string}{"Alice"};~\qc{type}{team}:~\qc{string}{"Crypto"}~\}}\\
\qtt{~~~~\qc{class}{ListElement}~\{~\qc{type}{name}:~\qc{string}{"Bob"};~\qc{type}{team}:~\qc{string}{"Crypto"}~\}}\\
\qtt{~~~~\qc{class}{ListElement}~\{~\qc{type}{name}:~\qc{string}{"Jane"};~\qc{type}{team}:~\qc{string}{"QA"}~\}}\\
\qtt{~~~~\qc{class}{ListElement}~\{~\qc{type}{name}:~\qc{string}{"Victor"};~\qc{type}{team}:~\qc{string}{"QA"}~\}}\\
\qtt{~~~~\qc{class}{ListElement}~\{~\qc{type}{name}:~\qc{string}{"Wendy"};~\qc{type}{team}:~\qc{string}{"Graphics"}~\}}\\
\qtt{\}}
\end{qml}

\end{slide}

%----------------------------------------------------------------------

\begin{slide}{1474}\frametitle{Displaying Sections}

Using the \qic{class}{ListView}

\begin{itemize}
\item Set \qic{type}{section.property}
  \begin{itemize}
  \item refer to the \qic{class}{ListElement} property holding the section name
  \end{itemize}
\item Set \qic{type}{section.criteria} to control what to show
  \begin{itemize}
  \item \qtt{ViewSection.FullString} for complete section names
  \item \qtt{ViewSection.FirstCharacter} for alphabetical groupings
  \end{itemize}
\item Set \qic{type}{section.delegate}
  \begin{itemize}
  \item create a delegate for section headings
  \item either include it inline or reference it
  \end{itemize}
\end{itemize}

\end{slide}

%----------------------------------------------------------------------

\begin{slide}{1473}\frametitle{Displaying Sections}

\begin{qml}
\qtt{\qc{class}{ListView}~\{}\\
\qtt{~~~~\qc{type}{model}:~nameModel}\\
\qtt{~~~~...}\\
\qtt{~~~~\qc{type}{section.property}:~\qc{string}{"team"}}\\
\qtt{~~~~\qc{type}{section.criteria}:~\qc{class}{ViewSection}.FullString}\\
\qtt{~~~~\qc{type}{section.delegate}:~\qc{class}{Rectangle}~\{}\\
\qtt{~~~~~~~~\qc{type}{color}:~\qc{string}{"\#b0dfb0"}}\\
\qtt{~~~~~~~~\qc{type}{width}:~parent.width}\\
\qtt{~~~~~~~~\qc{type}{height}:~childrenRect.height~\qc{operator}{+}~\qc{number}{4}}\\
\qtt{~~~~~~~~~~~~\qc{class}{Text}~\{~\qc{type}{anchors.centerIn}:~parent}\\
\qtt{~~~~~~~~~~~~~~~~~~~\qc{type}{font.pixelSize}:~\qc{number}{16}}\\
\qtt{~~~~~~~~~~~~~~~~~~~\qc{type}{font.bold}:~\qc{number}{true}}\\
\qtt{~~~~~~~~~~~~~~~~~~~\qc{type}{text}:~section~\}}\\
\qtt{~~~~~~~~\}}\\
\qtt{~~~~\}}\\
\qtt{\}}
\end{qml}

\begin{itemize}
\item The \qic{type}{section.delegate} is defined like the \qic{type}{highlight} delegate
\end{itemize}

\end{slide}

%----------------------------------------------------------------------

\begin{slide}{1472}\frametitle{Grid Views}

Set up a list model with items:

\vspace*{0.75em}
% declarative-uis/models-views/list-model-grid-view.qml
\begin{qml}
\qtt{\qc{class}{ListModel}~\{}\\
\qtt{~~~~\qc{type}{id}:~nameModel}\\
\qtt{~~~~\qc{class}{ListElement}~\{~\qc{type}{file}:~\qc{string}{"../images/rocket.svg"}}\\
\qtt{~~~~~~~~~~~~~~~~~~\qc{type}{name}:~\qc{string}{"rocket"}~\}}\\
\qtt{~~~~\qc{class}{ListElement}~\{~\qc{type}{file}:~\qc{string}{"../images/clear.svg"}}\\
\qtt{~~~~~~~~~~~~~~~~~~\qc{type}{name}:~\qc{string}{"clear"}~\}}\\
\qtt{~~~~\qc{class}{ListElement}~\{~\qc{type}{file}:~\qc{string}{"../images/arrow.svg"}}\\
\qtt{~~~~~~~~~~~~~~~~~~\qc{type}{name}:~\qc{string}{"arrow"}~\}}\\
\qtt{~~~~\qc{class}{ListElement}~\{~\qc{type}{file}:~\qc{string}{"../images/book.svg"}}\\
\qtt{~~~~~~~~~~~~~~~~~~\qc{type}{name}:~\qc{string}{"book"}~\}}\\
\qtt{\}}
\end{qml}

\begin{itemize}
\item Define string properties to use in the delegate
\end{itemize}

\demo{qml-presenting-data/ex-models-views/list-model-grid-view.qml}

\end{slide}

%----------------------------------------------------------------------

\begin{slide}{1471}\frametitle{Grid Views}

Set up a delegate:

\vspace*{0.75em}
% declarative-uis/models-views/list-model-grid-view.qml
\begin{qml}
\qtt{\qc{class}{Component}~\{}\\
\qtt{~~~\qc{type}{id}:~nameDelegate}\\
\qtt{~~~\qc{class}{Column}~\{}\\
\qtt{~~~~~~\qc{class}{Image}~\{}\\
\qtt{~~~~~~~~~\qc{type}{id}:~delegateImage}\\
\qtt{~~~~~~~~~\qc{type}{anchors.horizontalCenter}:~delegateText.horizontalCenter}\\
\qtt{~~~~~~~~~\qc{type}{source}:~file;~\qc{type}{width}:~\qc{number}{64};~\qc{type}{height}:~\qc{number}{64};~\qc{type}{smooth}:~\qc{number}{true}}\\
\qtt{~~~~~~~~~\qc{type}{fillMode}:~\qc{class}{Image}.PreserveAspectFit}\\
\qtt{~~~~~~\}}\\
\qtt{~~~~~~\qc{class}{Text}~\{}\\
\qtt{~~~~~~~~~\qc{type}{id}:~delegateText}\\
\qtt{~~~~~~~~~\qc{type}{text}:~name;~\qc{type}{font.pixelSize}:~\qc{number}{24}}\\
\qtt{~~~~~~\}}\\
\qtt{~~~\}}\\
\qtt{\}}
\end{qml}

\end{slide}

%----------------------------------------------------------------------

\begin{slide}{1470}\frametitle{Grid Views}

\flushedImageDoubleWidth{qml-presenting-data/images/list-model-grid-view.png}
% declarative-uis/models-views/list-model-grid-view.qml
\begin{qml}
\qtt{\qc{class}{GridView}~\{}\\
\qtt{~~~~\qc{type}{anchors.fill}:~parent}\\
\qtt{~~~~\qc{type}{model}:~nameModel}\\
\qtt{~~~~\qc{type}{delegate}:~nameDelegate}\\
\qtt{~~~~\qc{type}{clip}:~\qc{number}{true}}\\
\qtt{\}}
\end{qml}

\begin{itemize}
\item The same as \qic{class}{ListView} to set up
\item Uses data from a list model
  \begin{itemize}
  \item not like Qt's table view
  \item more like Qt's list view in icon mode
  \end{itemize}
\end{itemize}

\end{slide}

%----------------------------------------------------------------------

\begin{slide}{1469}\frametitle{Decoration and Navigation}

\flushedImageDoubleWidth{qml-presenting-data/images/grid-view-decoration.pdf}
Like \qic{class}{ListView}, \qic{class}{GridView} is

\begin{itemize}
\item undecorated
\item a flickable surface
\end{itemize}

\medskip
\begin{itemize}
\item To add decoration:
  \begin{itemize}
  \item define \qic{type}{header} and \qic{type}{footer}
  \item define \qic{type}{highlight} item to show\\
        the current item
  \end{itemize}

\medskip
\item To configure for navigation:
  \begin{itemize}
  \item set \qic{type}{focus} to allow keyboard navigation
  \item \qic{type}{highlight} also helps the user with navigation
  \item unset \qic{type}{interactive} to disable dragging and flicking
  \end{itemize}
\end{itemize}

\vspace*{0.5em}
\demo{qml-presenting-data/ex-models-views/grid-view-decoration.qml}

\end{slide}

%----------------------------------------------------------------------

\begin{slide}[fragile]{1468}\frametitle{Decoration and Navigation}

\flushedImageDoubleWidth{qml-presenting-data/images/grid-view-decoration.pdf}
% declarative-uis/models-views/grid-view-decoration.qml
\begin{qml}
\qtt{\qc{class}{GridView}~\{}\\
\qtt{~~~~...}\\
\qtt{~~~~\qc{type}{header}:~\qc{class}{Rectangle}~\{}\\
\qtt{~~~~~~~~\qc{type}{width}:~parent.width}\\
\qtt{~~~~~~~~\qc{type}{height}:~\qc{number}{10}}\\
\qtt{~~~~~~~~\qc{type}{color}:~\qc{string}{"pink"}}\\
\qtt{~~~~\}}\\
\qtt{~~~~\qc{type}{footer}:~\qc{class}{Rectangle}~\{}\\
\qtt{~~~~~~~~\qc{type}{width}:~parent.width}\\
\qtt{~~~~~~~~\qc{type}{height}:~\qc{number}{10}}\\
\qtt{~~~~~~~~\qc{type}{color}:~\qc{string}{"lightblue"}}\\
\qtt{~~~~\}}\\
\qtt{~~~~\qc{type}{highlight}:~\qc{class}{Rectangle}~\{}\\
\qtt{~~~~~~~~\qc{type}{width}:~parent.width}\\
\qtt{~~~~~~~~\qc{type}{color}:~\qc{string}{"lightgray"}}\\
\qtt{~~~~\}}\\
\qtt{~~~~\qc{type}{focus}:~\qc{number}{true};~\qc{type}{clip}:~\qc{number}{true}}\\
\qtt{\}}
\end{qml}

\end{slide}

%----------------------------------------------------------------------
%
%\begin{slide}{1467}\frametitle{Path Views}
%
%%% Path views
%\begin{center}
%\textbf{Not covered at the moment}
%\end{center}
%
%\end{slide}

%----------------------------------------------------------------------

\begin{slide}{1466}\frametitle{Lab~\textendash~Contacts}

\flushedImageDoubleWidth{qml-presenting-data/images/contacts.png}

\begin{itemize}
\item Create a \qic{class}{ListItemModel}, fill it with\\
      \qic{class}{ListElement} elements, each with
  \begin{itemize}
  \item a \qic{type}{name} property
  \item a \qic{type}{file} property referring to an image
  \end{itemize}
\item Add a \qic{class}{ListView} and a \qic{class}{Component}\\
      to use as a delegate
\item Add \qic{type}{header}, \qic{type}{footer} and \qic{type}{highlight}
      properties to the view
\item Add \qic{type}{states} and \qic{type}{transitions} to the delegate
  \begin{itemize}
  \item activate the state when the delegate item is current
  \item use a state condition with the \qic{type}{ListView.isCurrentItem}
        attached property
  \item make a transition that animates the height of the item
  \end{itemize}
\end{itemize}

\end{slide}
