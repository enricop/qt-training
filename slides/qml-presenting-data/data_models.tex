%%%%%%%%%%%%%%%%%%%%%%%%%%%%%%%%%%%%%%%%%%%%%%%%%%%%%%%%%%%%%%%%%%%%%%%%%%
%
% Copyright (c) 2008-2011, Nokia Corporation and/or its subsidiary(-ies).
% All rights reserved.
%
% This work, unless otherwise expressly stated, is licensed under a
% Creative Commons Attribution-ShareAlike 2.5.
%
% The full license document is available from
% http://creativecommons.org/licenses/by-sa/2.5/legalcode .
%
%%%%%%%%%%%%%%%%%%%%%%%%%%%%%%%%%%%%%%%%%%%%%%%%%%%%%%%%%%%%%%%%%%%%%%%%%%

\subsection{Data Models}

%----------------------------------------------------------------------

\begin{slide}{1461}\frametitle{Models and Views}

% Models and Views

Models and views provide a way to handle data sets

\begin{itemize}
\item Models hold data or items
\item Views display data or items
  \begin{itemize}
  \item using delegates
  \end{itemize}
\end{itemize}

\end{slide}

%----------------------------------------------------------------------

\begin{slide}{1460}\frametitle{Models}

Pure models provide access to data:

\begin{itemize}
\item \qic{class}{ListModel}
\item \qic{class}{XmlListModel}
\end{itemize}

Visual models provide information about how to display data:

\begin{itemize}
\item Visual item model: \qic{class}{VisualItemModel}
  \begin{itemize}
  \item contains child items that are supplied to views
  \end{itemize}
\item Visual data model: \qic{class}{VisualDataModel}
  \begin{itemize}
  \item contains an interface to an underlying model
  \item supplies a delegate for rendering
  \end{itemize}
\end{itemize}

\doc{qdeclarativemodels.html}{Data Models}
\end{slide}

%----------------------------------------------------------------------

\begin{slide}{1459}\frametitle{List Models}

\begin{itemize}
\item List models contain simple sequences of elements
\item Each \qic{class}{ListElement} contains
  \begin{itemize}
  \item one or more pieces of data
  \item defined using properties
  \item \textit{no information} about how to display itself
  \end{itemize}
\item \qic{class}{ListElement} does not have pre-defined properties
  \begin{itemize}
  \item all properties are custom properties
  \end{itemize}
\end{itemize}

\vspace*{1em}
\begin{qml}
\qtt{\qc{class}{ListModel}~\{}\\
\qtt{~~~~\qc{class}{ListElement}~\{~...~\}}\\
\qtt{~~~~\qc{class}{ListElement}~\{~...~\}}\\
\qtt{~~~~...}\\
\qtt{\}}
\end{qml}

\end{slide}

%----------------------------------------------------------------------

\begin{slide}{1458}\frametitle{Defining a List Model}

\flushedImage{qml-presenting-data/images/list-model-repeater.png}
% declarative-uis/models-views/visual-item-model.qml
\begin{qml}
\qtt{\qc{class}{ListModel}~\{}\\
\qtt{~~~~\qc{type}{id}:~name\_model}\\
\qtt{~~~~ListElement~\{~\qc{type}{name}:~\qc{string}{"Alice"}~\}}\\
\qtt{~~~~ListElement~\{~\qc{type}{name}:~\qc{string}{"Bob"}~\}}\\
\qtt{~~~~ListElement~\{~\qc{type}{name}:~\qc{string}{"Jane"}~\}}\\
\qtt{~~~~ListElement~\{~\qc{type}{name}:~\qc{string}{"Victor"}~\}}\\
\qtt{~~~~ListElement~\{~\qc{type}{name}:~\qc{string}{"Wendy"}~\}}\\
\qtt{\}}
\end{qml}

\begin{itemize}
\item Define a \qic{class}{ListModel}
  \begin{itemize}
  \item with an \qic{type}{id} so it can be referenced
  \end{itemize}
\item Define \qic{class}{ListElement} child objects
  \begin{itemize}
  \item each with a \qic{type}{name} property
  \item the property will be referenced by a delegate
  \end{itemize}
\end{itemize}

\demo{qml-presenting-data/ex-models-views/list-model-list-view.qml}

\end{slide}

%----------------------------------------------------------------------

\begin{slide}{1457}\frametitle{Defining a Delegate}

\flushedImage{qml-presenting-data/images/list-model-repeater.png}
% declarative-uis/models-views/visual-item-model.qml
\begin{qml}
\qtt{\qc{class}{Component}~\{}\\
\qtt{~~~~\qc{type}{id}:~name\_delegate}\\
\qtt{~~~~\qc{class}{Text}~\{}\\
\qtt{~~~~~~~~\qc{type}{text}:~name}\\
\qtt{~~~~~~~~\qc{type}{font.pixelSize}:~\qc{number}{32}}\\
\qtt{~~~~\}}\\
\qtt{\}}
\end{qml}

\begin{itemize}
\item Define a \qic{class}{Component} to use as a delegate
  \begin{itemize}
  \item with an \qic{type}{id} so it can be referenced
  \item describes how the data will be displayed
  \end{itemize}
\item Properties of list elements can be referenced
  \begin{itemize}
  \item use a \qic{class}{Text} item for each list element
  \item use the value of the \qic{type}{name} property from each element
  \end{itemize}
\item In the item inside a \qic{class}{Component}
  \begin{itemize}
  \item the \qic{type}{parent} property refers to the view
  \item a \qic{type}{ListView} attached property can also be used to access the view
  \end{itemize}
\end{itemize}

\end{slide}

%----------------------------------------------------------------------

\begin{slide}{1456}\frametitle{Using a List Model}

\flushedImage{qml-presenting-data/images/list-model-repeater.png}
% declarative-uis/models-views/visual-item-model.qml
\begin{qml}
\qtt{\qc{class}{Column}~\{}\\
\qtt{~~~~\qc{type}{anchors.fill}:~parent}\\
\qtt{~~~~\qc{class}{Repeater}~\{}\\
\qtt{~~~~~~~~\qc{type}{model}:~name\_model}\\
\qtt{~~~~~~~~\qc{type}{delegate}:~name\_delegate}\\
\qtt{~~~~\}}\\
\qtt{\}}
\end{qml}

\begin{itemize}
\item A \qic{class}{Repeater} fetches elements from \qtt{name\_model}
  \begin{itemize}
  \item using the delegate to display elements as \qic{class}{Text} items
  \end{itemize}
\item A \qic{class}{Column} arranges them vertically
  \begin{itemize}
  \item using anchors to make room for the items
  \end{itemize}
\end{itemize}

\end{slide}


%----------------------------------------------------------------------

\begin{slide}{1455}\frametitle{Working with Items}

\begin{itemize}
\item \qc{class}{ListModel} is a dynamic list of items
\item Items can be appended, inserted, removed and moved
  \begin{itemize}
  \item \textbf{append} item data using JavaScript dictionaries:\\
  \vspace*{0.25em}
  \begin{qml}
  \qtt{bookmarkModel.append(\{\qc{string}{"title"}:~lineEdit.text\})}
  \end{qml}
  \vspace*{0.5em}

  \item \textbf{remove} items by index obtained from a \qic{class}{ListView}:\\
  \vspace*{0.25em}
  \begin{qml}
  \qtt{bookmarkModel.remove(listView.currentIndex)}
  \end{qml}
  \vspace*{0.5em}

  \item \textbf{move} a number of items between two indices:\\
  \vspace*{0.25em}
  \begin{qml}
  \qtt{bookmarkModel.move(listView.currentIndex,}\\
  \qtt{~~~~~~~~~~~~~~~~~~~listView.currentIndex~\qc{operator}{+}~\qc{number}{1},~number)}
  \end{qml}
  \end{itemize}
\end{itemize}

\end{slide}

%----------------------------------------------------------------------

\begin{slide}{1454}\frametitle{List Model Hints}

\begin{itemize}
\item \textbf{Note:} Model properties cannot shadow delegate properties:
\end{itemize}
\begin{qml}
\qtt{~~\qc{class}{ListModel}~\{}\\
\qtt{~~~~~~ListElement~\{~\qc{type}{text}:~\qc{string}{"Alice"}~\}}\\
\qtt{~~\}}\\
\vspace*{0.5em}
\qtt{~~\qc{class}{Component}~\{}\\
\qtt{~~~~~~\qc{class}{Text}~\{}\\
\qtt{~~~~~~~~~~\qc{type}{text}:~text~\qc{comment}{// will not work}}\\
\qtt{~~~~~~\}}\\
\qtt{~~\}}
\end{qml}

\end{slide}

%----------------------------------------------------------------------

\begin{slide}{1453}\frametitle{Defining a Visual Item Model}

\flushedImage{qml-presenting-data/images/visual-item-model-repeater.png}
% declarative-uis/models-views/visual-item-model.qml
\begin{qml}
\qtt{\qc{class}{VisualItemModel}~\{}\\
\qtt{~~~~\qc{type}{id}:~labels}\\
\qtt{~~~~\qc{lightgray}{Rectangle~\{~color:~"\#cc7777";~radius:~10.0}}\\
\qtt{~~~~~~~~~~~~~~~~\qc{lightgray}{width:~300;~height:~50}}\\
\qtt{~~~~~~~~~~~~~~~~\qc{lightgray}{Text~\{~anchors.fill:~parent}}\\
\qtt{~~~~~~~~~~~~~~~~~~~~~~~\qc{lightgray}{font.pointSize:~32;~text:~"Books"}}\\
\qtt{~~~~~~~~~~~~~~~~~~~~~~~\qc{lightgray}{horizontalAlignment:~Qt.AlignHCenter~\}~\}}}\\
\qtt{~~~~\qc{lightgray}{Rectangle~\{~color:~"\#cccc55";~radius:~10.0}}\\
\qtt{~~~~~~~~~~~~~~~~\qc{lightgray}{width:~300;~height:~50}}\\
\qtt{~~~~~~~~~~~~~~~~\qc{lightgray}{Text~\{~anchors.fill:~parent}}\\
\qtt{~~~~~~~~~~~~~~~~~~~~~~~\qc{lightgray}{font.pointSize:~32;~text:~"Music"}}\\
\qtt{~~~~~~~~~~~~~~~~~~~~~~~\qc{lightgray}{horizontalAlignment:~Qt.AlignHCenter~\}~\}}}\\
\qtt{~~~~\qc{lightgray}{...}}\\
\qtt{\}}
\end{qml}

\begin{itemize}
\item Define a \qic{class}{VisualItemModel} item
  \begin{itemize}
  \item with an \qic{type}{id} so it can be referenced
  \end{itemize}
\end{itemize}

\end{slide}

%----------------------------------------------------------------------

\begin{slide}{1452}\frametitle{Defining a Visual Item Model}

\flushedImage{qml-presenting-data/images/visual-item-model-repeater.png}
% declarative-uis/models-views/visual-item-model.qml
\begin{qml}
\qtt{\qc{lightgray}{VisualItemModel~\{}}\\
\qtt{~~~~\qc{lightgray}{id:~labels}}\\
\qtt{~~~~\qc{class}{Rectangle}~\{~\qc{type}{color}:~\qc{string}{"\#cc7777"};~\qc{type}{radius}:~\qc{number}{10.0}}\\
\qtt{~~~~~~~~~~~~~~~~\qc{type}{width}:~\qc{number}{300};~\qc{type}{height}:~\qc{number}{50}}\\
\qtt{~~~~~~~~~~~~~~~~\qc{class}{Text}~\{~\qc{type}{anchors.fill}:~parent}\\
\qtt{~~~~~~~~~~~~~~~~~~~~~~~\qc{type}{font.pointSize}:~\qc{number}{32};~\qc{type}{text}:~\qc{string}{"Books"}}\\
\qtt{~~~~~~~~~~~~~~~~~~~~~~~\qc{type}{horizontalAlignment}:~\qc{class}{Qt}.AlignHCenter~\}~\}}\\
\qtt{~~~~\qc{class}{Rectangle}~\{~\qc{type}{color}:~\qc{string}{"\#cccc55"};~\qc{type}{radius}:~\qc{number}{10.0}}\\
\qtt{~~~~~~~~~~~~~~~~\qc{type}{width}:~\qc{number}{300};~\qc{type}{height}:~\qc{number}{50}}\\
\qtt{~~~~~~~~~~~~~~~~\qc{class}{Text}~\{~\qc{type}{anchors.fill}:~parent}\\
\qtt{~~~~~~~~~~~~~~~~~~~~~~~\qc{type}{font.pointSize}:~\qc{number}{32};~\qc{type}{text}:~\qc{string}{"Music"}}\\
\qtt{~~~~~~~~~~~~~~~~~~~~~~~\qc{type}{horizontalAlignment}:~\qc{class}{Qt}.AlignHCenter~\}~\}}\\
\qtt{~~~~...}\\
\qtt{\qc{lightgray}{\}}}
\end{qml}

\begin{itemize}
\item Define child items
  \begin{itemize}
  \item these will be shown when required
  \end{itemize}
\end{itemize}

\end{slide}

%----------------------------------------------------------------------

\begin{slide}{1451}\frametitle{Using a Visual Item Model}

\flushedImage{qml-presenting-data/images/visual-item-model-repeater.png}
% declarative-uis/models-views/visual-item-model.qml
\begin{qml}
\qtt{\qc{keyword}{import}~\qc{class}{QtQuick}~\qc{number}{2.0}}\\
\vspace*{0.5em}
\qtt{\qc{class}{Rectangle}~\{}\\
\qtt{~~~~\qc{type}{width}:~\qc{number}{400};~\qc{type}{height}:~\qc{number}{200};~\qc{type}{color}:~\qc{string}{"black"}}\\
\vspace*{0.5em}
\qtt{~~~~\qc{class}{VisualItemModel}~\{}\\
\qtt{~~~~~~~~\qc{type}{id}:~labels}\\
\qtt{~~~~~~~~...}\\
\qtt{~~~~\}}\\
\vspace*{0.5em}
\qtt{~~~~\qc{class}{Column}~\{}\\
\qtt{~~~~~~~~\qc{type}{anchors.horizontalCenter}:~parent.horizontalCenter}\\
\qtt{~~~~~~~~\qc{type}{anchors.verticalCenter}:~parent.verticalCenter}\\
\qtt{~~~~~~~~\qc{class}{Repeater}~\{~\qc{type}{model}:~labels~\}}\\
\qtt{~~~~\}}\\
\qtt{\}}\\
\end{qml}

\begin{itemize}
\item A \qic{class}{Repeater} fetches items from the \qtt{labels} model
\item A \qic{class}{Column} arranges them vertically
\end{itemize}

\end{slide}
