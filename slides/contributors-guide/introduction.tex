\subsection{Introduction}

\begin{slide}
  \frametitle{The Training Vision}
  \begin{enumerate}
    \item Help new users getting started coding Qt
    \begin{itemize}
      \item Help getting started with Qt's advanced UI design concepts
      \item Steering around common pitfalls
      \item Introducing new users to the Qt world and related resources
    \end{itemize}
    \item Teach experienced users how to use Qt efficiently
    \begin{itemize}
      \item Give advice on common practise
      \item Explain new Qt related technologies
      \item Demonstrate scalability of Qt technologies
    \end{itemize}
    \item Promote Qt and exchange ideas
    \begin{itemize}
      \item Collect user feedback from training sessions
      \item Feed back ideas into the development process
    \end{itemize}
  \end{enumerate}
\end{slide}

\begin{slide}
  \frametitle{Qt Training Partners and Certified Trainers}
  \begin{itemize}
    \item Training Partners
    \begin{itemize}
      \item Organizations actively engaged in the Qt community
      \item Propagade Qt to their customers
      \item Provide logistics for training events
    \end{itemize}
    \item Certified Trainers
    \begin{itemize}
      \item Know the ins and outs of UI design using Qt and/or other toolkits
      \item Keep themselves up-to-date about latest Qt technologies
      \item Keep track of new API revisions
    \end{itemize}
  \end{itemize}
\end{slide}

\begin{slide}
  \frametitle{Training Courses}
  \begin{itemize}
    \item Comprises of about 8-12 modules
    \item Each module takes about 90min, including 30min hands-on training
    \item Each module has about 4 thematic subsections and provides about 30 slides
    \item On average a speaker would talk 2 mins on each slide
    \item Depending on the trainer time spend on hands-on training can vary
  \end{itemize}
\end{slide}

\begin{slide}
  \frametitle{The Course Material}
  \begin{itemize}
    \item \LaTeX-Beamer
    \item Central Repository: git@gitorious.org:qt-training/course-material.git
    \item Accumulated training experience of Qt Training Partners and Nokia Qt Development Frameworks
    \item Qt has grown into a complete application framework
    \item The training material covers most of Qt's API, but maintainance is a major effort
    \item You can keep your private forks, but always have to refer to the original source
    \item If you want to contribute your changes, take a look at the Contributor's Guide
  \end{itemize}
\end{slide}
