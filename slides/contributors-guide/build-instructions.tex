%%%%%%%%%%%%%%%%%%%%%%%%%%%%%%%%%%%%%%%%%%%%%%%%%%%%%%%%%%%%%%%%%%%%%%%%%%
%
% Copyright (c) 2008-2011, Nokia Corporation and/or its subsidiary(-ies).
% All rights reserved.
%
% This work, unless otherwise expressly stated, is licensed under a
% Creative Commons Attribution-ShareAlike 2.5.
%
% The full license document is available from
% http://creativecommons.org/licenses/by-sa/2.5/legalcode .
%
%%%%%%%%%%%%%%%%%%%%%%%%%%%%%%%%%%%%%%%%%%%%%%%%%%%%%%%%%%%%%%%%%%%%%%%%%%

\subsection{Build Instructions}

\begin{slide}[fragile]
  \frametitle{Prequisites}
  \begin{itemize}
    \item Installing required packages on Debian/Ubuntu:
    \item[] \begin{shell}
apt-get install texlive-science texlive-pictures
apt-get install texlive-latex-extra texlive-formats-extra
apt-get install latex-beamer pdfjam transfig graphicsmagick
    \end{shell}
    \item[] {\footnotesize (Use TexLive 2009 for best compatibility.)}
    \item Obtain the source from gitorious:
    \item[] \begin{shell}
git clone git@gitorious.org:qt-training/course-material.git
    \end{shell}
  \end{itemize}
\end{slide}

\begin{slide}[fragile]
  \frametile{Building Training Modules}
  \begin{itemize}
    \item Building separate modules:
    \item[] \begin{shell}
cd course-material/slides
pdflatex fundamentals/fundamentals.tex
    \end{shell}
    {\footnotesize(For correct table of contents and page numbers call the above command a second time.)}
    \item Building entire courses:
    \item[] \begin{shell}
cd course-material/slides
pdflatex courses/essential-widgets.tex
    \end{shell}
  \end{itemize}
\end{slide}

\begin{slide}
  \frametitle{Customization}
  \begin{itemize}
    \item Customize the title page: slides/config.tex
    \item Customize the background:
    \begin{itemize}
      \item images/titlepage-background.png
      \item images/background.png
    \end{itemize}
    \item Change fonts and styles: slides/beamerthemeQtTraining.sty
  \end{itemize}
\end{slide}

\begin{slide}[fragile]
  \frametitle{Print Production}
  \begin{itemize}
    \item Beamer slides are generate in 4:3 format
    \item Size of a slide: 128mm x 96mm (5.04in x 3.78in)
    \item Creating handouts using pdfjam:
    \item[] \begin{shell}
pdfnup \
  --nup "2x2" \
  --offset ".0cm -.25cm" \
  --delta ".1cm .2cm" \
  --frame true \
  --scale 0.9 \
  contributors-guide.pdf --outfile contributors-guide-handout.pdf
    \end{shell}
    {\footnotesize (The handouts are meant to scale to both A4 and US Letter.)}
  \end{itemize}
\end{slide}
