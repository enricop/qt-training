%%%%%%%%%%%%%%%%%%%%%%%%%%%%%%%%%%%%%%%%%%%%%%%%%%%%%%%%%%%%%%%%%%%%%%%%%%
%
% Copyright (c) 2008-2011, Nokia Corporation and/or its subsidiary(-ies).
% All rights reserved.
%
% This work, unless otherwise expressly stated, is licensed under a
% Creative Commons Attribution-ShareAlike 2.5.
%
% The full license document is available from
% http://creativecommons.org/licenses/by-sa/2.5/legalcode .
%
%%%%%%%%%%%%%%%%%%%%%%%%%%%%%%%%%%%%%%%%%%%%%%%%%%%%%%%%%%%%%%%%%%%%%%%%%%

\subsection{Painting on Widgets}

\begin{slide}{1514}\frametitle{Low-level painting with QPainter}
  \begin{itemize}
  \item Paints on paint devices (\iCls{QPaintDevice})
  \item \iCls{QPaintDevice} implemented by
    \begin{itemize}
    \item On-Screen: \iCls{QWidget}
    \item Off-Screen: \iCls{QImage}, \iCls{QPixmap}
    \item And others ...
    \end{itemize}
  \item Provides drawing functions
    \begin{itemize}
    \item Lines, shapes, text or pixmaps
    \end{itemize}
  \item Controls
    \begin{itemize}
    \item Rendering quality
    \item Clipping
    \item Composition modes
    \end{itemize}
  \end{itemize}
\end{slide}

%----------------------------------------------------------------------
\begin{slide}[fragile]{1513}\frametitle{Painting on Widgets}
  \begin{itemize}
  \item Override \texttt{paintEvent(QPaintEvent*)}
  \item[]
    \begin{cpp}
void CustomWidget::paintEvent(QPaintEvent *) {
   QPainter painter(this);
   painter.drawRect(0,0,100,200); // x,y,w,h
}
    \end{cpp}
 \item Schedule painting
    \begin{itemize}
    \item \texttt{update()}: schedules paint event
    \item \texttt{repaint()}: repaints directly
    \end{itemize}
  \item Qt handles double-buffering
  \item To enable filling background:
    \begin{itemize}
    \item \texttt{QWidget::setAutoFillBackground(true)}
    \end{itemize}
 \end{itemize}
\end{slide}

%----------------------------------------------------------------------
\begin{slide}{1512}\frametitle{Coordinate System - Surface to render} 
  \begin{itemize}
  \item Controlled by QPainter
  \item Origin: Top-Left
 \item Rendering
    \begin{itemize}
    \item Logical - mathematical
    \item Aliased - right and below
    \item Anti-aliased - smoothing
   \end{itemize}
\end{itemize}
 \begin{center}
   \begin{tabular}{c c c}
     \image{painting/images/coordinatesystem-line} &
     \image{painting/images/coordinatesystem-line-raster} &
     \image{painting/images/coordinatesystem-line-antialias}
   \end{tabular}
   \begin{itemize}
   \item Rendering quality switch
     \begin{itemize}
     \item \iClsFn{QPainter}{setRenderHint}
     \end{itemize}
   \end{itemize}

 \end{center}
\end{slide}

%----------------------------------------------------------------------
\begin{slide}[fragile]{1012}\frametitle{Geometry Helper Classes}
\begin{itemize}
\item \texttt{QSize(w,h)}
  \begin{itemize}
  \item scale, transpose
  \end{itemize}
\item \texttt{QPoint(x,y)}
\item \texttt{QLine(point1, point2)}
  \begin{itemize}
  \item translate, dx, dy
  \end{itemize}
\flushedImage{painting/images/qrect-coordinates.png}
\item \texttt{QRect(point, size)}
  \begin{itemize}
  \item adjust, move
  \item translate, scale, center
  \end{itemize}
\item[] \begin{cpp}
QSize size(100,100);
QPoint point(0,0);
QRect rect(point, size);
rect.adjust(10,10,-10,-10);
QPoint center = rect.center();    
  \end{cpp}

\end{itemize}
\end{slide}                             
