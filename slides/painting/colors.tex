%%%%%%%%%%%%%%%%%%%%%%%%%%%%%%%%%%%%%%%%%%%%%%%%%%%%%%%%%%%%%%%%%%%%%%%%%%
%
% Copyright (c) 2008-2011, Nokia Corporation and/or its subsidiary(-ies).
% All rights reserved.
%
% This work, unless otherwise expressly stated, is licensed under a
% Creative Commons Attribution-ShareAlike 2.5.
%
% The full license document is available from
% http://creativecommons.org/licenses/by-sa/2.5/legalcode .
%
%%%%%%%%%%%%%%%%%%%%%%%%%%%%%%%%%%%%%%%%%%%%%%%%%%%%%%%%%%%%%%%%%%%%%%%%%%

\subsection{Color Handling}


%----------------------------------------------------------------------
\begin{slide}[fragile]{0927}\frametitle{Creating Color Values}
\xConcept{Colors}
\begin{itemize}
  \item Using different color models:
  \begin{itemize}
    \item \texttt{QColor(255,0,0)} // RGB
    \item \texttt{QColor::fromHsv(h,s,v)} // HSV
    \item \texttt{QColor::fromCmyk(c,m,y,k)} // CMYK
  \end{itemize}
  \item Defining colors:
  \item[] \begin{cpp}
QColor(255,0,0); // red in RGB
QColor(255,0,0, 63); // red 25% opaque (75% transparent)
QColor("#FF0000"); // red in web-notation
QColor("red"); // by svg-name
Qt::red; // predefined Qt global colors
    \end{cpp}
  \item Many powerful helpers for manipulating colors
  \item[] \begin{cpp}
QColor("black").lighter(150); // a shade of gray
    \end{cpp}
  \item \iCls{QColor} always refers to device color space
\end{itemize}
  \doc{qcolor.html\#details}{QColor Details}
\end{slide}

%----------------------------------------------------------------------
\begin{slide}[fragile]{1006}\frametitle{Drawing lines and outlines
    with QPen}
\begin{itemize}
  \item A pen (\iCls{QPen}) consists of:
    \begin{itemize}
    \item \textbf{a color or brush}
    \item \textbf{a width}
    \item \textbf{a style} (e.g. \texttt{NoPen} or \texttt{SolidLine})
    \item \textbf{a cap style} (i.e. line endings)
    \item \textbf{a join style} (connection of lines)
    \end{itemize}
  \item Activate with \iClsFn{QPainter}{setPen}. 
  \item[] \begin{cpp}
QPainter painter(this);
QPen pen = painter.pen();
pen.setBrush(Qt::red);
pen.setWidth(3);
painter.setPen(pen);          
// draw a rectangle with 3 pixel width red outline
painter.drawRect(0,0,100,100);
  \end{cpp}
  \end{itemize}
\end{slide}    

%----------------------------------------------------------------------
\begin{slide}[fragile]{1011}
\frametitle{The Outline}
\begin{block}{Rule}
The outline equals the size plus half the pen width on each side.
\end{block}

\begin{itemize}
\item For a pen of width 1:
\begin{cpp}
QPen pen(Qt::red, 1); // width = 1
float hpw = pen.widthF()/2; // half-pen width
QRectF rect(x,y,width,height);
QRectF outline = rect.adjusted(-hpw, -hpw, hpw, hpw);
\end{cpp}
\item  \textit{Due to integer rounding on a
    non-antialiased grid,
    the outline is shifted by 0.5 pixel towards the bottom right.}
\item \demo{painting/ex-rectoutline}
\end{itemize}
\end{slide}

%----------------------------------------------------------------------
\begin{slide}\frametitle{Coordinates and Pen Widths}
\begin{itemize}
\item Logical coordinates must map to a display of square pixels
  \begin{itemize}
  \item Half the pen width will be rendered on either side
  \item ... and rounded to the nearest pixel
  \end{itemize}
\item []
  \begin{tabular}{c c c}
  \image{painting/images/qrect-diagram-zero}
  \end{tabular}
\item Rectangles rendered with pens of width one, two, and three
\item []
  \begin{tabular}{c c c}
  \image{painting/images/qrectf-diagram-one}
  \image{painting/images/qrectf-diagram-two}
  \image{painting/images/qrectf-diagram-three}
  \end{tabular}
\end{itemize}
\end{slide}

%----------------------------------------------------------------------
\begin{slide}[fragile]{1007}\frametitle{Filling shapes with QBrush}
\begin{itemize}
  \item \iCls{QBrush} defines fill pattern of shapes
\flushedImageDoubleWidth{painting/images/brush}
  \item Brush configuration
    \begin{itemize}
    \item \texttt{setColor(color)}
    \item \texttt{setStyle(Qt::BrushStyle)}
      \begin{itemize}
      \item NoBrush, SolidPattern, ...
      \end{itemize}
    \item \texttt{QBrush(gradient)}  // QGradient's
    \item \texttt{setTexture(pixmap)}
   \end{itemize}
    \item Brush with solid red fill    
    \item[] \begin{cpp}
painter.setPen(Qt::red);
painter.setBrush(QBrush(Qt::yellow, Qt::SolidPattern));
painter.drawRect(rect);
    \end{cpp}
\end{itemize}
\end{slide}

%----------------------------------------------------------------------
\begin{slide}[fragile]{1009}
\frametitle{Drawing gradient fills}
\begin{itemize}                    
  \item Gradients used with \iCls{QBrush}
  \flushedImage{painting/images/gradients}
  \item Gradient types
  \begin{itemize}
  \item \iCls{QLinearGradient}
  \item \iCls{QConicalGradient}
  \item \iCls{QRadialGradient}
  \end{itemize} 
  \item Gradient from P1(0,0) to P2(100,100)
  \item[] \begin{cpp}   
QLinearGradient gradient(0, 0, 100, 100);
// position, color: position from 0..1
gradient.setColorAt(0, Qt::red); 
gradient.setColorAt(0.5, Qt::green);
gradient.setColorAt(1, Qt::blue);
painter.setBrush(gradient);
// draws rectangle, filled with brush
painter.drawRect(0, 0, 100, 100 );
  \end{cpp}                  
  \item \demo{painting/ex-gradients}
\end{itemize}
\end{slide}

%----------------------------------------------------------------------
\begin{slide}[fragile]{1010}\frametitle{Brush on \iCls{QPen}}
\begin{itemize}
\item Possible to set a brush on a pen
\item Strokes generated will be filled with the brush
\vfill
  \centeredImage{painting/images/pen-with-brush}
  \vfill
\item \demo{painting/ex-penwithbrush}
\end{itemize}
\end{slide}

%----------------------------------------------------------------------
\begin{slide}{0932}\frametitle{Color Themes and \iConcept{Palettes}}
\begin{itemize}
  \item To support widgets color theming
  \begin{itemize}
    \item \texttt{setColor(blue)} not recommended
    \item Colors needs to be managed
  \end{itemize} 
  \item \iCls{QPalette} manages colors
  \begin{itemize}
    \item Consist of color groups
  \end{itemize}     
\end{itemize}     
\vspace{5mm}
\begin{itemize}
  \item \texttt{enum QPalette::ColorGroup}
  \item Resemble widget states 
  \begin{itemize}
    \item \iClsEnum{QPalette}{Active} 
    \begin{itemize}    
    	\item Used for window with keyboard focus
    \end{itemize}    
    \item \iClsEnum{QPalette}{Inactive}
    \begin{itemize}    
    	\item Used for other windows
    \end{itemize}    
    \item \iClsEnum{QPalette}{Disabled}
    \begin{itemize}    
    	\item Used for disabled widgets
    \end{itemize}    
  \end{itemize}     
\end{itemize}
\end{slide}

%----------------------------------------------------------------------
\begin{slide}[fragile]{0934}\frametitle{Color Groups and Roles}
\begin{itemize}
  \item Color group consists of color roles
  \item enum \iClsEnum{QPalette}{ColorRole}
  \item Defines symbolic color roles used in UI
\end{itemize}
\centeredImage{painting/images/palette}
\begin{itemize}
  \item[] \begin{cpp}
QPalette pal = widget->palette();               
QColor color(Qt::red);
pal.setColor(QPalette::Active, QPalette::Window, color);
// for all groups
pal.setBrush(QPalette::Window, QBrush(Qt::red));
widget->setPalette(pal);
  \end{cpp}  
  \item \iClsFn{QApplication}{setPalette}
  \begin{itemize}
    \item Sets application wide default palette
  \end{itemize} 
\end{itemize}
\end{slide} 

%----------------------------------------------------------------------
\begin{slide}[fragile]{0934}\frametitle{Color Roles}
\begin{itemize}
\item Foreground and background roles
  \begin{itemize}
  \item Window, WindowText
  \item Base, AlternateBase, Text
  \item Button, ButtonText
  \item Highlight, HighlightedText
  \item ToolTipBase, ToolTipText
  \end{itemize}
\item 3D bevel roles
  \begin{itemize}
  \item Light, Midlight, Mid, Dark, Shadow
  \end{itemize}
\item Special roles
  \begin{itemize}
  \item Link, LinkVisited, BrightText
  \end{itemize}
\end{itemize}
\end{slide} 

