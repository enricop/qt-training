%%%%%%%%%%%%%%%%%%%%%%%%%%%%%%%%%%%%%%%%%%%%%%%%%%%%%%%%%%%%%%%%%%%%%%%%%%
%
% Copyright (c) 2008-2011, Nokia Corporation and/or its subsidiary(-ies).
% All rights reserved.
%
% This work, unless otherwise expressly stated, is licensed under a
% Creative Commons Attribution-ShareAlike 2.5.
%
% The full license document is available from
% http://creativecommons.org/licenses/by-sa/2.5/legalcode .
%
%%%%%%%%%%%%%%%%%%%%%%%%%%%%%%%%%%%%%%%%%%%%%%%%%%%%%%%%%%%%%%%%%%%%%%%%%%

\subsection{Painting Operations}

%----------------------------------------------------------------------
\begin{slide}\frametitle{Painter State}
\begin{itemize}
\item The painter has a state
  \begin{itemize}
  \item Context such as pen, brush, and font
  \item Settings such as render hints
  \item Coordinate transforms
  \item Clipping
  \end{itemize}
\item \iClsFn{QPainter}{save} - Pushes the current state onto a stack
\item \iClsFn{QPainter}{restore} - Pops the state back
\end{itemize}
\end{slide}                             


%----------------------------------------------------------------------
\begin{slide}{1521}\frametitle{Drawing Figures}
\flushedImage{painting/images/figures}
\begin{itemize}
\item Painter configuration
  \begin{itemize}
  \item pen width: 2
  \item pen color: red
  \item font size: 10
  \item brush color: yellow
  \item brush style: solid
  \end{itemize}

\item \demo{painting/ex-figures}
\end{itemize}
\end{slide}                             


%----------------------------------------------------------------------
\begin{slide}[fragile]{1520}\frametitle{Drawing Text}
  \begin{itemize}
  \item \texttt{QPainter::drawText(rect, flags, text)}
\flushedImageDoubleWidth{painting/images/drawtext.png}
  \item[] \begin{cpp}
QPainter painter(this);
painter.drawText(rect, Qt::AlignCenter, tr("Qt"));
painter.drawRect(rect);      
    \end{cpp}
 \item \iCls{QFontMetrics}
      \begin{itemize}
      \item calculate size of strings
      \end{itemize}
    \item[] \begin{cpp}
QFont font("times", 24);
QFontMetrics fm(font);
int pixelsWide = fm.width("Width of this text?");
int pixelsHeight = fm.height();        
      \end{cpp}
  \end{itemize}
\end{slide}

%----------------------------------------------------------------------
\begin{slide}[fragile]{1517}\frametitle{Painter Path - QPainterPath}
  \begin{itemize}
  \item Container for painting operations
  \item Enables reuse of shapes
\flushedImageDoubleWidth{painting/images/qpainterpath-construction}
  \item[] \begin{cpp}
QPainterPath path;
path.addRect(20, 20, 60, 60);
path.moveTo(0, 0);
path.cubicTo(99, 0,  50, 50,  99, 99);
path.cubicTo(0, 99,  50, 50,  0, 0);      
painter.drawPath(path);
    \end{cpp}
  \item Path information
    \begin{itemize}
    \item \hClsFn{QPainterPath}{controlPointRect} - rect containing all points
    \item \hClsFn{QPainterPath}{contains} - test if given shape is inside path
    \item \hClsFn{QPainterPath}{intersects} - test given shape intersects path
    \end{itemize}

  \item[] \qtdemo{examples/painting/painterpaths}

  \end{itemize}

\end{slide}

%----------------------------------------------------------------------
\begin{slide}{1519}\frametitle{Simple Transformations}
\flushedImage{painting/images/transform}
  \begin{itemize}
  \item Manipulating the coordinate system
    \begin{itemize}
    \item translate(x,y)
    \item scale(sx,sy)
    \item rotate(a)
    \item shear(sh,sv)
    \item reset()
    \end{itemize}
 \item[] \demo{painting/ex-transform}
  \end{itemize}
\end{slide}

%----------------------------------------------------------------------
\begin{slide}[fragile]{1518}\frametitle{Transform and Center}
  \begin{itemize}
  \item scale(sx, sy)
    \begin{itemize}
    \item scales around QPoint(0,0)
    \end{itemize}
  \item Same applies to all transform operations
  \item Scale around center?
\flushedImage{painting/images/scale-center}
  \item [] \begin{cpp}
painter.drawRect(r);
painter.translate(r.center());
painter.scale(sx,sy);
painter.translate(-r.center());
// draw center-scaled rect      
painter.drawRect(r); 
    \end{cpp}
 \item[] \demo{painting/ex-transform (scale center)}
  \end{itemize}
\end{slide}


%----------------------------------------------------------------------
\begin{slide}[fragile]\frametitle{Two-Dimensional Transforms}
\begin{itemize}
\item More complex transforms can be created using \iCls{QTransform}
  \begin{itemize}
  \item \iClsFn{QPainter}{transform} - getter
  \item \iClsFn{QPainter}{setTransform} - setter
  \end{itemize}
\item Transforms can be composed
\begin{cpp}
void drawRotatedText( QPainter *painter, float degrees,
int x, int y, const QString& text)
{
    // Precondition: painter->begin() has been called.
    QTransform oldTransform = painter->transform();
    QTransform transform = oldTransform;
    transform.translate( x, y );
    transform.rotate( degrees );
    painter->setTransform( transform );
    painter->drawText( 0, 0, text );
    painter->setTransform( oldTransform );
}
\end{cpp}
\end{itemize}
\end{slide}

%----------------------------------------------------------------------
\begin{slide}[fragile]\frametitle{Transformation Matrix}
\begin{itemize}
\item \iCls{QTransform} is a 3x3 matrix describing the transformation
\image{painting/images/qtransform-representation}
\item Not necessary to know transformation matrices
\item \iCls{translate()}, \iCls{rotate()}, \iCls{scale()}, \iCls{shear()} are conveniences
\item Perspective projections are also possible using \iCls{rotate()}
  \begin{itemize}
  \item \iClsFnPar{QTransform}{rotate}{double angle, Qt::Axis axis}
  \item The second optional parameter specifies the axis of rotation
  \item Z axis is "simple 2D rotation"
  \item Non-Z axis rotations are "perspective" projections
  \end{itemize}
\end{itemize}
\end{slide}

%----------------------------------------------------------------------
\begin{slide}\frametitle{Window and Viewport Transformations}
\begin{itemize}
\item The viewport represents the physical coordinates of a rectangle
\item A "window" describes this rectangle in logical coordinates.
\item Together they allow you to set custom coordinate systems
  \begin{itemize}
  \item \hClsFnPar{QPainter}{setWindow}{-100, -100, 200, 200}
  \item \hClsFnPar{QPainter}{setViewport}{cx-radius, cy-radius, radius*2, radius*2}
  \end{itemize}
\image{painting/images/analogclockwindow-viewport}
\end{itemize}
\end{slide}

%----------------------------------------------------------------------
\begin{slide}{1516}\frametitle{Other Painter Concepts}
  \begin{itemize}
  \item Clipping
    \begin{itemize}
    \item Clip drawing operation to shape
    \end{itemize}
    \flushedImage{painting/images/qpainter-compositiondemo}
  \item Composition modes:
    \begin{itemize}
    \item Rules for digital image compositing
    \item Combining pixels from source to destination
    \item[] \qtdemo{demos/composition}
    \end{itemize}
  \item Rubber Bands - \iCls{QRubberBand}
    \begin{itemize}
    \item Rectangle or line that indicate selection or boundary
    \item \doc{qrubberband.html}{QRubberband}
    \end{itemize}
  \end{itemize}  
\end{slide}

%----------------------------------------------------------------------
\begin{slide}{1515}\frametitle{Lab: Compass Widget}
\flushedImage{painting/images/compass}
\begin{itemize}
\item Implement a compass widget with dial and needle
\item Start with the stubbed version in the lab
\item Connect the compass widget to the \iCls{QSpinBox}
\item See the lab directions for details
\item \textit{Optional: Set the direction using mouse clicks on the compass}
\item[] \lab{painting/lab-compass}
\end{itemize}
\end{slide}

%----------------------------------------------------------------------
\begin{slide}{1515}\frametitle{Lab: Pie Chart Widget}
    \flushedImage{painting/images/piechart}
  \begin{itemize}
  \item Task to implement a pie chart
  \item Draw pies with painters based on data.
  \item Data Example: Population of 4 countries
    \begin{itemize}
    \item Sweden
    \item Germany
    \item Norway
    \item Italy
   \end{itemize}
  \item Guess the population in millions of citizens ;-)
  \item \textbf{Legend is optional}
  \item See lab description for details
  \item[] \lab{painting/lab-piechart}
  \end{itemize}
\end{slide}
