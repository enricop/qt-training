%%%%%%%%%%%%%%%%%%%%%%%%%%%%%%%%%%%%%%%%%%%%%%%%%%%%%%%%%%%%%%%%%%%%%%%%%%
%
% Copyright (c) 2008-2011, Nokia Corporation and/or its subsidiary(-ies).
% All rights reserved.
%
% This work, unless otherwise expressly stated, is licensed under a
% Creative Commons Attribution-ShareAlike 2.5.
%
% The full license document is available from
% http://creativecommons.org/licenses/by-sa/2.5/legalcode .
%
%%%%%%%%%%%%%%%%%%%%%%%%%%%%%%%%%%%%%%%%%%%%%%%%%%%%%%%%%%%%%%%%%%%%%%%%%%

\section{Painting and Styling}

%----------------------------------------------------------------------
\begin{slide}{1522}\frametitle{Custom painting or stylesheets}
  \begin{itemize}
  \item Painting
    \begin{itemize}
    \item You paint with a painter on a paint device during a paint event
    \item Qt widgets know how to paint themselves
    \item Often widgets look like we want
    \item Painting allows device independent 2D visualization
    \item Allows to draw pie charts, line charts and many more
    \end{itemize}
  \item StyleSheets
    \begin{itemize}
    \item Fine grained control over the look and feel
    \item Easily applied using style sheets in CSS format
    \end{itemize}
  \end{itemize}
\end{slide}

%----------------------------------------------------------------------
\begin{slide}{1217}
\label{painting-and-drawing}
\frametitle{Module Objectives}

Covers techniques for general 2D graphics and styling applications.
\begin{itemize}
\item \textbf{Painting}
  \begin{itemize}
  \item Painting infrastructure
  \item Painting on widget
  \end{itemize}
\item \textbf{Color Handling}
  \begin{itemize}
  \item Define and use colors
  \item Pens, Brushes, Palettes
  \end{itemize}
\item \textbf{Shapes}
  \begin{itemize}
  \item Drawing shapes
  \end{itemize}
\item \textbf{Transformation}
  \begin{itemize}
  \item 2D transformations of a coordinate system
  \end{itemize}
\item \textbf{Style Sheets} 
  \begin{itemize}
  \item How to make small customizations
  \item How to apply a theme to a widget or application
  \end{itemize}
% \item For GraphicsView intensive course you may skip Style Sheets
\end{itemize}

\end{slide}

%----------------------------------------------------------------------

%%%%%%%%%%%%%%%%%%%%%%%%%%%%%%%%%%%%%%%%%%%%%%%%%%%%%%%%%%%%%%%%%%%%%%%%%%
%
% Copyright (c) 2008-2011, Nokia Corporation and/or its subsidiary(-ies).
% All rights reserved.
%
% This work, unless otherwise expressly stated, is licensed under a
% Creative Commons Attribution-ShareAlike 2.5.
%
% The full license document is available from
% http://creativecommons.org/licenses/by-sa/2.5/legalcode .
%
%%%%%%%%%%%%%%%%%%%%%%%%%%%%%%%%%%%%%%%%%%%%%%%%%%%%%%%%%%%%%%%%%%%%%%%%%%

\subsection{Painting on Widgets}

\begin{slide}{1514}\frametitle{Low-level painting with QPainter}
  \begin{itemize}
  \item Paints on paint devices (\iCls{QPaintDevice})
  \item \iCls{QPaintDevice} implemented by
    \begin{itemize}
    \item On-Screen: \iCls{QWidget}
    \item Off-Screen: \iCls{QImage}, \iCls{QPixmap}
    \item And others ...
    \end{itemize}
  \item Provides drawing functions
    \begin{itemize}
    \item Lines, shapes, text or pixmaps
    \end{itemize}
  \item Controls
    \begin{itemize}
    \item Rendering quality
    \item Clipping
    \item Composition modes
    \end{itemize}
  \end{itemize}
\end{slide}

%----------------------------------------------------------------------
\begin{slide}[fragile]{1513}\frametitle{Painting on Widgets}
  \begin{itemize}
  \item Override \texttt{paintEvent(QPaintEvent*)}
  \item[]
    \begin{cpp}
void CustomWidget::paintEvent(QPaintEvent *) {
   QPainter painter(this);
   painter.drawRect(0,0,100,200); // x,y,w,h
}
    \end{cpp}
 \item Schedule painting
    \begin{itemize}
    \item \texttt{update()}: schedules paint event
    \item \texttt{repaint()}: repaints directly
    \end{itemize}
  \item Qt handles double-buffering
  \item To enable filling background:
    \begin{itemize}
    \item \texttt{QWidget::setAutoFillBackground(true)}
    \end{itemize}
 \end{itemize}
\end{slide}

%----------------------------------------------------------------------
\begin{slide}{1512}\frametitle{Coordinate System - Surface to render} 
  \begin{itemize}
  \item Controlled by QPainter
  \item Origin: Top-Left
 \item Rendering
    \begin{itemize}
    \item Logical - mathematical
    \item Aliased - right and below
    \item Anti-aliased - smoothing
   \end{itemize}
\end{itemize}
 \begin{center}
   \begin{tabular}{c c c}
     \image{painting/images/coordinatesystem-line} &
     \image{painting/images/coordinatesystem-line-raster} &
     \image{painting/images/coordinatesystem-line-antialias}
   \end{tabular}
   \begin{itemize}
   \item Rendering quality switch
     \begin{itemize}
     \item \iClsFn{QPainter}{setRenderHint}
     \end{itemize}
   \end{itemize}

 \end{center}
\end{slide}

%----------------------------------------------------------------------
\begin{slide}[fragile]{1012}\frametitle{Geometry Helper Classes}
\begin{itemize}
\item \texttt{QSize(w,h)}
  \begin{itemize}
  \item scale, transpose
  \end{itemize}
\item \texttt{QPoint(x,y)}
\item \texttt{QLine(point1, point2)}
  \begin{itemize}
  \item translate, dx, dy
  \end{itemize}
\flushedImage{painting/images/qrect-coordinates.png}
\item \texttt{QRect(point, size)}
  \begin{itemize}
  \item adjust, move
  \item translate, scale, center
  \end{itemize}
\item[] \begin{cpp}
QSize size(100,100);
QPoint point(0,0);
QRect rect(point, size);
rect.adjust(10,10,-10,-10);
QPoint center = rect.center();    
  \end{cpp}

\end{itemize}
\end{slide}                             

%%%%%%%%%%%%%%%%%%%%%%%%%%%%%%%%%%%%%%%%%%%%%%%%%%%%%%%%%%%%%%%%%%%%%%%%%%
%
% Copyright (c) 2008-2011, Nokia Corporation and/or its subsidiary(-ies).
% All rights reserved.
%
% This work, unless otherwise expressly stated, is licensed under a
% Creative Commons Attribution-ShareAlike 2.5.
%
% The full license document is available from
% http://creativecommons.org/licenses/by-sa/2.5/legalcode .
%
%%%%%%%%%%%%%%%%%%%%%%%%%%%%%%%%%%%%%%%%%%%%%%%%%%%%%%%%%%%%%%%%%%%%%%%%%%

\subsection{Color Handling}


%----------------------------------------------------------------------
\begin{slide}[fragile]{0927}\frametitle{Creating Color Values}
\xConcept{Colors}
\begin{itemize}
  \item Using different color models:
  \begin{itemize}
    \item \texttt{QColor(255,0,0)} // RGB
    \item \texttt{QColor::fromHsv(h,s,v)} // HSV
    \item \texttt{QColor::fromCmyk(c,m,y,k)} // CMYK
  \end{itemize}
  \item Defining colors:
  \item[] \begin{cpp}
QColor(255,0,0); // red in RGB
QColor(255,0,0, 63); // red 25% opaque (75% transparent)
QColor("#FF0000"); // red in web-notation
QColor("red"); // by svg-name
Qt::red; // predefined Qt global colors
    \end{cpp}
  \item Many powerful helpers for manipulating colors
  \item[] \begin{cpp}
QColor("black").lighter(150); // a shade of gray
    \end{cpp}
  \item \iCls{QColor} always refers to device color space
\end{itemize}
  \doc{qcolor.html\#details}{QColor Details}
\end{slide}

%----------------------------------------------------------------------
\begin{slide}[fragile]{1006}\frametitle{Drawing lines and outlines
    with QPen}
\begin{itemize}
  \item A pen (\iCls{QPen}) consists of:
    \begin{itemize}
    \item \textbf{a color or brush}
    \item \textbf{a width}
    \item \textbf{a style} (e.g. \texttt{NoPen} or \texttt{SolidLine})
    \item \textbf{a cap style} (i.e. line endings)
    \item \textbf{a join style} (connection of lines)
    \end{itemize}
  \item Activate with \iClsFn{QPainter}{setPen}. 
  \item[] \begin{cpp}
QPainter painter(this);
QPen pen = painter.pen();
pen.setBrush(Qt::red);
pen.setWidth(3);
painter.setPen(pen);          
// draw a rectangle with 3 pixel width red outline
painter.drawRect(0,0,100,100);
  \end{cpp}
  \end{itemize}
\end{slide}    

%----------------------------------------------------------------------
\begin{slide}[fragile]{1011}
\frametitle{The Outline}
\begin{block}{Rule}
The outline equals the size plus half the pen width on each side.
\end{block}

\begin{itemize}
\item For a pen of width 1:
\begin{cpp}
QPen pen(Qt::red, 1); // width = 1
float hpw = pen.widthF()/2; // half-pen width
QRectF rect(x,y,width,height);
QRectF outline = rect.adjusted(-hpw, -hpw, hpw, hpw);
\end{cpp}
\item  \textit{Due to integer rounding on a
    non-antialiased grid,
    the outline is shifted by 0.5 pixel towards the bottom right.}
\item \demo{painting/ex-rectoutline}
\end{itemize}
\end{slide}

%----------------------------------------------------------------------
\begin{slide}[fragile]{1007}\frametitle{Filling shapes with QBrush}
\begin{itemize}
  \item \iCls{QBrush} defines fill pattern of shapes
\flushedImageDoubleWidth{painting/images/brush}
  \item Brush configuration
    \begin{itemize}
    \item \texttt{setColor(color)}
    \item \texttt{setStyle(Qt::BrushStyle)}
      \begin{itemize}
      \item NoBrush, SolidPattern, ...
      \end{itemize}
    \item \texttt{QBrush(gradient)}  // QGradient's
    \item \texttt{setTexture(pixmap)}
   \end{itemize}
    \item Brush with solid red fill    
    \item[] \begin{cpp}
painter.setPen(Qt::red);
painter.setBrush(QBrush(Qt::yellow, Qt::SolidPattern));
painter.drawRect(rect);
    \end{cpp}
\end{itemize}
\end{slide}

%----------------------------------------------------------------------
\begin{slide}[fragile]{1009}
\frametitle{Drawing gradient fills}
\begin{itemize}                    
  \item Gradients used with \iCls{QBrush}
  \flushedImage{painting/images/gradients}
  \item Gradient types
  \begin{itemize}
  \item \iCls{QLinearGradient}
  \item \iCls{QConicalGradient}
  \item \iCls{QRadialGradient}
  \end{itemize} 
  \item Gradient from P1(0,0) to P2(100,100)
  \item[] \begin{cpp}   
QLinearGradient gradient(0, 0, 100, 100);
// position, color: position from 0..1
gradient.setColorAt(0, Qt::red); 
gradient.setColorAt(0.5, Qt::green);
gradient.setColorAt(1, Qt::blue);
painter.setBrush(gradient);
// draws rectangle, filled with brush
painter.drawRect(0, 0, 100, 100 );
  \end{cpp}                  
  \item \demo{painting/ex-gradients}
\end{itemize}
\end{slide}

%----------------------------------------------------------------------
\begin{slide}[fragile]{1010}\frametitle{Brush on \iCls{QPen}}
\begin{itemize}
\item Possible to set a brush on a pen
\item Strokes generated will be filled with the brush
\vfill
  \centeredImage{painting/images/pen-with-brush}
  \vfill
\item \demo{painting/ex-penwithbrush}
\end{itemize}
\end{slide}

%----------------------------------------------------------------------
\begin{slide}{0932}\frametitle{Color Themes and \iConcept{Palettes}}
\begin{itemize}
  \item To support widgets color theming
  \begin{itemize}
    \item \texttt{setColor(blue)} not recommended
    \item Colors needs to be managed
  \end{itemize} 
  \item \iCls{QPalette} manages colors
  \begin{itemize}
    \item Consist of color groups
  \end{itemize}     
\end{itemize}     
\vspace{5mm}
\begin{itemize}
  \item \texttt{enum QPalette::ColorGroup}
  \item Resemble widget states 
  \begin{itemize}
    \item \iClsEnum{QPalette}{Active} 
    \begin{itemize}    
    	\item Used for window with keyboard focus
    \end{itemize}    
    \item \iClsEnum{QPalette}{Inactive}
    \begin{itemize}    
    	\item Used for other windows
    \end{itemize}    
    \item \iClsEnum{QPalette}{Disabled}
    \begin{itemize}    
    	\item Used for disabled widgets
    \end{itemize}    
  \end{itemize}     
\end{itemize}
\end{slide}

%----------------------------------------------------------------------
\begin{slide}[fragile]{0934}\frametitle{Color Groups and Roles}
\begin{itemize}
  \item Color group consists of color roles
  \item enum \iClsEnum{QPalette}{ColorRole}
  \item Defines symbolic color roles used in UI
\end{itemize}
\centeredImage{painting/images/palette}
\begin{itemize}
  \item[] \begin{cpp}
QPalette pal = widget->palette();               
QColor color(Qt::red);
pal.setColor(QPalette::Active, QPalette::Window, color);
// for all groups
pal.setBrush(QPalette::Window, QBrush(Qt::red));
widget->setPalette(pal);
  \end{cpp}  
  \item \iClsFn{QApplication}{setPalette}
  \begin{itemize}
    \item Sets application wide default palette
  \end{itemize} 
\end{itemize}
\end{slide}  
 
%%%%%%%%%%%%%%%%%%%%%%%%%%%%%%%%%%%%%%%%%%%%%%%%%%%%%%%%%%%%%%%%%%%%%%%%%%
%
% Copyright (c) 2008-2011, Nokia Corporation and/or its subsidiary(-ies).
% All rights reserved.
%
% This work, unless otherwise expressly stated, is licensed under a
% Creative Commons Attribution-ShareAlike 2.5.
%
% The full license document is available from
% http://creativecommons.org/licenses/by-sa/2.5/legalcode .
%
%%%%%%%%%%%%%%%%%%%%%%%%%%%%%%%%%%%%%%%%%%%%%%%%%%%%%%%%%%%%%%%%%%%%%%%%%%

\subsection{Painting Operations}

%----------------------------------------------------------------------
\begin{slide}{1521}\frametitle{Drawing Figures}
\flushedImage{painting/images/figures}
\begin{itemize}
\item Painter configuration
  \begin{itemize}
  \item pen width: 2
  \item pen color: red
  \item font size: 10
  \item brush color: yello
  \item brush style: solid
  \end{itemize}

\item \demo{painting/ex-figures}
\end{itemize}
\end{slide}                             


%----------------------------------------------------------------------
\begin{slide}[fragile]{1520}\frametitle{Drawing Text}
  \begin{itemize}
  \item \texttt{QPainter::drawText(rect, flags, text)}
\flushedImageDoubleWidth{painting/images/drawtext.png}
  \item[] \begin{cpp}
QPainter painter(this);
painter.drawText(rect, Qt::AlignCenter, tr("Qt"));
painter.drawRect(rect);      
    \end{cpp}
 \item \iCls{QFontMetrics}
      \begin{itemize}
      \item calculate size of strings
      \end{itemize}
    \item[] \begin{cpp}
QFont font("times", 24);
QFontMetrics fm(font);
int pixelsWide = fm.width("Width of this text?");
int pixelsHeight = fm.height();        
      \end{cpp}
  \end{itemize}
\end{slide}

%----------------------------------------------------------------------
\begin{slide}{1519}\frametitle{Transformation}
\flushedImage{painting/images/transform}
  \begin{itemize}
  \item Manipulating the coordinate system
    \begin{itemize}
    \item translate(x,y)
    \item scale(sx,sy)
    \item rotate(a)
    \item shear(sh,sv)
    \item reset()
    \end{itemize}
 \item[] \demo{painting/ex-transform}
  \end{itemize}
\end{slide}

%----------------------------------------------------------------------
\begin{slide}[fragile]{1518}\frametitle{Transform and Center}
  \begin{itemize}
  \item scale(sx, sy)
    \begin{itemize}
    \item scales around QPoint(0,0)
    \end{itemize}
  \item Same applies to all transform operations
  \item Scale around center?
\flushedImage{painting/images/scale-center}
  \item [] \begin{cpp}
painter.drawRect(r);
painter.translate(r.center());
painter.scale(sx,sy);
painter.translate(-r.center());
// draw center-scaled rect      
painter.drawRect(r); 
    \end{cpp}
 \item[] \demo{painting/ex-transform (scale center)}
  \end{itemize}
\end{slide}


%----------------------------------------------------------------------
\begin{slide}[fragile]{1517}\frametitle{Painter Path - QPainterPath}
  \begin{itemize}
  \item Container for painting operations
  \item Enables reuse of shapes
\flushedImageDoubleWidth{painting/images/qpainterpath-construction}
  \item[] \begin{cpp}
QPainterPath path;
path.addRect(20, 20, 60, 60);
path.moveTo(0, 0);
path.cubicTo(99, 0,  50, 50,  99, 99);
path.cubicTo(0, 99,  50, 50,  0, 0);      
painter.drawPath(path);
    \end{cpp}
  \item Path information
    \begin{itemize}
    \item \hClsFn{QPainterPath}{controlPointRect} - rect containing all points
    \item \hClsFn{QPainterPath}{contains} - test if given shape is inside path
    \item \hClsFn{QPainterPath}{intersects} - test given shape intersects path
    \end{itemize}

  \item[] \qtdemo{examples/painting/painterpaths}

  \end{itemize}

\end{slide}

%----------------------------------------------------------------------
\begin{slide}{1516}\frametitle{Other Painter Concepts}
  \begin{itemize}
  \item Clipping
    \begin{itemize}
    \item Clip drawing operation to shape
    \end{itemize}
    \flushedImage{painting/images/qpainter-compositiondemo}
  \item Composition modes:
    \begin{itemize}
    \item Rules for digital image compositing
    \item Combining pixels from source to destination
    \item[] \qtdemo{demos/composition}
    \end{itemize}
  \item Rubber Bands - \iCls{QRubberBand}
    \begin{itemize}
    \item Rectangle or line that indicate selection or boundary
    \item \doc{qrubberband.html}{QRubberband}
    \end{itemize}
  \end{itemize}  
\end{slide}

\begin{slide}{1515}\frametitle{Lab: Pie Chart Widget}
    \flushedImage{painting/images/piechart}
  \begin{itemize}
  \item Task to implement a pie chart
  \item Draw pies with painters based on data.
  \item Data Example: Population of 4 countries
    \begin{itemize}
    \item Sweden
    \item Germany
    \item Norway
    \item Italy
   \end{itemize}
  \item Guess the population in millions of citizens ;-)
  \item \textbf{Legend is optional}
  \item See lab description for details
  \item[] \lab{painting/lab-piechart}
  \end{itemize}
\end{slide}

%%%%%%%%%%%%%%%%%%%%%%%%%%%%%%%%%%%%%%%%%%%%%%%%%%%%%%%%%%%%%%%%%%%%%%%%%%
%
% Copyright (c) 2008-2011, Nokia Corporation and/or its subsidiary(-ies).
% All rights reserved.
%
% This work, unless otherwise expressly stated, is licensed under a
% Creative Commons Attribution-ShareAlike 2.5.
%
% The full license document is available from
% http://creativecommons.org/licenses/by-sa/2.5/legalcode .
%
%%%%%%%%%%%%%%%%%%%%%%%%%%%%%%%%%%%%%%%%%%%%%%%%%%%%%%%%%%%%%%%%%%%%%%%%%%

\subsection{Style Sheets}\label{style_sheets}

%----------------------------------------------------------------------
\begin{slide}{0450}\frametitle{Qt Style Sheets}
\begin{itemize}
\item Mechanism to customize appearance of widgets
  \begin{itemize}
  \item Alternative to subclassing \texttt{QStyle}
  \end{itemize}
\item Inspired by HTML CSS
\item Textual specifications of styles
\flushedImage{painting/images/style-sheets-simple}
\item Applying Style Sheets
  \begin{itemize}
  \item \texttt{QApplication::setStyleSheet(sheet)}
    \begin{itemize}
    \item On whole application
    \end{itemize}
  \item \texttt{QWidget::setStyleSheet(sheet)}
    \begin{itemize}
    \item On a specific widget (incl. child widgets)
    \end{itemize}
 \end{itemize}
\item Meant for restyling a few widgets
  \begin{itemize}
  \item ...not for theming the entire application
  \end{itemize}
\item[] \demo{painting/ex-simpleqss}
\end{itemize}
\end{slide}

%----------------------------------------------------------------------
\begin{slide}[fragile]{0452}\frametitle{CSS Rules}
\begin{block}{CSS Rule}
selector \{ property : value; property : value \}    
  \end{block}
\begin{itemize}
\item Selector: specifies the  widgets
\item Property/value pairs: specify properties to change.
\item[] \begin{cpp}
QPushButton {color:red; background-color:white}    
  \end{cpp}
\item Examples of stylable elements
  \begin{itemize}
  \item Colors, fonts, pen style, alignment.
  \item Background images.
  \item Position and size of sub controls.
  \item Border and padding of the widget itself.
  \end{itemize}
\item Reference of stylable elements
\item[] \doc{stylesheet-reference.html}{Qt Style Sheets Reference}
\end{itemize}
\end{slide}

%----------------------------------------------------------------------
\begin{slide}[fragile]{0454}\frametitle{The Box Model}
\flushedImage{painting/images/stylesheet-boxmodel}
\begin{itemize}
\item Every widget treated as box
\item Four concentric rectangles
  \begin{itemize}
  \item Margin, Border, Padding, Content
  \end{itemize}
\item Customizing QPushButton
  \item[] \begin{cpp}
QPushButton {
  border-width: 2px;
  border-radius: 10px;
  padding: 6px;
  // ...
 }    
  \end{cpp}
\flushedImage{painting/images/stylesheet-redbutton}
\item By default, \texttt{margin}, \texttt{border-width}, and \texttt{padding} are 0
\end{itemize}
\end{slide}


%----------------------------------------------------------------------
\begin{slide}{0455}\frametitle{Selector Types}
\begin{itemize}
\item \texttt{*\{ \}} // Universal selector
  \begin{itemize}
  \item All widgets
  \end{itemize}
\item \texttt{QPushButton \{ \}} // Type Selector
  \begin{itemize}
  \item All instances of class
  \end{itemize}
\item\texttt{.QPushButton \{ \}} // Class Selector
  \begin{itemize}
  \item All instances of class, but not subclasses
  \end{itemize}
\item \texttt{QPushButton\#objectName} // ID Selector
  \begin{itemize}
  \item All Instances of class with objectName
  \end{itemize}
\item\texttt{QDialog QPushButton \{ \} } // Descendant Selector
  \begin{itemize}
  \item All instances of QPushButton which are child of QDialog
  \end{itemize}
\item QPushButton[enabled="true"] // Property Selector
  \begin{itemize}
  \item All instances of class which match property
  \end{itemize}
\end{itemize}
\end{slide}

%----------------------------------------------------------------------
\begin{slide}{0457}\frametitle{Selector Details}
\begin{itemize}
\item Property Selector
  \begin{itemize}
  \item If property changes it is required to re-set style sheet
  \end{itemize}
\item Combining Selectors
  \begin{itemize}
  \item \texttt{QLineEdit, QComboBox, QPushButton \{ color:\ red \}}
  \end{itemize}
\item Pseudo-States
  \begin{itemize}
  \item Restrict selector based on widget's state
  \item Example: \texttt{QPushButton:hover \{color:red\}}
  \end{itemize}
\item \demo{painting/ex-qssselector}
\item Selecting Subcontrols
  \begin{itemize}
  \item Access subcontrols of complex widgets such as QComboBox, QSpinBox, etc.
  \item \texttt{QComboBox::drop-down \{ image: url(dropdown.png) \}}
 \end{itemize}
  \item Subcontrols positioned relative to other elements
    \begin{itemize}
  \item Change using \texttt{subcontrol-origin} and \texttt{subcontrol-position}
  \end{itemize}

\end{itemize}
\end{slide}


%----------------------------------------------------------------------
\begin{slide}[fragile]{0460}\frametitle{Conflict Resolution - Cascading}\label{style_sheet_conflict_resolution}
  \begin{itemize}
  \item Effective style sheet obtained by merging
    \begin{enumerate}
    \item Widgets's ancestor (parent, grandparent, etc.)
    \item Application stylesheet
    \end{enumerate}
  \item On conflict: widget own style sheet preferred
  \item[]
    \begin{cpp}
qApp->setStyleSheet("QPushButton { color: white }");
button->setStyleSheet("* { color: blue }");
    \end{cpp}
  \item Style on button forces button to have blue text
    \begin{itemize}
    \item In spite of more specific application rule
    \end{itemize}
\item[] \demo{painting/ex-qsscascading}
\end{itemize}
\end{slide}

%----------------------------------------------------------------------
\begin{slide}[fragile]{0461}\frametitle{Conflict Resolution - Selector Specifity}
\begin{itemize}
\item Conflict: When rules on same level specify same property
  \begin{itemize}
  \item Specificity of selectors apply
\item[]
  \begin{cpp}
QPushButton:hover { color: white }
QPushButton { color: red }    
  \end{cpp}
\item Selectors with pseudo-states are more specific
  \end{itemize}
\item Follows the CSS2 rules for specificity
  \begin{itemize}
  \item [] \externalLink{http://www.w3.org/TR/REC-CSS2/cascade.html}
  \end{itemize}
  
\item[] \demo{painting/ex-qssconflict}
\end{itemize}
\end{slide}


%----------------------------------------------------------------------
\begin{slide}{0462}\frametitle{Qt Designer Integration}
\flushedImage{painting/images/designer-stylesheet-options}
\begin{itemize}
\item Excellent tool to preview style sheets
\item Right-click on any widget
  \begin{itemize}
  \item Select \textit{Change styleSheet..}
  \end{itemize}
\item Includes syntax highlighter and validator
\item[] \demo{Editing Style Sheets in Designer}
\end{itemize}
\end{slide}

%----------------------------------------------------------------------
\begin{slide}{0463}\frametitle{Project Task}
\flushedImage{painting/images/stylesheet-project}
\begin{itemize}
\item Tasks
  \begin{itemize}
  \item Investigate style sheet
  \item Modify style sheet
  \item Remove style sheet \\ and implement your own
 \end{itemize}
\item Example does not save changes. \\ Use designer for this.
\item Edit style sheet using \\ \texttt{File -> Edit StyleSheet}
\item[] \lab{\$QTDIR/examples/widgets/stylesheet}
\end{itemize}
\end{slide}

