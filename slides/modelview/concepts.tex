%%%%%%%%%%%%%%%%%%%%%%%%%%%%%%%%%%%%%%%%%%%%%%%%%%%%%%%%%%%%%%%%%%%%%%%%%%
%
% Copyright (c) 2008-2011, Nokia Corporation and/or its subsidiary(-ies).
% All rights reserved.
%
% This work, unless otherwise expressly stated, is licensed under a
% Creative Commons Attribution-ShareAlike 2.5.
%
% The full license document is available from
% http://creativecommons.org/licenses/by-sa/2.5/legalcode .
%
%%%%%%%%%%%%%%%%%%%%%%%%%%%%%%%%%%%%%%%%%%%%%%%%%%%%%%%%%%%%%%%%%%%%%%%%%%

\subsection{Model/View Concept}


% ----------------------------------------------------------------------
\begin{slide}{1602}\frametitle{Why Model/View?}
  \begin{itemize}
  \item \textbf{Isolated domain-logic}
    \begin{itemize}
    \item From input and presentation
    \end{itemize}
  \item \textbf{Makes Components Independent}
    \begin{itemize}
    \item For Development
    \item For Testing
    \item For Maintenance
    \end{itemize}
  \item \textbf{Foster Component Reuse}
    \begin{itemize}
    \item Reuse of Presentation Logic
    \item Reuse of Domain Model
    \end{itemize}
  \end{itemize}
\end{slide}

%----------------------------------------------------------------------
\begin{slide}{1016}\frametitle{Model/View-Components}\label{model_view_concepts}
\xConcept{Model/View!Concept}
% \oooFlushedImage{model-view-overview}
\centeredImageDoubleWidth{modelview/images/modelview-overview}\\
\medskip
\demo{modelview/ex-simple}
  
\end{slide}

%----------------------------------------------------------------------
\begin{slide}{1601}\frametitle{Model Structures}
 \begin{itemize}
 \item \textbf{List} - One-Dimensional
    \begin{itemize}
    \item Rows
   \end{itemize}
  \item \textbf{Table} - Two-Dimensional
    \begin{itemize}
    \item Rows
    \item Columns
    \end{itemize}
  \item \textbf{Tree} - Three-Dimensional
    \begin{itemize}
    \item Rows
    \item Columns
    \item Parent/Child
    \end{itemize}
  \end{itemize}
\medskip
\centeredImage{modelview/images/modelview-models}
\end{slide}


%----------------------------------------------------------------------
\begin{slide}{1600}\frametitle{Display the Structure - View Classes}
\begin{itemize}
 \item \textbf{QtQuick ItemView}
   \begin{itemize}
   \item Abstract base class for scrollable views
   \end{itemize}
\item \textbf{QtQuick ListView}
  \begin{itemize}
  \item Items of data in a list
  \end{itemize}
\item \textbf{QtQuick GridView}
  \begin{itemize}
  \item Items of data in a grid
  \end{itemize}
\item \textbf{QtQuick PathView}
  \begin{itemize}
  \item Items of data along a specified path
  \end{itemize}
  \centeredImage{qml-presenting-data/images/standard-quick-views}
\end{itemize}
\end{slide}

%----------------------------------------------------------------------
\begin{slide}{1599}\frametitle{Adapts the Data - Model Classes}
  \begin{itemize}
\flushedImageDoubleWidth{modelview/images/modelclasses}
  \item \textbf{\iCls{QAbstractItemModel}}
    \begin{itemize}
    \item Abstract interface of models
   \end{itemize}
  \item \textbf{Abstract Item Models}
    \begin{itemize}
    \item Implement to use
 \end{itemize}
  \item \textbf{Ready-Made Models}
    \begin{itemize}
    \item Convenient to use
  \end{itemize}
 \item \textbf{Proxy Models}
   \begin{itemize}
   \item Reorder/filter/sort your items
  \end{itemize}
   \item[] \doc{model-view-model.html}{Model Classes}
 \end{itemize}
\end{slide}

% ----------------------------------------------------------------------
\begin{slide}{1598}
  \frametitle{Data - Model - View Relationships}
  \begin{itemize}
 \item \textbf{Standard Item Model}
\flushedImage{modelview/images/datamodelview2}
    \begin{itemize}
    \item Data+Model combined
    \item View is separated
    \item Model is your data
    \end{itemize}
 \end{itemize}
 \begin{itemize}
 \item \textbf{Custom Item Models}
\flushedImage{modelview/images/datamodelview3}
    \begin{itemize}
    \item Model is adapter to data
    \item View is separated
    \end{itemize}
 \end{itemize}
\end{slide}



% ----------------------------------------------------------------------
\begin{slide}{1597}\frametitle{Addressing Data - \iCls{QModelIndex}}
  \begin{itemize}
  \item Refers to item in model
  \item Contains all information to specify location
  \item Located in given row and column
    \begin{itemize}
    \item May have a parent index
    \end{itemize}
    \medskip
  \item \textbf{QModelIndex API}
    \begin{itemize}
    \item \texttt{row()} - row index refers to
    \item \texttt{column()} - column index refers to
    \item \texttt{parent()} - parent of index
      \begin{itemize}
      \item or \texttt{QModelIndex()} if no parent
      \end{itemize}
    \item \texttt{isValid()}
      \begin{itemize}
      \item Valid index belongs to a model
      \item Valid index has non-negative row and column numbers
      \end{itemize}
    \item \texttt{model()} - the model index refers to
    \item \texttt{data( role )} - data for given role
    \end{itemize}
  \end{itemize}
\end{slide}

% ----------------------------------------------------------------------
\begin{slide}[fragile]{1596}\frametitle{\iCls{QModelIndex} in Table/Tree
  Structures}
\flushedImage{modelview/images/modelview-tablemodel}
  \begin{itemize}
  \item \textbf{Rows and columns}
    \begin{itemize}
    \item Item location in table model
    \item Item has no parent (parent.isValid() == false)
   \end{itemize}
   \begin{cpp}
indexA = model->index(0, 0, QModelIndex());
indexB = model->index(1, 1, QModelIndex());
indexC = model->index(2, 1, QModelIndex());      
    \end{cpp}
\flushedImage{modelview/images/modelview-treemodel}
\medskip
\item \textbf{Parents, rows, and columns}
  \begin{itemize}
  \item Item location in tree model
  \end{itemize}

    \begin{cpp}
indexA = model->index(0, 0, QModelIndex());
indexC = model->index(2, 1, QModelIndex());
// asking for index with given row, column and parent
indexB = model->index(1, 0, indexA);      
    \end{cpp}
  \end{itemize}
\doc{model-view-model.html\#model-indexes}{Model Indexes}
  
\end{slide}


% ----------------------------------------------------------------------
\begin{slide}[fragile]{1595}\frametitle{Item and Item Roles}
\flushedImage{modelview/images/modelview-roles}
  \begin{itemize}
  \item \textbf{Item performs various roles}
    \begin{itemize}
    \item for other components (delegate, view, ...)
    \end{itemize}
  \item \textbf{Supplies different data}
    \begin{itemize}
    \item for different situations
    \end{itemize}
  \item \textbf{Example:}
    \begin{itemize}
    \item \iClsEnum{Qt}{DisplayRole} used displayed string in view
    \end{itemize}
  \item \textbf{Asking for data}
    \begin{cpp}
QVariant value = model->data(index, role);      
// Asking for display text
QString text = model->data(index, Qt::DisplayRole).toString()        
    \end{cpp}
  \item \textbf{Standard roles}
    \begin{itemize}
    \item Defined by \iClsEnum{Qt}{ItemDataRole}
    \item \doc{qt.html\#ItemDataRole-enum}{enum Qt::ItemDataRole}
    \end{itemize}
  \end{itemize}
\end{slide}

% ----------------------------------------------------------------------
\begin{slide}[fragile]{}\frametitle{Exporting models to QML}
  \begin{itemize}
    \item Export model instance
    \begin{itemize}
        \item Create model instance in C++
        \item Set as a context property on the view's engine
        \item[]
        \begin{cpp}
CustomModel *model = new CustomModel;        
QQuickView view;
view.engine()->rootContext("_model", model);
        \end{cpp}
        \item Use in QML by id
        \item[]
        \begin{qml}
\qtt{\qc{class}{ListView}~\{~\qc{type}{model}:~\_model~\}}\\
        \end{qml}
    \end{itemize}
    \vspace*{1.0em}
    \item Export model type
    \begin{itemize}
        \item Register custom model class with QML type system
        \item[]
        \begin{cpp}
qmlRegisterType<CustomModel>("Models", 1, 0, "CustomModel");        
        \end{cpp}
        \item Use in QML like any other QML element
        \item[]
        \begin{qml}
\qtt{\qc{keyword}{import}~\qc{class}{Models}~\qc{number}{1.0}}\\
\qtt{\qc{class}{ListView}~\{}\\
\qtt{~~~~\qc{type}{model}:~\qc{class}{CustomModel}~\{\}}\\
\qtt{\}}\\
        \end{qml}
    \end{itemize}
  \end{itemize}
\end{slide}

% ----------------------------------------------------------------------
\begin{slide}[fragile]{}\frametitle{Mapping Item Roles from/to QML}
  \begin{itemize}
      \item Item Roles in C++
      \item[]
      \begin{cpp}
// Asking for display text
QString text = model->data(index, Qt::DisplayRole).toString()
      \end{cpp}
      \vspace*{1.0em}
      \item Item properties in QML
      \item[]
      \begin{qml}
\qtt{\qc{type}{onCurrentIndexChanged}:~\{}\\
\qtt{~~~~var text = model.get(index).display}\\
\qtt{\}}\\
      \end{qml}
      \item Default mappings
      \begin{itemize}
          \item \iClsEnum{Qt}{DisplayRole} in C++ is \qc{type}{display} in QML
          \item \iClsEnum{Qt}{DecorationRole} in C++ is \qc{type}{decoration} in QML
      \end{itemize}
      \item Add additional mappings by reimplementing \iClsFn{QAbstractItemModel}{roleNames}
  \end{itemize}
\end{slide}

% ----------------------------------------------------------------------
\begin{slide}{1594}\frametitle{Recap of Model/View Concept}
\begin{columns}
  \begin{column}{50mm}
    \begin{itemize}
    \item \textbf{Model Structures}
      \begin{itemize}
      \item List, Table and Tree
      \end{itemize}
    \item \textbf{Components}
      \begin{itemize}
      \item Model - Adapter to Data
      \item View - Displays Structure
      \item Delegate - Paints Item
      \item Index - Location in Model
      \end{itemize}
    \item \textbf{Views}
      \begin{itemize}
      \item \iCls{ListView}
      \item \iCls{GridView}
      \item \iCls{PathView}
      \end{itemize}
    \end{itemize}
  \end{column}
  \begin{column}{55mm}
    \begin{itemize}
    \item \textbf{Models}
      \begin{itemize}
      \item \iCls{QAbstractItemModel}
      \item Other Abstract Models
      \item Ready-Made Models
      \item Proxy Models
      \end{itemize}
    \item \textbf{Index}
      \begin{itemize}
      \item \texttt{row(),column(),parent()}
      \item \texttt{data( role )}
      \item \texttt{model()}
      \end{itemize}
    \item \textbf{Item Role}
      \begin{itemize}
      \item \iClsEnum{Qt}{DisplayRole}
      \item Standard Roles in \iClsEnum{Qt}{ItemDataRoles}
      \end{itemize}
   \end{itemize}
  \end{column}
\end{columns}
\end{slide}


%----------------------------------------------------------------------
\begin{slide}[fragile]{1019}\frametitle{Things you may want to customize}
\textbf{Models} - \iCls{QAbstractItemModel}
\begin{itemize}
\item If model is your data
  \begin{itemize}
  \item \iCls{QStandardItemModel}
  \end{itemize}
\item Otherwise
  \begin{itemize}
  \item Adapt own data by subclassing a model
  \end{itemize}
\end{itemize}
\textbf{Delegate} - \iCls{QtQuick Component}
\begin{itemize}
\item Encapsulation of QtQuick elements
\item Even complex element trees possible
  \begin{itemize}
  \item To control display/edit of data
  \end{itemize}
\end{itemize}
\end{slide}
