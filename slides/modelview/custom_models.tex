%%%%%%%%%%%%%%%%%%%%%%%%%%%%%%%%%%%%%%%%%%%%%%%%%%%%%%%%%%%%%%%%%%%%%%%%%%
%
% Copyright (c) 2008-2011, Nokia Corporation and/or its subsidiary(-ies).
% All rights reserved.
%
% This work, unless otherwise expressly stated, is licensed under a
% Creative Commons Attribution-ShareAlike 2.5.
%
% The full license document is available from
% http://creativecommons.org/licenses/by-sa/2.5/legalcode .
%
%%%%%%%%%%%%%%%%%%%%%%%%%%%%%%%%%%%%%%%%%%%%%%%%%%%%%%%%%%%%%%%%%%%%%%%%%%

\subsection{Custom Models}

%----------------------------------------------------------------------                                                                                                         
\begin{slide}{1593}\frametitle{Implementing a Model}
\begin{itemize}
\item Variety of classes to choose from
  \begin{itemize}
  \item \iCls[\textbf]{QAbstractListModel}
    \begin{itemize}
    \item One dimensional list
    \end{itemize}
  \item \iCls[\textbf]{QAbstractTableModel}
    \begin{itemize}
    \item Two-dimensional tables
    \end{itemize}
  \item \iCls[\textbf]{QAbstractItemModel}
    \begin{itemize}
    \item Generic model class
    \end{itemize}
  \item \iCls[\textbf]{QStringListModel}
    \begin{itemize}
    \item One-dimensional model
    \item Works on string list
    \end{itemize}
  \item \iCls[\textbf]{QStandardItemModel}
    \begin{itemize}
    \item Model that stores the data
    \end{itemize}
 \end{itemize}
\item[]
\item \textbf{Notice:} Need to subclass \textit{abstract} models
\end{itemize}
\end{slide}

% ----------------------------------------------------------------------
\begin{slide}[fragile]{1592}\frametitle{Step 1: Read Only List Model}
  \begin{cpp}
class MyModel: public QAbstractListModel {
public:
  // return row count for given parent
  int rowCount( const QModelIndex &parent) const;
  // return data, based on current index and requested role
  QVariant data( const QModelIndex &index, 
                 int role = Qt::DisplayRole) const;
};    
  \end{cpp}
\demo{modelview/ex-stringlistmodel}
\end{slide}


% ----------------------------------------------------------------------
\begin{slide}[fragile]{1591}\frametitle{Step 2: Suppling Header Information}
 \begin{cpp}
QVariant MyModel::headerData(int section, 
                             Qt::Orientation orientation, 
                             int role) const
{
  // return column or row header based on orientation
}
   \end{cpp}
\demo{modelview/ex-stringlistmodel-2}
\end{slide}


% ----------------------------------------------------------------------
\begin{slide}[fragile]{1590}\frametitle{Step 3: Enabling Editing}
  \begin{cpp}
// should contain Qt::ItemIsEditable
Qt::ItemFlags MyModel::flags(const QModelIndex &index) const 
{
 return QAbstractListModel::flags() | Qt::ItemIsEditable;
}

// set role data for item at index to value
bool MyModel::setData( const QModelIndex & index, 
                const QVariant & value, 
                int role = Qt::EditRole)
{
  ... = value; // set data to your backend
  emit dataChanged(topLeft, bottomRight); // if successful
}
\end{cpp}
\demo{modelview/ex-stringlistmodel-3}
\end{slide}


% ----------------------------------------------------------------------
\begin{slide}[fragile]{1589}\frametitle{Step 4: Row Manipulation}
    \begin{cpp}
// insert count rows into model before row
bool MyModel::insertRows(int row, int count, parent) {      
   beginInsertRows(parent, first, last);
   // insert data into your backend
   endInsertRows();
}

// removes count rows from parent starting with row
bool MyModel::removeRows(int row, int count, parent) {
   beginRemoveRows(parent, first, last);
   // remove data from your backend
   endRemoveRows();
}      
    \end{cpp}
\demo{modelview/ex-stringlistmodel-4}
\end{slide}


% ----------------------------------------------------------------------
\begin{slide}{1587}\frametitle{Lab: City List Model}
  \begin{itemize}
  \item Please implement a City List Model
  \item Given:
    \begin{itemize}
    \item Start with solution of \texttt{modelview/lab-cities-standarditem}
    \end{itemize}
  \item Your Task:
    \begin{itemize}
    \item Rebase \iCls{CityModel} to \iCls{QAbstractListModel}
    \end{itemize}
  \item \textbf{Optional}
    \begin{itemize}
    \item Make the model editable
    \item Enable adding/removing cities
   \end{itemize}
 \end{itemize}
\lab{modelview/lab-cities-standarditem}
 
\end{slide}



