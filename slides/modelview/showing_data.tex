%%%%%%%%%%%%%%%%%%%%%%%%%%%%%%%%%%%%%%%%%%%%%%%%%%%%%%%%%%%%%%%%%%%%%%%%%%
%
% Copyright (c) 2008-2011, Nokia Corporation and/or its subsidiary(-ies).
% All rights reserved.
%
% This work, unless otherwise expressly stated, is licensed under a
% Creative Commons Attribution-ShareAlike 2.5.
%
% The full license document is available from
% http://creativecommons.org/licenses/by-sa/2.5/legalcode .
%
%%%%%%%%%%%%%%%%%%%%%%%%%%%%%%%%%%%%%%%%%%%%%%%%%%%%%%%%%%%%%%%%%%%%%%%%%%

\subsection{Showing Simple Data}


% ----------------------------------------------------------------------
\begin{slide}[fragile]{1584}
\frametitle{Convenience Model - \iCls{QStandardItemModel}}
\flushedImage{modelview/images/standarditemmodel.png}
  \begin{itemize}
  \item \iCls{QStandardItemModel}
    \begin{itemize}
   \item Classic item-based approach
    \item Only practical for small sets of data
 \end{itemize}
   \begin{cpp}
model = new QStandardItemModel(parent);
item = new QStandardItem("A (0,0)");
model->appendRow(item);
model->setItem(0, 1, new QStandardItem("B (0,1)"));
item->appendRow(new QStandardItem("C (0,0)"));
\end{cpp}
\item \textit{"C (0,0)" - Not visible because table view is only 2-dimensional}
\end{itemize}
\doc{qstandarditemmodel.html}{QStandardItemModel}
\demo{modelview/ex-QStandardItemModel}
\end{slide}

% ----------------------------------------------------------------------
\begin{slide}
\frametitle{Other Ready-Made Models}
\begin{itemize}
\item \iCls{QStringListModel} - Provides a model of a \iCls{QStringList}
\item \iCls{QFileSystemModel} - Provides a model of a local filesystem
\item \iCls{QSqlQueryModel} - Provides a model of a SQL result set
\item \iCls{QSqlTableModel} - Provides a model of a single SQL table
\end{itemize}
\end{slide}



% ----------------------------------------------------------------------
\begin{slide}[fragile]{}
\frametitle{Sharing Models and Selection}
  \flushedImage{modelview/images/simpleviews}
  \begin{cpp}// create a model
model = ... // with some items
// and several views
list = ...; table = ...; tree = ...;
// You can share model with views
list->setModel(model);
table->setModel(model);
tree->setModel(model);
// Even can share selection
list->setSelectionModel(tree->selectionModel());
table->setSelectionModel(tree->selectionModel());
\end{cpp}
\end{slide}

% ----------------------------------------------------------------------
\begin{slide}[fragile]{}
\frametitle{Finishing Touch}
\begin{itemize}
  \item Customizing view headers
  \begin{cpp}// set horizontal headers
model->setHorizontalHeaderItem(0, new QStandardItem("Column 1");
model->setHorizontalHeaderItem(1, new QStandardItem("Column 2");
// hide vertical headers on table
table->verticalHeader().hide();
\end{cpp}

  \item Customizing items
  \begin{cpp}item->setEditable(false); // disable edit
item->setCheckable(true); // checkbox on item
\end{cpp}

  \item Customizing selection
  \begin{cpp}// allow only to select single rows on table
table->setSelectionBehavior(QAbstractItemView::SelectRows);
table->setSelectionMode(QAbstractItemView::SingleSelection);
\end{cpp}

  \item \demo{modelview/ex-showdata}
\end{itemize}
\end{slide}

% ----------------------------------------------------------------------
\begin{slide}{}
\frametitle{Selections - \iCls{QItemSelectionModel}}
\begin{itemize}
  \item Keeps track of selected items in a views
  \item Not a \iCls{QAbstractItemModel}, just \iCls{QObject}
  \item \iCls{QItemSelectionModel} API
  \begin{itemize}
    \item \iClsFn{QItemSelectionModel}{currentIndex}
    \item \textit{signal} \iClsFnPar{QItemSelectionModel}{currentChanged}{current, previous}
    \item \iCls{QItemSelection} \iClsFn{QItemSelectionModel}{selection}
      \begin{itemize}
        \item List of selection ranges
      \end{itemize}
      \item \textit{signal} \iClsFnPar{QItemSelectionModel}{selectionChanged}{selected, deselected}
  \end{itemize}
\end{itemize}
\end{slide}


% ----------------------------------------------------------------------
\begin{slide}{1580}
\frametitle{Meet the City Engine}
  \flushedImage{modelview/images/citymodelcsv}
  \begin{itemize}
  \item Sample Model Data
    \begin{itemize}
    \item 62 most populous cities of the world
    \item Data in CSV file
   \end{itemize}
 \item Data columns
    \begin{itemize}
    \item \textit{City | Country | Population | Area | Flag}
   \end{itemize}
 \medskip
 \item Provided in \texttt{CityEngine} class
   \begin{itemize}
   \item Internal implementation is hidden
   \item Code located in \href{run:modelview/cityengine}{\beamerbutton{modelview/cityengine}}
   \end{itemize}
 \end{itemize}
\end{slide}

\begin{slide}[fragile]{1579}
\frametitle{The Backend CityEngine API}
 \begin{cpp}
class CityEngine : public QObject {
  // returns all city names
  QStringList cities() const;
  // returns country by given city name 
  QString country(const QString &cityName) const;
  // returns population by given city name
  int population(const QString &cityName) const;
  // returns city area by given city name
  qreal area(const QString &cityName) const;
  // returns country flag by given country name
  QIcon flag(const QString &countryName) const;
  // returns all countries
  QStringList countries() const;    
  // returns city names filtered by country
  QStringList citiesByCountry(const QString& countryName) const;
};
 \end{cpp}
\end{slide}

% ----------------------------------------------------------------------
\begin{slide}{1578}
\frametitle{Lab: Tree Model for CityEngine}
  \begin{itemize}
    \item Using \iCls{QStandardItemModel}:
    \begin{itemize}
      \item Implement \texttt{setupTreeModel()} in \texttt{mainwindow.cpp}
      \item Display cities grouped by countries
    \end{itemize}
    \item[]
\centeredImage{modelview/images/sol-treecityengine}


 \end{itemize}
  \lab{modelview/lab-treecityengine}
\end{slide}


