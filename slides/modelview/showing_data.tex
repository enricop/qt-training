%%%%%%%%%%%%%%%%%%%%%%%%%%%%%%%%%%%%%%%%%%%%%%%%%%%%%%%%%%%%%%%%%%%%%%%%%%
%
% Copyright (c) 2008-2011, Nokia Corporation and/or its subsidiary(-ies).
% All rights reserved.
%
% This work, unless otherwise expressly stated, is licensed under a
% Creative Commons Attribution-ShareAlike 2.5.
%
% The full license document is available from
% http://creativecommons.org/licenses/by-sa/2.5/legalcode .
%
%%%%%%%%%%%%%%%%%%%%%%%%%%%%%%%%%%%%%%%%%%%%%%%%%%%%%%%%%%%%%%%%%%%%%%%%%%

\subsection{Showing Simple Data}


% ----------------------------------------------------------------------
\begin{slide}[fragile]{1584}\frametitle{\iCls{QStandardItemModel} -
    Convenient Model}
\flushedImage{qml-presenting-data/images/standarditemmodel.png}
  \begin{itemize}
  \item \iCls{QStandardItemModel}
    \begin{itemize}
   \item Classic item-based approach
    \item Only practical for small sets of data
 \end{itemize}
   \begin{cpp}
model = new QStandardItemModel(parent);
item = new QStandardItem("A (0,0)");
model->appendRow(item);
model->setItem(0, 1, new QStandardItem("B (0,1)"));
item->appendRow(new QStandardItem("C (0,0)"));
\end{cpp}
\item \textit{"B (0,1)" and "C (0,0)" - Not visible. (list view is only 1-dimensional)}
\end{itemize}
\doc{qstandarditemmodel.html}{QStandardItemModel}
\end{slide}

% ----------------------------------------------------------------------
\begin{slide}[fragile]{1582}\frametitle{Finishing Touch}
  \begin{itemize}
  \item Customizing header, footer and highlight
  \item[]
\begin{qml}
\qtt{ListView~\{}\\
\qtt{~~~~\qc{type}{header}:~\qc{class}{Component}~\{}\\
\qtt{~~~~~~~~\qc{class}{Rectangle}~\{}\\
\qtt{~~~~~~~~~~~~~\qc{type}{width}:~parent.width;~\qc{type}{height}:~\qc{number}{10};~\qc{type}{color}:~\qc{string}{"pink"}}\\
\qtt{~~~~~~~~\}}\\
\qtt{~~~~\}}\\
\qtt{~~~~\qc{type}{footer}:~\qc{class}{Component}~\{}\\
\qtt{~~~~~~~~\qc{class}{Rectangle}~\{}\\
\qtt{~~~~~~~~~~~~~\qc{type}{width}:~parent.width;~\qc{type}{height}:~\qc{number}{10};~\qc{type}{color}:~\qc{string}{"lightblue"}}\\
\qtt{~~~~~~~~\}}\\
\qtt{~~~~\}}\\
\qtt{~~~~\qc{type}{highlight}:~\qc{class}{Component}~\{}\\
\qtt{~~~~~~~~\qc{class}{Rectangle}~\{}\\
\qtt{~~~~~~~~~~~~~\qc{type}{width}:~parent.width;~\qc{type}{color}:~\qc{string}{"lightgray"}}\\
\qtt{~~~~~~~~\}}\\
\qtt{~~~~\}}\\
\qtt{\}}\\
\end{qml}
\item[] \demo{qml-presenting-data/ex-model-views/ex-view-current-item.qml}
\end{itemize}
\end{slide}


% ----------------------------------------------------------------------
\begin{slide}[fragile]{1581}\frametitle{Selections - \iCls{QItemSelectionModel}}
  \begin{itemize}
   \item Keeps track of selected items in view
    \item Not a \iCls{QAbstractItemModel}, just \iCls{QObject}
 \item \texttt{QItemSelectionModel API}
    \begin{itemize}
    \item \texttt{currentIndex()}
    \item \textit{signal} \texttt{currentChanged(current, previous)}
    \item \texttt{QItemSelection selection()}
      \begin{itemize}
      \item List of selection ranges
      \end{itemize}
    \item \texttt{select( ... )}
    \item \textit{signal} \texttt{selectionChanged(selected, deselected)}
  \end{itemize}
  \begin{cpp}
// selecting a range
selection = new QItemSelection(topLeft, bottomRight);
view->selectionModel()->select(selection);    
  \end{cpp}
  \end{itemize}
\end{slide}

% ----------------------------------------------------------------------
\begin{slide}{1580}\frametitle{Meet the City Engine}
  \flushedImage{modelview/images/citymodelcsv}
  \begin{itemize}
  \item Our Demo Model
    \begin{itemize}
    \item 62 most populous cities of the world
    \item Data in CSV file
   \end{itemize}
 \item Data Columns
    \begin{itemize}
    \item \textit{City | Country | Population | Area | Flag}
   \end{itemize}
 \item Implemented as data backend
   \begin{itemize}
   \item Internal implementation is hidden
   \item Code in \texttt{CityEngine} class
  \end{itemize}
 \end{itemize}
\centeredImage{modelview/images/citymodel}
\end{slide}

\begin{slide}[fragile]{1579}\frametitle{Our Backend CityEngine API}
 \begin{cpp}
public CityEngine : public QObject {
  // returns all city names
  QStringList cities() const;
  // returns country by given city name 
  QString country(const QString &cityName) const;
  // returns population by given city name
  int population(const QString &cityName) const;
  // returns city area by given city name
  qreal area(const QString &cityName) const;
  // returns country flag by given country name
  QIcon flag(const QString &countryName) const;
  // returns all countries
  QStringList countries() const;    
  // returns city names filtered by country
  QStringList citiesByCountry(const QString& countryName) const;
};
 \end{cpp}
 \begin{itemize}
 \item \demo{/modelview/ex-standarditemmodel}  
\end{itemize}
\end{slide}

% ----------------------------------------------------------------------
\begin{slide}{1578}\frametitle{Lab: Tree Model for CityEngine}
  \begin{itemize}
    \item Implement \texttt{createTreeModel()} in \texttt{mainwindow.cpp}
    \item Display cities grouped by countries
    \item \textbf{Optional}
      \begin{itemize}
     \item Provide a find for country field.
     \item Found countries shall be selected
      \end{itemize}
\centeredImage{modelview/images/sol-treecityengine}


 \end{itemize}
  \lab{modelview/lab-treecityengine}
\end{slide}


