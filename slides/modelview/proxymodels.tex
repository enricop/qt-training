%%%%%%%%%%%%%%%%%%%%%%%%%%%%%%%%%%%%%%%%%%%%%%%%%%%%%%%%%%%%%%%%%%%%%%%%%%
%
% Copyright (c) 2008-2011, Nokia Corporation and/or its subsidiary(-ies).
% All rights reserved.
%
% This work, unless otherwise expressly stated, is licensed under a
% Creative Commons Attribution-ShareAlike 2.5.
%
% The full license document is available from
% http://creativecommons.org/licenses/by-sa/2.5/legalcode .
%
%%%%%%%%%%%%%%%%%%%%%%%%%%%%%%%%%%%%%%%%%%%%%%%%%%%%%%%%%%%%%%%%%%%%%%%%%%

\subsection{Proxy Models}
% ----------------------------------------------------------------------
\begin{slide}[fragile]{1586}\frametitle{Proxy Model - \iCls{QSortFilterProxyModel}}
  \begin{itemize}
  \item \iCls{QSortFilterProxyModel}
    \begin{itemize}
    \item Transforms structure of source model
    \item Maps indexes to new indexes
    \end{itemize}
\begin{cpp}
view = new QListView(parent);
// insert proxy model between model and view
proxy = new QSortFilterProxyModel(parent);
proxy->setSourceModel(model);
view->setModel(proxy);
\end{cpp}
  \end{itemize}
\textit{Note:} Need to load all data to sort or filter
\end{slide}
% ----------------------------------------------------------------------
\begin{slide}[fragile]{1585}\frametitle{Sorting/Filtering - \iCls{QSortFilterProxyModel}}
  \begin{itemize}
  \item \textbf{Filter with Proxy Model}
\begin{cpp}
// filter column 1 by "India"
proxy->setFilterWildcard("India");
proxy->setFilterKeyColumn(1); 
\end{cpp}
\item \textbf{Sorting with Proxy Model}
\begin{cpp}
// sort column 0 ascending
view->setSortingEnabled(true);
proxy->sort(0, Qt::AscendingOrder);
\end{cpp}
  \end{itemize}
\medskip
\begin{itemize}
\item \textbf{Filter via \iCls{QLineEdit} signal}
\begin{cpp}
connect(m_edit, SIGNAL(textChanged(QString)), 
    proxy, SLOT(setFilterWildcard(QString)));      
\end{cpp}
\end{itemize}
\demo{modelview/ex-sortfiltertableview}
\end{slide}

